% Copyright 2006 INRIA

\chapter{Introduction}\label{chap:intro}

Zenon is an automatic theorem-prover whose main claim to fame is that
it produces actual proofs of theorems.
Correctness is an explicit non-goal in the design of Zenon: the proof
output needs to be checked by another program before the theorem is
considered proved.
  As a consequence, Zenon is not
designed for direct use by humans, but rather for interfacing into
formal proof systems, such as interactive proof assistants and
non-interactive proof checkers.

This document gives a detailed description of all the information
needed by users of Zenon.  Chapter~\ref{chap:install} describes how to
compile and install Zenon from source code.  Chapter~\ref{chap:options}
describes all the command-line options accepted by Zenon.
Chapter~\ref{chap:input-zen} describes the native input syntax of
Zenon; chapter~\ref{chap:input-tptp} describes the TPTP input syntax;
chapter~\ref{chap:input-coq} describes the Focal/Coq-style input
syntax.  Chapter~\ref{chap:messages} describes the error and warning
messages that Zenon may output when searching for a proof.
