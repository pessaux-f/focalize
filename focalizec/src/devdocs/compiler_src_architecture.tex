%$

\section{\focalizec\ source tree}
The compiler source tree is split into several directories and files
described below. The root of the compiler sources is located in
{\tt focalize/focalizec}. All directories and files given below are
relative to this root. In some of the directories is a sub-directory
called {\tt odoc}. It can be safely ignored and is only used when
invoking verb+make doc+ to store the \ocamldoc\ output (HTML
documentation extracted from the source code of \focalize). In the
below enumeration (d x) stands for ''directory'' whose content is at
$x$ nesting level of the root (i.e. of {\tt focalize/focalizec}) and
(f) stands for ''file".

\begin{itemize}
\item {\tt Makefile} (f) : The toplevel Makefile building the
  compiler, the standard library, the extra libraries, the
  documentation generator, the \focalize\ documentations and the
  contributions. This Makefile indeed trigger the build process in all
  the deeper directories.
\item {\tt Makefile.common} (f) : Defines once for all the implicit
  rules and suffixes that will be used in deeper Makefiles.
\item {\tt .config\_var} (f) : File generated during the ``configure''
  process (invoked by the file {\tt configure} below and that records
  the various things among installation paths, installed commands and
  tools, \ldots
\item {\tt Makefile.config} (f) : Defines once for all the commands
  according to the information the ``configure'' process recorded. It
  especially include the {\tt .config\_var} file that kept trace of what
  the ``configure'' process defined.
\item {\tt TAGS} (f) : Documentation file where developers are
  invited to note all the tags added to the development tree under CVS,
  with a description of the status of the development tree when the
  tag was set (or the reason why to put a tag).
\item {\tt TODO} (f) : More or less describes points that are still
  pending.
\item {\tt configure} (f) : Script used to detect available commands
  and tools, to ask the user where to install \focalize\ components,
  \ldots Parts of its output is stored in the above file
  {\tt .config\_var}.
\item {\tt doc\_src} (d 1) : The directory containing all the
  documentations at destination of the users.
  \begin{itemize}
  \item {\tt Makefile} (f) : Makefile triggering the documentations
    build.
  \item {\tt html} (d 2) : The \focalize\ WEB site in HML format.
    \begin{itemize}
    \item {\tt Includes} (d 3)
      \begin{itemize}
      \item {\tt aftertitle-eng.html} (f)
      \item {\tt aprestitre-fra.html} (f)
      \item {\tt avanttitre-fra.html} (f)
      \item {\tt basdepage-fra.html} (f)
      \item {\tt beforetitle-eng.html} (f)
      \item {\tt bottomofpage-eng.html} (f)
      \item {\tt copyright-eng.html} (f)
      \item {\tt copyright-fra.html} (f)
      \item {\tt doctype} (f)
      \item {\tt endofpage-eng.html} (f)
      \item {\tt findepage-fra.html} (f)
      \item {\tt hautdepage-fra.html} (f)
      \item {\tt htmlc-version.html} (f)
      \item {\tt maquette-eng.html} (f)
      \item {\tt maquette-fra.html} (f)
      \item {\tt powered\_by\_caml.html} (f)
      \item {\tt topofpage-eng.html} (f)
      \end{itemize}
    \item {\tt Makefile} (f)
    \item {\tt Makefile.html} (f)
    \item {\tt images} (d 3) : Directory containing images for the WEB
      site.
      \begin{itemize}
      \item {\tt focal\_picture.jpg} (f) : A niiiice 3D picture done with
        Povray \smiley.
      \end{itemize}
    \end{itemize}
  \item {\tt man} (d 2) : Directory containing the ``man'' manual.
    \begin{itemize}
    \item {\tt Makefile} (f)
    \item {\tt focalizec.env} (f)
    \item {\tt focalizec.man} (f) : ``man'' for \focalizec.
    \item {\tt focalizedep.man} (f) : ``man'' for \focalizedep.
    \end{itemize}
  \item {\tt  tex} (d 2) : Documentations in \latex\ format.
    \begin{itemize}
    \item {\tt Makefile} (f)
    \item {\tt refman} (d 3) : Directory containing the reference manual.
      \begin{itemize}
      \item {\tt Makefile} (f)
      \item {\tt basic\_concepts.tex} (f)
      \item {\tt bibli.bib} (f) : Contains bibliography references.
      \item {\tt building\_species.tex} (f)
      \item {\tt compiler\_err\_msgs.tex} (f) : About the \focalizec\
        compiler error messages.
      \item {\tt compiler\_opts.tex} (f) : About the \focalizec\
        compiler command line options.
      \item {\tt constructs\_syntax.tex} (f)
      \item {\tt doc\_gen.tex} (f) : About the \focalize\ automated
        documentation generation (option {\tt focalize-doc} of
        \focalizec).
      \item {\tt fullpage.sty} (f)
      \item {\tt glimpse.tex} (f) : Short presentation of the language
        to give a first an rapid taste.
      \item {\tt header\_html\_snapshot.ps} (f)
      \item {\tt introduction.tex} (f)
      \item {\tt lexical\_conventions.tex} (f) : About \focalize\ lexical
        conventions, description of the tokens and syntax.
      \item {\tt macros.hva} (f)
      \item {\tt macros.tex} (f) : \latex macros used in the reference
        manual.
      \item {\tt mathml\_snapshot.ps} (f)
      \item {\tt more\_on\_meths.tex} (f)
      \item {\tt motivations.tex} (f)
      \item {\tt output.tex} (f) : About what the \focalizec\ compiler
        generates.
      \item {\tt pres\_requir.tex} (f)
      \item {\tt proofs.tex} (f)
      \item {\tt refman.html} (f)
      \item {\tt refman.hva} (f)
      \item {\tt refman.pdf} (f)
      \item {\tt refman.tex} (f) : Entry point of the reference manual.
      \item {\tt syntaxdef.hva} (f)
      \item {\tt syntaxdef.sty} (f)
      \end{itemize}
    \end{itemize}
  \end{itemize}
\item {\tt src} (d 1) : Root of the sources of \focalizec\ and \focalizedep.
  \begin{itemize}
  \item {\tt attic} (d) : Contains miscellaneous interesting parts of
    code that have been written one day, that are not used anymore, but
    that we didn't want to trash in case it could serve in the future.
    \begin{itemize}
    \item {\tt ast.mli}, {\tt lib\_pp.ml}, {\tt lib\_pp.mli},
      {\tt misc.ml}, {\tt new2old.ml}, {\tt new2old.mli},
      {\tt oldsourcify.ml}, {\tt oldsourcify.mli},
      {\tt printer.ml}, {\tt printtree.ml},\\
      {\tt string\_search\_stuff.ml}
    \item May be more\ldots
    \end{itemize}
  \item {\tt basement} (d 2) : Contains the early basic bricks of the
    compiler.
    \begin{itemize}
    \item {\tt Makefile} (f) : Trigger build in this directory.
    \item {\tt configuration.ml} (f) : Manages version number (and
      by side effect the {\tt -help} option output text) and the
      command line options flags.
    \item {\tt configuration.mli} (f)
    \item {\tt files.ml} (f) : Contains primitives dealing with
      files, taking into account search path, files suffixes,
      read/write object files, \ldots
    \item {\tt files.mli} (f)
    \item {\tt handy.ml} (f) : Various general functions, on lists,
      pretty-printers, font setting\ldots
    \item {\tt handy.mli} (f)
    \item {\tt installation.ml} (f) : File automatically generated
      by the ``configure'' process and record as \ocaml\ source, the
      various installation paths and commands the \focalizec\
      compiler need to know when dealing with \focalize\ source
      files.
    \item {\tt location.ml} (f) : Defines the type of
      ``locations'', i.e. point in a source file and gives
      primitives to work with them.
    \item {\tt location.mli} (f)
    \item {\tt miscHelpers.ml} (f) : Mostly contains 1 function
      used during several passes to bind formal names to their
      types from a type scheme and a list of formal names. May be
      could somewhere else to save one file\ldots
    \item {\tt miscHelpers.mli} (f)
    \item {\tt parsetree.mli} (f) : The description of the AST.
    \item {\tt parsetree\_utils.ml} (f) : General utilities to process
      the AST.
    \item {\tt parsetree\_utils.mli} (f)
    \item {\tt types.ml} (f) : The description of the types structure
      and operations to work with them. Because types are a complex
      structure with invariants, they are exported as opaque to
      prevent breaking these invariants. Hence, any function dealing
      with the intimate representation of a type must be in this file
      since his is the only location where this representation is visible.
    \item {\tt types.mli} (f) :
    \end{itemize}
  \item {\tt ccodegen} (d) : Dedicated to the C code generation
    back-end. Maintained by Julien Blond and is not currently included in
    \focalize.
    \begin{itemize}
    \item \ldots not stable.
    \end{itemize}
  \item {\tt commoncodegen} (d 2) : Contains processing common to all
    target languages that is performed in any case before emitting the
    target code.
    \begin{itemize}
    \item {\tt Makefile} (f) : 
    \item {\tt abstractions.ml} (f) : Performs the synthesis of
      $\lambda$-lifted things, summarising def and decl-dependencies
      and dependencies on collection parameters methods. It especially
      performs the final (complete) computation of dependencies on
      collection parameters, applying the various completion rules on
      the dependencies found previously by the rules [TYPE] and [BODY].
    \item {\tt abstractions.mli} (f)
    \item {\tt context.mli} (f) : The structure of the ``context'',
      i.e. the structure inductively passed to each function during
      this pass and that group into one single record type all the
      information needed by the various functions. This a handy way to
      prevent from having numerous parameters to pass each time to the
      functions.
    \item {\tt externals\_generation\_errs.ml} (f) : Exceptions that
      can be raised when generating code for ``external'' definitions,
      i.e. definitions that are not written in native \focalize
      language and used to interface with other programming languages.
    \item {\tt externals\_generation\_errs.mli} (f)
    \item {\tt minEnv.ml} (f) : Compute the ``\coq minimal typing
      environment''.
    \item {\tt minEnv.mli} (f)
    \item {\tt misc\_common.ml} (f) : Various type definitions and
      functions used in several points during this pass.
    \item {\tt misc\_common.mli} (f)
    \item {\tt recursion.ml} (f) : Deals the the recursive
      functions processing.
    \item {\tt recursion.mli} (f)
    \item {\tt visUniverse.ml} (f) : Compute the ``visible
      universe''.
    \item {\tt visUniverse.mli} (f)
    \end{itemize}
  \item {\tt contribs} (d 2) : Contains contribution source codes written
    by users that are not developers of the compiler and its framework.
    \begin{itemize}
    \item {\tt Makefile} (f)
    \item {\tt automata} (d 3) : Hierarchical automata. Maintained by
      Philippe Ayrault.
      \begin{itemize}
      \item {\tt Makefile} (f)
      \item {\tt gen\_def.fcl} (f)
      \item {\tt main.fcl} (f)
      \item {\tt request.fcl} (f)
      \item {\tt switch\_automata.fcl} (f)
      \item {\tt switch\_recovery\_automata.fcl} (f)
      \item {\tt switch\_recovery\_normal\_automata.fcl} (f)
      \item {\tt switch\_recovery\_reverse\_automata.fcl} (f)
      \end{itemize}
    \item {\tt utils} (d 3) : Various utilities. Maintained by Philippe
      Ayrault.
      \begin{itemize}
      \item {\tt Makefile} (f)
      \item {\tt pair.fcl} (f)
      \item {\tt peano.fcl} (f)
      \end{itemize}
    \item {\tt voter} (d 3) : Model of a generic voter. Maintained by
      Philippe Ayrault.
      \begin{itemize}
      \item {\tt Makefile} (f)
      \item {\tt etat\_vote.fcl} (f)
      \item {\tt main.fcl} (f)
      \item {\tt num\_capteur.fcl} (f)
      \item {\tt value.fcl} (f)
      \item {\tt vote.fcl} (f)
      \end{itemize}
    \end{itemize}
  \item {\tt coqcodegen} (d 2) : Contains \coq\ code generation back-end.
    \begin{itemize}
    \item {\tt FOCAL\_COQ\_MAPPINGS} (f) : Short (incomplete) description
      of how \focalize\ names are mapped onto \coq\ names in the
      generated source code.
    \item {\tt Makefile} (f)
    \item {\tt main\_coq\_generation.ml} (f) : Entry point of \coq\ code
      generation. Start code generation for a compilation unit, deals
      with toplevel entities ({\tt let}, theorems, expressions and
      type definitions) and triggers processing of species,
      collections.
    \item {\tt rec\_let\_gen.ml} (f) : Manages parts of the code
      generation for recursive functions with stuff dedicated to
      \coq.. Especially, generates the termination lemmas and
      transforms arguments of a recursive into a tuple.
    \item {\tt rec\_let\_gen.mli} (f)
    \item {\tt species\_coq\_generation.ml} (f) : Main part of code
      generation for species and collections.
    \item {\tt species\_coq\_generation.mli} (f)
    \item {\tt species\_record\_type\_generation.ml} (f) : Deals with the
      generation of the record type representing species and
      collections. Also generates code for expressions.
    \item {species\_record\_type\_generation.mli} (f)
    \item {\tt type\_coq\_generation.ml} (f) : Deals with code generation
      for (toplevel) type definitions.
    \item {\tt type\_coq\_generation.mli} (f)
    \end{itemize}
  \item {\tt devdocs} (d 2) : The current documentation explaining the
    compiler's architecture and mechanisms.
    \begin{itemize}
    \item {\tt foc2ocaml.tex}, {\tt pending.txt} (f) :
      Oldies. Should disappear.
    \item {\tt legacy.tex} (f) : The main \latex file. Compile it to
      get the current documentation.
    \item {\tt macros.tex} (f) : Miscellaneous macros to make life
      easier \smiley.
    \item {\tt mathpartir.sty} : Inference rules package.
    \item {\tt phd\_changes.tex} : Things we changed, corrected,
      enhanced since Virgile Prevosto's PhD Thesis.
    \end{itemize}
  \item {\tt docgen} (d 2) : Source for \focalizec\ automated
    documentation generation ({\tt -focalize-doc} option of the
    compiler) as XML files.
    \begin{itemize}
    \item {\tt Makefile} (f) :
    \item {\tt doc\_lexer.mll} : Lexer to scan ``@''-markups in a
      \focalize\ source code.
    \item {\tt env\_docgen.ml} : Contains the documentation environment
      mechanisms. This environment maps methods names onto some optional MathML
      and LaTeX code.
    \item {\tt env\_docgen.mli} (f)
    \item {\tt focdoc.css} (f) : \focalizedoc-XML style sheet.
    \item {\tt focdoc.dtd} (f) :
    \item {\tt focdoc.rnc} (f) :
    \item {\tt focdoc.xsd} (f) :
    \item {\tt focdoc2html.xsl} : Transforms some \focalizedoc-XML into
        HTML + MathML.
    \item {\tt focdoc2tex.xsl} (f) : Transforms some \focalizedoc-XML
        into \latex.
    \item {\tt main\_docgen.ml} (f) : Engine extracting information from
        a \focalize\ source code to produce \focalizedoc-XML.
    \item {\tt main\_docgen.mli} (f)
    \item {\tt mmlctop2\_0.xsl} (f) : Transforms HTML + MathML into
      HTML.
    \item {\tt proposition.xsl} (f) : : Processing of logical
      expressions from \focalizedoc-XML to HTML + MathML.
    \item {\tt proposition2tex.xsl} (f) : Processing of logical
      expressions from \focalizedoc-XML to \latex.
    \item {\tt utils\_docgen.ml} (f) : Various helpers used for XML
      production.
    \item {\tt utils\_docgen.mli} (f)
    \end{itemize}
  \item {\tt extlib} (d 2) : Libraries higher level than the basic
    standard library. Especially contains formal calculus structures.
    \begin{itemize}
    \item {\tt Makefile} (f)
    \item {\tt access\_control} (d 3) : Access control
      policies. Maintained by Lionel Habib and Mathieu Jaume.
      \begin{itemize}
      \item {\tt Makefile} (f)
      \item {\tt access\_control.fcl} (f)
      \item {\tt ensembles\_finis.fcl} (f)
      \item {\tt hru.fcl} (f)
      \item {\tt rbac.fcl} (f)
      \item {\tt ticket.fcl} (f)
      \item {\tt tm} (d 4)
        %% \begin{itemize}  Latex says "Too deeply nested"...
        \item \hspace{0.7cm} {\tt graph.ml} (f)
        \item \hspace{0.7cm} {\tt trust\_management.fcl} (f)
        %% \end{itemize}
      \item{\tt unix.fcl} (f)
      \end{itemize}
    \item {\tt algebra} (d 3) : Formal calculus library. Maintained by
      Renaud Rioboo.
      \begin{itemize}
      \item {\tt Makefile} (f)
      \item {\tt additive\_law.fcl} (f)
      \item {\tt arrays.fcl} (f)
      \item {\tt arrays\_externals.v} (f)
      \item {\tt big\_integers.fcl} (f)
      \item {\tt constants.fcl} (f)
      \item {\tt integers.fcl} (f)
      \item {\tt iterators.fcl} (f)
      \item {\tt multiplicative\_law.fcl} (f)
      \item {\tt parse\_poly.fcl} (f)
      \item {\tt polys\_abstract.fcl} (f)
      \item {\tt product\_structures.fcl} (f)
      \item {\tt quotient\_structures.fcl} (f)
      \item {\tt randoms.fcl} (f)
      \item {\tt randoms\_externals.ml} (f)
      \item {\tt randoms\_externals.v} (f)
      \item {\tt rings\_fields.fcl} (f)
      \item {\tt small\_integers.fcl} (f)
      \item {\tt weak\_structures.fcl} (f)
      \item {\tt weak\_structures\_externals.ml} (f)
      \item {\tt weak\_structures\_externals.v} (f)
      \end{itemize}
    \end{itemize}
  \item {\tt focalizedep} (d 2) : The dependencies generator to create
    Makefiles.
    \begin{itemize}
    \item {\tt Makefile} (f)
    \item {\tt directive\_lexer.mll} (f) : Lexer to scan {\tt open}
      and {\tt use} directives in a \focalize\ source.
    \item {\tt make\_depend.ml} (f) : Source of the generator.
    \end{itemize}
  \item {\tt mlcodegen} (d 2) : Contains \ocaml\ code generation back-end.
    \begin{itemize}
    \item {\tt FOCAL\_ML\_MAPPINGS} (f) : Short (incomplete) description
      of how \focalize\ names are mapped onto \ocaml\ names in the
      generated source code.
    \item {\tt Makefile} (f)
    \item {\tt base\_exprs\_ml\_generation.ml} (f)
    \item {\tt base\_exprs\_ml\_generation.mli} (f)
    \item {\tt main\_ml\_generation.ml} (f)
    \item {\tt main\_ml\_generation.mli} (f)
    \item {\tt misc\_ml\_generation.ml} (f)
    \item {\tt misc\_ml\_generation.mli} (f)
    \item {\tt species\_ml\_generation.ml} (f)
    \item {\tt species\_ml\_generation.mli} (f)
    \item {\tt type\_ml\_generation.ml} (f)
    \item {\tt type\_ml\_generation.mli} (f)
    \end{itemize}
  \item {\tt parser} (d 2) : Contains the lexical and syntactic
    analysers to process \focalize\ source code.
    \begin{itemize}
    \item {\tt Makefile} (f)
    \item {\tt dump\_ptree.ml} (f) : Dumps the AST structure of a file
      in raw text mode. Used for debugging purpose.
    \item {\tt dump\_ptree.mli} (f)
    \item {\tt lex\_file.ml} (f) : Initial attempt to have a small
      ``main'' to check the behaviour of the lexer on a file at the
      early stages of development. This is not used anymore.
    \item {\tt lexer.mll} (f)
    \item {\tt lexer.spec} (f) : Trace of thoughts we had all along
      the time. Initially contained ideas about the language syntax,
      (at lexical level) to make it smooth.
    \item {\tt parse\_file.ml} (f) : Initial attempt to have a small
      ``main'' to check the behaviour of the parser on a file at the
      early stages of development. This is not used anymore.
    \item {\tt parse\_file.mli} (f)
    \item {\tt parser.mly} (f) : The parser description in \ocamlyacc.
    \item {\tt parser.spec} (f) : Trace of thoughts we had all along
      the time. Initially contained ideas about the language syntax,
      (at syntactic level) to make it smooth. Next other ideas were
      added, beyond syntax, more oriented toward the AST structure.
    \item {\tt sourcify.ml} (f) : Dumps the AST structure of a file
      in \focalize\ syntax. Gives back a source code from a parsed
      code. This is mostly useful for debugging purpose. May be used
      bu customers to pretty-print their source, but I don't
      recommand. This more specially serves to report pieces of code in
      a good-looking form in error messages. Each time we need to tell
      the user something involving a piece of code, we pretty-print it
      with functions provided here.
    \item {\tt sourcify.mli} (f)
    \item {\tt sourcify\_token.ml} (f) : Not used anymore. Was used to
      test the lexer at the early stages of development. It was used
      to print the found token.
    \item {\tt test} (d 3) : Oldies originally used to test the
      lexer/parser in the early times. No used anymore.
      \begin{itemize}
      \item {\tt sets\_orders.fcl} (f) : First translation of a file of
        the standard library used ti test the lexer/parser. Not used
        anymore.
      \end{itemize}
    \end{itemize}
  \item {\tt scoper} (d 2) : Old stuff. Was discarded a long time
    ago. The directory is empty by the way.
  \item {\tt stdlib} (d 2) : The standard library. Most of these
    sources are maintained by Renaud Rioboo. A few ones (basics and
    built-ins) are low-level code close to the compiler and maintained
    by the core compiler development team.
    \begin{itemize}
    \item {\tt Makefile} (f)
    \item {\tt basics.fcl} (f) : Basic bricks and internal-related
      stuff. Contains the primitive type definitions and some very
      basic functions for \focalize.
    \item {\tt coq\_builtins.v} (f) : Basic bricks and internal-related
      stuff especially dedicated to \coq\ code.
    \item {\tt generic\_proof\_cases.v} (f) : Contains built-in stuff
      for \coq\ about built-in lists and some tactics.
    \item {\tt lattices.fcl} (f)
    \item {\tt ml\_builtins.ml} (f) : Basic bricks and internal-related
      stuff especially dedicated to \ocaml\ code.
    \item {\tt orders.fcl} (f)
    \item {\tt orders\_and\_lattices.fcl} (f)
    \item {\tt products.fcl} (f)
    \item {\tt quotients.fcl} (f)
    \item {\tt sets.fcl} (f)
    \item {\tt sets\_externals.ml} (f) : External definitions to link
      for \ocaml\ related to {\tt sets.fcl}.
    \item {\tt sets\_orders.fcl} (f)
    \item {\tt sets\_orders\_externals.ml} (f) : External definitions
      to link for \ocaml\ related to {\tt sets\_orders.fcl}.
    \item {\tt strict\_orders.fcl} (f)
    \item {\tt sums.fcl} (f)
    \item {\tt wellfounded.fcl} (f) : Define well foundation,
      termination orders for recursive function definitions for \coq.
    \item {\tt wellfounded\_externals.v} (f) : External definitions
      to link for \ocaml\ related to {\tt wellfounded.fcl}.
    \end{itemize}
  \item {\tt toplevel} (d 2)
    \begin{itemize}
    \item {\tt Makefile} (f)
    \item {\tt exc\_wrapper.ml} (f) : Firewall designed to recover all
      the exceptions that can be raised during a source compilation
      and issues the related error message. Roughly, this code wraps
      the main call to the compilation engine.
    \item {\tt focalizec.ml} (f) : The compiler entry point.
    \item {\tt focalizec.mli} (f)
    \item {\tt fodump.ml} (f) : For debug only, dumps (partially) the
      content of a \focalize\ object file (``.fo'' file).
    \end{itemize}
  \item {\tt typing} (d 2) : Performs the type-checking of the
    language, normal form of species, checks well-formation of
    species, compute def and decl-dependencies, initiate computation
    of dependencies on collection parameters' methods, resolves
    inheritance.
    \begin{itemize}
    \item {\tt Makefile} (f)
    \item {\tt ast\_equal.ml} (f) : Implements equality test between
      logical expressions (indeed, between expressions since logical
      statements fully embed expressions).
    \item {\tt ast\_equal.mli} (f)
    \item {\tt depGraphData.mli} (f) : Defines the structure of the
      dependencies graph used to compute well-formation from def and
      decl-dependencies.
    \item {\tt dep\_analysis.ml} (f) : Handles the dependencies
      analyses on methods of {\tt Self}, i.e. def and decl-dependencies.
    \item {\tt dep\_analysis.mli} (f)
    \item {\tt env.ml} (f) : Structure of the environments used all
      along the compilation. Contains a generic environment
      structure, the specialised scoping environment the specialised
      typing environment, the specialised \ocaml\ code generation
      environment and the specialised \coq\ code generation
      environment. All these environments are instantiations of the
      generic environment structure. Also contains data-structures
      used from type-checking up to code generation to record the
      various information about analysed species, collections,
      methods, types, \ldots Some of these data-structures will appear
      in later passes, picked-up from these environments.
    \item {\tt env.mli} (f)
    \item {\tt infer.ml} : Core engine performing type-checking,
      inheritance resolution, normal form of species, check of
      well-formation, def and decl-dependencies computation and
      triggers the initial computation of dependencies on collection
      parameters' methods.
    \item {\tt infer.mli} (f)
    \item {\tt param\_dep\_analysis.ml} (f) : Handles the initial
      computation of dependencies on collection parameters'
      methods. The computed dependencies will have to be completed
      later in the next pass.
    \item {\tt param\_dep\_analysis.mli} (f)
    \item {\tt scoping.ml} (f) : Performs scoping analysis.
    \item {\tt scoping.mli}
    \item {\tt substColl.ml} (f) : Performs substitution of a
      collection (name/carrier) by another one or {\tt Self} in an
      AST.
    \item {\tt substColl.mli} (f)
    \item {\tt substExpr.ml} (f) : Performs substitution of an
    expression (i.e. {\tt Parsetree.expr}) by another in an AST.
    \item {\tt substExpr.mli} (f)
    \end{itemize}
  \end{itemize}
\item {\tt tests} (d 1) : Various little tests we wrote to shake the
  compiler or to ensure that some particular features/processing
  required by the compiler were really working. This code is most of
  the time meaningless and only design to target a particular point of
  the compilation process. The naming scheme is: a source file name
  starting by ``ko\_\_'' must lead to an error (an error message, not
  a failure of the compiler), a source file name starting by
  ``ok\_\_'' must be accepted by the compiler and lead to effective
  code.
  \begin{itemize}
  \item {\tt Makefile} (f)
  \item {\tt ko\_\_bad\_self\_use.fcl} (f)
  \item {\tt ko\_\_param\_toy.fcl} (f)
  \item {\tt ko\_\_test\_error.fcl} (f)
  \item {\tt ko\_\_test\_rec.fcl} (f)
  \item {\tt ok\_\_baby\_toy.fcl} (f) : A bit of everything, without
    any order.
  \item {\tt ok\_\_baby\_toy\_externals.ml} : External code to link
    for \ocaml\ related to {\tt ok\_\_baby\_toy.fcl}.
  \item {\tt ok\_\_caveat.fcl} (f) : Illustrates the pitfall of sum
    type constructors that have {\bf 1} argument that is a tuple or
    {\bf several} arguments but are hardly different from the syntax
    point of view.
  \item {\tt ok\_\_coll\_outside.fcl} (f) : Example showing how now
    use a collection defined outside the current one (i.e. the used
    one not a parameter) in a collection or a species.
  \item {\tt ok\_\_definition\_72\_rule\_PRM.fcl} (f) : Illustrates
    the need for rule PRM in definition 72 page 153 in Virgile
    Prevosto's PhD.
    \item {\tt ok\_\_in\_example.fcl} (f) : Example showing how now
      use the {\tt in} parameter without allowing their type to be
      anymore a ``ml type''.
    \item {\tt ok\_\_in\_example2.fcl} (f) : Same purpose than for
      {\tt ok\_\_in\_example.fcl}.
    \item {\tt ok\_\_list.fcl} (f) : Example showing how to make lists
      as species in \focalize. It also illustrates the need for the
      extra unit argument to the collection generator
      ``{\tt collection\_create}'' in case it has no arguments, what
      would prevent \ocaml\ from generalising the type of this
      generator. This  example must pass in \ocaml\ but is not yet
      accepted in \coq.
    \item {\tt ok\_\_multiple\_inherit.fcl} (f) : Used to track the
      provenance of methods in case of multiple inheritance.
    \item {\tt ok\_\_need\_inspect\_self.fcl} (f) : Example showing
      how now use the need to inspect the structure of
      {\tt representation} while computing the \\
      ``{\tt used\_species\_parameters\_ty}'' in the case where a
      method has a  def-dependency on {\tt Self}.
    \item {\tt ok\_\_need\_re\_ordering.fcl} (f) : Example showing the
      need of the final re-organisation of methods due to the
      collapsing procedure of properties and {\tt proof of}s.
    \item {\tt ok\_\_odd\_even.fcl} (f) : Example showing simple
      mutually recursive methods that must pass for \ocaml\ code
      generation. Note that \coq\ code generation does not yet handle
      mutual recursion.
    \item {\tt ok\_\_phd\_def\_deps.fcl} (f) : Example of Virgile
      Prevosto's Phd, section 3.9.6, page 56. Illustrates the need for
      the erasing procedure.
    \item {\tt ok\_\_phd\_meths\_gen.fcl} (f) : Example from Virgile
      Prevosto's Phd, section 6.4.3, page 115.
    \item {\tt ok\_\_phd\_sample.fcl} (f) : The initial example given
      in Virgile Prevosto's Phd, section 2.2.2 starting page 14.
    \item {\tt ok\_\_scoping\_tricky.fcl} (f) : Example showing
      scoping occurrences of a same name in case where this name is not
      recursively defined. Combines the stuff with late-binding.
    \item {\tt ok\_\_term\_measure.fcl} (f) : Is designed to be a
      successful use of termination proofs. Still doesn't pass.
    \item {\tt ok\_\_toplevel\_odd\_even.fcl} (f) : Example showing
      simple  mutually recursive toplevel functions that must pass for
      \ocaml\ code generation. Note that \coq\ code generation does
      not yet handle mutual recursion.
    \item {\tt ok\_\_torture\_params.fcl} (f) : Example showing how to
      torture parametrised species. It exhibits various non-trivial
      cases that must pass and lead to effective \ocaml\/\coq\
      code. These examples make heavy use of {\tt in}, {\tt is}
      parameters and inheritance to shake the compiler's instanciation
      of parameters during inheritance.
  \end{itemize}
\end{itemize}

\section{Other tools}
Aside \focalizec\ itself and its library, 2 others tools explicitly
dedicated to the \focalize\ packages exist.

\subsection{zenon}
Maintained by Damien Doliger, he is the man in the place to describe
\smiley.

\subsection{zvtov}
Maintained by Damien Doliger, he is the man in the place to describe
\smiley.
