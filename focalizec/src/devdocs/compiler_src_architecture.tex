%$

\section{\focalizec\ source tree}
The compiler source tree is split into several directories and files
described below. The root of the compiler sources is located in
{\tt focalize/focalizec}. All directories and files given below are
relative to this root. In some of the directories is a sub-directory
called {\tt odoc}. It can be safely ignored and is only used when
invoking verb+make doc+ to store the \ocamldoc\ output (HTML
documentation extracted from the source code of \focalize). In the
below enumeration (d) stands for ''directory'' and (f) stands for
''file".

\begin{itemize}
\item {\tt Makefile} (f) : The toplevel Makefile building the
  compiler, the standard library, the extra libraries, the
  documentation generator, the \focalize\ documentations and the
  contributions. This Makefile indeed trigger the build process in all
  the depper directories.
\item {\tt Makefile.common} (f) : Defines once for all the implicit
  rules and suffixes that will be used ni deeper Makefiles.
\item {\tt .config\_var} (f) : File generated during the ``configure''
  process (invoked by the file {\tt configure} below and that records
  the various things among installation paths, installed commands and
  tools, \ldots
\item {\tt Makefile.config} (f) : Defines once for all the commands
  according to the information the ``configure'' process recorded. It
  especially include the {\tt .config\_var} file that kept trace of what
  the ``configure'' process defined.
\item {\tt TAGS} (f) : Documnentation file where developpers are
  invited to note all the tags added to the development tree under CVS,
  with a description of the status of the development tree when the
  tag was set (or the reason why to put a tag).
\item {\tt TODO} (f) : More or less describes points that are still
  pending.
\item {\tt configure} (f) : Script used to detect available commands
  and tools, to ask the user where to install \focalize\ components,
  \ldots Parts of its output is stored in the above file
  {\tt .config\_var}.
\item {\tt doc\_src} (d) : The directory containing all the
  documentations at destination of the users.
  \begin{itemize}
  \item {\tt Makefile} (f) : Makefile trigering the documentations
    build.
  \item {\tt html} (d) : The \focalize\ WEB site in HML format.
    \begin{itemize}
    \item {\tt Includes} (d)
      \begin{itemize}
      \item {\tt aftertitle-eng.html} (f)
      \item {\tt aprestitre-fra.html} (f)
      \item {\tt avanttitre-fra.html} (f)
      \item {\tt basdepage-fra.html} (f)
      \item {\tt beforetitle-eng.html} (f)
      \item {\tt bottomofpage-eng.html} (f)
      \item {\tt copyright-eng.html} (f)
      \item {\tt copyright-fra.html} (f)
      \item {\tt doctype} (f)
      \item {\tt endofpage-eng.html} (f)
      \item {\tt findepage-fra.html} (f)
      \item {\tt hautdepage-fra.html} (f)
      \item {\tt htmlc-version.html} (f)
      \item {\tt maquette-eng.html} (f)
      \item {\tt maquette-fra.html} (f)
      \item {\tt powered\_by\_caml.html} (f)
      \item {\tt topofpage-eng.html} (f)
      \end{itemize}
    \item {\tt Makefile} (f)
    \item {\tt Makefile.html} (f)
    \item {\tt images} (d) : Directoty containing images for the WEB
      site.
      \begin{itemize}
      \item {\tt focal\_picture.jpg} (f) : A niiiice 3D picture done with
        Povray \smiley.
      \end{itemize}
    \end{itemize}
  \item {\tt man} (d) : Directory containing the ``man'' manual.
    \begin{itemize}
    \item {\tt Makefile} (f)
    \item {\tt focalizec.env} (f)
    \item {\tt focalizec.man} (f) : ``man'' for \focalizec.
    \item {\tt focalizedep.man} (f) : ``man'' for \focalizedep.
    \end{itemize}
  \item {\tt  tex} (d) : Documentations in \latex\ format.
    \begin{itemize}
    \item {\tt Makefile} (f)
    \item {\tt refman} (d) : Directory containing the reference manual.
      \begin{itemize}
      \item {\tt Makefile} (f)
      \item {\tt basic\_concepts.tex} (f)
      \item {\tt bibli.bib} (f) : Contains bibliograpy references.
      \item {\tt building\_species.tex} (f)
      \item {\tt compiler\_err\_msgs.tex} (f) : About the \focalizec\
        compiler error messages.
      \item {\tt compiler\_opts.tex} (f) : About the \focalizec\
        compiler command line options.
      \item {\tt constructs\_syntax.tex} (f)
      \item {\tt doc\_gen.tex} (f) : About the \focalize\ automated
        documentation generation (option {\tt focalize-doc} of
        \focalizec).
      \item {\tt fullpage.sty} (f)
      \item {\tt glimpse.tex} (f) : Short presentation of the language
        to give a first an rapid taste.
      \item {\tt header\_html\_snapshot.ps} (f)
      \item {\tt introduction.tex} (f)
      \item {\tt lexical\_conventions.tex} (f) : About \focalize\ lexical
        conventions, description of the tokens and syntax.
      \item {\tt macros.hva} (f)
      \item {\tt macros.tex} (f) : \latex macros used in the reference
        manual.
      \item {\tt mathml\_snapshot.ps} (f)
      \item {\tt more\_on\_meths.tex} (f)
      \item {\tt motivations.tex} (f)
      \item {\tt output.tex} (f) : About what the \focalizec\ compiler
        generates.
      \item {\tt pres\_requir.tex} (f)
      \item {\tt proofs.tex} (f)
      \item {\tt refman.html} (f)
      \item {\tt refman.hva} (f)
      \item {\tt refman.pdf} (f)
      \item {\tt refman.tex} (f) : Entry point of the reference manual.
      \item {\tt syntaxdef.hva} (f)
      \item {\tt syntaxdef.sty} (f)
      \end{itemize}
    \end{itemize}
  \end{itemize}
\item {\tt src} (d) : Root of the sources of \focalizec\ and \focalizedep.
  \begin{itemize}
  \item {\tt attic} (d) : Contains miscellaneous interesting parts of
    code that have been writen one day, that are not used anymore, but
    that we didn't want to trash in case it could serve in the future.
    \begin{itemize}
    \item {\tt ast.mli}, {\tt lib\_pp.ml}, {\tt lib\_pp.mli},
      {\tt misc.ml}, {\tt new2old.ml}, {\tt new2old.mli},
      {\tt oldsourcify.ml}, {\tt oldsourcify.mli},
      {\tt printer.ml}, {\tt printtree.ml},\\
      {\tt string\_search\_stuff.ml}
    \item May be more\ldots
    \end{itemize}
    \item {\tt basement} (d) : Contains the early basic bricks of the
      compiler.
      \begin{itemize}
        \item {\tt Makefile} (f) : Trigger build in this directory.
        \item {\tt configuration.ml} (f) : Manages version number (and
          by side effect the {\tt -help} option output text) and the
          command line options flags.
        \item {\tt configuration.mli} (f)
        \item {\tt files.ml} (f) : Contains primitives dealing with
          files, taking into account search path, files suffixes,
          read/write object files, \ldots
        \item {\tt files.mli} (f)
        \item {\tt handy.ml} (f) : Various general functions, on lists,
          pretty-printers, font setting\ldots
        \item {\tt handy.mli} (f)
        \item {\tt installation.ml} (f) : File automatically generated
          by the ``configure'' process and record as \ocaml\ source, the
          various installation paths and commands the \focalizec\
          compiler need to know when dealing with \focalize\ source
          files.
        \item {\tt location.ml} (f) : Defines the type of
          ``locations'', i.e. point un a source file and gives
          primitives to work with them.
        \item {\tt location.mli} (f)
        \item {\tt miscHelpers.ml} (f) : Mostly contains 1 fonction
          used during several passes to bind formal names to their
          types from a type scheme and a list of formal names. May be
          could somewhere else to save one file\ldots
        \item {\tt miscHelpers.mli} (f)
        \item {\tt parsetree.mli} (f) : The description of the AST.
        \item {\tt parsetree\_utils.ml} (f) : General utilities to process
          the AST.
        \item {\tt parsetree\_utils.mli} (f)
        \item {\tt types.ml} (f) : The description of the types
          structure and operations to work with them. Because types
          are a complex structure with invariants, they are exported
          as opaque to prevent breaking these invariants. Hence, any
          function dealing with the intime representation of a type
          must be in this file since his is the only location where
          this representation is visible.
        \item {\tt types.mli} (f) :
        \end{itemize}
      \item {\tt ccodegen} (d) : Dedicated to the C code generation
        back-end. Managed by Julien Blond and is not currently
        included in \focalize.
        \begin{itemize}
        \item \ldots not stable.
        \end{itemize}
      \item {\tt commoncodegen} (d) : Contains processing common to
        all target languages that is performed in any case before
        emitting the target code.
        \begin{itemize}
        \item {\tt Makefile} (f) : 
        \item {\tt abstractions.ml} (f) : Performs the synthesis of
          $\lambda$-lifted things, summarising def and
          decl-depdendencies and dependencies on collection
          parameters methods. It especially performs the final
          (complete) computation of dependencies on collection
          parameters, applying the various completion rules on the
          dependencies found previously by the rules [TYPE] and
          [BODY].
        \item {\tt abstractions.mli} (f)
        \item {\tt context.mli} (f) : The structure of the
          ``context'', i.e. the structure inductively passed to each
          function during this pass and that group into one single
          record type all the information needed by the various
          functions. This a handy way to prevent from having numerous
          parameters to pass each time to the functions.
        \item {\tt externals\_generation\_errs.ml} (f) : Exceptions
          that can be raised when generating code for ``external''
          definitions, i.e. definitions that are not writen in native
          \focalize language and used to interface with other
          programming languages.
        \item {\tt externals\_generation\_errs.mli} (f)
        \item {\tt minEnv.ml} (f) : Compute the ``\coq minimal typing
          environment''.
        \item {\tt minEnv.mli} (f)
        \item {\tt misc\_common.ml} (f) : Various type definitions and
          functions used in several points during this pass.
        \item {\tt misc\_common.mli} (f)
        \item {\tt recursion.ml} (f) : Deals the the recursive
          functions processing.
        \item {\tt recursion.mli} (f)
        \item {\tt visUniverse.ml} (f) : Compute the ``visible
          universe''.
        \item {\tt visUniverse.mli} (f)
        \end{itemize}
      \item {\tt contribs} (d) : Contains contribution source codes
        writen by users that are not developpers of the compiler and
        its framework.
        \begin{itemize}
          \item {\tt Makefile} (f)
          \item {\tt automata} (d) : Hierarchical automata. Managed by
            Philippe Ayrault.
            \begin{itemize}
              \item {\tt Makefile} (f)
              \item {\tt gen\_def.fcl} (f)
              \item {\tt main.fcl} (f)
              \item {\tt request.fcl} (f)
              \item {\tt switch\_automata.fcl} (f)
              \item {\tt switch\_recovery\_automata.fcl} (f)
              \item {\tt switch\_recovery\_normal\_automata.fcl} (f)
              \item {\tt switch\_recovery\_reverse\_automata.fcl} (f)
            \end{itemize}
          \item {\tt utils} (d) : Various utilities. Managed by
            Philippe Ayrault.
            \begin{itemize}
            \item {Makefile} (f)
            \item {pair.fcl} (f)
            \item {peano.fcl} (f)
            \end{itemize}
          \item {\tt voter} (d) : Model of a generic voter. Managed by
            Philippe Ayrault.
            \begin{itemize}
              \item {Makefile} (f)
              \item {etat\_vote.fcl} (f)
              \item {main.fcl} (f)
              \item {num\_capteur.fcl} (f)
              \item {value.fcl} (f)
              \item {vote.fcl} (f)
            \end{itemize}
         \end{itemize}
|   |-- coqcodegen
|   |   |-- FOCAL\_COQ\_MAPPINGS
|   |   |-- Makefile
|   |   |-- main\_coq\_generation.ml
|   |   |-- rec\_let\_gen.ml
|   |   |-- rec\_let\_gen.mli
|   |   |-- species\_coq\_generation.ml
|   |   |-- species\_coq\_generation.mli
|   |   |-- species\_record\_type\_generation.ml
|   |   |-- species\_record\_type\_generation.mli
|   |   |-- type\_coq\_generation.ml
|   |   `-- type\_coq\_generation.mli
|   |-- devdocs
|   |   |-- foc2ocaml.tex
|   |   |-- legacy.tex
|   |   |-- macros.tex
|   |   |-- mathpartir.sty
|   |   |-- pending.txt
|   |   `-- phd\_changes.tex
|   |-- docgen
|   |   |-- Makefile
|   |   |-- doc\_lexer.mll
|   |   |-- env\_docgen.ml
|   |   |-- env\_docgen.mli
|   |   |-- focdoc.css
|   |   |-- focdoc.dtd
|   |   |-- focdoc.rnc
|   |   |-- focdoc.xsd
|   |   |-- focdoc2html.xsl
|   |   |-- focdoc2tex.xsl
|   |   |-- main\_docgen.ml
|   |   |-- main\_docgen.mli
|   |   |-- mmlctop2\_0.xsl
|   |   |-- proposition.xsl
|   |   |-- proposition2tex.xsl
|   |   |-- utils\_docgen.ml
|   |   `-- utils\_docgen.mli
|   |-- extlib
|   |   |-- Makefile
|   |   |-- access\_control
|   |   |   |-- Makefile
|   |   |   |-- access\_control.fcl
|   |   |   |-- ensembles\_finis.fcl
|   |   |   |-- hru.fcl
|   |   |   |-- rbac.fcl
|   |   |   |-- ticket.fcl
|   |   |   |-- tm
|   |   |   |   |-- graph.ml
|   |   |   |   `-- trust\_management.fcl
|   |   |   `-- unix.fcl
|   |   `-- algebra
|   |       |-- Makefile
|   |       |-- additive\_law.fcl
|   |       |-- arrays.fcl
|   |       |-- arrays\_externals.v
|   |       |-- big\_integers.fcl
|   |       |-- constants.fcl
|   |       |-- integers.fcl
|   |       |-- iterators.fcl
|   |       |-- multiplicative\_law.fcl
|   |       |-- parse\_poly.fcl
|   |       |-- polys\_abstract.fcl
|   |       |-- product\_structures.fcl
|   |       |-- quotient\_structures.fcl
|   |       |-- randoms.fcl
|   |       |-- randoms\_externals.ml
|   |       |-- randoms\_externals.v
|   |       |-- rings\_fields.fcl
|   |       |-- small\_integers.fcl
|   |       |-- weak\_structures.fcl
|   |       |-- weak\_structures\_externals.ml
|   |       `-- weak\_structures\_externals.v
|   |-- focalizedep
|   |   |-- Makefile
|   |   |-- directive\_lexer.ml
|   |   |-- directive\_lexer.mll
|   |   `-- make\_depend.ml
|   |-- mlcodegen
|   |   |-- FOCAL\_ML\_MAPPINGS
|   |   |-- Makefile
|   |   |-- base\_exprs\_ml\_generation.ml
|   |   |-- base\_exprs\_ml\_generation.mli
|   |   |-- main\_ml\_generation.ml
|   |   |-- main\_ml\_generation.mli
|   |   |-- misc\_ml\_generation.ml
|   |   |-- misc\_ml\_generation.mli
|   |   |-- species\_ml\_generation.ml
|   |   |-- species\_ml\_generation.mli
|   |   |-- type\_ml\_generation.ml
|   |   `-- type\_ml\_generation.mli
|   |-- parser
|   |   |-- Makefile
|   |   |-- dump\_ptree.ml
|   |   |-- dump\_ptree.mli
|   |   |-- lex\_file.ml
|   |   |-- lexer.mll
|   |   |-- lexer.spec
|   |   |-- parse\_file.ml
|   |   |-- parse\_file.mli
|   |   |-- parser.mly
|   |   |-- parser.spec
|   |   |-- sourcify.ml
|   |   |-- sourcify.mli
|   |   |-- sourcify\_token.ml
|   |   `-- test
|   |       `-- sets\_orders.fcl
|   |-- scoper
|   |-- stdlib
|   |   |-- Makefile
|   |   |-- basics.fcl
|   |   |-- coq\_builtins.v
|   |   |-- generic\_proof\_cases.v
|   |   |-- lattices.fcl
|   |   |-- ml\_builtins.ml
|   |   |-- orders.fcl
|   |   |-- orders\_and\_lattices.fcl
|   |   |-- products.fcl
|   |   |-- quotients.fcl
|   |   |-- sets.fcl
|   |   |-- sets\_externals.ml
|   |   |-- sets\_orders.fcl
|   |   |-- sets\_orders\_externals.ml
|   |   |-- strict\_orders.fcl
|   |   |-- sums.fcl
|   |   |-- wellfounded.fcl
|   |   `-- wellfounded\_externals.v
|   |-- toplevel
|   |   |-- Makefile
|   |   |-- exc\_wrapper.ml
|   |   |-- focalizec.ml
|   |   |-- focalizec.mli
|   |   |-- fodump.ml
|   `-- typing
|       |-- Makefile
|       |-- ast\_equal.ml
|       |-- ast\_equal.mli
|       |-- depGraphData.mli
|       |-- dep\_analysis.ml
|       |-- dep\_analysis.mli
|       |-- env.ml
|       |-- env.mli
|       |-- infer.ml
|       |-- infer.mli
|       |-- param\_dep\_analysis.ml
|       |-- param\_dep\_analysis.mli
|       |-- scoping.ml
|       |-- scoping.mli
|       |-- substColl.ml
|       |-- substColl.mli
|       |-- substExpr.ml
|       `-- substExpr.mli
`-- tests
    |-- Makefile
    |-- ko\_\_bad\_self\_use.fcl
    |-- ko\_\_param\_toy.fcl
    |-- ko\_\_test\_error.fcl
    |-- ko\_\_test\_rec.fcl
    |-- ok\_\_baby\_toy.fcl
    |-- ok\_\_baby\_toy\_externals.ml
    |-- ok\_\_caveat.fcl
    |-- ok\_\_coll\_outside.fcl
    |-- ok\_\_definition\_72\_rule\_PRM.fcl
    |-- ok\_\_in\_example.fcl
    |-- ok\_\_in\_example2.fcl
    |-- ok\_\_list.fcl
    |-- ok\_\_multiple\_inherit.fcl
    |-- ok\_\_need\_inspect\_self.fcl
    |-- ok\_\_need\_re\_ordering.fcl
    |-- ok\_\_odd\_even.fcl
    |-- ok\_\_phd\_def\_deps.fcl
    |-- ok\_\_phd\_meths\_gen.fcl
    |-- ok\_\_phd\_sample.fcl
    |-- ok\_\_scoping\_tricky.fcl
    |-- ok\_\_term\_measure.fcl
    |-- ok\_\_toplevel\_odd\_even.fcl
    `-- ok\_\_torture\_params.fcl
  \end{itemize}
\end{itemize}

\section{Other tools}
Aside \focalizec\ itself and its library, 2 others tools explicitely
dedicated to the \focalize\ packages exist.

\subsection{zenon}
Damien is the man in the place to describe ;)

\subsection{zvtov}
Damien is the man in the place to describe ;)
