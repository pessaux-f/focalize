% $Id: typing.tex,v 1.1 2009-04-27 15:26:25 pessaux Exp $

The type-checking pass performs in fact several important tasks. It
obviously infer the type of each expression and construct, but it also
performs inheritance resolution, compute dependencies on methods of
{\tt Self} (def and decl-dependencies), ensure that the species are
well-formed and sort the methods in order they are properly ordered.
Once the type-checking pass is ended, a processed species gets in
normal form, i.e. with all its methods present once, inherited and
defined ones having been consistently put together.

This especially means that at each inheritance step, any species
issued by the type-checker has all its methods: the inheritance
disappears from the species structure. Obviously, the still have means
to know about the inheritance history somewhere, but all the methods,
inherited, defined, declared are always all together in a species in
normal form.

In other words, considering only the bunch of methods a species has,
there is no difference between a species having them via inheritance
and a species having them in its own current body with no
inheritance.

This point is especially important since it allows to inductively be
sure that if one inherits of a species, then in just one shot we know
which methods we have inherited: there is no need to walk again along
all the inheritance steps. Then, we can says that we inductively build
the normal form of species all along the inheritance tree. This point
allows faster searches and prevent from having information
disseminated in several place which would be more difficult to
maintain.

In fact, the scoping pass already used a similar way to proceed,
keeping for each species the list of all the methods it had, either in
its own body or by arbitrary inheritance.

\medskip
We will now examine various points of this typing pass.




\subsection{Type inference}
\subsubsection{Where to record type information in the AST ?}
Type inference is the process of guessing the type of each expression,
each definition of the source code. In fact, in \focalize, types are
partly inferred, partly given by the programmer. Signatures are a way
to make types explicit by giving annotations. However, at each node of
the AST, types must be infer-ed to finally label the node. In effect,
the first output of the type-checking pass it a ``typed AST'',
i.e. the initial AST with each node now having its type recorded in
the node itself.

As defined in the source file {\tt basement/parsetree.mli}, an AST
node is a generic data-structure containing a specific description:
{\footnotesize
\begin{lstlisting}[language=MyOCaml, title=Generic AST node]
type 'a ast = {
   (** The location in the source of the AST node. *)
   ast_loc : Location.t;
   (** The description of the node. *)
   ast_desc : 'a;
   (** The support for documentation. *)
   ast_doc : documentation;
   (** The type of the node. *)
   mutable ast_type : ast_node_type_information;
}
\end{lstlisting}
}
Hence, a node containing an expression (hence of type
{\tt Parsetree.expr}) will be built by something like:
{\footnotesize
\begin{lstlisting}[language=MyOCaml, title=An ``expression'' AST node]
type expr_desc =
  | E_self
  | E_const of constant
  | E_fun of vname list * expr
  | ...

type expr = expr_desc ast
\end{lstlisting}
}

In the generic data-structure of the node, the field
{\tt ast\_type} is used to record the type inferred for this
node. Since we want to keep the same AST structure all along the
compilation process, we use a mutable field for the type because
initially, after lexing/parsing and scoping, the type is not yet
known. So, the type-checking pass will modify the value contained in
this field for each node of the AST.

According to the source file {\tt basement/parsetree.mli}, values for
this field can be:
{\footnotesize
\begin{lstlisting}[language=MyOCaml, title=An ``expression'' AST node]
type ast_node_type_information =
   | ANTI_non_relevant   (** The node has no meaningful type information.
			     However, it was processed by the type-checker. *)
   | ANTI_none      (** The node was not yet processed by the type-checker.
		        Clearly, after the type-checking pass, no AST node
			should remain with this tag in the [ast_type] field of
			the node ! *)
   | ANTI_type of Types.type_simple  (** The type information is a type. Mostly
					 used to label expressions. *)
   | ANTI_scheme of Types.type_scheme (** The type information is a type scheme.
					  Mostly used to label definitions. *)
\end{lstlisting}
}

Before type-checking is done, all the node of the AST have their
{\tt ast\_type} field worthing {\tt ANTI\_none}. Once the
type-checking pass is done, no AST node should remain with this
tag. If some do, then is must be considered as a bug (node forgotten
during processing) of the compiler. At least, when a node does not
require a type information (for instance, the AST node of a {\tt open}
directive), it must however be traversed by the type-checker and must
be updated with the {\tt ANTI\_non\_relevant} value.



\subsubsection{Types and type schemes}
Like in regular ML-like type-checkers, expressions are assigned
``types'' although definitions are assigned ``type schemes''. The
difference is due to the ability for definitions to be polymorphic and
to be instantiated differently at each usage occurrence. In effect, an
expression exists in only one point. So it has one {\bf type}, that's
all. A definition leads to a ``template'' of types, where polymorphic
type variables car be instantiated as wished each time the identifier
bound by the definition is used. For this reason, a definition is
bound to a ``model'' of types, a ``family'' of type, that is usually
called a {\bf type scheme}.

Then a type scheme is in fact a list of polymorphic type variables and
a body that is a type expression. For instance, the ML-like type scheme
(possibly bound to a {\tt List.map} function)
{\tt ('a -> 'b) -> 'a list -> 'b list} will be represented by the list
of it's 2 type variables {\tt 'a} and {\tt 'b} and its body that is
the expression {\tt ('a -> 'b) -> 'a list -> 'b list}. In fact, in a
type scheme all the polymorphic type variables are {\bf implicitly
quantified universally}. The ``implicitly'' is the reason why it is
so difficult for the programmer to really see the difference between a
type and a type scheme. For the above type scheme, we should be more
explicit and better write:

$\forall$ {\tt 'a, 'b . ('a -> 'b) -> 'a list -> 'b\ list}

If we now have a look at the following expression:
{\tt List.map (fun x -> x + 1)}, then the expression (i.e. the
identifier node) {\tt List.map} will have the {\bf type}
{\tt ('t -> 'u) -> 't list -> 'u list} (in which we intentionally
changed the names of the types variables to show that they are {\bf not
the same} that those of the type scheme). This type expression
contains 2 variables {\tt 't} and {\tt 'u} that will be unified during
the type-checking of the whole application expression (unified with
{\tt int} in the current example).



\subsubsection{Working ``in place'' for substitutions}
The type-checking algorithm of ML-like languages is often stated using
the notion of MGU and using substitutions. In \focalize, the effective
inference algorithm uses techniques more efficient in practice than
regular substitutions to manually apply and combine on the type
terms. Instead, we work ``in place'', by taking benefits of sharing
between type sub-terms to simulate the substitutions by direct
physical modifications inside the terms.

\medskip
The full description of this technique is outside the scope of the
present document, but a clear and efficient explanation can be found
in ``Le langage Caml'' written by Pierre Weis and Xavier Leroy.

\medskip
The idea is to represent the types by terms that can be physically
shared, with type variables that can be directly assign a
value. Hence, as described in the source file {\tt basement/type.ml},
our type algebra is:
{\footnotesize
\begin{lstlisting}[language=MyOCaml, title=An ``expression'' AST node]
type type_simple =
  | ST_var of type_variable                   (** Type variable. *)
  | ST_arrow of (type_simple * type_simple)   (** Functional type. *)
  | ST_tuple of type_simple list              (** Tuple type. *)
  | ST_sum_arguments of type_simple list      (** Type of sum type value
                                                  constructor's arguments. To
                                                  prevent them from being
                                                  confused with tuples. *)
  | ST_construct of
      (** Type constructor, possibly with arguments. Encompass the types
          related to records and sums. Any value of these types are typed as
          a [ST_construct] whose name is the name of the record (or sum)
          type. *)
      (type_name * type_simple list)
  | ST_self_rep     (** Carrier type of the currently analysed species. *)
  | ST_species_rep of
      (** Carrier type of a collection hosted in the specified module. *)
      (fname * collection_name)


(** Variable of type. Must be repr'ed. *)
and type_variable = {
  (** Binding level of the type. *)
  mutable tv_level : int ;
  (** Value of the type variable. *)
  mutable tv_value : type_variable_value
}


(** Value of a type link (generalization principle of type variable's value. *)
and type_variable_value =
  | TVV_unknown
  | TVV_known of type_simple
\end{lstlisting}
}
The algebra describes the built-in type constructors and more
interestingly for our explanation, the {\bf type variables}. We can
see that in the type algebra, all the variables look pretty
structurally the same: the constructor {\tt ST\_var} and a
{\tt type\_variable} containing a few information. Hence, for instance
2 variables whose value are unknown will look exactly the same. To
make the difference, we must consider physical equality. Hence, if 2
variables are physically equal, ``they are {\bf the} same''
variable(s?), otherwise they are really different. The aim of this
mechanism is to share the same physical data to represent all the
occurrences of a variable in a term, so that when we want to assign it
a value, we just need to modify it in place and all the shared
occurrences will be updated for free.

A type variable is initially an unknown of the unification equation
induced by the type-checking process. Hence it starts with its field
{\tt tv\_value} worthing {\tt TVV\_unknown}. If during unification, a
variable needs to be assigned a value (i.e. a constraint was found on
this variable), then its {\tt tv\_value} that is mutable will be
assigned {\tt TVV\_known} ``of something''. This ``something'' is
itself a type and this allows indeed to instantiate a type variable by
a type expression.

Obviously, this mechanism to represent instantiations will create
``strings'' of links between variables and their effective value. To
be sure that the value of a variable is know ``equal to something'' or
really unknown, we must use a mechanism that returns the canonical
representation of a type. For instance, let's imagine that in our
inference problem, we arrived to have 4 variables $\alpha$, $\beta$,
$\gamma$ and $\delta$, with $\alpha = \beta$, $\beta = {\tt int}$,
$\delta = \gamma$ and $\gamma = \beta$. Hence, we have the following
picture:

\noindent
\begin{math}
\begin{array}{ccccc}
\alpha & \rightarrow & \beta & \rightarrow & {\tt int} \\
       &             & \uparrow & & \\
\delta & \rightarrow & \gamma & &
\end{array}
\end{math}

\noindent representing the system: where
{\footnotesize\lstinline!'a = ST_var (TVV_known (...))!} where
{\footnotesize\lstinline!...!} represents {\tt 'b} and has the
structure
{\footnotesize\lstinline!ST_var (TVV_known (ST_construct (``int'', [])))!}.
We have the same kind of thing for $\delta$ and $\gamma$, with
$\gamma$ worthing
{\footnotesize\lstinline!TVV_known (...)!} with
{\footnotesize\lstinline!...!} representing the structure of $\beta$.
In fact, despite all the variables we see above, all are equal and are
instantiated by {\tt int}. This means than must not trust the first
value constructor seen for a type to know what it is equal to. One
must ``follow'' the links.

Moreover, to avoid the loss of efficiency induced by walking each
time along these ``strings'' of links, the operation of getting the
canonical representation of a type will use the ``path compression''
operation in order to suppress indirections a soon as they are
encountered a first time.

The operation returning the canonical representation is the guardian
of the correct structure of the types. {\bf Any operation working /
relying / walking on the type structure must call this operation to be
sure that the structural view of the type is has is really the
canonical view of the type}. This operation called {\tt repr} is
located in the {\tt basement/types.ml} source file of the
compiler. The presence of such a strong invariant is the reason why
the types are exported as {\bf abstract}. This ensures that outside
the module manipulating the type algebra, nobody will forget to get
the canonical representation of a type.


\subsubsection{Unification}
Based on the representation of our types, since the unification tends
to return a substitution to apply on the 2 unified types in order to
make them equal, instead of getting this substitution to later apply
it to each type (and combine this substitution with the other
substitutions the types may be subject to), we will directly make the
type terms equal by instantiating their type variables. The
``instanciation'' is then made in place, by changing in place the
mutable field {\tt tv\_value} of the unknown variables from
{\tt TVV\_unknown} to {\tt TVV\_known} ``of'' the type required to
have equality.

This is especially fast since all the occurrences of a variable share
the same physical location. Hence, changing the value of the field
{\tt tv\_value} at this location is equivalent to simultaneously
instantiate all the occurrences of this variable (past and future) in
type terms.


\subsubsection{Polymorphism}
Binding level, specialize et generalize.









Parler du hack de Self\_manifest.

\subsection{Environments structures}
\subsection{Inheritance resolution}
\subsection{Dependency on Self's methods calculus}
Parler de comment on d�tecte les d�pendances sur Self.

\subsection{Normal form}
