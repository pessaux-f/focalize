% $Id: typing.tex,v 1.4 2009-05-04 09:04:45 pessaux Exp $

The type-checking pass performs in fact several important tasks. It
obviously infer the type of each expression and construct, but it also
performs inheritance resolution, compute dependencies on methods of
{\tt Self} (def and decl-dependencies), ensure that the species are
well-formed and sort the methods in order they are properly ordered.
Once the type-checking pass is ended, a processed species gets in
normal form, i.e. with all its methods present once, inherited and
defined ones having been consistently put together.

\label{cool-incremental-nf}
This especially means that at each inheritance step, any species
issued by the type-checker has all its methods: the inheritance
disappears from the species structure. Obviously, the still have means
to know about the inheritance history somewhere, but all the methods,
inherited, defined, declared are always all together in a species in
normal form.

In other words, considering only the bunch of methods a species has,
there is no difference between a species having them via inheritance
and a species having them in its own current body with no
inheritance.

This point is especially important since it allows to inductively be
sure that if one inherits of a species, then in just one shot we know
which methods we have inherited: there is no need to walk again along
all the inheritance steps. Then, we can says that we inductively build
the normal form of species all along the inheritance tree. This point
allows faster searches and prevent from having information
disseminated in several place which would be more difficult to
maintain.

In fact, the scoping pass already used a similar way to proceed,
keeping for each species the list of all the methods it had, either in
its own body or by arbitrary inheritance.

\medskip
We will now examine various points of this typing pass.




\subsection{Type inference}
\subsubsection{Where to record type information in the AST ?}
Type inference is the process of guessing the type of each expression,
each definition of the source code. In fact, in \focalize, types are
partly inferred, partly given by the programmer. Signatures are a way
to make types explicit by giving annotations. However, at each node of
the AST, types must be infer-ed to finally label the node. In effect,
the first output of the type-checking pass it a ``typed AST'',
i.e. the initial AST with each node now having its type recorded in
the node itself.

As defined in the source file {\tt basement/parsetree.mli}, an AST
node is a generic data-structure containing a specific description:

{\footnotesize
\begin{lstlisting}[language=MyOCaml, title=Generic AST node]
type 'a ast = {
   (** The location in the source of the AST node. *)
   ast_loc : Location.t;
   (** The description of the node. *)
   ast_desc : 'a;
   (** The support for documentation. *)
   ast_doc : documentation;
   (** The type of the node. *)
   mutable ast_type : ast_node_type_information;
}
\end{lstlisting}
}
Hence, a node containing an expression (hence of type
{\tt Parsetree.expr}) will be built by something like:

{\footnotesize
\begin{lstlisting}[language=MyOCaml, title=An ``expression'' AST node]
type expr_desc =
  | E_self
  | E_const of constant
  | E_fun of vname list * expr
  | ...

type expr = expr_desc ast
\end{lstlisting}
}

In the generic data-structure of the node, the field
{\tt ast\_type} is used to record the type inferred for this
node. Since we want to keep the same AST structure all along the
compilation process, we use a mutable field for the type because
initially, after lexing/parsing and scoping, the type is not yet
known. So, the type-checking pass will modify the value contained in
this field for each node of the AST.

According to the source file {\tt basement/parsetree.mli}, values for
this field can be:

{\footnotesize
\begin{lstlisting}[language=MyOCaml, title=An ``expression'' AST node]
type ast_node_type_information =
   | ANTI_non_relevant   (** The node has no meaningful type information.
			     However, it was processed by the type-checker. *)
   | ANTI_none      (** The node was not yet processed by the type-checker.
		        Clearly, after the type-checking pass, no AST node
			should remain with this tag in the [ast_type] field of
			the node ! *)
   | ANTI_type of Types.type_simple  (** The type information is a type. Mostly
					 used to label expressions. *)
   | ANTI_scheme of Types.type_scheme (** The type information is a type scheme.
					  Mostly used to label definitions. *)
\end{lstlisting}
}

Before type-checking is done, all the node of the AST have their
{\tt ast\_type} field worthing {\tt ANTI\_none}. Once the
type-checking pass is done, no AST node should remain with this
tag. If some do, then is must be considered as a bug (node forgotten
during processing) of the compiler. At least, when a node does not
require a type information (for instance, the AST node of a {\tt open}
directive), it must however be traversed by the type-checker and must
be updated with the {\tt ANTI\_non\_relevant} value.



\subsubsection{Types and type schemes}
\index{type!scheme}
\label{type-scheme}
Like in regular ML-like type-checkers, expressions are assigned
``types'' although definitions are assigned ``type schemes''. The
difference is due to the ability for definitions to be polymorphic and
to be instantiated differently at each usage occurrence. In effect, an
expression exists in only one point. So it has one {\bf type}, that's
all. A definition leads to a ``template'' of types, where polymorphic
type variables car be instantiated as wished each time the identifier
bound by the definition is used. For this reason, a definition is
bound to a ``model'' of types, a ``family'' of type, that is usually
called a {\bf type scheme}.

Then a type scheme is in fact a list of polymorphic type variables and
a body that is a type expression. For instance, the ML-like type scheme
(possibly bound to a {\tt List.map} function)
{\tt ('a -> 'b) -> 'a list -> 'b list} will be represented by the list
of it's 2 type variables {\tt 'a} and {\tt 'b} and its body that is
the expression {\tt ('a -> 'b) -> 'a list -> 'b list}. In fact, in a
type scheme all the polymorphic type variables are {\bf implicitly
quantified universally}. The ``implicitly'' is the reason why it is
so difficult for the programmer to really see the difference between a
type and a type scheme. For the above type scheme, we should be more
explicit and better write:

$\forall$ {\tt 'a, 'b . ('a -> 'b) -> 'a list -> 'b\ list}

If we now have a look at the following expression:
{\tt List.map (fun x -> x + 1)}, then the expression (i.e. the
identifier node) {\tt List.map} will have the {\bf type}
{\tt ('t -> 'u) -> 't list -> 'u list} (in which we intentionally
changed the names of the types variables to show that they are {\bf not
the same} that those of the type scheme). This type expression
contains 2 variables {\tt 't} and {\tt 'u} that will be unified during
the type-checking of the whole application expression (unified with
{\tt int} in the current example).

\medskip
In \focalize, type schemes are compound of the list of the polymorphic
type variables that are its parameters and the type expression that is
its body:
{\footnotesize
\begin{lstlisting}[language=MyOCaml, title=An ``expression'' AST node]
type type_scheme = {
  ts_vars : type_variable list ;          (** Parameters in the scheme. *)
  ts_body : type_simple    (** Body of the scheme where generalized types
                               have a level equal to [generic_level]. *)
}
\end{lstlisting}
}



\subsubsection{Working ``in place'' for substitutions}
The type-checking algorithm of ML-like languages is often stated using
the notion of MGU and using substitutions. In \focalize, the effective
inference algorithm uses techniques more efficient in practice than
regular substitutions to manually apply and combine on the type
terms. Instead, we work ``in place'', by taking benefits of sharing
between type sub-terms to simulate the substitutions by direct
physical modifications inside the terms.

\medskip
The full description of this technique is outside the scope of the
present document, but a clear and efficient explanation can be found
in ``Le langage Caml'' written by Pierre Weis and Xavier Leroy.

\medskip
The idea is to represent the types by terms that can be physically
shared, with type variables that can be directly assign a
value. Hence, as described in the source file {\tt basement/type.ml},
our type algebra is:

{\footnotesize
\begin{lstlisting}[language=MyOCaml, title=An ``expression'' AST node]
type type_simple =
  | ST_var of type_variable                   (** Type variable. *)
  | ST_arrow of (type_simple * type_simple)   (** Functional type. *)
  | ST_tuple of type_simple list              (** Tuple type. *)
  | ST_sum_arguments of type_simple list      (** Type of sum type value
                                                  constructor's arguments. To
                                                  prevent them from being
                                                  confused with tuples. *)
  | ST_construct of
      (** Type constructor, possibly with arguments. Encompass the types
          related to records and sums. Any value of these types are typed as
          a [ST_construct] whose name is the name of the record (or sum)
          type. *)
      (type_name * type_simple list)
  | ST_self_rep     (** Carrier type of the currently analysed species. *)
  | ST_species_rep of
      (** Carrier type of a collection hosted in the specified module. *)
      (fname * collection_name)


and type_variable = {
  (** Binding level of the type. *)
  mutable tv_level : int ;
  (** Value of the type variable. *)
  mutable tv_value : type_variable_value
}


and type_variable_value =
  | TVV_unknown
  | TVV_known of type_simple
\end{lstlisting}
}

The algebra describes the built-in type constructors and more
interestingly for our explanation, the {\bf type variables}. We can
see that in the type algebra, all the variables look pretty
structurally the same: the constructor {\tt ST\_var} and a
{\tt type\_variable} containing a few information. Hence, for instance
2 variables whose value are unknown will look exactly the same. To
make the difference, we must consider physical equality. Hence, if 2
variables are physically equal, ``they are {\bf the} same''
variable(s?), otherwise they are really different. The aim of this
mechanism is to share the same physical data to represent all the
occurrences of a variable in a term, so that when we want to assign it
a value, we just need to modify it in place and all the shared
occurrences will be updated for free.

A type variable is initially an unknown of the unification equation
induced by the type-checking process. Hence it starts with its field
{\tt tv\_value} worthing {\tt TVV\_unknown}. If during unification, a
variable needs to be assigned a value (i.e. a constraint was found on
this variable), then its {\tt tv\_value} that is mutable will be
assigned {\tt TVV\_known} ``of something''. This ``something'' is
itself a type and this allows indeed to instantiate a type variable by
a type expression.

Obviously, this mechanism to represent instantiations will create
``strings'' of links between variables and their effective value. To
be sure that the value of a variable is know ``equal to something'' or
really unknown, we must use a mechanism that returns the canonical
representation of a type. For instance, let's imagine that in our
inference problem, we arrived to have 4 variables $\alpha$, $\beta$,
$\gamma$ and $\delta$, with $\alpha = \beta$, $\beta = {\tt int}$,
$\delta = \gamma$ and $\gamma = \beta$. Hence, we have the following
picture:

\noindent
\begin{math}
\begin{array}{ccccc}
\alpha & \rightarrow & \beta & \rightarrow & {\tt int} \\
       &             & \uparrow & & \\
\delta & \rightarrow & \gamma & &
\end{array}
\end{math}

\noindent representing the system: where
{\footnotesize\lstinline!'a = ST_var (TVV_known (...))!} where
{\footnotesize\lstinline!...!} represents {\tt 'b} and has the
structure
{\footnotesize\lstinline!ST_var (TVV_known (ST_construct (``int'', [])))!}.
We have the same kind of thing for $\delta$ and $\gamma$, with
$\gamma$ worthing
{\footnotesize\lstinline!TVV_known (...)!} with
{\footnotesize\lstinline!...!} representing the structure of $\beta$.
In fact, despite all the variables we see above, all are equal and are
instantiated by {\tt int}. This means than must not trust the first
value constructor seen for a type to know what it is equal to. One
must ``follow'' the links.

Moreover, to avoid the loss of efficiency induced by walking each
time along these ``strings'' of links, the operation of getting the
canonical representation of a type will use the ``path compression''
operation in order to suppress indirections a soon as they are
encountered a first time.

The operation returning the canonical representation is the guardian
of the correct structure of the types. {\bf Any operation working /
relying / walking on the type structure must call this operation to be
sure that the structural view of the type is has is really the
canonical view of the type}. This operation called {\tt repr} is
located in the {\tt basement/types.ml} source file of the
compiler. The presence of such a strong invariant is the reason why
the types are exported as {\bf abstract}. This ensures that outside
the module manipulating the type algebra, nobody will forget to get
the canonical representation of a type.

Basically, this function receives a type. If this type is not a
variable, then it returns it directly. This means that the type
constructor is already known to be something else than a variable for
which we should investigate further. On the other side, if the
received type is a variable whose ``value'' (i.e. field
{\tt tv\_value} is ``known to be equal to something'' (i.e. is
{\tt TVV\_known (...)}, then we will ask to get the canonical
representation of this ``something''. This is typically a recursive
call on this ``something''. This way, if this ``something'' is itself
a variable ``known to be equal to something else'', then we will
inductively know each step of indirection. So, once we get our
canonical representation of our ``something'', {\bf this is} in fact
the canonical representation of our type, since it was a variable
``known to be equal to {\bf this} something''. By the way, before
returning, since now we know that the variable is in fact ``pointing''
onto a type (i.e. is not anymore a variable, was instantiated), we
take benefit to cut the string of indirections by directly
establishing a link between the variable and the canonical
representation we obtained. Hence, next time we will access this
variable via a type sub-term shared somewhere else, we won't have
anymore to walk along the whole ``string'' of links to know the
variable's value. Then, the {\tt repr} function simply looks like:

{\footnotesize
\begin{lstlisting}[language=MyOCaml,
                   title=Getting the canonical representation of a type]
let rec repr = function
  | ST_var ({ tv_value = TVV_known ty1 } as var) ->
      let val_of_ty1 = repr ty1 in
      var.tv_value <- TVV_known val_of_ty1 ;
      val_of_ty1
  | ty -> ty
\end{lstlisting}
}


\subsubsection{Unification}
The computation of the most general type of an expression, called type
inference, strongly rely on unification of type terms. We will say
that 2 terms $\tau_1$ and $\tau_2$ can be {\em unified} if there exists
a substitution $\phi$ so that $\phi(\tau_1) = \phi (\tau_2)$. The
substitution $\phi$ is then called {\em unifier}.
\index{unifier} of the terms $\tau_1$ and $\tau_2$. Hence, two terms
can be unified if it is possible to instantiate all or part of their
variables by a same substitution, so that they become structurally
equal.

Based on the representation of our types, since the unification tends
to return a substitution to apply on the 2 unified types in order to
make them equal, instead of getting this substitution to later apply
it to each type (and combine this substitution with the other
substitutions the types may be subject to), we will directly make the
type terms equal by instantiating their type variables. The
``instanciation'' is then made in place, by changing in place the
mutable field {\tt tv\_value} of the unknown variables from
{\tt TVV\_unknown} to {\tt TVV\_known} ``of'' the type required to
have equality.

This is especially fast since all the occurrences of a variable share
the same physical location. Hence, changing the value of the field
{\tt tv\_value} at this location is equivalent to simultaneously
instantiate all the occurrences of this variable (past and future) in
type terms.

Hence, ideally unification doesn't return any result and makes the 2
unified type equal by side effect or fails because there exist no
unifier for the 2 types. And then, in the type-checking algorithm,
instead of using one of the type on which we apply its related
substitution, we can directly use any one of the 2 types once unified
since they are now equal.

In fact, in our case this is not completely the case since we have an
additional problem that forces us to return a type. This is due to
the fact that when unifying a type and {\tt Self}, \focalize's rules
require to have {\tt Self} as unification result rather than any of
one the two types. This is a problem when the unification used the
known representation of {\tt Self} to achieve finding the mgu. In
effect, in this case, one of the 2 type is {\tt Self} and this other
is a type expression that is compatible with {\tt Self}'s
representation. And in this fact, choosing any of the 2 types as result
is wrong: the result must always chose (prefer) {\tt Self}. Hence our
unification routine returns the preferred unifier in addition to make
the physical modifications in place when required. We then have a
unification function described in the compiler's source file
{\tt basement/types.ml} looking like (explanations follow):

{\footnotesize
\begin{lstlisting}[language=MyOCaml,numbers=left, firstnumber=0,
                   title=The unification algorithm]
let unify ~loc ~self_manifest type1 type2 =
  let rec rec_unify ty1 ty2 =
    let ty1 = repr ty1 in
    let ty2 = repr ty2 in
    if ty1 == ty2 then ty1 else
    match (ty1, ty2) with
     | (ST_var var, _) ->
         (* BE CAREFUL: [occur_check] performs the setting of decl-dependencies
            on the carrier ! In effect, if [ty2] involved Self then we have a
            dependency on the carrier and that must be taken into account !
            The interest to make [occur_check] doing this work is that it
            walk all along the type so it's a good idea to take benefit of this
            walk to avoid one more walk. *)
         occur_check ~loc var ty2 ;
         lowerize_levels var.tv_level ty2 ;
         var.tv_value <- TVV_known ty2 ;
         ty2
     | (_, ST_var var) ->
         (* BE CAREFUL: Same remark than above for [occur_check]. *)
         occur_check ~loc var ty1 ;
         lowerize_levels var.tv_level ty1 ;
         var.tv_value <- TVV_known ty1 ;
         ty1
     | ((ST_arrow (arg1, res1)), (ST_arrow (arg2, res2))) ->
         let arg3 = rec_unify arg1 arg2 in
         let res3 = rec_unify res1 res2 in
         ST_arrow (arg3, res3)
     | ((ST_sum_arguments tys1), (ST_sum_arguments tys2)) ->
         let tys3 =
           (try List.map2 rec_unify tys1 tys2 with
           | Invalid_argument "List.map2" ->
               (* In fact, that's an arity mismatch on the types. There is a
                  strange case appearing when using a sum type constructor that
                  requires arguments without arguments. The type of the
                  constructor's arguments is an ampty list. Then the conflict is
                  reported as "Types and ... are not compatible". Hence one of
                  the type is printed as nothing (c.f. bub report #180).
                  In this case, we generate a special error message. *)
               if (List.length tys1) = 0 || (List.length tys2) = 0 then
                 raise (Arity_mismatch_unexpected_args (loc))
               else raise (Conflict (ty1, ty2, loc))) in
         ST_sum_arguments tys3
     | ((ST_sum_arguments _), (ST_tuple _))
     | ((ST_tuple _), (ST_sum_arguments _)) ->
         (* Special cases to handle confusion between sum type value
            constructor's that take SEVERAL arguments and not 1 argument that
            is a tuple. *)
         raise (Arity_mismatch_unexpected_args (loc))
     | ((ST_tuple tys1), (ST_tuple tys2)) ->
         let tys3 =
           (try List.map2 rec_unify tys1 tys2 with
           | Invalid_argument "List.map2" ->
               (* In fact, that's an arity mismatch on the tuple. *)
               raise (Conflict (ty1, ty2, loc))) in
         ST_tuple tys3
     | (ST_construct (name, args), ST_construct (name', args')) ->
         (if name <> name' then raise (Conflict (ty1, ty2, loc))) ;
         let args'' =
           (try List.map2 rec_unify args args' with
           | Invalid_argument "List.map2" ->
               (* In fact, that's an arity mismatch. *)
               raise
                 (Arity_mismatch
                    (name, (List.length args), (List.length args'), loc))) in
         ST_construct (name, args'')
     | (ST_self_rep, ST_self_rep) ->
         (begin
         (* Trivial, but anyway, proceed as everywhere else. *)
         set_decl_dep_on_rep () ;
         ST_self_rep
         end)
     | (ST_self_rep, _) ->
         (begin
         match self_manifest with
          | None -> raise (Conflict (ty1, ty2, loc))
          | Some self_is_that ->
              ignore (rec_unify self_is_that ty2) ;
              set_def_dep_on_rep () ;
              (* Always prefer Self ! *)
              ST_self_rep
         end)
     | (_, ST_self_rep) ->
         (begin
         match self_manifest with
          | None -> raise (Conflict (ty1, ty2, loc))
          | Some self_is_that ->
              ignore (rec_unify self_is_that ty1) ;
              set_def_dep_on_rep () ;
              (* Always prefer Self ! *)
              ST_self_rep
         end)
     | ((ST_species_rep c1), (ST_species_rep c2)) ->
         if c1 = c2 then ty1 else raise (Conflict (ty1, ty2, loc))
     | (_, _) -> raise (Conflict (ty1, ty2, loc)) in
  (* ****************** *)
  (* Now, let's work... *)
  rec_unify type1 type2
;;
\end{lstlisting}
}

First of all, we see that as previously said, since we intend to work
on the structure of the types, we start by computing their canonical
representation by calling {\tt repr} (lines 2 and 3).

If the 2 types are already the same (physically, note the usage of
{\tt ==} and not {\tt =}), then there is nothing more to to and we can
return any one of the 2 types as unifier. We can really return any one
since they are really equal and the problem of preferring {\tt Self}
doesn't apply here: either the 2 types are both {\tt Self} or they are
both something else.

Then the algorithm considers all the cases of two types. When one is a
variable (lines 6 and 17), we assign to the variable the other type
hence telling that the variable is not anymore unknown (lines 15 and
21).

Before assigning the variable, we perform an ``occur check''
\index{occur check}
(lines 13 and 19). This ensures that the variable we assign doesn't
appear in the type is it assigned. This is to prevent types from being
cyclic. In effect, if we try to unify $\alpha$ with $\alpha
\rightarrow \alpha$, we get in the following situation:

\noindent
\begin{math}
\begin{array}{ccc}
\alpha     & \rightarrow & \alpha \\
 \uparrow  &             & \downarrow  \\
           & \leftarrow &
\end{array}
\end{math}

\noindent with the arrow on the same line than the $\alpha$s
represent the functional type constructor and the arrows below the
link the unification creates between variables and types. Hence, if
the unification falls in this case, an error is raised telling that
the type of the expression leads to cyclic types and that this
expression is rejected.

For the moment, the lines 14 and 20 (and the comment lines 7-12) can
be forgotten since they will be explained when we will be dealing
with polymorphism.

When unifying with one type being a variable, the returned type is
always the type assigned to the variable (lines 16 and 22). We could
also choose to return the variable but this would be less efficient
since no {\tt repr} is yet applied since we assigned it a value, then
to use this variable as type, one {\tt repr} will immediately be
required. Choosing to return the other type that has been
``{\tt repr}-ed'' we can save one call. This may seems a tiny
advantage, but unification in place is efficient for several tiny
advantages put together ! An this one is part of them.

\medskip
In term of difficulty, we now have the unification of {\tt Self} and a
type. The simplest case (line 65) is the unification of {\tt Self} and
itself. Trivially, the unification succeed. Forget the line 66 for the
moment. There remain 2 symmetric cases: unifying {\tt Self} and another
type that is not {\tt Self} and not a variable (lines 71 and 81).
In these cases, we are in the rules [{\tt Self1}] and [{\tt Self2}] of
Virgile Prevosto's PhD, page 27, definition 9. These are the cases
where, if the structure of the carrier (i.e. of the {\tt representation}
method) is visible then an occurrence of {\tt Self} is allowed to be
unified with a type having a structure compatible with the one given
by the {\tt representation} method.

To be able to make so, in the typing context, we record if the
structure of the carrier is known of not. This information if passed
to the unification function via the parameter
{\tt $\sim$self\_manifest}. This parameter has type
{\tt simple\_type option}. If the value is {\tt None}, this means that
the structure of the carrier is not visible (hence {\tt Self} can only
be unified with {\tt Self}). If the value is {\tt Some ...}, then this
means that we know that the carrier was defined as the type ``...''
(hence unifying {\tt Self} with a type expression compatible with
``...'' must be successful).

So, when unifying {\tt Self} with a type, if {\tt $\sim$self\_manifest}
is {\tt None} we raise an error because the carrier is abstract. If 
{\tt  $\sim$self\_manifest} is {\tt Some (}$\tau${\tt )}, then we must
ensure that $\tau$ can be unified (and if so, the possible side
effects induced by this unification must be done) with the other type
(lines 76 and 96): it is then simply a recursive call. Since if the
unification succeed we want to return {\tt Self} as unifier, we throw
the result an return {\tt ST\_self\_rep}, meaning the type
``{\tt Self}''.

{\bf Attention}: With this mechanism, we directly unify the type that
represents the structure of the carrier. This means that we directly
apply modification in place on it. This especially means that, because
this is not a type scheme but a type, there is a cumulative effect of
all the unifications that are made between {\tt Self} and other
types. Hopefully, this is not a problem since methods are not allowed
to be polymorphic. And {\tt representation}, defining the structure
of {\tt Self} is a method. So it can't be polymorphic. This especially
means that the structure of {\tt Self} will never have remaining
variables that could be instantiated by a unification. Hence, there is
no risk that a unification pollutes the type used as reference for the
structure of {\tt Self} by instantiating a type variable by a type
that would be incompatible for a later unification of this variable by
another type. Sooooo, this means that {\tt $\sim$self\_manifest} is a type
and not a type scheme because we don't have polymorphism on
methods. If we had some, we would need another mechanism to ``prefer
{\tt Self}'' during unification (problem of specialisation,
generalisation, putting the right binding level, and so on\ldots).


\medskip
The other cases of unification are simpler and structural. The
unification can now only be successful if the 2 types have the same
constructor. So we check the types 2 by 2 with each time the same
constructor and if the matching is right, we recurse structurally on
the types structures.

\medskip
Finally remains the all cases where the 2 types do not have the same
constructor (line 93). This leads to a type-checking error by raising
an exception.

\subsubsection{Polymorphism}
Like introduced in \ref{type-scheme}, type schemes are used to
implement the polymorphism in type inference. They act as ``models''
of types. But the question is how to make such ``models'' and how
extract types from such ``models''. The first point is handled by the
{\tt generalize} function and the second by the {\tt specialize}
function of the source file {\tt basement/types.ml}.

We must start our explanation by introducing the notion of
{\bf binding level}. The aim is to keep trace of type variables
introduced in let-definitions that are deeper than our level. The
binding level counts the number of nested let-definitions. Hence, the
current level must be incremented each {\bf before} every potentially
generalisable definition. This increase will enable the generalisation
once we will go back to a lower level. And it must be decreased at the
end of this definition. To increase the current binding level, the
function {\tt begin\_definition} must be used; to decrease it the
function {\tt end\_definition} must be used.

We said ``potentially generalisable'' definition since all
let-definitions can't be generalised (non-expansivity problem).

There remains a little problem with the level of variables. The higher
the level, the closest is the definition that introduced this type
variable. When a variable is put in contact (i.e. unified) with a type
containing variables with a lower level (i.e. created in outer
let-definitions), hence that may not be generalisable when we will
leave the current definition, we must reflect this onto our variable !
In effect, unification means equality. So if our variable is ``equal''
to a variable than can be generalised, so it must be for our
variable. Hence, when unifying a variable with a type, we must hunt in
this type for variables that have a lower level and if we find some,
we must lower the level of our variable to this found level. That's
what the function {\tt lowerize\_levels} does. And in effect, this
function is called when unifying a type variable with something (lines 
14 and 20 in the {\tt unify} function listed above).


\medskip
So, the generalisation process consists in considering as
generalisable, all the variables having a level strictly greater than
the current binding level. We then search for all the generalisable
variables, and remind them in a list. By the way, we change in place
the level {\tt tv\_level} field) of the found variables by setting a
level meaning ``generalised'' (technically, we choose a value to big,
that there is no chance that anybody nests so many let-definitions to
reach this level). Finally, to build the type scheme, we simply
create a {\tt type\_scheme} structure with the list of found variables
and the type itself as body.

\medskip
Now, taking an instance of a type scheme is done by the function
{\tt specialize} that simply create a fresh type variable for each
generalised variable of the type scheme, then copies the body of the
type scheme, replacing each generalised variable by the corresponding
fresh one.


\subsubsection{Type inference among other typing things to do}
The type inference we spoke about deals with expressions and
definitions. Hence it is only one part of the analyses performed
during the ``typing pass''. More accurately, it is the first analysis
carried out after the scoping. Each expression of a program must be
type-checked anyway where it appears, in entity parameters, in
methods, in toplevel definitions. Some other constructs of the
\focalize\ language require some other kinds of type-checking: that is
the case of species, collections, collection parameters, inheritance
species expressions. However, during these other kinds of
type-checking, we use the type-checking of expressions to make the
``glue'' ensuring consistency. Rules given in Virgile' PhD in figure
3.2, page 43, section 3.8 are example of this idea (these rules deal
with with species parameters).

\medskip
Aside this notion of type-checking, we also have other analyses to
ensure a program is sound and also to extract the basic shape of a
species by resolving inheritance and late binding. All these things
are performed during the typing pass, the type-checking being the
first step.

\medskip
The typing pass is driven by the source file {\tt typing/infer.ml} and
more especially by the entry point function {\tt
  typecheck\_file}. This function triggers the type-checking of each
phrase of the compilation unit. Be aware that in the source of the
compiler, ``type-check'' is widely overloaded and denotes the process
of infering type for entities and then ensuring they are well-formed
and building the first pieces of information that will ease the code
generation. In particular, the fact that each species is given a
normal form containing all its methods is handy to recover the methods
of parameters or inheritance species expressions since there is no
need to walk along all the inheritance level (point already mentionned
in the introduction of this chapter, at \ref{cool-incremental-nf}).



\subsection{Environment and structures for typing pass}
The structure of the environment used during the typing pass has been
presented in \ref{typing-environment} and is coded in {typing/env.ml}.

The only other structure used during this pass is the typing context
that group various information to pass to the functions instead of
having to pass them individually hence preventing from having tons of
arguments for each call. This structure, shown below is local to
the {\tt typing/infer.ml} and passed to each function under the
parameter name {\tt ctx}.

{\footnotesize
\begin{lstlisting}[language=MyOCaml, title=Typing context]
type typing_context = {
  (** The name of the currently analysed compilation unit (i.e. the name
      of the file without extension and not capitalized). *)
  current_unit : Types.fname ;
  (** The name of the current species if relevant. *)
  current_species : Parsetree.qualified_species option ;
  (** Optional type Self is known to be equal to. *)
  self_manifest : Types.type_simple option ;
  (** Mapping between 'variables [vname]s and the [simple_type] they are
      bound to. Used when creating a polymorphic type definition. *)
  tyvars_mapping : (Parsetree.vname * Types.type_simple) list
}
\end{lstlisting}
}

\section{Typing a species definition}
The process of typing a species will be now examined step by step by
step. This is done by typing its parameters, resolving inheritance,
type-checking the methods, merging properties and proofs, normalising
the species, computing its dependencies graph and then inserting it in
the environment. A few other administrative tasks are performed and
will be detailled below.

\subsection{Dealing with the species parameters}
For each parameter, this process builds the species type of the
parameter and get the list of methods it has. Each parameter will be
inserted in the environment as species and type (the type representing
the carrier of this parameter).

\medskip
There are 2 cases: entity and collection parameters. Both of them lead
to a ``parameter description'' that will be recorded in the
{\tt Env.TypeInformation.species\_param} that will appear in the
hosting species {\tt spe\_sig\_params} field.


\subsection{Typinf a species expression}
A species expression is either a simple species/collection identifier
like {\tt Setoid} or an application of a species/collection identifier
to one or more {\bf species/collection identifiers}. This means that
we have no expressions like {\tt A(B(C))}.

The first thing is to find in the environment the name of the ``main''
species (i.e. the unique only if there is no application, or the name
in ``applicative'' position if there is an application). From this
search, we get a species description of a previously existing species.

The aim is to finally get (among other things, but this is the most
obvious result we want) the list of methods this species expression
has. If there is no parameter, then the list of methods is trivially
the one obtained in the structure found in the environment. The most
interesting point is when the expression is an application. We then
must apply the species to its effective argument(s) before being able
to know the methods it has. In effect, let's imagine we have:

{\footnotesize
\begin{lstlisting}
species A (B is Setoid) =
  let eq (x in B, y in B) = B!bla (x, y) && B!bli (y, x) ... ;;

species C inherits A (D) ... ;
\end{lstlisting}
}

\noindent {\tt C} inherits {\tt eq} but not with the body
{\footnotesize\lstinline*let eq (x in B, y in B) = B!bla (x, y) && B!bli (y, x) ...*}
but with the body
{\footnotesize\lstinline*let eq (x in D, y in D) = D!bla (x, y) && D!bli (y, x) ...*}
where {\tt B} was replaced by {\tt D}.

\subsection{Inheritance resolution}
The expected result of inheritance resolution is to load the inherited
methods in the environment and get their signatures and methods
information. Hence, combined with the defined methods, we will be able
to know all the methods of the species. We also want to get a possibly
new context where the fact that {\tt Self} is now manifest is
updated, in case we inherited a {\tt representation} method.

By the way, since in the inheritance clause, we can have parametrised
collection applied to {\tt Self} as effective argument, we remind all
the species type that {\tt Self} must be compatible with (called
{\tt self\_must\_be} in the source code). In effect, since we are
building the current species, the species type of {\tt Self} (i.e. the
currently built species) is not yet known. Then to ensure consistency
we must check afterwards, i.e. once we got all the methods of the
species if it compliant with all the species types expected as
effective argument fo the used collections on which we applied
{\tt Self}.
\label{cadaver-self-must-be}
{\bf {\large Note that currently the test is not performed
and to remind we have to do this, the compiler emits a warning}}.

\medskip
When processing each inherited species expression, we type it like we
did for the collection parameters. This gives us the list of methods
the species has (hence the list of methods the current species inherits
via the processed species expression), the list of species types that
{\tt Self} must be compatible with, and finally the substitutions
applied on the formal parameters of parametrised species used in the
{\tt inherits} clause. Such substitutions are those replacing formal
parameters by effective collection expressions when the species
expression used in the {\tt inherits} clause uses application.
This information will be recorded in the history information of the
inherited methods for later use.

For instance, considering the \focalize\ sample code:
{\footnotesize
\begin{lstlisting}
species Foo0 (A0 is Sp0) inherits ... = let v = 1 end ;;
species Foo1 (A1 is Sp1) inherits Foo0 (A1) = let v = 2 end ;;
species Foo2 (A2 is Sp1) inherits Foo1 (A2) = end ;;
species Foo3 (A3 is Sp1, A4 is A3) inherits Foo2 (A4) = end ;;
\end{lstlisting}
}

\noindent we will remind that in {\tt Foo3}, we inherited from
{\tt Foo2} applied to {\tt A4} and that in {\tt Foo2}, we inherited
from {\tt Foo1} applied to {\tt A2}.

These substitutions will be used to represent the fact that in rule
[{\tt COLL-INST}], page 43, figure 3.2, section 3.8 in Virgile
Prevosto's PhD, the substutition $[C_1 \leftarrow C_2]$ is performed
on $t_s$.

% (
%         - [accu_substs] : The list of substitutions to apply to each
%              effective parameter before processing it. This is the way
%              to represent the fact that in rule COLL-INST, page 43,
%              fig 3.2, section 3.8 in Virgile Prevosto's Phd, the
%              substutition "[c1 <- c2]" is performed on ts, i.e. in
%              it's methods types but also in the remaining effective
%              species parameters (yep, indeed, the ts signature of a
%              species contains both the methodes and the parameters).
%              Because these parameters will be "evaluated" after the
%              current one, we need to delay the substitution until they
%              are really processed.
%              This list contains the in *REVERSE* order of the
%              application order. This means that the first required
%              substitution is in tail.           
% )



\subsection{Type-checking methods}
\subsection{Dependency on Self's methods calculus}
Parler de comment on d�tecte les d�pendances sur Self.



\subsection{Normal form}
proof collapsing.
fusion
