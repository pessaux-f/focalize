% $Id: ocamlcodegen.tex,v 1.5 2009-06-16 13:24:00 pessaux Exp $

The code generation starts from the {\tt Infer.please\_compile\_me}
structure returned by the type-checking pass. It will examine each
phrase of the program, call the abstractions computation previously
described (c.f. \ref{intermediate-abstractions}) if needed, before
starting generating some target code.

The most interesting parts of the code generator are those dealing
with species and collection. Toplevel functions do not pose particular
problems, as well as type definitions since \ocaml\ has similar type
definitions. One may note that because \ocaml\ doesn't have logical
features, toplevel theorems trivially lead to no produced
code. Generation for {\tt use} and {\tt open} directives doesn't
produce any code, only {\tt open} has a significant effect, loading
the definitions of the related module in the environment (exactly like
for the other passes).


\section{Species generation}

\subsection{The collection carrier mapping}
To be able to properly map the species parameters' carriers to
type variables in the generated code, we start by creating a
``collection carrier mapping''. This mapping is the correspondance
between the collection type of the species definition parameters and
the type variables names to be used later during the OCaml
translation. For a species parameter {\tt A is/in ...}, the type
variable that will be used is ``''' + the lowercased name of the
species parameter + an integer unique in this type +
``{\tt \_as\_carrier}''.

We need to add an extra integer (in fact, a stamp) to prevent a same
type variable from appearing several time in the tricky case where a
{\tt in} and a {\tt is} parameters wear the same lowercased name. For
instance in
{\footnotesize\lstinline!species A (F is B, f in F)!}
where {\tt F} and {\tt f} will lead to a same name of \ocaml\ type
variable: ``{\tt 'f\_as\_carrier}''.

Hence, each time we will need to generate a type expression involving
a species parameter carrier, by a simple look-up in the mapping we
will get the type variable's name to emit. This mapping primarily
serves to the type pretty-print function. It also have few minor
usages that will be explained when we will encounter the case.

\medskip
The mapping gets stored in a compilation context (a bit like the
context we already saw for type-checking) with various other
structures that will be always passed to the compilation
functions. This way, grouping all in one unique argument makes the
code clearer.

\subsection{The record type}
As described in \label{code-gen-model}, a species starts by a record
type definition. This record contains one field per method. This type
is named ``{\tt me\_as\_species}'' to reflect the point that it
represents the \ocaml\ structure representing the \focal\ species.
Depending on whether the species has parameters, this record type also
has parameters. In any case, it at least has a parameter representing
``self as it will be once instanciated'' once ``we'' (i.e. the
species) will be really living as a collection.

\medskip
If the carrier is defined, then before the record definition, we
generate the type definition ``{\tt me\_as\_carrier}'' that shown the
constraints due to the {\tt representation} definition
(c.f. \ref{ocaml-tydef-for-defined-repr}). This is done by the
function {\tt generate\_rep\_constraint\_in\_record\_type}.

\medskip
Next, we must start the generation of the record type itself. Be
careful that it will have at least one type parameter (the one
representing our carrier and named {\tt 'me\_as\_carrier}). It can
have several parameters if the species has species parameters. So, we
start generating the {\tt 'me\_as\_carrier}, and we use the collection
carriers mapping to generate the other type parameters corresponding
to the carriers of the species parameters.

\medskip
Now comes the point where we must generate the record fields. The
first weird thing is that we must extend the collections carrier mapping
with ourselve known. This is required when {\tt representation} is defined. 
Hence, if we refer to our {\tt representation} (i.e.
{\tt me\_as\_carrier}), not to {\tt Self}, I mean to a type-collection
that is ``(our compilation unit, our species name)'' (that is the case
when creating a collection where {\tt Self} gets especially abstracted
to ``(our compilation unit, our species name)'', we  will be known and
we wont get the fully qualified type name, otherwise this would lead
to a dependency with ourselve in term of \ocaml\ module.

Indeed, we now may refer to our carrier explicitely here in the scope
of a collection (not species, really collection)  because there is no
more late binding: here when one say ``me'', it's not anymore ``what I
will be finally'' because we are already ``finally''. Before, as long
a species is not a collection, it always refers to itself's type as
``{\tt 'me\_as\_carrier}'' because late binding prevents known until
the last moment who ``we will be''. But because now it's the end of
the species specification, we know really ``who we are'' and
``{\tt 'me\_as\_carrier}'' is definitely replaced by ``who we really
are'' : ``{\tt me\_as\_carrier}''.

We can now iterate through the list of fields of the species, to
create the record's fields. We take care to not generate {\tt let}
fields whose type involves {\tt prop} since they can only be created
by {\tt logical let} that are discarded in \ocaml. We also skip
theorem and property fields.

\medskip
Once done, we close the record definition.


\subsection{Abstraction computation}
As previously explained, each back-end triggers the computation of
abstractions (i.e. things -- carriers, methods -- to abstract due to
dependencies). From this computation we get the list of fields of the
species with the information explaining what to $\lambda$-lift.


\subsection{Definitions' code generation}
We then iterate code generation on each field, depending on its kind.
Signatures, properties and theorem are purely discarded in \ocaml\
since they have no mapping and no use.

It then remains the {\tt let}-definitions. The generation consists in
2 parts: the generation of the extra parameters due to $\lambda$-lifts
and the translation of the function's body into \ocaml. This last part
is quite straightforward, so we will take more time on the first one.

The first thing is that only methods defined in the current species
must be generated. Inherited methods {\bf are not} generated again.
So we start generating the the {\tt let} and name of the
function.

\medskip
Next come the $\lambda$-lifts that abstract according to the species's
parameters the current method depends on. We process each species
parameter. For each of them, aach abstracted method will be named like
``{\tt \_p\_}'', followed by the species parameter name, followed by
``{\tt \_}'', followed by the method's name. We don't care here about
whether the species parameters is {\tt in} or {\tt IS}.

\medskip
Next come the extra arguments due to methods of ourselves we depend
on. They are always present in the species under the name
``{\tt abst\_...}''. These $\lambda$-lifts are only done for methods
that are in the minimal coq environment because they computational and
only declared.

\medskip
Next come the parameters of the {\tt let}-binding with their type. We
ignore the result type of the {\tt let} if it's a function because we
never print the type constraint on the result of the {\tt let}. We
only print them in the arguments of the {\tt let}-bound identifier.
We also ignore the variables used to instanciate the polymorphic ones
of the scheme because in \ocaml\ polymorphism is not explicit.   
Note by the way thet we do not have anymore information about
{\tt Self}'s structure...

Becareful that while printing the type of the function's arguments,
since they belong to a same type scheme they may share variables
together. For this reason, we first purge the printing variable
mapping and after, activate its persistence between each parameter
printing. This allows the type pretty-print function to ``remember''
the variables already seen and print the same name if it see one of
them again. And this, until we release the persistence.

\medskip
Finally, we can dump the code corresponding to the translation of the
function's body. The only tricky part is to generate code for
identifiers because we must take care of wether the identifier
represents a local variable, an entity parameter, a method of
{\tt Self}, a toplevel identifier or a collection parameter's method.
We must also be able to detect occurrences of recursive calls to be
sure that at application time, we will really provide the arguments
for the $\lambda$-lifts.

\subsection{If the species is complete}
\ldots then we must manage the fact that a collection generator must
be created. The generator is created from the list of compiled fields.
The idea is to make a function that will be parametrised by all the
dependencies from species parameters and returning a value of the
species record type.

\medskip
The generic name of the collection generator:
``{\tt collection\_create}''. Be careful, if the collection generator
has no extra parameter then the ``{`\tt collection\_create}'' will not
be a function but directly the record representing the species. In
this case, if some fields of this record are polymorphic, \ocaml\
won't generalize because it is unsound to generalise a value that is
expansive (and record values are expansives). So, to ensure this won't
arise, we always add one {\tt unit} argument to the generator. We
could add it only if there is no argument to the generator, but it is
pretty boring and we prefer just to {\bf K}eep {\bf I}t {\bf S}imple
and {\bf S}tupid \smiley.

\medskip
After the name of the generator, we must generate the parameters the
collection generator needs to build the each of the current species's
local function (functions corresponding to the actual method stored in
the collection record). These parameters of the generator come from
the abstraction of methods coming from our species parameters we
depend on. By the way, we want to recover the list of species
parameters linked together with their methods we need to instanciate
in order to apply the collection generator. To do so, we first build
by side effect the list for each species parameter of the methods we
depend on (using the dependency information previously computed). Then
we simply dump this list, using the naming scheme ``{\tt \_p\_''} +
the species parameter name + ``{\tt \_}'' + the called method
name. Since that also this way species parameters methods are called
in the bodies of the local functions of {\tt Self} (see below), this
will really lead to bind this abstracted identifiers in these bodies.

\medskip
At this point comes the moment to generate the local functions that
will be used to fill the record value. We then iterate on the list of
compiled fields of the species, skipping the {\tt logical let}s, to
find the method generator of the field and apply it to all it
needs. ``All it needs'' means stuff abstracted because of dependencies
on species parameters and things abstracted because of dependencies on
other methods of ourselves.

To get the method generator, we must check at which inheritance level
it is, in other words in which species its code was defined. To do so,
since we know that the species is complete, we just need to look in
the {\tt from\_history} of the field and the generator is defined in
the {\tt fh\_initial\_apparition} (since by construction, we record
the first declaration is no definition or the definition of a method
here, and subsequent apparition due to inheritance are recorded
somewhere else).

We first start by abstractions for species parameters. Depending on if
the method generator is inherited or directly defined in the current
species, we have 2 behaviours.

In the simples case, it is defined in the current species and we just
need to apply the method generator to each of the extra arguments
induced by the various lambda-lifting we previously performed for
species parameters: here we will not use them to $\lambda$-lift them
this time, but to apply them ! The name used for application is formed
according to the same scheme we used at $\lambda$-lifting time:
``{\tt \_p\_}'' + the species parameter name + ``{\tt \_}'' + the
called method name.

However, if the method is inherited, the things are more complex. We
must apply the method generator to each of the extra arguments induced
by the various lambda-lifting we previously in the species from which
we inherit, i.e. where the method was defined. During the inheritance,
parameters have been instanciated. We must track these instanciations
to know to what apply the method generator.


\subsubsection{Following parameters instantiations}
This work is performed by
{\tt species\_ml\_generation.instanciate\_parameter\_through\_inheritance}.
We search to instanciate the parameters ({\tt is} and {\tt in}) of the
method generator of one {\tt field\_memory} (i.e. the data-structure that
represents what we know about a method that was previously
generated). The parameters we deal with are those coming from the
$\lambda$-lifts we did to abstract dependencies of the method
described by the {\tt field\_memory] on species parameters of the species
where this method is {\bf defined}.
Hence we deal with the species parameters of the species where the
method was {\bf defined} ! It must be clear that we do not matter of
the parameters of the species who inherited !!! We want to trace by
what the parameters of the original hosting species were instanciated
along the inheritance.

So we want to generate the \ocaml\ code that enumerates the arguments to
apply to the method generator. These arguments are the methods coming
from species parameters on which the current method has dependencies
on. The locations from where these methods come depend on the
instanciations that have be done during inheritance.

This function trace these instanciations to figure out exactly from
 where these methods come. Process sketch follows:
\begin{enumerate}
\item Find at the point (i.e. the species) where the method generator
 of the {\tt field\_memory} was defined, the species parameters that
 were existing at this point. 
\item Find the dependencies the original method had on these (its) species
    parameters.
\item For each of these parameters, we must trace by what it was
    instanciated along the inheritance history, (starting from oldest
    species where the method appeared to most recent) and then generate
    the corresponding \ocaml\ code.
    \begin{enumerate}
    \item Find the index of the parameter in the species's signature
    from where the method was {\bf really} defined (not the one where
    it is inherited).
    \item Follow instanciations that have been done on the parameter
    from past to now along the inheritance history.
    \item If it is a {\tt in} parameter then we must generate the code
    corresponding to the \focal\ expression that instanciated the
    parameter. This expression is built by applying
    effective-to-formal arguments substitutions.
    \item If it is a IS parameter, then we must generate for each
    method we have dependencies on, the \ocaml\ code accessing the
    \ocaml\ code of the method inside its module structure (if
    instanciation is done by a toplevel species/collection) or
    directly use an existing collection generator parameter (if
    instanciation is done by a parameter of the species where the
    method is found inherited, i.e. the species we are currently
    compiling).
    \end{enumerate}
\end{enumerate}


\subsubsection{Ending the collection henerator}
Finally, we must apply the method generator to each of the extra
arguments induced by the methods of our inheritance tree we depend on
and that were only declared when the method generator was
created. These methods leaded to ``local'' functions defined above (by
the same process we describe here). Hence, for each  method only
declared of ourselves we depend on, its name is ``{\tt local\_}'' +
the method's name. Since we are in \ocaml, we obviously skip the logical
methods.

We were at the point of generating the ``local'' functions
corresponding to our methods. This is now done, and we only have to
create now the record value representing the collection returned by
the collection generator. To do this, we only assign each record
fields corresponding to the current species's method the corresponding
``local'' function we defined just above. Remind that the record
field's is simply the method's name. The local function corresponding
to the method is ``{\tt local\_}'' + the method's name.


\subsection{Ending the code of a species}
We ar enow done with the process handling the case a species is fully
defined, hence has a collection generator. We just now need to create
the data that will be recorded in the \ocaml\ code generation
environment. Remember that while creating the collection generator, we
returned the list of arguments it need. This info will be part of what
is stored in the environment. We also build the list of the
{\tt compiled\_field\_memory}'s of the methods and the information
about the species parameters.


\section{Collection generation}
