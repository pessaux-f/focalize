Currently, a recursive function is handled in 3 ways in \coq\ depending
on whether it is a:
\begin{itemize}
  \item recusive method: usage of {\tt Function}.
  \item local recursive function: usage of {\tt fix} with assumption
  that the function is structurally recursive on its {\bf first} argument.
  \item toplevel recursive function: usage of {\tt Fixpoint} with assumption
  that the function is structurally recursive on its {\bf first} argument.
\end{itemize}

The entry point for recursive function code generation in \coq\ is
{\tt generate\_recursive\_let\_definition} from module {\tt Species\_coq\_generation} .


\subsubsection{Recursive method}
{\footnotesize
\begin{lstlisting}
type t = | D | C (t) ;;

species A =
  signature toto : int ;

  let rec foo(x, y) =
    match x with
    | D ->
        if syntactic_equal(y, 0) then 1
        else 1 + foo(D, (y + 1))
    | C(D) -> 2
    | C(a) -> foo(a, (y - toto)) ;
end ;;
\end{lstlisting}
}

Entry point: {\tt generate\_defined\_recursive\_let\_definition\_With\_Function}.
In this configuration, the generated code uses {\tt Function}, creates
a {\tt Module} and a {\tt Section} to embed the termination
proof. Note that logical recursive methods are not yet handled.

{\footnotesize
\begin{lstlisting}[language=MyCoq]
Module Termination_foo_namespace.
  Section foo.
\end{lstlisting}
}

Then, is creates {\tt Variables} for methods on which the recursive
function depends (like always, using \\
{\tt generate\_field\_definition\_prelude}). Since we are in a {\tt Section}
for \zenon, we tell instead of abstracting dependencies by adding
extra arguments to  the current definition, we generate {\tt Variable},
{\tt Let} and {\tt Hypothesis}. This prelude inserts abstractions due
to the types of the species parameters, then dues to methods of the
species parameters the function depends on, then finally methods of
{\tt Self} on which the function depends on.

{\footnotesize
\begin{lstlisting}[language=MyCoq]
    Variable abst_toto : basics.int__t.
\end{lstlisting}
}

A {\tt Variable} always called ``{\tt  \_\_term\_order} is now generated
to represent the order. It always has 2 arguments having the same
type. This type is a {\bf tuple} if the method has several arguments.
We see below that our function takes 2 parameters, one of type {\tt t}
the other of type {\tt int}. So the order takes 2 tuples of type
{\tt (t * int)}, i.e. will have to compare 2 ``sets'' of call
arguments. It always return {\tt Prop}.

{\footnotesize
\begin{lstlisting}[language=MyCoq]
    (* Abstracted termination order. *)
    Variable __term_order :
      (((t__t) * (basics.int__t))%type) -> (((t__t) * (basics.int__t))%type) -> Prop.
\end{lstlisting}
}

Then the proof obligation that this order is well-founded come. The
{\tt Variable} representing the termination proof obligation is always
called ``{\tt \_\_term\_obl}''.
It's now time to generate the lemmas proving that each recursive call
decreases. Each of then will be followed by a {\tt /\textbackslash} to make the
conjunction of all of them. And the latest one will be used to add 
the final ``{\tt well\_founded \_\_term\_order}'' to this big conjunction.
This is big job is performed by {\tt generate\_termination\_lemmas} of
the module {\tt Rec\_let\_gen}.

This function takes the list (initial variable of the function * expression provided
in the recursive calls). The expression must hence be $<$ to the
initial variable for the function to terminate. In fact that's the tuple of
initial variables that must be $<$ to the tuple of expressions provided
in each recursive call. Basically, the algorithm is the following:

\noindent For each recursive call:\\
\verb+  +(* For each binding, bind the variable by a forall. \\
\verb+  +Done by {\tt Rec\_let\_gen.generate\_variables\_quantifications} *) \\
\verb+    + For each binding \\
\verb+      + match binding with \\
\verb+        + | Recursion.B\_let (name, scheme) -> \\
\verb+          + "forall $\%$a : $\%$a, " name (specialize scheme) \\
\verb+        + | Recursion.B\_match (\_, pattern) -> \\
\verb+          + (* Get all bound variable of the pattern. *) \\
\verb+          + (* Generate a forall for each bound variable. *) \\
\verb+          + "forall $\%$a : $\%$a, " var\_name pat-var-ty \\
\verb+        + | Recursion.B\_condition (\_, \_) -> \\
\verb+          +  (* No possible variable bound, so nothing to do. *) \\
\verb+  +(* We must generate the hypotheses and separate them by ->. *) \\
\verb+  +List.iter \\
\verb+    +(function \\
\verb+      +| Recursion.B\_let ->\\
\verb+        +(* A "let x = e" induces forall x: ..., x "=" e. *) \\
\verb+        +{\tt Rec\_let\_gen.generate\_binding\_let} \\
\verb+      +| Recursion.B\_match (expr, pattern) ->\\
\verb+        +(* Induces "forall variables of the pattern, pattern = expr". *) \\
\verb+        +{\tt Rec\_let\_gen.generate\_binding\_match} \\
\verb+      +| Recursion.B\_condition (expr, bool\_val) ->\\
\verb+        +(* Induces "Is\_true (expr)" if bool\_val is true,\\
\verb+        +else "$\sim$ Is\_true (expr)" if bool\_val is false. *) \\
\verb+    +) \\
\verb+    +bindings ;\\
\verb+  +(* Now, generate the expression that tells the decreasing by applying \\
\verb+  +the "\_\_term\_order" {\tt Variable} or the really defined order if we were \\
\verb+  +provided one by the argument [~explicit\_order] with its arguments to \\
\verb+  +apply due to lambda-liftings. *) \\
\verb+  +(* Now, generate the arguments to provide to the order. They are 2 tuples, \\
\verb+  + the first one being the one containing arguments of the recursive call, \\
\verb+  + the second one being the arguments of the original call. *) \\
\verb+  +(* Separate by a /\textbackslash *)

{\footnotesize
\begin{lstlisting}[language=MyCoq]
    Variable __term_obl :
      (forall x : t__t, forall y : basics.int__t,
       (@D = x) -> ~ Is_true ((basics.syntactic_equal _ y 0)) ->
         __term_order ((@D ), ((basics._plus_ y 1))) (x, y))
      /\
      (forall x : t__t, forall y : basics.int__t,
       forall a : t__t,
       (@C  a = x) ->
         __term_order (a, ((basics._dash_ y abst_toto))) (x, y))
\end{lstlisting}
}

Above, we the 2 cases of recursive calls with in one, {\tt
  \_\_term\_order} applies to $({\tt D}, (y+1))$ and $(x, y)$ and in
the other case, applied to $(a, (y - {\tt toto}))$ and $(x, y)$.

{\footnotesize
\begin{lstlisting}[language=MyCoq]
      /\
       (well_founded __term_order).
\end{lstlisting}
}

Now comes the translation of the real body of the recursive function,
using the \coq\ {\tt Function} construct. We generate its arguments as
a tuple if there are several. Before the return type is inserted the
verbatim code ``{\tt \{wf \_\_term\_order \_\_arg\}}''. We need then
to recover the individual arguments of the function. Unfortunately, we
can't simply generate {\tt let (x, y, ..) := \_\_arg} because \coq
only allows pairs as let-binding pattern. So instead,  we generate a
{\tt match}.

{\footnotesize
\begin{lstlisting}[language=MyCoq]
    Function foo (__arg: (((t__t) * (basics.int__t))%type))
      {wf __term_order __arg}: basics.int__t
      :=
      match __arg with
        | (x,
        y) =>
\end{lstlisting}
}

Before being able to generate the effective body of the function, we
must transform the recursive function's body so that all the recursive
calls send their arguments as a unique tuple rather than as several
arguments. This is because we ``tuplified'' the arguments of the
recursive function in order to be able to exhibit a lexicographic
order if needed. This job is done by
{\tt transform\_recursive\_calls\_args\_into\_tuple} of the module
{\tt Rec\_let\_gen}. Once we got this modified body expression, we can
now generate the code, telling that we must not apply recursive calls to the extra
arguments due to lambda-liftings because we are in the scope of a {\tt Section}
containing {\tt Variable}s representing dependencies. Code generation
for this body is as usual, using of the module {\tt generate\_expr}
{\tt Species\_record\_type\_generation}.

{\footnotesize
\begin{lstlisting}[language=MyCoq]
        match x with
         | D  =>
             (if (basics.syntactic_equal _ y 0) then 1
              else (basics._plus_ 1 (foo ((@D ), ((basics._plus_ y 1))))))
         | C D  => 2
         | C a => (foo (a, ((basics._dash_ y abst_toto))))
         end
        end.
\end{lstlisting}
}

Now comes the termination proof. We first enforce {\tt Variable}s to
be used to prevent \coq from removing them (we generate {\tt assert}
for this sake).
Anf finally we apply {\tt magic\_prove} for each recursive call + once
for the {\tt (well\_founded \_\_term\_order)}. In our example, this is
2 recursive calls (so 2 proof goals) + 1 = 3.

{\footnotesize
\begin{lstlisting}[language=MyCoq]
      Proof.
        assert (__force_use_abst_toto := abst_toto).
        apply coq_builtins.magic_prove.
        apply coq_builtins.magic_prove.
        apply coq_builtins.magic_prove.
        Qed.
\end{lstlisting}
}

We must now create the curried version of the function, i.e. having
several arguments and not 1 being a tuple. Remember that to be able to
create an order, we transformed the initial (and ``real'') function by
making it taking a tuple of arguments instead of several
arguments. It's then now time to put thing back in place because code
using this function relies on it having several arguments and not one
tuple of arguments ! Finally we close the open {\tt Section} and {\tt Module}.

{\footnotesize
\begin{lstlisting}[language=MyCoq]
      Definition A__foo x  y := foo (x, y).
      End foo.
    End Termination_foo_namespace.
\end{lstlisting}
}

TO BE CONTINUED.

{\footnotesize
\begin{lstlisting}[language=MyCoq]
  Definition foo (abst_toto : basics.int__t) :=
    Termination_foo_namespace.A__foo abst_toto coq_builtins.magic_order.
\end{lstlisting}
}


\subsubsection{Local recursive function}
{\footnotesize
\begin{lstlisting}
species B =
  let bar(a, b : int) =
    let rec rec_bar(x, y) =
      match x with
      | D -> 0
      | C(z) -> 1 + rec_bar(z, y) in
    rec_bar(a, b) ;
end ;;
\end{lstlisting}
}
They are trivially compiled into a {\tt let fix} construct,
hard-wiring that the the structurally decreasing argument is the first one.
{\footnotesize
\begin{lstlisting}[language=MyCoq]
Definition bar (a : t__t) (b : basics.int__t) : basics.int__t :=
  let fix rec_bar (__var_a : Set) (x : t__t) (y : __var_a) {struct x} :
    basics.int__t :=
    match x with
     | D  => 0
     | C z => (basics._plus_ 1 (rec_bar _ z y))
     end
  in (rec_bar _ a b).
\end{lstlisting}
}


\subsubsection{Toplevel recursive function}
The function:
{\footnotesize
\begin{lstlisting}
let rec gee(x, y) =
    if syntactic_equal(x, 0) then 1
    else 1 + gee((x - 1), (y + 1))
;;
\end{lstlisting}
}

\noindent into:

{\footnotesize
\begin{lstlisting}[language=MyCoq]
Fixpoint gee (x : basics.int__t) (y : basics.int__t) {struct x} : basics.int__t :=
  (if (basics.syntactic_equal _ x 0) then 1
   else (basics._plus_ 1
             (gee ((basics._dash_ x 1)) ((basics._plus_ y 1))))).
\end{lstlisting}
}

This fails since this function is not recursive structural. Instead,
consider the following recursive structural function:

{\footnotesize
\begin{lstlisting}
let rec buzz(x) =
  match x with
  | A -> 0
  | C(y) -> 1 + buzz(y)
;;
\end{lstlisting}
}

\noindent which is compiled into:

{\footnotesize
\begin{lstlisting}[language=MyCoq]
Fixpoint buzz (x : t__t) {struct x} : basics.int__t :=
  match x with
   | A  => 0
   | C y => (basics._plus_ 1 (buzz y))
   end.
\end{lstlisting}
}
