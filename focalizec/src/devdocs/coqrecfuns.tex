Currently, a recursive function is handled in 3 ways in \coq\ depending
on whether it is a:
\begin{itemize}
  \item recusive method: usage of {\tt Function}.
  \item local recursive function: usage of {\tt fix} with assumption
  that the function is structurally recursive on its {\bf first} argument.
  \item toplevel recursive function: usage of {\tt Fixpoint} with assumption
  that the function is structurally recursive on its {\bf first} argument.
\end{itemize}

\subsubsection{Recursive method}
{\footnotesize
\begin{lstlisting}
species A =
  let rec foo(x) =
    if syntactic_equal(x, 0) then 1
    else 1 + foo(x - 1) ;
end ;;
\end{lstlisting}
}

{\footnotesize
\begin{lstlisting}[language=MyCoq]
Module Termination_foo_namespace.
  Section foo.
   (* Abstracted termination order. *)
    Variable __term_order : (basics.int__t) -> (basics.int__t) -> Prop.
    Variable __term_obl :(forall x : basics.int__t, ~
      Is_true ((basics.syntactic_equal _ x 0)) -> __term_order
      (basics._dash_ x 1) x)
      /\
       (well_founded __term_order).
      
    Function foo (__arg: (basics.int__t)) {wf __term_order __arg}:
      basics.int__t :=
      match __arg with
        | (x) =>
        (if (basics.syntactic_equal _ x 0) then 1
          else (basics._plus_ 1 (foo (basics._dash_ x 1))))
        end.
      Proof.
        apply coq_builtins.magic_prove.
        apply coq_builtins.magic_prove.
        Qed.
      Definition A__foo x := foo (x).
      End foo.
    End Termination_foo_namespace.
  Definition foo := Termination_foo_namespace.A__foo
    coq_builtins.magic_order.
\end{lstlisting}
}


\subsubsection{Local recursive function}
{\footnotesize
\begin{lstlisting}
type t = | A | C (t) ;;

species B =
  let bar(a) =
    let rec rec_bar(x) =
      match x with
      | A -> 0
      | C(y) -> 1 + rec_bar(y) in
    rec_bar(a) ;
end ;;
\end{lstlisting}
}
They are trivially compiled into a {\tt let fix} construct,
hard-wiring that the the structurally decreasing argument is the first one.
{\footnotesize
\begin{lstlisting}[language=MyCoq]
Definition bar (a : t__t) : basics.int__t :=
  let fix rec_bar (x : t__t) {struct x} : basics.int__t :=
    match x with
     | A  => 0
     | C y => (basics._plus_ 1 (rec_bar y))
     end
  in (rec_bar a).
\end{lstlisting}
}


\subsubsection{Toplevel recursive function}
The function:
{\footnotesize
\begin{lstlisting}
let rec gee(x) =
  if syntactic_equal(x, 0) then 1
  else 1 + gee(x - 1)
;;
\end{lstlisting}
}

\noindent into:

{\footnotesize
\begin{lstlisting}[language=MyCoq]
Fixpoint gee (x : basics.int__t) {struct x} : basics.int__t :=
  (if (basics.syntactic_equal _ x 0) then 1
    else (basics._plus_ 1 (gee (basics._dash_ x 1)))).
\end{lstlisting}
}

This fails since this function is not recursive structural. Instead,
consider the following recursive structural function:

{\footnotesize
\begin{lstlisting}
let rec buzz(x) =
  match x with
  | A -> 0
  | C(y) -> 1 + buzz(y)
;;
\end{lstlisting}
}

\noindent which is compiled into:

{\footnotesize
\begin{lstlisting}[language=MyCoq]
Fixpoint buzz (x : t__t) {struct x} : basics.int__t :=
  match x with
   | A  => 0
   | C y => (basics._plus_ 1 (buzz y))
   end.
\end{lstlisting}
}
