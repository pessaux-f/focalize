% $Id: phd_changes.tex,v 1.2 2009-03-13 16:57:57 pessaux Exp $
\subsection{Type unification}
\noindent Section 3.3, definition 9, page 27.

\noindent Rule [SELF1] should be: $mg(t, {\bf Self}, t) = {\bf Self}, id$

\noindent Rule [SELF2] should be: $mg(t, t, {\bf Self}) = {\bf Self}, id$



\subsection{Normal form algorithm}
\noindent Section 3.7.1, page 36.

\noindent In the algorithm, line 10 should be:
$\mathbb{W}_1 \leftarrow ((\psi_{i_{0}} \varogreaterthan \phi), \mathbb{X}).$

\smallskip

\noindent In the algorithm, line 13 should be:
$\mathbb{W}_2 \leftarrow (\mathbb{W}_2, \phi)$.

\smallskip
\noindent In the running text, page 37, line 8, the same modification
must be done to get ``on garde $\psi_{i_{0}} \varogreaterthan \phi$
dans $\mathbb{W}_1$ \ldots''.



\subsection{Typing rules for parametrised species}
\noindent Section 3.8, figure 3.2, page 43.

\noindent Rule [COL-PRM] should be:

\inferrule
  {
  {\cal C}, \Omega \vdash e^S : a \\
  {\cal C} + C : {\cal A} (a, C), \Omega \vdash
    {\bf species}\ S (prms)\ {\bf inherits}\ e^{S}_1, \ldots
    e^{S}_{h_f} = \Phi_1 \ldots \Phi_n : t_S
  }
  {
  {\cal C}, \Omega \vdash {\bf species}\ S(C \ {\bf is}\ e^S, prms)
  \ {\bf inherits}\ e^{S}_1, \ldots e^{S}_{h_f} =
  \Phi_1 \ldots \Phi_n : (C\ {\bf is}\ a)t_S
  }



\subsection{Dependency on the carrier}
\noindent Section 3.9.4, definition 28, page 50.

\noindent The definition should be: ``Soit une expression $e$, si une
sous-expression de $e$ \rlap{----------}a le type {\bf fait r�f�rence �
(``contient'')} {\tt Self}, il y a une
 decl-d�pendance
vis-�-vis du type support.''.

\noindent In English: ''Let's have $e$ an expression, if a
sub-expression of $e$ \rlap{-----------}has type {\bf makes reference
to (``contains'')} {\tt Self}, then there is a decl-dependency
on the carier''.

\medskip
This is more  accurate since an expression having type
${\tt int} \rightarrow {\tt Self}$ does not have type {\tt Self}, but
when we state its type, {\tt Self} occurs in the type and must be
bound somewhere (and the decl-dependency on the carrier is just there
for this purpose).



% MERDE !!! Je ne me souviens plus pourquoi. L'explication que j'ai
% foutue sur la version papier de la th�se est merdique. Bon, �
% investiguer plus tard...
%
% \subsection{Inference rules for logical statements}
% \noindent Section 3.9.5, figure 3.3, page 52.

% \noindent The rule [EXPR] states that one can infer that an expression
% $expr$ has type {\tt prop} in $\Sigma$ if one can infer that is has
% type {\tt bool} in the ``reduced'' $\Sigma^{*}$.
% This is correct in the context of type checking. But this is not exact
% in the context of type {\bf inference}. In fact, it doen't work since
% the rules [EX] and [ALL] need to enter the bound variables 



\subsection{Dependencies in a species}
\noindent Section 3.9.5, definition 31, page 53.

\noindent Lines 3 and 4 should be:

\hspace{1cm} $\forall j < n,
   \ y_{j} \in \decldeps{y_{i}}_S \cup \defdeps{y_{j+1}}_S$

\noindent $x_1 <^{def}_{S} x_2 \ \widehat{=}\ \exists \{y_i\}_{i=1\ldots n}
  \ tel\ que \ y_1 \circlearrowright_S x_1, y_n \circlearrowright_S x_2,
  \forall j < n,\ y_j \in \defdeps{y_{j+1}}_S$



\subsection{Parameters used by a method}
\noindent Section 3.9.5, definition 66, page 124.

\noindent To understand the rules [BODY], [TYPE], [DEF-DEP], [UNIVERS]
and [PRM], it should be stated that implicitely the parameter $C_{p'}$
has the form: $$C_{p'} {\tt is/in\ } \tau_{p'}$$.


\subsection{Instanciation of species parameters}
\noindent Section 3.9.5, definition 67, page 124.

\noindent Second rule should be:
\inferrule
  {
  x \Lsh S = S_h \\
  {\cal E}(S_h) = (C_1 \centerdot \tau_1, \ldots,
                   C_{p_f} \centerdot \tau_{p_f}) \\
  l_h = e_1 \ldots\ e_{p_f}
  }
  {
  {\tt Inst}_S (x) = \{ {\tt Inst}_{C_p}(e_p)_{C_p\in {\cal U}_{S_h}(x)}\}
  }


\subsection{Traduction example in \ocaml}
\noindent Section 8.2, page 152.

The code sample shown and the explanation about {\tt create} at the
top of the page is wrong or at least not complete. In effect, if
presented this way, we don't know from where {\tt create} comes. In
fact, we must used the one coming from the species we
``implement''. So a qualified notation (i.e. module name + function
name) is required.


\subsection{Coq code generation model}
\subsection{Missing rule for dependencies on parameters}
\subsection{Missing order for dependencies on parameters}

