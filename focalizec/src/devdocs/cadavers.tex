% $Id: cadavers.tex,v 1.1 2009-05-04 09:04:45 pessaux Exp $

Here is the holly section of things that are not yet done and for
which we know there is still something to do. Currently, most of them
are not very intricate and do not impact the general sanity of the
compiler. However, it is possible to create some cases of programs
falling in these open points. Here are the cadavers we hide under the
carpet\ldots\smiley


\begin{itemize}
\item The verification that {\tt Self} is really compatible with the
  list of species types encountered during species expression typing
  is not currently done. This deals with inheritance from parametrised
  species applied to {\tt Self} as described in
  \ref{cadaver-self-must-be}. Most of the points in the code are
  tagged by a comment {\tt [Unsure]} and involve an identifier
  named {\tt self\_must\_be} to represent the unused list of these
  species types {\tt Self} must be compatible with. In fact, the
  function performing compatibility test already exists and is called
  {\tt is\_sub\_species\_of} in the source file {\tt typing/infer.ml}
  but we do not use it for this task.

\item Recursive functions with termination proofs using \coq's
  {\tt Function} construct are to be finished. I have some notes
  about this and must scan them instead of rewriting all. $A
  priori$, there is not tons of work to do for this and major
  identified points are:
  \begin{itemize}
  \item Adding $\alpha$ conversion in some parts of the generated
    code to prevent name conflicts in \coq.
  \item Rewrite the {\tt LTac} of William that should provide a mean
    to prove well foundation of orders from simple basic orders.
  \item Instead of using {\tt magic\_order} at the end of the proof,
    use the defined order (can be done only when the above {\tt LTac}
    will work since it will provide the well foundation of the
    order.
  \item Verify the {\tt measure} kind of proof and implement the
    {\tt structural} kind based on the primary brick {\tt order}.
  \end{itemize}

\item Check record label exhaustivity when dealing with expressions of
  type record. Currently the compiler only issues a warning to say
  that it is not done.
  
\item Have an automated and transparent renaming mechanism to prevent
  identifiers used in \focalize\ to make syntax errors in the target
  languages if they are token of these languages. For instance,
  defining an identifier ``{\tt module}'' will make a syntax error in
  the generated \ocaml\ code since ``{\tt module}'' is a keyword in
  \ocaml.

\item Local collections are not yet re-implemented. One must
  understand the semantics of the structure we want before.

\item And most generally, the few points in the code tagged by a
  comment  {\tt [Unsure]}. These are the points where I strongly
  wondered without having yet found a solution in which I have a
  strong, indubitable confidence.
\end{itemize}
