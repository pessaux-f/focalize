% $Id: scoping.tex,v 1.6 2009-05-12 13:10:19 pessaux Exp $
The scoping process aims to link each identifier occurrence to its
related definition according to the rules that drive the visibility of
identifiers. It avoids name confusion and ensure that each occurrence
of identifier refers to one and one unique effective definition.

Ideally, once scoped, a program should have unique names for each
identifier (i.e. a name plus a stamp that ensure the unicity). For
example in the following piece of code:
{\footnotesize
\begin{lstlisting}
let y = 0 ;;
let x =
  let x = 1 in
  let x = x + 1 in
  let y = x + x in
  y + x in
y + x ;;
\end{lstlisting}
}
scoping rules allow to uniquely identify identifiers by renaming them
in:
{\footnotesize
\begin{lstlisting}
let y_0 = 0 ;;
let x_0 =
  let x_1 = 1 in
  let x_2 = x_1 + 1 in
  let y_1 = x_2 + x_2 in
  y_1 + x_2 in
y_0 + x_0 ;;
\end{lstlisting}
}

This way, each occurrence has a clearly identified definition since if
two identifiers have the same name then they refer to the same
definition. Moreover, this ensure that all the occurrences have one
and one unique related definition, hence preventing unbound
identifiers.

Moreover, since compilation unit can be ``{\tt open}-ed'', some
identifiers can be used without explicit qualification, then looking
like belonging to the current compilation unit although they belong to
another one. Scoping also allow to ``rename'' identifiers belonging to
an ``{\tt open}-ed'' unit to make their hosting unit clear by adding a
qualification.

In \focalize, we do not rename the identifiers with indices. Instead,
we make they qualification explicit. The case where two identifiers
with the same qualification are present is handled by the environment
mechanism that will hide the oldest identifier definition by the
newest according to their order in the source code. In fact, in the
case of toplevel definitions in a same compilation unit and methods in
a same species, the environment mechanism refuses to have several
times the same names. This is not a technical problem, this is only a
choice to prevent the programmer from masking these fundamental kind
of definitions.

Hence the output of the scoping pass is an AST where all the
identifiers occurrences received a qualification if they are not
locally bound. This means that by default, the parser must parse
identifiers that are not qualified (i.e. with no \#-notation and/or no
!-notation) as {\bf local identifiers} (i.e. as
{\tt Parsetree.I\_local}). This means that during scoping, only
{\tt Parsetree.I\_local} identifiers will be affected by the scoping
transformation. Local identifiers will be looked-up to determine
whether they are really local or are method names or toplevel (of a
file) names. The transformation is not performed in place. Instead, we
return a fresh AST (still possibly having sharing with the original
one) that will be suitable for the typechecking phase.

For identifiers already disambiguated by the parser, there are 2 cases:
``\#-ed'' and ``!-ed'' identifiers. The scoper will still work by
ensuring that these identifiers are really related to an existing
definition.
\begin{itemize}
\item For ``\#-ed'' identifiers, the look-up is performed and they are
  always explicitly replaced with the name of the hosting file where
  they are bound. Hence in a current compilation unit ``Kikoo''", then
  {\tt \#test ()} will be replaced by {\tt Kikoo\#test ()} if the
  function {\tt test} was really found inside this unit. If it was not
  found, then an exception is raised.

\item For ``!-ed'' identifiers, the look-up is performed but no change
  is done. If it is like {\tt !test()}, then it is {\bf not} changed to
  {\tt Self!test} !!! Only a verification is done that the method
  exists in {\tt Self}. If it is like {\tt Coll!test}, then also only
  a verification is done that the method exists in {\tt Coll}.
\end{itemize}

The scoping heavily uses the ``scoping environment'' structure
described in \ref{scoping-environment-presentation}. Scoping is
performed on each phrase of the source text, in the order of
apparition of these phrases. Hence we have to scope phrases among
documentation title, {\tt use} directive, {\tt open} directive,
{\tt coq\_require} directive, species definition, collection
definition, type definition, toplevel value definition, toplevel
theorem definition and toplevel expression. For each scoped phrase,
the {\bf new environment} made of the initially received one extended
by the new scoped definitions is returned. We don't return only a
delta: we return a complete usable new environment.

Most of the scoping functions use a parameter named a
``context''. This structure is intended to group into 1 unique
parameter various values (that would otherwise be as many parameters)
the scoping functions will mostly always use.
{\footnotesize
\begin{lstlisting}[language=MyOCaml]
type scoping_context = {
  (** The name of the currently analysed compilation unit. *)
  current_unit : Types.fname ;
  (** The optional name of the currently analysed species. *)
  current_species : string option ;
  (** The list of "use"-d (or open-ed since "open" implies "use") modules.
      Not file with paths and extension : just module name (ex: "basics"). *)
  used_modules : Types.fname list
} ;;
\end{lstlisting}
}


\subsection{Scoping an {\tt use} directive}
Scoping a {\tt use} directive returns an unchanged environment.
It simply adds the ``used'' module to the list modules allowed to be
used of the current context and return a new context with this
extended list. The point is only to mention for the rest of the
scoping passe that qualified identifiers with this module as
qualification is now allowed.


\subsection{Scoping an {\tt open} directive}
\label{scoping-open-directive}
This directive has no to be really scoped. Instead, it has an impact
on the scoping process and more accurately on the scoping environment.
In load in the environment the scoping information of the identifiers
contained in the opened compilation unit, tagging then as
{\tt BO\_opened} like seen in \ref{tag-BO-opened}. Hence all the
imported identifiers will be known as possible definitions to use as
``origin'' of an identifier occurrence, according to the scoping
rules.


\subsection{Scoping a species definition ({\tt scope\_species\_def})}
Before scoping a species, the first thing is to pass a modified
context in which we record that we are inside this species (field
{\tt current\_species} on the context.

The scoping environment of a species will be gradually extended as
long as we process its components: we must first add the parameters of
collection and entity in their order of apparition then import the
bindings from the inheritance tree, and finally local bindings will be
added while scoping the species' body (fields). When scoping involves
searches in the environment, searches must be done in the following
order:
\begin{enumerate}
\item Try to find the identifier in local environment.
\item Check if it's a parameter of entity (``{\tt in}'') or collection
  (``{\tt is}'').
\item Try to find the identifier throughout the hierarchy.
\item Try to find if the identifier is a global identifier..
\item ``Else'' \ldots not found !
\end{enumerate}
So the order in which the identifiers are inserted into the
environment used to scope the species must respect extensions in the
reverse order in order to find the most recently added bindings first.

Hence we first scope the species parameters. Once they are scoped, we
get an environment where they are bound to their scoping information.

Using this new environment, we now scope the {\tt inherits} clause,
i.e. the inheritance. In addition to the scoped inheritance
expression, this will give us bake the names of the methods we
inherits. This is especially useful because we must add them into our
environment before scoping the methods defined in this
species. Indeed, the bodies of these defined methods may make
reference to identifiers corresponding to inherited methods.

Once these inherited methods are added to the current environment, we
can scope the defined methods in this new environment. Each scoped
method is added to the environment used to scope the remaining
methods.

Once all is scoped, we create the scoping information for the species
itself and add it to our initial environment (not to the one where we
gradually added parameters, methods etc, since they do not need to be
visible outside the species). And finally, in the type bucket of this
environment, we add the type corresponding the the species carrier.

We then return the scoped species definition and the new environment
where the species and it carrier are bound. This new environment is
not a delta, it is a complete ''all-in-one'' environment suitable to
scope the remaining phrases of the source file.

\subsubsection{Scoping the species parameters ({\tt scope\_species\_params\_types})}
The species parameters (i.e. collection and entity parameters) are
scoped in their apparition order, adding the scoping information of
each one to scope the next ones. We have two cases of parameters.
\begin{itemize}
\item Entity parameter {\tt a in C(...)}: we must scope the collection
  expression {\tt C(...)} the parameter {\tt a} is ``{\tt in}''. To
  do so, we use the current scoping environment. Once this collection
  expression is scoped, we must insert in the environment the name of
  the parameter (here, {\tt A}. Since it is a {\bf value} (whose type
  will be the carrier of the collection expression), we add it in the
  values bucket of the environment.
\item Collection parameter {\tt P is C(...)} we must scope the
  collection expression {\tt C(...)} . This gives back the scoping
  information (with methods names list) of this expression. Because
  collections and species are not first-class values, the environment
  extension will not be done in the values bucket. Instead, we must
  extend the environment with a ``locally defined'' collection having
  the same methods than those coming from the collection expression
  {\tt C(...)}. So from the obtained scoping information (with methods
  list) of this expression, we create a fresh species scoping
  information that we bind to the parameter name {\tt P} in the
  environment. Then, because a species induces a type by its carrier,
  we insert in the type bucket of the environment a type constructor
  representing the carrier of the parameter (here a type {\tt P}).
\end{itemize}
The scoping of a parameter returns the extended environment and the
scoped version of the list of parameters.


\subsubsection{Scoping the inheritance clause ({\tt scope\_inheritance})}
Scoping the inheritance is done by scoping each collection expression
in their order of apparition. Like we saw for species parameters
scoping, scoping a collection expression returns the scoped expression
and the list of the methods names this expression has.
Hence, for each parent, we collect the obtained scoped methods. They
will be later inserted in the environment to scope the remaining of
the species. Note that we don't need to add the methods found of a
parent to scope the next parent in the inheritance list. When all
these collected will have to be inserted in the environment, we must
take care to insert first those of the ``left-most'' parent in the
{\tt inherits} clause, going on from left to right. This ensure
searches in the environment will comply the inheritance resolution
order of \focalize. During insertion in the environment, because these
methods are inherited, they now are methods of ourselves, hence
methods of {\tt Self} and they must be inserted as
{\tt SBI\_method\_of\_self} ! The change needed to say that one of
these methods is a method of the ``current collection'''s name,
i.e. toggling the flag to {\tt SBI\_method\_of\_coll} will be done
during accesses in the environment (by {\tt find\_value}).


\subsubsection{Scoping the defined methods ({\tt scope\_species\_fields})}
To scope all the defined methods of the species, we scope each field
in its order of apparition, then add the obtained information about it
in the environment used to scope the remaining methods.

Scoping a method is done by scoping its body (a
{\tt Parsetree.expression}) then returning this scoped method and the
scoping environment extended by a binding between this method and the
computed scoping information. One must note that depending on if the
method is recursive, we pre-insert its name in the scoping environment
before scoping its body or not. If the method is recursive, we
pre-insert, otherwise no.


\subsubsection{Scoping in general, scoping other constructs}
 For each used ``identifier'' occurrences (i.e. structures that
 represent identifiers since we have several kinds in \focalize), we
 look-up in the scoping environment for information about this
 identifier.

Depending on the form of the identifier, we either re-build explicitly
a scoped identifier with explicit qualification (case of
{\tt Parsetree.expr\_ident}) or we simply check that the identifier is
really bound (case of identifiers where qualification is already
built-in in its structure).

Depending on the class of the identifier (value, type, record
field\ldots) we look-up in the related bucket in the scoping
environment to get the required information.

The interesting function in scoping is
{\tt scoped\_expr\_ident\_desc\_from\_value\_binding\_info} that in
fact re-build an {\tt expr\_ident\_desc} from the inner simple name (i.e.
{\tt vname}) found in an {\tt expr\_ident} and the scoping
information bound to this name in the scoping environment.