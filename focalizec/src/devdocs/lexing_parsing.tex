% $Id: lexing_parsing.tex,v 1.2 2009-05-28 08:43:26 pessaux Exp $

\section{Lexing}
\index{lexing}
The lexing process is performed as usual, by a lexer created by
\ocamllex. The description of the lexer is located in
{\tt src/parser/lexer.mll}. The structure of this file is like any
regular lexer for a realistic language.

The keywords of the language is stored in a hash-table to
prevent the automaton from being too big. Each time a stream of
alpha-numeric characters that can be either an identifier or a keyword
is found, we lookup in this hash-table. If the word make of this
stream of characters belongs to the hash-table, then we return the
corresponding token, else we return an identifier based on the form of
the word (capitalised or not).

As any lexer, each time it is called (by the lexer), it returns one
and only one lexem.



\section{Parsing}
\index{parsing}
The lexing process is performed as usual, by a lexer created by
\ocamlyacc. The description of the parser is located in
{\tt src/parser/parser.mly}. The structure of this file is like any
regular parser for a realistic language. The parser doesn't try to
implements recovery on error, hence, it describes the syntax of the
language without any addition (conversely to \ocaml\ or \gcc\'s
parsers).

\index{AST!node structure}
The important rule applied in the parser is that at each AST node it
creates, it record the corresponding location (extend) in the source
file and the corresponding (optional) documentation (i.e. special
comments kept in the AST). This mechanism is performed by a set of
basic functions with names derivated from {\tt mk} ({\tt mk\_loc},
{\tt mk\_doc\_elem}, {\tt mk\_doc}, {\tt mk\_local\_ident},
{\tt mk}, \ldots). This especially
means that one never must create an AST node by hand. Any created AST
node must be created via these basic functions.

The parser, returns a complete AST corresponding to the parse of the
whole submited source file. It currently doesn't have an entry rule
returning only an AST for one definition. This especially means that
the result of the parsing is always an AST node of the form
{\tt Parsetree.file}.
