% $Id: codegen_model.tex,v 1.4 2009-04-01 06:16:29 pessaux Exp $

The code generation model is the now closely the same for both \ocaml\
and \coq\ generated source files. The main difference comes from the
fact that in \ocaml, logical methods are discarded. However, except
for the case of \zenon\ proofs where we introduce \coq\ {\tt Section},
the generation model for ``computation'' and logical methods are
exactly the the same. This especially means that dependencies are
``manually'' abstracted by explicit $\lambda$-lifting instead of using
(like in the previous compiler) the \coq's {\tt Section}
mechanism. In the same order of idea, in order to have a common model,
both \ocaml\ and \coq\ code are based on a record-oriented structure
for the species and collection, with explicit record fields accesses.

\section{Species}

\subsection{Species header}
The generated code for a species is hosted by a module whose name is
the species' name. This way, it is possible to have species having the
same names of methods without conflict. The module hence defines the
name-space of ``things'' contained in and induced by a species.


\subsection{Carrier representation}
In \ocaml, if the structure of the carrier of the species is known
(i.e. if the method {\tt representation} was defined), then we
generate a type definition whose name is {\tt me\_as\_carrier} and
body is the type translation of the corresponding \focalize\ type
expression.

In such a type definition, carriers of species parameters appearing in
the species' carrier are abstracted by type variables. The naming
scheme of these variable is ''{\tt '}'' + the species parameter's name
un-capitalised + ``{\tt \_as\_carrier}'' + an integer stamp that is
unique (inside this type definition). Since species names in
\focalize\ are capitalised and capitalised identifiers in \ocaml\ are
reserved for modules and sum type value constructors, we need to
un-capitalise the name when generating \ocaml\ code. For the stamp, it
is required to prevent several type variable from having the same name
in case the species is parameterised by both a collection parameter
and an entity parameter whose names differ only by the capitalisation
of the first letter.

For instance, the following header of a species definition:
{\scriptsize
\begin{lstlisting}
species Cartesian_product(A is Setoid, B is Setoid) =
  representation = A * B;
  ...
\end{lstlisting}
}
will generate the \ocaml\ type definition:
{\scriptsize
\begin{lstlisting}[language=MyOCaml]
module Cartesian_product =
  struct
  (* Carrier's structure explicitly given by "rep". *)
  type ('a0_as_carrier, 'b1_as_carrier) me_as_carrier =
    'a0_as_carrier * 'b1_as_carrier
  ...
\end{lstlisting}
}

In fact, {\em a posteriori}, I think now that it's useless in case
of a species, even closed. This type definition is only used in the
case of a collection.

\medskip
In \coq, no definition generated, the knowledge of the structure of
the carrier being reflected directly is needed in the methods (see
later).



\subsection{The record type}
The type of data representing a species is a record type. We first
examine its header, i.e. stuff before this type definition's body,
then it's body.


\subsubsection{The record header}
The name of this type is always {\tt me\_as\_carrier}. It can be
parametrised due to various abstraction requirements induced by the
late-binding feature of \focalize. \ocaml\ requires a parameter not
needed for \coq: this is the only fundamental difference. We will see
that abstractions (hence, parameters) required by \coq\ can involve
methods but this is only because some of the dependencies that are
present in the \coq code are always trivially absent in the
\ocaml\ code.


\begin{enumerate}
\item First come all the species parameter carriers appearing in the
  {\bf types} of the methods of the species. Each carrier will be
  abstracted by one type variable. This allows to ``late-bind'' the
  {\tt representation} of the species parameters. We say here 
  ``appearing in the {\bf types}\ldots'': be aware that the type of
  ``computational'' methods are ML-like types and the ones of logical
  methods are their {\bf statement} ! We do not explain here how these
  parameters are found: this is the role of dependency computation on
  species parameters.

  The naming scheme of these type variable in \ocaml\ is the same than
  described above for the carrier representation.

  In \coq\, these parameters of the type are not ``really type
  variables'' but arguments of type {\tt Set} (simply a technical
  question). Their naming scheme is the species parameter's name +
  ``{\tt \_T}''.

\item Next, {\bf only in the \ocaml\ code}, the record type is always
  parametrised by a type variable representing the carrier of
  (i.e. the internal representation of the type encapsulated in this)
  species. By convention, this type variable is always named
  {\tt 'me\_as\_carrier} (don't confuse with {\tt me\_as\_carrier}
  that is the name of the type definition representing the effective
  structure of the carrier when it is known).

  In \coq\ we don't have this mandatory type parameter, but instead of
  it, we will have one extra field in the body of the record. This
  variable enable to ``late-bind'' the {\tt representation} of the
  species.

\item Finally come all the methods of the species parameter appearing
  in the {\bf types} of the methods of the species. Since in \ocaml\
  the type of a method can only involve type constructors (a ML-like
  type), it is clear that we won't have any such parameters. In
  effect, in \ocaml\ the logical methods are discarded. However,
  since for the type of a logical method is its logical statement, to
  \coq's side, we can have any expression inside their type.. In
  particular, we can have calls to some collection parameters'
  methods. Having these parameters in the record type allows the
  ``late-binding'' on the collection parameter itself (i.e. on by
  which effective collection will be used to instantiate the
  parameter).
\end{enumerate}

Let's now change our previous example to illustrate the header of our
record type in \ocaml\ and \coq:
{\scriptsize
\begin{lstlisting}
species Cartesian_product(A is Setoid, B is Setoid) =
  representation = A * B;
  let make (x in A, y in B) = (x, y) ;
  let equal (x in Self) = ... A!equal (...) && B!equal (...) ;
  theorem thm : all x in A, A!tst (x) -> ...
    proof = ... by property A!commutes ... ;
  ...
\end{lstlisting}
}


\subsubsection{Fields and their types}
\subsection{Methods}
\subsubsection{Inherited methods}
\subsubsection{Defined methods}
\subsection{Fully defined species}
\subsection{Collection generator function's header}
\subsubsection{Local functions}
\subsubsection{Creating the record value}
\section{Collection}
\section{Toplevel functions}
\section{Toplevel theorems}
\section{Type definitions}
\section{External definitions}
