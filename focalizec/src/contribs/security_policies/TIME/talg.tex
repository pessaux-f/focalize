\documentclass[a4paper,10pt]{article}

\usepackage{amssymb,amsmath,stmaryrd,mathrsfs}

\usepackage{tikz}
\usetikzlibrary{arrows,automata,positioning}
\usetikzlibrary{automata,patterns,topaths,shapes,calc}
\tikzstyle{every picture}+=[>=stealth',initial text=]

\tikzstyle{ent}=[circle,draw,thick,inner sep=0pt,minimum size=2.5mm]

\newtheorem{definition}{Definition}
\newtheorem{proposition}{Proposition}
\newtheorem{example}{Example}
\newtheorem{remark}{Remark}
\newtheorem{lemma}{Lemma}
\newtheorem{theorem}{Theorem}

\def\A{\ensuremath{\mathcal{A}}}
\def\C{\ensuremath{\mathcal{C}}}
\def\T{\ensuremath{\mathcal{T}}}
\def\E{\ensuremath{\mathcal{E}}}
\def\bbbr{{\rm I\!R }} 

\newenvironment{proof}{\noindent {\it Proof. }}
{\hspace{\fill} $\blacktriangleleft$\vspace{.2cm}}


\title{Timed-automata \& Landing gear system ?}

\begin{document}

\maketitle

\section{Timed transition systems and Timed automata}


\begin{definition}[Timed transition system (TTS)]
{\rm
A timed transition system over an alphabet $\Sigma$ (of actions) is a
transition system $\T=(S,s_0,E)$ over the set of labels $L = \Sigma
\cup \{ \varepsilon \} \cup \bbbr_{+}$ where:
\begin{itemize}
\item $S$ is a set of configurations,
\item $s_0 \in S$ is the initial configuration,
\item $E \subseteq S \times L \times S$,
\end{itemize}
such that:
\begin{itemize}
\item (Zero delay) $\forall s_1,s_2 \in S \, \, s_1 \xrightarrow{0}_{E} s_2
  \Leftrightarrow s_1=s_2$
\item (Additivity) 

$\forall s_1, s_2 ,s_3 \in S \, \, \forall d,d' \in
  \bbbr_{+} \, \, (s_1 \xrightarrow{d}_E s_2 \land s_2 \xrightarrow{d'}_E
  s_3) \Rightarrow s_1 \xrightarrow{d+d'}_E s_3$
\end{itemize}
}
\end{definition}
A \emph{run} of a TTS $\T=(S,s_0,E)$ over $\Sigma$ is a sequence:
\[
s_0 \xrightarrow{d_1}_E s'_0 \xrightarrow{a_1}_E s_1
\xrightarrow{d_2}_E s'_1 \xrightarrow{a_2}_E s_2 \xrightarrow{d_3}_E \cdots
\]
starting from the initial configuration and where durations and
actions striclty alternate.

We consider a finite set $X$  of real valued variables called
clocks. These clocks evolve synchronously with time and can be reset
or compared with constant values.
For a set $X$ of clocks, $\C(X)$ denotes the set of \emph{clocks constraints}
which are conjunctions of atomic constraints of the form $x \bowtie c$
where $x \in X$, $c$ is a constant, and $\bowtie \in \{ <,\le,= , \geq,>
\}$.

We also consider a finite set $Y$ of variables over finite domains: we
write $D_y$ the finite domain of the variable $y \in Y$. 
For a set $Y$ of variables,
$\C(Y)$ denotes the set of \emph{variable constraints}
which are 
obtained by combining atomic constraints of the form $y_i = k_i$ (where
$k_i \in D_{y_i}$) with
$\lor$ and $\land$.
An update $u \in Up(Y)$ is a set of assignments of the form $y_i :=
k_i$ such that for $i \neq j$, $y_i \neq y_j$.


\begin{definition}[Timed automaton (TA)]
{\rm
A timed automaton over an alphabet $\Sigma$ (of actions) is a tuple $\A=(X,Y,Q,q_0,\Delta,I)$ where:
\begin{itemize}
\item $X$ is a finite set of clocks,
%($X = \bigcup_{i=1}^{|X|} \{ x_i \in \bbbr_+ \}$),
\item $Y$ is a finite set of variables (with finite range),
\item $Q$ is a finite set of states,
\item $q_0 \in Q$ is the initial state,
\item $\Delta \subseteq Q \times \C(X) \times \C(Y) \times (\Sigma
  \cup \{ \varepsilon \}) \times \wp(X) \times Up(Y) \times Q$
\item $I : Q \to \C(X)$ (clocks invariants over states).
%\item $M : Q \to \prod_{i=1}^{|V|} D_i$ where $D_i$ is the finite
%  domain of $v_i \in V$.
\end{itemize}
}
\end{definition}

Given a set $X$ of clocks, a \emph{clock valuation} is a mapping $v : X \to
\bbbr_+$, with $\mathbf{0}$ the null valuation assigning zero to all
clocks in $X$. 
A valuation can be viewed as a tuple $(v(x))_{x \in X}$ defining a
point in $\bbbr_{+}^{|X|}$
(we write $\bbbr_{+}^{|X|}$ the set of valuations over $X$).
For $d \in \bbbr_+$, we write $v+d$ the valuation such
that for all $x \in X$, $(v+d)(x) = v(x) + d$. For $r \subseteq X$, we
write $v[r \mapsto 0]$ the valuation such
that for all $x \in X$, $v[r \mapsto 0](x) =0$ if $x \in r$ and $v(x)$
otherwise. Clocks constraints are interpreted on clock valuations: a
valuation $v$ satisfies the atomic constraint $x \bowtie c$, denoted by
$v \models x \bowtie c$, iff $v(x) \bowtie c$ is true. The notation is
extended to general clock constraints. 

Similarly,
given a finite set $Y$ of variables over finite domains,
a \emph{variable valuation} is a mapping $w : Y \to
\cup_{y \in Y} D_y$ (such that for all $y \in Y$, $w(y) \in D_y$).
We write $Y^D$ the set of valuations over $Y$.
Given a variable valuation $w$ and an update $u$, we write $w[u]$ the
valuation such that $w[u](y) = k$ if $y := k$ belongs to $u$, and
$w(y)$ otherwise.
Variables constraints are interpreted on variables valuations: a
valuation $w$ satisfies the atomic constraint $y=k$, denoted by
$w \models y=k$, iff $w(y) = k$ is true. The notation is
extended to general variable constraints. 



The \emph{semantics} of a timed automaton $\A=(X,Y,Q,q_0,\Delta,I)$
over an alphabet $\Sigma$ is defined from a initial variable valuation $w_0$
by a timed transition system
$\T_{\A}[w_0] = (S,s_0,E)$ over $\Sigma$ where:
\begin{itemize}
\item $S = \left \{ (q,v,w) \in Q \times \bbbr_{+}^{|X|}  \times
    Y^D \mid v \models I(q) \right \}$
\item $s_0 = (q_0,\mathbf{0},w_0)$
\item $E = \begin{array}{ll}
& \left \{ 
(q,v,w) \xrightarrow{d}_E (q,v+d,w) \mid d \in \bbbr_+ 
%\land v + d \models I(q) 
\right \} \\
\cup & 
\left \{ \begin{array}{l}
(q_1,v_1,w_1) \xrightarrow{a}_E (q_2,v_2,w_2) \mid  
\exists q_1 \xrightarrow{g_c,g_v,a,r,u}_{\Delta} q_2 \\
v_1 \models g_c \, \land w_1 \models g_v \, \land v_2 = v_1[r \mapsto
0] \land w_2 = w_1[u]
\end{array}
\right \}
\\
\end{array}$
\end{itemize}

\section{Time-out mechanism}




\section{Landing gear system}

\subsection{Without time-out features}

questions :

l'id\'ee c'est de raconter la spec. du landing gear system avec des
automates temporis\'es ....

Les variables servent \`a mod\'eliser les capteurs de l'avion et les
actions du pilote ... le changement de valeurs des variables est aussi
mod\'elis\'e par des updates sur les transitions. Pour le
moment seules les parties noires, vertes et bleues de l'automate sont
mod\'elisables avec les d\'efinitions ci-dessus .... pour les
transitions avec un ``temps'' , par exemple $e_1 \xrightarrow{200ms}
e_2$ dans l'automate de Fran\c{c}ois (qui mod\'elise le fait que
l'on reste au moins 200ms dans l'\'etat $e_1$ avant de passer \`a
$e_2$), on peut utiliser une horloge $x_t$ et repr\'esenter cette
transition par
$e_1 \xrightarrow{x_t \geq 200 , \top , \varepsilon , \{ x_t \},\emptyset} e_2$.
d'un automate temporis\'e.




\bibliographystyle{plain}
\bibliography{fbibli}



\end{document}