This manual describes the main constructions of the \focal\ language,
as well as the main tools given by the \focal\ environment. Basic
expressions of the \focal\ language are quite similar to the core
expressions of \ocaml\
(\verb+http://caml.inria.fr+). A prior knowledge of some functional
languages might be helpful to better understand the \focal's core
language concepts. Last a good knowledge of \coq\ is needed if you
intend to manually write proofs in \focal{}. However, this task may be
helped by the \zenon\ automated theorem prover, as long as the user
write his proofs using the \focal\ Proof Language embedded in the
\focal\ language.



%%%%%%%%%%%%%%%%%%%%%%%%%%%%%%%%%%%%%%%%%%%%%%%%%%%%%%%%%%%%%%%%%%%%%%%
%%%%%%%%%%%%%%%%%%%%%%%%%%%%%%%%%%%%%%%%%%%%%%%%%%%%%%%%%%%%%%%%%%%%%%%
%%%%%%%%%%%%%%%%%%%%%%%%%%%%%%%%%%%%%%%%%%%%%%%%%%%%%%%%%%%%%%%%%%%%%%%
\section{Required software}
To be able to develop with the \focal\ environment, a few third party
tools are required. All of them can be freely downloaded from their
related website.
\begin{itemize}
  \item The Objective Caml compiler (version $\geq$ 3.09pl3). \\
    Available
    at \verb+http://caml.inria.fr+. This will be used to compile both
    the \focal\ system at installation stage from the tarball and
    the \focal\ compiler's output generated by the compilation of
    your \focal\ programs.

  \item The Coq Proof Assistant (version $\geq$ 8.1pl3). \\
     Available at
    \verb+http://coq.inria.fr+. This will be used to compile both
    the \focal\ libraries at installation stage from the tarball and
    the \focal\ compiler's output generated by the compilation of
    your \focal\ programs.
\end{itemize}



%%%%%%%%%%%%%%%%%%%%%%%%%%%%%%%%%%%%%%%%%%%%%%%%%%%%%%%%%%%%%%%%%%%%%%%
%%%%%%%%%%%%%%%%%%%%%%%%%%%%%%%%%%%%%%%%%%%%%%%%%%%%%%%%%%%%%%%%%%%%%%%
%%%%%%%%%%%%%%%%%%%%%%%%%%%%%%%%%%%%%%%%%%%%%%%%%%%%%%%%%%%%%%%%%%%%%%%
\section{Optional software}
The \focal\ compiler can generate dependencies graphs from compiled
source code. It generates them in the format suitable to be processed
and displayed by the \dotty\ tools suit of the ``Graphwiz'' package. If
you plan to examine these graphs, you also need to install this
software.



%%%%%%%%%%%%%%%%%%%%%%%%%%%%%%%%%%%%%%%%%%%%%%%%%%%%%%%%%%%%%%%%%%%%%%%
%%%%%%%%%%%%%%%%%%%%%%%%%%%%%%%%%%%%%%%%%%%%%%%%%%%%%%%%%%%%%%%%%%%%%%%
%%%%%%%%%%%%%%%%%%%%%%%%%%%%%%%%%%%%%%%%%%%%%%%%%%%%%%%%%%%%%%%%%%%%%%%
\section{Installation}
\label{installation}
\index{installation}
\focal\ is currently distributed as a tarball contaning the whole
source code of the development environment. You must first deflate the
archive (a directory will be created) by:
\begin{center}
{\tt tar xvzf focalize-x.x.x.tgz}
\end{center}
Next, go in the sources directory:
\begin{center}
{\tt cd focalize-x.x.x/}
\end{center}
You now must configure the build process by:
\begin{center}
{\tt ./configure}
\end{center}
The configuration script then asks for directories where to install
the \focal\ components. You may just press enter to keep the default
installation directories.
\begin{verbatim}
diddle:~/devel/ssurf/focal/focalize$ ./configure 
Where to install FoCaL binaries ?
Default is /usr/local/bin.
Just press enter to use default location.

Where to install FoCaL libraries ?
Default is /usr/local/lib/focalize.
Just press enter to use default location.
\end{verbatim}
After the configuration ends, just build the system:
\begin{center}
{\tt make all}
\end{center}
And finally, be root to install the built \focal\ system:
\begin{center}
{\tt su}\\
{\tt make install}
\end{center}
