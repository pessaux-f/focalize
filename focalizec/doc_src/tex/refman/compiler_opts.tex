\index{compiler option}
When invoking the \focal\ compiler with the {\tt focalizec} command,
various command line options can be provided. If no option is
specified, the default behaviour is to compile the source file
provided as argument and generate both the \ocaml\ and the
\coq\ codes. The result files names wear the same prefix than the
\focal\ source file and are suffixed by ``.ml'' for the generated
\ocaml\ code and ``.zv'' for the generated pre-\coq\ code.
To generate the final \coq\ code, the ``.zv'' file must be send to
\zenon\ via the {\tt zvtov} command.

\begin{itemize}
  \item[*] {\bf $--$dot-non-rec-dependencies} {\em directory name}.
    Dumps non-let-rec dependencies of the species present in the
    compiled source file. The output format is suitable to be
    graphically displayed by \dotty\ (free software available via the
    \graphviz\ package). Each species will lead to a \dotty\ file into
    the argument directory. Files are names by ``deps\_'' + the source
    file base name (i.e. without path and suffix) + the species name +
    the suffix ``.dot''.

  \item[*] {\bf $-$focal-doc} Generates documentation..... TODO.

  \item[*] {\bf $-$i}. Prints the interfaces of the species present in
    the compiled source file. Result is sent to the standard output.

  \item[*] {\bf $-$I} {\em directory name}. Adds the specified
    directory to the path list where to search for compiled
    interfaces. Several $-$I options can be used. The search order is
    in the standard library directory first (unless the
    $-$no-stdlib-path the option is used, see below), then in the
    directories specified by the $-$I options in their apparition
    order on the command line.

 \item[*] {\bf $--$methods-history-to-text} {\em directories
   name}. Dumps the methods' inheritance history of the species
   present in the compiled. The result is sent as plain text files
   into the argument directory. For each method of each species a file
   is generated wearing the name made of ``history\_'' + the source
   file base name (i.e. without path and suffix) + ``\_'' + the
   hosting species name + the suffix ``.txt''.

  \item[*] {\bf $--$no-ansi-escape}. Disables ANSI escape sequences in
    the error messages. By default, when an error is reported, bold,
    italic, underline fonts are used to make easier reading the
    message. Using this option removes all these text attributes and
    may be used if your terminal doesn't support ANSI escape sequences
    or, for example, if compiling under \emacs.

  \item[*] {\bf $--$no-coq-code}. Disables the \coq\ code
    generation. By default \coq\ code is always generated.

  \item[*] {\bf $--$no-ocaml-code}. Disables the \ocaml\ code
    generation. By default \ocaml\ code is always generated.

  \item[*] {\bf $-$no-stdlib-path}. Does not include the standard
    library installation directory in the libraries seacrch path. This
    option is rarely useful and mostly dedicated to the
    \focal\ compiler build process.

  \item[*] {\bf $--$pretty} {\em file name}. (Undocumented: mostly for
    debug purpose). Pretty-prints the parse tree of the \focal\ file
    as a \focal\ source into the argument file.

  \item[*] {\bf $--$raw-ast-dump}. (Undocumented: mostly for debug
    purpose). Prints on stderr the raw AST structure after parsing
    stage.

 \item[*] {\bf $--$scoped\_pretty} {\em file name}. (Undocumented:
   mostly for debug purpose). Pretty-prints the parse tree of the
   \focal\ file once scoped as a \focal\ source into the argument
   file.

  \item[*] {\bf $--$verbose}. Sets the compiler in verbose mode. It
    will then generate the trace of the steps and operations is does
    during, the compilation. This feature is mostly used for debugging
    purpose but can also explain the elaboration of the model during
    compilation for people interested in \focal's compilation
    process.

  \item[*] {\bf $-$v}. Prints the focalize version then exits.

  \item[*] {\bf $--$version}. Prints the full \focal\ version,
    sub-version and release date, then exits.

  \item[*] {\bf $--$where}. Prints the binaries and libraries
    installation directories then exits.

  \item[*] {\bf $-$help} {\bf $--$help}. Prints the summary of command
    line options (i.e. this documentation) on the standard output.
\end{itemize}
