% $Id: constructs_syntax.tex,v 1.28 2008-12-15 22:50:38 hardin Exp $


\section{Language constructs and syntax}
%%%%%%%%%%%%%%%%%%%%%%%%%%%%%%%%%%%%%%%%%%%%%%%%%%%%%%%%%%%%%%%%%%%%%%%
%%%%%%%%%%%%%%%%%%%%%%%%%%%%%%%%%%%%%%%%%%%%%%%%%%%%%%%%%%%%%%%%%%%%%%%
\subsection{Types}
Before dealing with expressions and in general, constructs that allow to
compute, let us first examine data-type definitions since, to emit its
result, an algorithm must manipulate data that are more or less specific to
the algorithm. Hence we must know about type definitions to define data that
have a convenient shape and carry the necessary information to model the
problem at hand.

Type definitions allow to build new types or more complex types by
combining previously existing types. They always appear as
toplevel-definitions, in other words, outside species and
collections. Hence a type definition is visible in the whole
compilation unit (and also in other units by using the {\tt open}
directive or by qualifying the type name as described in section
\ref{qualified-name}).

%%%%%%%%%%%%%%%%%%%%%%%%%%%%%%%%%%%%%
\subsubsection{Type constructors}
A {\bf type constructor} is, roughly speaking, a type name. 

\focal\ provides the basic built-in
types (constructors):
\begin{itemize}
  \item {\tt int} for signed machine integers,
  \item {\tt bool} for boolean values ({\tt true} and {\tt false} that
    are hardwired in the syntax or {\tt True} and {\tt False} that are
    defined in ``basics.fcl''),
  \item {\tt float} for floating point numbers,
  \item {\tt unit} for the trivial type whose only value is {\tt ()},
  \item {\tt char} for characters literals,
  \item {\tt string} for strings literals.
\end{itemize}

 New   type constructors are introduced by  {\bf type
  definitions}. 
Types constructors can be parametrised by  {\bf type expressions} separated
by commas and between parentheses.

%%%%%%%%%%%%%%%%%%%%%%%%%%%%%%%%%%%%%%%%%%%%%%%%%%%%%%%%%%%%%%%%%%%%%%%
\subsubsection{Type expressions}
\index{type!expression}
\index{expression!type}
Type definitions require {\bf type expressions} to build more complex
data-types.
\vspace{0.2cm}
\begin{syntax}
\syntaxclass{Type expressions:}
\tau & ::= & lowercase\ ident  & Type constructor \\
     & \mid & uppercase\ ident & Species representation \\
     & \mid & \terminal{'}lowercase\ ident & Type variable \\
     & \mid & uppercase\ ident
              \terminal{(}\tau \{\terminal{,} \tau\}+\terminal{)} &
               Parameterised type constructor \\
     & \mid & \tau \terminal{\rightarrow} \tau & Functional type \\
     & \mid & \terminal{(}\tau \terminal{*} \tau
              \{\terminal{*}\ \tau\}+\terminal{)} & Tuple type \\
     & \mid & \terminal {Self} & Current species representation \\
     & \mid & \terminal{(}\tau\terminal{)} & Parenthesised type expression
\end{syntax}
\vspace{0.2cm}

A type expression can be a type constructor. 

A type expression can denote the representation of a species or a
collection by using their name, thus a capitalized name. The special
case of {\tt Self} denotes the representation of the current
species. Hence, obviously {\tt Self} is only bound in the scope of a
species.

Type expressions representing function types are written using the
arrow notation ({\tt ->} in which the type of the argument of the
function is the left type expression and its return type is the right
one. As usual in functional languages, a function with several (say
$n$) arguments is considered as a function with {\bf 1} argument
returning a function with $n-1$ arguments. Hence,
{\tt int -> int -> bool} is the type of a function taking 2 integers
and returning a boolean.

\focal\ provides native tuples (generalisation of pairs). The type of
a tuple is the type of each of its components separated by a *
character and surrounded by parentheses. Hence,
{\tt (int * bool * string) } is the type of triplets whose first
component is an integer, second component is a boolean and third
component is a string.


Finally, type expressions can be written between parentheses without
changing their semantics.




%%%%%%%%%%%%%%%%%%%%%%%%%%%%%%%%%%%%%%%%%%%%%%%%%%%%%%%%%%%%%%%%%%%%%%%
\subsubsection{Type definitions}
\index{type!definition}
\label{type-definition}
A type {\bf definition} introduces  a new type
 constructor (the name of the type), which becomes available to build 
new type expressions. Hence, defining a type is the
way to give a name to a new type structure. \focal\ proposes 3 kinds
of type definitions: aliases, sum types and record types.



\vspace{0.5cm}\noindent {\bf Aliases}

\index{type!definition!alias}

\bigskip

Aliases provide a way to create type abbreviations. It is not handy to
manipulate large {\bf type expressions} like for instance, a tuple of
5 components: {\tt (int * int * int * int * int)}. Moreover, several kind of information can be represented by such a
tuple. For instance, x, y, z 3D-coordinates and temperature and
pressure. For another example, year, month, day, hours, minutes. In
these two cases, the manipulated type expression is the same and
the two uses cannot be easily differentiated. Type aliases allows to give a name to a
(complex) type expression, for sake of readability or to shorten the
code. Example:
{\scriptsize
\begin{lstlisting}
type experiment_conditions = alias (int * int * int * int * int) ;;
type date = alias (int * int * int * int * int) ;;
\end{lstlisting}
}

\vspace{0.2cm}
\begin{syntax}
\syntaxclass{Alias type definitions:}
alias\_type\_def & ::= & \terminal{type}\ ident\ \terminal{=}
   \ \terminal{alias}\ \tau
\end{syntax}
\vspace{0.2cm}

In the remaining of the development, the type names
{\tt  experiment\_conditions} and {\tt date} will be known to be
tuples of 5 integers and will be compatible with any other type being
also a tuple of 5 integers. This especially means that a {\em type alias
does not create a really ``new'' type, it only gives a name to a type
expression and this name is type-compatible with any occurrence of the
type expression it is bound to}. Obviously, it is possible to use
aliases with and in any type expression or type definition.

%  For
% instance:
% Je ne vois pas ce que cet exemple illustre. Th. 
% {\scriptsize
% \begin{lstlisting}
% type t1 =
%  | A
%  | B
% ;;

% type t2 = (t1 * bool) ;

% type t3 ('a) =
%  | C (t2)
%  | D ('a)
% ;;

% type t4 = t3 (int * bool) ;;
% \end{lstlisting}
% }




\vspace{0.5cm}\noindent{\bf Sum types}
\index{type!definition!sum}


Sum types provide the way to create new {\bf values} that
 belong to the same {\bf type}. Like {\tt 1} or {\tt 42} are
{\bf values} of {\bf type} {\tt int}, one may want to have {\tt Red},
{\tt Blue} and {\tt Green} as the {\bf only} values of a new type
called  {\tt color}. The {\bf only}  means that the created type {\tt color}
is inhabited only by these 3 values. To define
such a type, we itemize its value names (that are always capitalized
identifiers) by  preceeding them by a ``$\mid$'' character :

{\scriptsize
\begin{lstlisting}
type color =
  | Red
  | Blue
  | Green
;;
\end{lstlisting}
}
Note that the first ``$\mid$'' character is required: it is not a
separator. This especially means that when writing a sum type
definition on a single line, the first ``$\mid$'' must be written:
{\scriptsize
\begin{lstlisting}
type color = | Red | Blue | Green ;;
\end{lstlisting}
}

{\bf Values} of a sum type  are built from the
{\bf value constructors}, i.e. from the names enumerated in the
definition (that must not be confused with the {\bf type constructor}
which is the name of the type. For, instance,
{\tt Red} is a {\bf value} of the type constructor {\tt color}.


 Value constructors of sum
types can be {\bf parametrised} by a type expression, corresponding
values being obtained by applying the value constructor to a value of
the parameter type. For
instance, let's define the type of playing cards as king, queen, jack
and simply numbered cards:

{\scriptsize
\begin{lstlisting}
type card =
  | King
  | Queen
  | Jack
  | Numbered (int)
;;
\end{lstlisting}
} Hence, the {\tt Numbered} constructor ``carries'' the integer value
written on the card. Some values of type {\tt card} are: {\tt King},
{\tt (Numbered 4)}, {\tt (Numbered 42)}.


Any type
expression, even recursive, can be used as a  parameter of value
contructors\index{type!recursive}. 
For instance, the type of lists of boolean $\times$ integer pairs
could be defined like:
{\scriptsize
\begin{lstlisting}
type b_i_list =
  | Empty
  | Cons ((bool * int) * b_i_list)
;;
\end{lstlisting}
} From this type definition, a value of type {\tt b\_i\_list} is
either empty (constructor {\tt Empty}) or has a head (the first
component of the {\tt Cons} constructor) and a trailing list (the
second component of this constructor): {\tt Cons ((false, 2), (Cons
  ((true, 1), Empty)))}. The length of this list is 2 and its
elements are {\tt (false, 2)} followed by {\tt (true, 1)}.

\bigskip



\vspace{0.2cm}
\begin{syntax}
\syntaxclass{Sum types definitions:}
opt\_params & ::= & \epsilon
    \mid \terminal{(}\ \terminal{'}ident \{\terminal{,} \terminal{'}ident \}*
    \ \terminal{)} & \\
opt\_args & ::= & \epsilon
    \mid \terminal{(}\ \tau\ \{\terminal{,} \tau \}*\ \terminal{)} & \\
constructor & ::= & \terminal{\mid}\ uident\ opt\_args & \\
sum\_type\_def & ::= &
    \terminal{type}\ ident\ opt\_params\  \terminal{=} \  constructor+
\end{syntax}
\vspace{0.2cm}


\vspace{0.5cm}\noindent {\bf Record types}
\label{record-type-definition}
\index{type!definition!record}

\bigskip

Record types provide a way to aggregate data of various types, a bit like
tuples, but naming the components of the group, instead of
differentiating them by their position like in tuples. A record is
a sequence of names and types between braces. For example:
{\scriptsize
\begin{lstlisting}
type experiment_conditions = {
  x : int ;
  y : int ;
  z : int ;
  temperature : int ;
  pressure : int
} ;;

type identity = {
  name : string ;
  birth : int ;
  living : bool
} ;;
\end{lstlisting}
}

\vspace{0.2cm}
\begin{syntax}
\syntaxclass{Record types definitions:}
field & ::= & ident\ \terminal{:}\ \tau\ \terminal{;} & \\
opt\_params & ::= & \epsilon
    \mid \terminal{(}\ \terminal{'}ident \{\terminal{,} \terminal{'}ident \}*
    \ \terminal{)} & \\
record\_type\_def & ::= &
    \terminal{type}\ ident\ opt\_params\  \terminal{=}
    \ \terminal{\{} field+ \terminal{\}}
\end{syntax}
\vspace{0.2cm}

To create a {\bf value} of a record type, a value of the related
type must be provided for each field of the record.
{\scriptsize
\begin{lstlisting}
{ name = "Benjamin" ; birth = 2003 ; living = true }
\end{lstlisting}
}
Like in tuples, records can mix types of fields.

\vspace{0.5cm}\noindent{\bf Parameterised type definitions}

It is possible, {\em at toplevel}, to {\bf parametrise a type
  definition}, i.e. to create a type with a {\bf type variable} that
can be instantiated by any type expression. A type variable is written
as an identifier preceded by a {\tt '} (quote) character.

 For instance, the type
definition of generic (polymorphic) lists may be defined by:

{\scriptsize
\begin{lstlisting}
type list ('a) =
  | Empty
  | Cons ('a * list ('a))
;;
\end{lstlisting}
}

The value
constructor {\tt Cons} carries a value of type ``unknown'' (of type
``variable'') and the tail of the list, i.e. a value of type
{\tt list} with its parameter instantiated by the same type
variable. This explicitly says that all the elements of such a list
have the same type. It is now possible to use the {\tt list} type in
type {\bf expressions} by providing a type {\bf expression} as
argument of the {\bf type constructor} {\tt list}. For instance,
{\tt list (int)} is the type of lists containing integers,
{\tt list (list (char))} is the type of lists containing lists of
characters.

Parametrised record types can also be introduced, as in the following example:
{\scriptsize
\begin{lstlisting}
type pair ('a, 'b) = {
  first : 'a ;
  second : 'b
} ;;

type int_bool_pair = pair (int, bool) ;;
\end{lstlisting}
}



%%%%%%%%%%%%%%%%%%%%%%%%%%%%%%%%%%%%%%%%%%%%%%%%%%%%%%%%%%%%%%%%%%%%%%%
%%%%%%%%%%%%%%%%%%%%%%%%%%%%%%%%%%%%%%%%%%%%%%%%%%%%%%%%%%%%%%%%%%%%%%%
\subsection{Representations}
As further explained (see section \ref{rep-is-method}) the
representation is a method of a species that describes the internal
data structure that the species manages. Hence, it is a kind of {\bf
  type definition}, more accurately an {\bf alias type
  definition}. This means that a representation does not introduce a
new type, it only ``assigns'' to the representation a {\bf type
  expression} defining the type of the manipulated entities of the
species. Moreover, like for any other methods (c.f. section
\ref{no-polymorphism-for-methods}), {\em the representation must not
  be a polymorphic type}. Thus its definition cannot
contain type variables (but may contain collection parameter
names). Defining a species'representation is simply done by adding the
{\tt representation} method:
 {\scriptsize
\begin{lstlisting}
open "basics" ;;

species IntPair =
  representation = (int * int) ;
end ;;
\end{lstlisting}
}

Recall that the type introduced by the method {\tt representation} is
denoted by {\tt Self} within the species.

\vspace{0.2cm}
\begin{syntax}
\syntaxclass{Representation:}
representation & ::= &
    \terminal{representation} \terminal{=}
    {type expression}
\end{syntax}
\vspace{0.2cm}


%%%%%%%%%%%%%%%%%%%%%%%%%%%%%%%%%%%%%%%%%%%%%%%%%%%%%%%%%%%%%%%%%%%%%%%
%%%%%%%%%%%%%%%%%%%%%%%%%%%%%%%%%%%%%%%%%%%%%%%%%%%%%%%%%%%%%%%%%%%%%%%
\subsection{Expressions}
\label{expressions}
\index{expression} Expressions are constructs of the language that are
evaluated into a {\bf value} of a certain {\bf type}. Hence values and
types are not at the same level. Types serve to classify values into
categories.  Although proofs may contain expressions, we describe them
in the \ref{making-proofs}. Indeed proofs are not expressions, they do
not lead to \focal\ values thus live at another level.

\vspace{0.2cm}
\begin{syntax}
\syntaxclass{Expressions:}
exp & ::= & integer\_literal & \\
    & \mid & string\_literal & \\
    & \mid & character\_literal & \\
    & \mid & float\_literal & \\
    & \mid & {\tt true}\ \mid\ {\tt false} & Boolean constant \\
    & \mid & [[ident] \terminal{\#}]uident & sum type value constructor
                                       or species/collection identifier \\
    & \mid & [\terminal{Self}] \terminal{!} uident &
                       Method of the current species \\
    & \mid & [[ident] \terminal{\#}] [uident \terminal{!}] lident &
               Method from specified species/collection \\
    & \mid & [[ident] \terminal{\#}] [uident \terminal{!}]
               \terminal{(} operator \terminal{)} & Infix or prefix
            operator used in functional position \\
    & \mid & \terminal{let}\ [\terminal{rec}]\ let\_binding
               \ \{\terminal{and}\ let\_binding \}* &  Let bound definition \\
    &      & \ \ \ \terminal{in}\ exp & \\
    & \mid & \terminal{if}\ exp\ \terminal{then}\ exp
                 \ \terminal{else}\ exp & Conditional \\
    & \mid & \terminal{match}\ exp\ \terminal{with}\ match\_binding+ &
                     Pattern matching \\
    & \mid & exp\ \terminal{(} [exp*] \terminal{)} & Function application \\
    & \mid & unary\_operator\ exp & Application of unary operator \\
    & \mid & exp\ binary\_operator\ exp & Application of binary operator \\
    & \mid & \terminal{\{}\ record\_field\_value & Record value \\
    &      &     \hspace{0.6cm}  \{\ \terminal{;}\ record\_field\_value\ \}+
                \ \terminal{\}} & \\
    & \mid & \terminal{\{}\ exp\ \terminal{with}\ record\_field\_value &
                  Record value clone \\
    &      &      \hspace{1.0cm}  \{\ \terminal{;}\ record\_field\_value\ \}+
                  \ \terminal{\}} & \\
    & \mid & expr\ \terminal{.}[ident\terminal{\#}]lident &
                 Record field access \\
    & \mid & \terminal{(} exp \terminal{)} & Parenthesised expression
\end{syntax}

\begin{syntax}
\syntaxclass{Record field value:}
record\_field\_value & ::= &
     [[ident] \terminal{\#}] lident\ \terminal{=}\ exp &
\end{syntax}

\begin{syntax}
\syntaxclass{Let bindings:}
let\_binding & ::= & lident\ [\terminal{in}\ type\_expression]
           \ \terminal{=}\ exp & Definition without parameter \\
   & \mid & lident\
                \terminal{(}lident\ [\terminal{in}\ type\_expression] &
                 Definition with parameter(s) \\
   &     &      \hspace{1.3cm} \{\ \terminal{,}\ lident
                  \ [\terminal{in}\ type\_expression]\}* \terminal{)} & \\
   &     &       \ \ \ \ [\terminal{in}\ type\_expression]\ \terminal{=}\ exp &
\end{syntax}

\begin{syntax}
\syntaxclass{Match bindings:}
 match\_binding & ::= &
   \terminal{\mid}\ pattern\ \terminal{-}\/\terminal{>}\ exp &
\end{syntax}

\begin{syntax}
\syntaxclass{Patterns:}
pattern & ::= & integer\_literal & \\
    & \mid & string\_literal & \\
    & \mid & character\_literal & \\
    & \mid & float\_literal & \\
    & \mid & {\tt true}\ \mid\ {\tt false} & Boolean constant \\
    & \mid & lident & Variable \\
    & \mid & [[ident] \terminal{\#}]uident & 0-ary sum type value constructor \\
    & \mid  & [[ident] \terminal{\#}]uident
               \ \terminal{(}pattern \{\terminal{,}\ pattern\}*\terminal{)}&
               N-ary sum type constructor \\
    & \mid & \_ & ``Catch-all'' pattern \\
    & \mid & \terminal{\{}
               record\_field\_pattern\
               \{ \terminal{;}\ record\_field\_pattern \}*
                 \terminal{\}} & Record \\
    & \mid & \terminal{(} pattern\ \{ \terminal{,}\ pattern\}+
                  \terminal{)} & Tuple \\
    & \mid & \terminal{(} pattern \terminal{)} & Parenthesised pattern
\end{syntax}

\begin{syntax}
\syntaxclass{Record field pattern:}
record\_field\_pattern & ::= &
     [[ident] \terminal{\#}] lident\ \terminal{=}\ lident &
\end{syntax}
\vspace{0.2cm}



%%%%%%%%%%%%%%%%%%%%%%%%%%%%%%%%%%%%%%%%%%%%%%%%%%%%%%%%%%%%%%%%%%%%%%%
\subsubsection{Literal expressions}
\index{expression!literal} \index{expression!constant} 

The literal
expressions  of type integer, string, character,
float and boolean) are evaluated into the constant represented by the
literal. The expression {\tt 25} denotes the value 25 of type
{\tt int}. 



%%%%%%%%%%%%%%%%%%%%%%%%%%%%%%%%%%%%%%%%%%%%%%%%%%%%%%%%%%%%%%%%%%%%%%%
\subsubsection{Sum type  value constructor expressions}
\index{expression!sum type constructor}
We presented in section \ref{type-definition} the way to define sum
types. We saw that {\bf values} of such a {\bf type} are built using
its {\bf value} constructors.

Hence, for {\bf value} constructors with no argument, the constructor itself
is an expression that gets evaluated in a value wearing the same name.

For {\bf value} constructors with parameters, a value is created by
evaluating an expression applying the constructor to as many expressions as
the constructor's arity. Obviously, sub-expressions used as arguments of
the constructor must we well-typed according to the type of the
constructor. The resulting value is written by the
name of the constructor followed by the tuple of values given as
arguments. For instance, with the following type definition:
{\scriptsize
\begin{lstlisting}
type t =
  | A
  | B (int * bool)
;;
\end{lstlisting}
}
the expression {\tt A} is evaluated into $A$, the expression
{\tt B ((2 + 3), true)} is evaluated into the value $B (5, true)$.



%%%%%%%%%%%%%%%%%%%%%%%%%%%%%%%%%%%%%%%%%%%%%%%%%%%%%%%%%%%%%%%%%%%%%%%
\subsubsection{Identifier expressions}
\label{identifier-expressions}
\index{identifier}
\index{expression!identifier}


An identifier expression is either a basic identifier, an extended
identifier or a qualified identifier (see section
\ref{qualified-name}), which denotes the value of this identifier in
the scope of the expression. The identifier is said to be {\bf bound}
to this value.

The value bound to an identifier can be of any type. A value having a
functional type, that is a {\bf functional
  value}\label{functional-value}\index{functional value} also called a
{\bf closure}, is created by a function definition. Such a value,
obtained by the evaluation of the body of the function, is slightly
different from other ones since it embeds both the code of the
function (i.e. a kind of evaluation of its body expression) and its
environment (i.e. bindings between identifiers occuring in the body of
the function and their value in the definition scope). This closure
will be kept untouched until it appears in a functional application
expression as described further in \ref{function-application}.

There are several possibilities to bind an identifier.  Definitions
introduce a basic or extended identifier and {\bf binds} it to the
value of the expression stated in the definition.  There are three
ways to introduce and directly bind an identifier:
\begin{itemize}
  \item By a {\tt let-in} construct,
  \item By a toplevel-definition ({\tt let} or {\tt theorem}),
  \item by a method definition  ({\tt let}),
 \end{itemize}
 Each of these three cases will
be described in their related section.


There are two ways to introduce basic identifiers as parameters:
\begin{itemize}
\item in a function definition
 \item by a pattern inside a {\tt match-with} construct
\end{itemize}
Then the binding of the parameter is differed until the application of
the function or the pattern-matching mechanism.
 Each of these two cases will
be described in their related section.


Suppose that an expression $exp$ contains several occurrences of an
identifier {\tt my\_var}.  Assume that, in the scope of $exp$, {\tt
  my\_var} is bound to a {\bf value} $v$, then each occurrence of {\tt
  my\_var} in $exp$ is substituted by $v$ during the evaluation of
$exp$. This is basically the principle of the so-called {\bf eager} or
{\bf call by-value} evaluation regime.


\medskip
\noindent{\bf Identifier resolution}
Remember that identifiers forms differ depending on the syntactic
class of entity they refer to, capitalized identifiers being used for
species and colections.  To evaluate an identifier expression, the
\focal\ compiler tries to find its definition from the current scoping
context.  \index{name!resolution} \index{name!qualification}
\index{scoping} It searches for the closest definition with this name,
starting by the parameters present in the current definition
(i.e. formal parameters in a function and in a {\tt match-with}
construction and {\tt let-in} bound identifiers). If no identifier
definition with this name is found, the search goes on among the
methods of the current species. If a method is found with this name,
it will be retained, otherwise the identifier is looked in the
preceding toplevel-definitions of the current compilation unit.
\label{identifier-scoping-and-open}
\index{directive!open} If no suitable definition is found, then the
ones imported by the {\tt open} directives are examined to find one
with the searched name. Finally if no definition is found, the
identifier is {\bf reported unbound} by an error message.

 Note that an
{\tt open} directive may arise anywhere at toplevel in the source
code. Hence, the order of search between the current file's toplevel-definitions and the imported ones by {\tt open} is not really
separated: the name resolver looks for the most recent definition
considering that the toplevel-definitions and the imported ones are
ordered according to the apparition of the effective definitions in
the file themselves and the imported ones. In other words, if a
toplevel-definition exists for an entity {\tt foo}, if later an
{\tt open}�directive imports another {\tt foo}, then this last one
will be the retained one.

\medskip

\medskip
\noindent{\bf Identifier qualification}

Identifiers can manually be disambiguated in term of compilation unit
location using the sharp (\#) notation as explained in section
\ref{qualified-name}.

\medskip \index{method!qualification} As further presented in section
\ref{method-qualification}, species methods identifiers are made
explicit using the ``!'' notation. The notation {\tt Spe!meth} stands
for ``the method {\tt meth} of the species {\tt Spe}''. By extension,
{\tt !meth} stands for the method {\tt meth} of the current
species. It is possible to explicit {\tt Self} in the naming scheme
using {\tt Self!meth}. This is useful when a more recently defined
identifier hides a method of the species at hand: {\scriptsize
\begin{lstlisting}
species S =
  let m (x in ...) = ... ;
  let n (y in ...) =
    ...
    let m = ... in
    (* Want to call the *method* "m" with argument "m" !!! *)
    !m (n) ;
end ;;
\end{lstlisting}
}

Hence, the name resolution mechanism allows to omit the ``!'' but
making it explicit can help for conflicts resolution. Moreover, when
invoking species parameters' methods, the name resolution never
searches among methods of collection parameters, hence the explicit ``!''
notation is required.

As the grammar shows, name qualification by compilation unit and
hosting species can be freely mixed. We can build identifiers like
{\tt my\_file\#My\_species!my\_method} to refer to the method
{\tt my\_method} hosted in the species {\tt My\_species} located in
the \focal\ source file ``my\_file.fcl''. These disambiguation
methods are indeed orthogonal.



\medskip
\noindent{\bf Extended identifier expressions}

Finally, infix/postfix operators can be used as regular
identifiers. Usually, an operator is syntactically used according to
it prefix or infix nature. For instance, the binary {\tt +} operator
is used between its arguments as in {\tt x + 4}, the unary operator {\tt
  $\sim$} is used before its argument as in {\tt $\sim$ x}. \focal\ allows to
refer to those operators as regular identifiers (for instance as function
parameters).
This allows to use operators as any other identifiers, and
\begin{itemize}
  \item using them as regular function (i.e. in functional position),
  \item bind them as arguments of functions,
  \item use them as regular identifiers in expressions, for example to
    pass them as arguments of other functions.
\end{itemize}
To get an identifier from an operator, its symbol
(c.f. \ref{extended-identifiers}) must be delimited by spaces and enclosed
into matching parentheses.
For example: {\tt (\ + )} is the regular identifier corresponding to the
infix symbol {\tt +}.

Note that spaces around the operator symbol are mandatory and part of the
syntax. If spaces are omitted, the parens get their usual meaning and the
interpretation can be completely different.
A specially puzzling error is to write {\tt (*)} to mean {\tt ( * )}:
{\scriptsize
\begin{lstlisting}
...
let (*) (x, y) = ...
\end{lstlisting}
}
Now, {\tt (*} is evidently parsed as the beginning of comment,
leading to a syntax error or any other cryptic error long after the faulty
{\tt (*} occurrence.
Conversely {\tt *)} is always considered as an end of comment by the lexical
analyzer.

%This pitfall exists indeed
%only for the ``{\tt *}-starting'' and ``{\tt *}-ending'' operators,
%but a good practice is to have a homogeneous naming scheme, hence
%always adding extra spaces.
% I'm not sure that this is true. I don't want to find a counter example
% though.
% In fact we do not need a counterexample:
% addind spaces is not a good practice: it is mandatory! In focalize, (+) never means the prefix
% version of + !!



%%%%%%%%%%%%%%%%%%%%%%%%%%%%%%%%%%%%%%%%%%%%%%%%%%%%%%%%%%%%%%%%%%%%%%%
\subsubsection{{\tt let-in} expression}
\index{let-in}
\index{expression!let-in}
{\tt let-in} expression binds an identifier to a value to
evaluate a trailing expression (the ``{\tt in-part}'' of the ``{\tt let-in}''
or ``body'') where this ident may appear. During the evaluation of the
trailing expression, any occurrence of the bound identifier is ``replaced''
by the value bound to this identifier. For instance:
{\scriptsize
\begin{lstlisting}
let x = 5 in (x, x)
\end{lstlisting}
}
 binds {tt x} to the evaluation of the {\bf expression} {\tt (3+2)}
(i.e. the integer {\bf value} $5$) and then, the evaluation of the
trailing expression returns the tuple {\bf value} $(5, 5)$.  From the
syntax, it is clear that {\tt let-in} constructs can be nested.  For
instance, {\scriptsize
\begin{lstlisting}
let x = 3+2 in
  let y = (x, x) in
  let z = true in
  (y, z, y, z)
\end{lstlisting}
}
returns the value $((5,\ 5),\ true,\ (5,\ 5),\ true)$ of type
{\tt ((int * int) * bool * (int * int) * bool)}.



\medskip
\index{identifier binding}
Note that the notion of ``binding an identifier to a value'' is essentially different from the
notion of assignment in imperative languages. In such languages (like
C, Java, Pascal,\ldots) a variable is first {\em declared}, then a value is
{\em assigned} to the variable. It is thus possible to assign a variable
several times to different values. For example in C:
{\scriptsize
\begin{lstlisting}
...
{
  int i ;
  ... ;
  i = 10 ;
  while (i > 0) i = i-- ;
}
...
\end{lstlisting}
}
The variable {\tt i} is declared, then assigned the initial value
$10$, then the {\tt while} loop makes it decreasing by successive
assignments.

In a {\tt let-in} binding construct, an identifier is given a value
once and for all: it is impossible to change its  value, once it
has been bound. 
Each new definition, binding an already bound identifier will just hide the
old definition. For instance:
{\scriptsize
\begin{lstlisting}
let x = 5 in
  let y = (x, x) in
  let x = true in
  let z = (x, x) in
  (y, x, y, x)
\end{lstlisting}
}
leads to the value $((5,\ 5),\ true, (5,\ 5),\ true)$
of type {\tt ((int * int) * bool * (int * int) * bool)}. Clearlythe
first value bound to {\tt x}  holds until {\tt x}
is bound again: $5$ is used to define {\tt y} but not to define
{\tt z}, since the value of {\tt x} is then the boolean $true$.

\medskip

The {\tt let-in} construct serves to bind an identifier to a value of
any type.  As a consequence, it can also bind an identifier to a
functional value.  This lead to the natural way to define {\bf
  functions}. For instance:

 {\scriptsize
\begin{lstlisting}
let f (x, y) = x + y in
f (6, 7)
\end{lstlisting}
}

The {\bf let} construct binds {\tt f} to a function which has 2
parameters {\tt x} and {\tt y}, and the body of {\tt f} is the
addition of these 2 parameters. Then the body of the {\tt let-in}
construct applies {\tt f} to 2 effective arguments {\tt 6} and {\tt
  7} (we obviously expect the result of this {\em application} to be
$13$).  (Function application is explained below in
\ref{application-expression}).

It is possible to provide a type constraint to precise the type of the
return value of a function, or the type of the {\tt let}-bound
variable or parameters:

{\scriptsize
\begin{lstlisting}
let f (x : int, y) = x + y in
f (6, 7)
\end{lstlisting}
}
{\scriptsize
\begin{lstlisting}
let f (x : int, y) in int = x + y in
f (6, 7)
\end{lstlisting}
}
{\scriptsize
\begin{lstlisting}
let a in int = 3 in
(a, a)
\end{lstlisting}
}



\medskip
It is possible to define several identifiers at the same time separating
each definition by the keyword {\tt and}.
{\scriptsize
\begin{lstlisting}
...
let f = exp_1
and g  = exp_2
and h = exp_3 in exp;
\end{lstlisting}
}

All the definitions are separately evaluated ``in parallel''.  As a
consequence, the identifiers introduced by a {\tt let ... and }
cannot be used in the right members of this construction (in the {\tt
  exp\_i}). Do not confuse this construct with nested {\tt let-in} as
the followig one, where {\tt exp\_2} can contain {\tt f} and {\tt
  exp\_3} can contain {\tt f} and {\tt g}. 

{\scriptsize
\begin{lstlisting} 
let f = exp_1 in
 g  = exp_2 in 
 h = exp_3 in exp
\end{lstlisting}
}

Mutually recursive functions need to know each other because their
bodies call these other functions and their definition require a
non-nested evaluation of each function.  In this case, the keyword
{\tt let} must be followed by the keyword {\tt rec}. 

 {\scriptsize
\begin{lstlisting}
...
let rec even (x) =
   if x = 0 then true else odd (x - 1)
and odd (y) =
   if y = 0 then false else even (y - 1) in
...
\end{lstlisting}
}

{\large {\bf Warning:}} in the current version of \focal\ mutually
recursive functions cannot be compiled into \coq\ code. Only
\ocaml\ code generation is available. Moreover, for \coq, recursive
functions imply termination proofs. This last point will be covered in
the section \ref{recursive-function} especially dedicated to
recursive (non-mutually) function definitions.

\medskip
{\color{red}To to: logical and local}.



%%%%%%%%%%%%%%%%%%%%%%%%%%%%%%%%%%%%%%%%%%%%%%%%%%%%%%%%%%%%%%%%%%%%%%%
\subsubsection{Conditional expression}
\label{conditional-expression}
\index{expression!if}
\index{if}
A conditional expression has the form:

{\tt if} $exp_1$ {\tt then}
$exp_2$ {\tt else} $exp_3$ 

Its evaluation starts by the evaluation of the $exp_1$ expression
which must be of type boolean. If its value is $true$ then the result
value of the whole expression is the value of $exp_2$, otherwise
(i.e. if its value is $false$) the value of $exp_3$. This obviously
implies that $exp_2$ and $exp_3$ must have the same type. This
construct is then a binary conditional expression (i.e. with 2
branches).  {\scriptsize
\begin{lstlisting}
let f (x) = if x then 1 else 0 in ...
\end{lstlisting}
}
The function {\tt f} will return $1$ if the effective argument
provided for {\tt x} is $true$, otherwise it will return $0$.
{\scriptsize
\begin{lstlisting}
let is_too_small (x) = ... in
let y = ... in
let y_corrected = if is_to_small (y) then 0 else y in ...
\end{lstlisting}
}
In this example, we assume we have a function {\tt is\_too\_small}
checking if a value is ``too small'' and an identifier {\tt y} bound
to a certain value. The result of the conditional expression bound to
{\tt y\_corrected} will be either $0$ if the condition is met or $y$
otherwise.



%%%%%%%%%%%%%%%%%%%%%%%%%%%%%%%%%%%%%%%%%%%%%%%%%%%%%%%%%%%%%%%%%%%%%%%
\subsubsection{Match expression}
\index{match} \index{expression!match} \index{pattern matching} 

The {\tt match-with} construct is a generalised conditional construct
with pattern-matching. By ``generalised'', we mean that unlike the
{\tt if-then-else} which has only 2 branches, the present expression
can have several branches. The notion of condition here is not anymore
a boolean value. Instead, the construct allows to discriminate on the
different values an expression is evaluated into. The basic structure
of a {\tt match-with} consists in a discriminating expression followed
by an enumeration of cases (called {\bf patterns}). The dicriminating
expression is evaluated and its value is matched against the patterns,
following the textual ordering of these patterns, until a match
succeeds. Then the expression associated with the matchng pattern is
evaluated to obtain the value of the whole expression {\tt
  match-with}.


{\scriptsize
\begin{lstlisting}
let a = ... ;
let x =
  match a + 5 with
   | 0 -> "zero"
   | 5 -> "five"
   | 1 -> "one"
   | 10 -> "ten"
   | _ -> "other" in
...
\end{lstlisting}
} The discriminating expression in this case is {\tt a + 5} of type
{\tt int}. We can then react to each (or some of the) values of this
expression. When {\tt a + 5}  is equal to $0$ the result of the
whole {\tt match-with} expression (bound to the identifier {\tt x}) is
the string ``zero''. When {\tt a + 5} is equal to $1$, the result is the string
``one'', and so on. The final pattern {\tt \_} stands for ``anything
that was not in the previous cases'' (also called ``catch-all
pattern''). Hence, the order of the patterns is important. If the
case {\tt $\mid$ \_ ->} was put before the case {\tt $\mid$ 1 ->},
then this last case would never be reached since the {\tt \_} pattern
would have caught the discriminated value.



\smallskip
As a consequence of the structure of this construct, type constraints
must be respected in order to have the whole expression well-typed:
\begin{itemize}
  \item The type of the discriminating expression must be compatible with the
    type of the patterns.
  \item thus all the patterns must have compatible
    types.
  \item The types of all the result expressions in the rightmost parts
    of the cases must be compatible.
\end{itemize}


\medskip In the example above, the patterns were constants. A value
matches a constant pattern if and only if it is equal to this
constant.  In addition to this multi-branches conditional feature, the
{\tt match-with} construct provides true {\bf pattern matching}. That
is, patterns may be built from constants, value constructors,
variables and the catch-all symbol {\tt \_}. Any value matches any
variable pattern and the {\tt \_} pattern. For general patterns built
from value constructors, variables, constants,{\tt \_}, roughly
speaking, a value matches a pattern if this pattern can be seen as a
prefix of this value. Then, the variables of the pattern get bound to
the parts of the discriminating expression that are ``at the same
place'' than those variables. For example: 

{\scriptsize
\begin{lstlisting}
let e = ... in
...
let x =
  match e with
   | (0, 0, 0) -> 1
   | (0, x, y) -> x + y
   | (1, 1, x) -> x
   | (x, y, z) -> x + y + z
...
\end{lstlisting}
}
According the the type-checking mechanism, the examined expression
{\tt e} must have here type {\tt (int * int * int)}. The first pattern
will be chosen if {\tt e} is  equal to the tuple $(0, 0, 0)$. We
say here ``equal'' since there is no variable in the pattern, hence
the only way to fit the pattern is to simply  be equal. If this pattern
is not fitted, the we examine the second pattern. It will be chosen if
{\tt e} has a $0$ as first component and any integer for the second
and the third ones. In this case, the result value will be the
evaluation of the expression {\tt x + y} where x will be bound to the
effective second component of the value of {\tt e} and {\tt y} will be
bound to its third component. We can notice that no ``catch-all pattern''
is needed since the enumerated patterns cover all the possible values
of tuples with 3 components (look at the last pattern that do not put
any constraint on the tuple components, hence will match all the
remaining cases).

The previous example used tuples as matched expression and patterns,
but patterns also contain sum type value constructors, hence allowing
to ``match'' on any sum type structure. For example: {\scriptsize
\begin{lstlisting}
type t =
  | A
  | B (int)
  | C (int * int)
;;
...
let e = ... in
let x =
  match e with
   | A -> 0
   | B (3) -> 4
   | B (_) -> 10
   | C (x, 10) -> 5
   | C (_, y) -> y
;;
\end{lstlisting}
}
This example shows different cases following the structure of the type
{\tt t}. Note the use of the ``catch-all'' pattern 
inside  patterns. In fact, the ``catch-all'' pattern acts like
a variable  unused in the rightmost part of the
case. It is however preferable to use ``{\tt \_}'' instead of a
variable since \ocaml\ generates warning for unused variables and the
generated \ocaml\ code generated by \focal\ will not change unused
variables into ``{\tt \_}''s.


\smallskip
Patterns also allow to match record values
(c.f. \ref{record-expression}), i.e. to match on values of the fields:
{\scriptsize
\begin{lstlisting}
type t = { name : string ; birth : int } ;;
let r = ... in
let x =
  match r with
   | { name = "Alexandre" } -> ...
   | { name = n ; birth = 2003 } -> ...
   | { name = n } -> ...
\end{lstlisting}
}
In such a pattern, fields not specified are considered as
``catch-all'' patterns. Hence, the last case catches all the record
values not caught before since the field {\tt name}'s value is bound
to a variable (so, any value can match it) and the field
{\tt birth} is absent (so, considered as {\tt birth = \_}).



%%%%%%%%%%%%%%%%%%%%%%%%%%%%%%%%%%%%%%%%%%%%%%%%%%%%%%%%%%%%%%%%%%%%%%%
\subsubsection{Application expression}
\label{function-application}
\index{functional value} \index{expression!application} We previously
saw that the {\tt let-in} construct allows to define functions by
binding an identifier to a functional value. Using a function by
providing it with effective arguments to get its result value is
called {\bf application}. Hence, in an application there are 2
distinct parts: the applicative part that must be an expression
leading to a functional value and the effective arguments that are
expressions whose values will be provided to the function to make its
computation. The syntax for application is simply the juxtaposition of
the applicative expression and the comma-separated expressions used as
arguments embraced by parentheses: {\scriptsize
\begin{lstlisting}
let f (x) = ... in
let g (x, y) = ... f (y) ... in
g (f (3), 4)
...
\end{lstlisting}
}
As described in \ref{functional-value}, the evaluation of an
application of a function to its effective arguments start by the
evaluation of these arguments (the order of the evaluation of several
arguments is left unspecified). Then these effective values are
substituted to the corresponding parameters inside the body of the
function and the so-obtained expression (the substituted body) is
evaluated.  For instance, having the
following function and application:
{\scriptsize
\begin{lstlisting}
let g (x, y) = (y, x) in
g (true, 1)
\end{lstlisting}
} The evaluation of this {\tt let-in} expression first binds the
identifier {\tt g} to a {\bf functional value} also called {\bf
  closure}. Then the application expression {\tt g (true, 1)} is
evaluated. So the values of {\tt g} and of the expression {\tt
  (true,1)} are elaborated: the evaluation of {\tt g} returns a
closure, {\tt true} is evaluated into the boolean {\bf value} $true$,
{\tt 1} into the integer {\bf value} $1$. The next step is to evaluate
the body of the {\bf closure} of {\tt g}, replacing the formal
parameter {\tt x} by the effective argument $true$ and {\tt y} by
$1$. The body of {\tt g} creates a tuple from its 2 arguments, putting
{\tt y} in the first component and {\tt x} in the second. Hence, the
result of the application is the tuple {\bf value} $(1,\ true)$.



%%%%%%%%%%%%%%%%%%%%%%%%%%%%%%%%%%%%%%%%%%%%%%%%%%%%%%%%%%%%%%%%%%%%%%%
\subsubsection{Operator application expression}
\index{expression!operator}
Since operators are designed to be used in infix or prefix position,
application of operators consists simply in providing arguments
according to the operator infix/prefix nature. For infix operators,
arguments are on left and right sides. For prefix operators, the
operator is in front of the argument expression.



%%%%%%%%%%%%%%%%%%%%%%%%%%%%%%%%%%%%%%%%%%%%%%%%%%%%%%%%%%%%%%%%%%%%%%%
\subsubsection{Record expression}
\label{record-expression}
\index{expression!record}
As stated in \ref{record-type-definition},
record types are defined by a list of labels with their types. As
usual a record
expression follows the same structure, replacing the type expressions
of the definition by values of these types. For instance, assuming the
given record type definition, the following example shows a possible
record value: 
{\scriptsize
\begin{lstlisting}
type identity = {
  name : string ;
  birth : int ;
  living : bool
} ;;

...
{ name = "Nobody" ; birth = 42 ; living = false }
...
\end{lstlisting}
}
If the record type definition is in a different compilation unit, you
may qualify the record fields by the ``{\tt\#}'' notation:
{\scriptsize
\begin{lstlisting}
{ my_file#name = "Nobody" ; my_file#birth = 42 ; my_file#living = false }
\end{lstlisting}
}



%%%%%%%%%%%%%%%%%%%%%%%%%%%%%%%%%%%%%%%%%%%%%%%%%%%%%%%%%%%%%%%%%%%%%%%
\subsubsection{Cloning a record expression}
\label{record-clone-expression}
\index{expression!record!clone}
 It is sometimes needed to create a new
value of a record type by modifying a few fields of an existing
record, leaving the other fields unchanged. If the record type
definition contains numerous fields, manually copying the old fields
values to create the new record value appears boring and error prone:
{\scriptsize
\begin{lstlisting}
type t = { a : int ; b : int ; c : int ; d : int ; e : int ; f : int } ;;
...
let v1 = { a = 1 ; b = 2 ; c = 3 d = 4 ; e = 5 ; f = 6 } in
...
let v2 = {
  a = v1.a ; b = v1.b ;
  c = 5 ;  (* Changed value. *)
  d = v1.c ;  (* an error since the requested value was  "v1.d". *)
  e = 6 ;  (* Changed value. *)
  f = v1.f } in
...
\end{lstlisting}
}
Instead of manually copy the unchanged fields, \focal\ provides a way
to clone a record value, that is to create a {\bf new}, a {\bf fresh}
value from an existing one,  only by specifying the fields whose values
differ from the old record value:
{\scriptsize
\begin{lstlisting}
type t = ... (* Like above. *)
let v1 = ... (* Like above. *)
...
let v2 = { v1 with c = 5 ; e = 6 } in
...
\end{lstlisting}
}
As for other record value expressions, if the record type definition
is in a different compilation unit, you may qualify the record fields
by the ``{\tt\#}'' notation.



%%%%%%%%%%%%%%%%%%%%%%%%%%%%%%%%%%%%%%%%%%%%%%%%%%%%%%%%%%%%%%%%%%%%%%%
\subsubsection{Record field access expression}
\label{record-field-access}
\index{expression!record!field}
Once a record value is created by aggregating values of its fields, it
is possible to recover the value of one field by a dot notation. For
instance, assuming the type definition and record values of the
previous example:
{\scriptsize
\begin{lstlisting}
... v1.a ...
... v2.c ...
\end{lstlisting}
}
respectively get the value of the fields {\tt a} of {\tt v1} and
{\tt c} of {\tt v2}, that is, $1$ and $5$. If the record type
definition is in a different compilation unit, you may qualify the
record fields by the ``{\tt\#}'' notation: {\tt t1.my\_source\#a}.



%%%%%%%%%%%%%%%%%%%%%%%%%%%%%%%%%%%%%%%%%%%%%%%%%%%%%%%%%%%%%%%%%%%%%%%
\subsubsection{Parenthesised expression}
The parentheses can be used around any expression, to enforce the
associativity or evaluation order of expressions. Simple expressions
(i.e. atomic) can also be parenthesised without changing their
values.



%%%%%%%%%%%%%%%%%%%%%%%%%%%%%%%%%%%%%%%%%%%%%%%%%%%%%%%%%%%%%%%%%%%%%%%
%%%%%%%%%%%%%%%%%%%%%%%%%%%%%%%%%%%%%%%%%%%%%%%%%%%%%%%%%%%%%%%%%%%%%%%
\subsection{Core language expressions and definitions}
In the previous sections, we described the syntax of
expressions. Expressions rarely appear outside any
definition but it is still possible to have top-level
expressions. They  will be directly evaluated and not bound to any
identifier, but this implies that these expressions use previously
written definitions.

\medskip
As   further explained  in (c.f. \ref{species-intro}) species are made
of methods. Some methods contain expressions (functions, properties,
theorems). Function-methods are introduced by the {\tt let} keyword,
using the same syntax (hence expressions) that the {\tt let-in}
construct except the fact they do not have a ``{\tt in}'' expression.
The idea is that the ``{\tt in}'' expression is implicitly the
remaining of the species. Properties and theorems are respectively
introduced by the keywords {\tt property} and {\tt theorem} and may
 contain expressions. The  section 
\ref{properties-theorems-proofs}  is dedicated to their detailed explanation.
{\scriptsize
\begin{lstlisting}
open "basics" ;;
species My_Setoid inherits Basic_object =
  signature ( = ) : Self -> Self -> bool ;
  signature element : Self ;
  let different (x, y) = basics#not_b (x = y) ;

  property refl : all x in Self, x = x ;
  property symm : all x y in Self, Self!( = ) (x, y) -> y = x ;
end ;;
\end{lstlisting}
}


\medskip

Toplevel-definitions are definitions introduced outside of any
species. General functions and general theorems, i.e. that do not
depend on a particular species can be introduced as
toplvel-definitions. Toplevel-functions are introduced by the {\tt
  let} keyword and don't have a ``{\tt in}'' expression, this part
being implicitly the remaining of the program (i.e. the current
compilation unit and those using the current).  Toplevel-theorems are
introduced by the {\tt theorem} keyword. These definitions must be ended by a double semi
(``{\tt ;;}'').  {\scriptsize
\begin{lstlisting}
let is_failed (x) =
  match x with
  | Failed -> true
  | Unfailed (_) -> false
;;

theorem int_plus_minus: all x y z in int,
  (*   x + y = z -> y = z - x *)
  #base_eq (#int_plus (x, y), z) -> #base_eq (y, #int_minus (z, x))
  proof:
    coq proof {*
      intros x y z;
      unfold int_plus, int_minus, base_eq, syntactic_equal in |- *;
      intros H;
      unfold bi__int_minus;
      apply EQ_base_eq; apply Zplus_minus_eq;
      symmetry  in |- *;
      apply (decidable _ _ _ (Z_eq_dec (x + y) z) H).
      Qed.
    *}
;;
\end{lstlisting}
}



%%%%%%%%%%%%%%%%%%%%%%%%%%%%%%%%%%%%%%%%%%%%%%%%%%%%%%%%%%%%%%%%%%%%%%%
%%%%%%%%%%%%%%%%%%%%%%%%%%%%%%%%%%%%%%%%%%%%%%%%%%%%%%%%%%%%%%%%%%%%%%%
\subsection{Files and uses directives}
\label{file directive}
\focal\ provides 3 directives that are not expressions. This means
that they do not lead to values or computation.



%%%%%%%%%%%%%%%%%%%%%%%%%%%%%%%%%%%%%%%%%%%%%%%%%%%%%%%%%%%%%%%%%%%%%%%
\subsubsection{The {\tt use} directive}
{\color{red}To to}.



%%%%%%%%%%%%%%%%%%%%%%%%%%%%%%%%%%%%%%%%%%%%%%%%%%%%%%%%%%%%%%%%%%%%%%%
\subsubsection{The {\tt open} directive}
\index{directive!open}


This directive is followed by the name of the file to open between
double quotes without the ``.fcl'' extension. As previously introduced
(c.f. \ref{identifier-scoping-and-open} and \ref{qualified-name}) the
{\tt open} directive loads in the current name resolution (scoping)
environment the definitions of the compilation unit named in the {\tt
  open} directive. This prevents the user from having to explicitly
qualify definitions of this unit by the ``{\tt\#}''
notation. Definitions imported by the directive hide (``mask'') those
wearing the same name already defined in the current compilation unit
from the point the directive appears. Remember that it is however
possible to recover the hidden definitions, using the ``{\tt\#}''
notation without compilation unit name.


{\scriptsize
\begin{lstlisting}
open "sets";;
\end{lstlisting}
}This directive load the definitions of the compilation unit
``sets.fcl'' in the current name resolution (scoping) environment.


The path of the compilation unit is never specified. The file
will be searched in the library search path specified with the
{\tt -I} option (c.f. \ref{compiler-options}).




%%%%%%%%%%%%%%%%%%%%%%%%%%%%%%%%%%%%%%%%%%%%%%%%%%%%%%%%%%%%%%%%%%%%%%%
\subsubsection{The {\tt coq\_require} directive}
\index{directive!coq\_require}
Some source files of a development may be directly written in \coq\ to
provide external definitions (more detailed further in
\ref{interfacing-other-languages}) to import and use in the
\focal\ source code. In this case, the \coq\ code generated for the
\focal\ source code must be aware of the need to import the external
definitions from the manually written \coq\ file. For this reason, the
\focal\ source must explicitly indicate by the {\tt coq\_require}
directive that it makes references to definitions hosted in this
\coq\ source file. For example, the file ``wellfounded.fcl'' of the
standard library needs ``wellfounded\_externals.v'' and signals this
fact in its early lines of code:
{\scriptsize
\begin{lstlisting}
...
open "basics";;
open "sets_orders";;
coq_require "wellfounded_externals";;
...
\end{lstlisting}
}



%%%%%%%%%%%%%%%%%%%%%%%%%%%%%%%%%%%%%%%%%%%%%%%%%%%%%%%%%%%%%%%%%%%%%%%
%%%%%%%%%%%%%%%%%%%%%%%%%%%%%%%%%%%%%%%%%%%%%%%%%%%%%%%%%%%%%%%%%%%%%%%
\subsection{Properties, theorems and proofs}
\label{properties-theorems-proofs}
\index{theorem}
\index{property}

Properties are first order logic propositions and theorems are
properties with their proofs. We will study here first the structure
of logical expressions used to express the statements, show properties
and theorems forms and shorty present the 3 available ways to write
proofs.



%%%%%%%%%%%%%%%%%%%%%%%%%%%%%%%%%%%%%%%%%%%%%%%%%%%%%%%%%%%%%%%%%%%%%%%
\subsubsection{Logical expressions}
\label{logical-expressions}
\index{expression!logical}
Logical expressions are those used to write first order logic
formulaes.
\begin{syntax}
\syntaxclass{Logical expressions:}
logical\_expr & := & \terminal{all}\ lident\ [,lident]*
          \ \terminal{in} \ type\_expr & Universal quantification \\
              & \mid & \terminal{ex}\ lident\ [,lident]*
          \ \terminal{in} \ type\_expr & Existential quantification \\
              & \mid & logical\_expr\ \terminal{-}\/\terminal{>}
                   \ logical\_expr & Implication \\
              & \mid & logical\_expr\ \terminal{<}\/\terminal{-}\/\terminal{>}
                   \ logical\_expr & Equivalence \\
              & \mid & logical\_expr\ \terminal{/\setminus} \ logical\_expr &
                              Conjunction \\
              & \mid & logical\_expr\ \terminal{\setminus/} \ logical\_expr &
                              Disjunction \\
              & \mid & \terminal{\sim}\ logical\_expr & Negation \\
              & \mid & expr & Arbitrary \focal\ expression (atom) \\
              & \mid & \terminal{(} logical\_expr \terminal{)} &
                              Parenthesised logical expression
\end{syntax}

 Logical expressions contain the usual logical connectors ``imply''
($\Rightarrow$), ``and'' ($\wedge$), ``or'' ($\vee$), ``there exists''
($\exists$), ``for all'' ($\forall$), ``is equivalent''
($\Leftrightarrow$) and ``not'' ($\sim$). Moreover, logical
expressions embed the \focal\ expressions used in computational
methods (i.e. identifiers, conditionals, application, \ldots). This
allows to have connected propositions using the previously defined
functions and species methods.
{\scriptsize
\begin{lstlisting}
species S ... =
  signature gt : Self -> Self -> bool ;  (* Greater than... *)
  signature geq : Self -> Self -> bool ;  (* Greater or equal... *)
  signature equal : Self -> Self -> bool ;  (* Equal to... *)
  signature different : Self -> Self -> bool ;  (* Different of... *)

  property gt_is_lt : all x y in Self,
    (!gt (x, y) -> (!geq (x, y) /\ !different(x, y)))
    /\
    (!geq (x, y) -> (!gt (x, y) \/ !equal(x, y))) ;
end ;;
\end{lstlisting}
}
Since  propositions in logical expressions are truth values, this
obviously imply that the arbitrary expressions used between connectors
must have type {\tt bool}.



%%%%%%%%%%%%%%%%%%%%%%%%%%%%%%%%%%%%%%%%%%%%%%%%%%%%%%%%%%%%%%%%%%%%%%%
\subsubsection{Properties}
A property is  a logical expression bound to an identifier. Its
form is the name of the property, a colon character (``:'') and the
logical expression being its statement. See the example given in
\ref{logical-expressions}.
\begin{syntax}
\syntaxclass{Properties:}
property & ::= & \terminal{property}\ lident\ \terminal{:}\ logical\_expr
\end{syntax}




%%%%%%%%%%%%%%%%%%%%%%%%%%%%%%%%%%%%%%%%%%%%%%%%%%%%%%%%%%%%%%%%%%%%%%%
\subsubsection{Proofs}
\label{proof-short-intro}
\focal\ currently provides 3 ways to write proofs. We only give here a
simple description of these 3 means without going deeply in the
technical mechanisms since this problem will be especially addressed
in section \ref{making-proofs} and \ref{coq-proofs}.

\begin{itemize}
\item {\bf Consider the proof as ``assumed''}. This way is the
  simplest but also the weakest one since it consists in saying that
  no proof is given and the system must trust the stated statement.
  {\scriptsize
\begin{lstlisting}
species S =
  representation = int;
  let equal = (=Ox);
  theorem symetry : all x y in Self, Self!equal (x, y) -> equal (y, x)
  proof : assumed 
    {*The equality of machine integers is admitted to be symetric  *} ;
  ...
end ;;
\end{lstlisting}
}

 Following the {\tt assumed} keyword is a mandatory message used for
sake of information, justification, traceability of the proof
absence. Although such a proof can introduce inconsistencies if the
``theorem'' is not a tautology and thus decrease confidence in the
correctness of the \focal\ program, there are several cases where
using this keyword may help.
  \begin{itemize}
      \item The first case is simply that the developer doesn't know
        (yet ?) how to make the proof, doesn't have time yet to write
        it, is not interested in proofs but still wants his program to
        compile to get the executable code.

      \item Second case deals with import of external code, i.e. code
        not written in \focal\ and considered as external. In this
        case, since the imported code does not fit the \focal\ model
        and more accurately, does not have formal properties, it is
        impossible to make any proof on \focal's side based on the
        structure of this code and its non-existing implementation
        properties. In other terms, things coming outside \focal\
        universe can not be modeled by \focal. The developer only can
        import them providing a binding is given and must trust them.

      \item Last case addresses ``well-known'' mathematical properties
        that do not actually hold in computers since they are finite
        machines, working on bounded arithmetics. The most obvious
        example is the fact that since an integer is coded on a
        machine word (often $2^{32}$ or $2^{64}$ bits nowadays), the
        mathematical property $\forall x \in \mbox{\cal{N}}, x + 1 >
        x$ does not hold anymore.
%%% I suppose that readers know about overflow. Th. 
%  In effect, taking a 32 bits-long
%         machine word, taking $x = 2^{32} = 4294967296$, the binary
%         configuration of $x$ is all 32 bits equal to 1. Then adding 1
%         one more time would give a binary configuration with the
%         33$^{\textnormal{th}}$ bit (the most significant bit) equal to
%         1 and all the 32 others equal to 0. Since the machine word
%         only contain 32 bits, the most significant of our computation
%         is dropped, leaving only 32 bits equal to 0. And this binary
%         configuration is the integer $0$. Hence we get a case where $x
%         + 1 = 0 < x$.

%         \vspace{0.1cm}
%         \begin{center}
%         {\scriptsize
%         \begin{tabular}[h]{|c|c||c|c|c|c|c|c|c|}
%           \hline
%           & & 1 & 1 & 1 & \ldots & 1 & 1 & 1 \\
%           \hline
%           + & & 0 & 0 & 0 & \ldots & 0 & 0 & 1 \\
%           \hline
%           \hline
%           = & {\color{red}1} & 0 & 0 & 0 & \ldots & 0 & 0 & 0 \\
%           \hline
%           = &  & 0 & 0 & 0 & \ldots & 0 & 0 & 0 \\
%           \hline
%         \end{tabular}
%         }
%         \end{center}

        \vspace{0.1cm}
        However, conceptually, except when dealing with boundaries,
        this property holds and we need to achieve further proofs. For
        this reason, assuming that the proof holds is legitimate, if the
        developer is able to guaranty that the integer computations
        never overflow. If he cannot guaranty non-overflow, then this
        is a true problem of specification or design which should be
        re-considered. 
    \end{itemize}

In any case,  we advice the reader to use 
the test tool (or another mean) to comfort the confidence in the
statement of the theorem.

\item {\bf Write an automated proof script}.  \focal\ provides a
  syntax, the \focal\ Proof Language, to split proofs into steps that
  may be proved by the \zenon\ theorem prover. Without entering deeply
  into the syntax further described in chapter \ref{making-proofs},
  the main features are the following. The user may state hypotheses,
  demonstrate subgoals that will serve as lemmas for a higher level
  goal and may give hints about definitions or declarations of
  methods. Then \zenon\ tries to automatically guess a proof of this
  goal, then tries to prove those lemmas, hence building a proof tree
  until the top goal (i.e. the theorem) is proved. Below follows an
  example of such proof.

 {\scriptsize
\begin{lstlisting}
 theorem zero_is_unique : all o in Self,
  (all x in Self, !equal (x, !plus (x, o))) -> !equal (o, !zero)
  proof =
         <1>1 assume o in Self,
              assume H1: all x in Self, !equal (x, !plus (x, o)),
              prove !equal (o, !zero)
             <2>1 prove !equal (!zero, !plus (!zero, o))
                by hypothesis H1
             <2>3 prove !equal (o, !zero)
                by step <2>1
                property zero_is_neutral, equal_transitive, equal_symmetric
             <2>4 qed.
        <1>2 qed.
  ;
\end{lstlisting}
}

  \item {\bf Write a \coq\ script} This way is the most difficult
     since it means to directly write \coq\ code. It 
    requires the understanding of both \coq\ and the mapping the
    \focal\ compiler does to generate \coq\ code from \focal\ source
    code. The section \ref{focal-coq-mapping} describes 
     how \focal\ definitions are mapped onto
    \coq\ names. 

The \coq\ script is introduced by the keywords
    {\tt coq proof} and surrounded by {\tt \{*} and {\tt *\}}. 
% But,
%     anyway, such a way to make proofs require some skills in \coq\ and we try
%     as far as possible to provide, in the \focal\ framework, enough
%     powerful tools to prevent the user to directly deal with
%     \coq\ proofs.
 Below follows an example of such proof.
{\scriptsize
\begin{lstlisting}
theorem int_minus_plus: all x y z in int,
  (* x - y = z -> x = y + z *)
  #base_eq (#int_minus (x, y), z) -> #base_eq (x, #int_plus (y, z))
  proof:
    coq proof {*
      intros x y z; unfold int_plus, int_minus, base_eq,
        syntactic_equal in |- *;
      intros H;
      unfold bi__int_minus;
      apply EQ_base_eq; rewrite <- (Zplus_minus y x);
      apply Zplus_eq_compat; trivial; apply decidable.
      apply Z_eq_dec. assumption.
      Qed.
    *} ;;
\end{lstlisting}
}
\end{itemize}



%%%%%%%%%%%%%%%%%%%%%%%%%%%%%%%%%%%%%%%%%%%%%%%%%%%%%%%%%%%%%%%%%%%%%%%
\subsubsection{Theorems}
Now we know how to write a logical statement and
(nearly $\stackrel{..}{\sim}$) how
to write a proof, the structure of a theorem appears simple since it
contains both the statement and the proof inside the same
construct. The theorem is introduced by the keyword {\tt theorem} and
the proof by the keyword {\tt proof} followed by a colon character
(``:'').
\begin{syntax}
\syntaxclass{Theorems:}
theorem & ::= & \terminal{theorem}\ lident\ \terminal{:}\ logical\_expr
                \ \terminal{proof =}\ proof
\end{syntax}

\noindent For instance:
{\scriptsize
\begin{lstlisting}
species Meet_semi_lattice inherits Setoid =
  ...
  theorem inf_right_substitution_rule : all x y z in Self,
     equal(y, z) -> equal(!inf(x, y), !inf(x, z))
     proof:
       by property
          inf_left_substitution_rule,
          inf_commutes,
          equal_transitive ;
  ...
end ;;
\end{lstlisting}
}
The kind of proof used here is written in \focal\ Proof Language and
must not be a matter of understanding at this point since this
particular point will be addressed with more details in chapter
\ref{making-proofs}.

Notice that theorems can be hosted in a species or can be
toplevel-theorems. 
Unlike theorems, properties cannot appear at
toplevel since there is no way to inherit at toplevel, hence no
way to give a proof after the property definition in a ``parent''.
