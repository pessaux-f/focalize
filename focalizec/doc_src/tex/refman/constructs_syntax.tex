\subsection{Types}
Before dealing with expressions and in general, constructs that allow
to compute, we first examine datatype definitions since to emit a
result an algorithm must manipulate data that have a certain type,
hence must know about their type definitions.

Type definitions allow to build new types or more complex types by
combining previously existing types. They always appear at
``toplevel'', in other words, outside species and collections. Hence
they are a kind of ``global'' definitions that are visible in the
whole compilation unit (and also in other units by using the
{\tt open} directive or by qualifying the type name as described in
\ref{qualified_name}).



%%%%%%%%%%%%%%%%%%%%%%%%%%%%%%%%%%%%%%%%%%%%%%%%%%%%%%%%%%%%%%%%%%%%%%%
\subsubsection{Type expressions}
\index{type!expression}
Type definitions require type expressions to build more complex
datatypes. For this reason we present here the shape of type
expressions.
%\begin{syntax}
%\end{syntax}



%%%%%%%%%%%%%%%%%%%%%%%%%%%%%%%%%%%%%%%%%%%%%%%%%%%%%%%%%%%%%%%%%%%%%%%
\subsubsection{Type definitions}
\index{type!definition}


\subsection{Carrier (type) definition}

\subsection{Expressions}

\subsection{Value and function definitions}

\subsection{Files and uses directives}

\subsection{Theorems and proofs}
