\documentclass{report}
\usepackage[francais]{babel}
%%\usepackage{dropping}
\usepackage[T1]{fontenc}
\usepackage{geometry}
\usepackage{alltt}
\usepackage[latin1]{inputenc}
\usepackage[french]{varioref}
\usepackage{amssymb}
\usepackage{amsmath}
\usepackage{amscd}
\usepackage{color}
\usepackage{listings}


\def\month{August}
\def\year{2009}
\def{\focalversion}{{\sf 0.1.5}}

\def{\focalversiondate}{{\sf {\month} {\year}}}

\newenvironment{citemize}{\begin{list}{$\bullet$}{\itemsep 0pt \topsep 5pt}}{\end{list}}

\def\bbbn{{\rm I\!N}}

\def\url#1{{#1}}
\def\email#1{{\tt#1}}
\def\ocaml{{\sf OCaml}}
\def\moca{{\sf Moca}}
\def\focal{{\sf FoCaLiZe}}
\def\coq{{\sf Coq}}
\def\cime{{\sf CiME}}
\def\zenon{{\sf Zenon}}
\def\zvtov{{\sf zvtov}}
\def\focdoc{{\sf FoCaLiZeDoc}}
\def\dotty{{\sf dotty}}
\def\graphviz{{\sf graphviz}}
\def\emacs{{\sf emacs}}
\def\cygwin{{\sf Cygwin}}
\def\xsltproc{{\sf xsltproc}}
\def\terminal#1{\mbox{\tt\color{red}#1}}
\def\foc{{\sf Foc}}
\def\oldfocal{{\sf Focal}}
\def\focalizec{{\sf focalizec}}

\def\backslash{{\char92}}
\def\chapeau{{\char94}}
\def\vertical{{\char124}}
\def\tilde{{\char126}}

%% Pour afficher les listings de mani�re plus sexy... ;)
\lstdefinelanguage{FoCaLiZe}
  {morekeywords={alias, all, and, as, assume, assumed, begin, by, caml,
      collection, coq, coq_require, definition, else, end,
      ex, external, false, function, hypothesis, if, in,
      inherit, internal, implement, is, let, lexicographic,
      local, logical, match, measure, not, notation, of, open,
      on, or, order, proof, prop, property, prove, qed, rec,
      representation, Self, signature, species, step,
      structural, termination, then, theorem, true, type, use,
      with, conclude},
    otherkeywords={->, :, /\\, \\/, =>, \~, \#, ;, \&, \|},
    sensitive=false,
    morecomment=[n]{(*}{*)},  %% Autoriser les nested comments
    morecomment=[n]{(**}{*)}, %% Autoriser les nested comments
    morestring=[b]",
  }

\lstset{
  language=FoCaLiZe, tabsize=2, frame=single, breaklines=true,
  basicstyle=\ttfamily, framexleftmargin=1mm, xleftmargin=1mm
}

% For syntax definitions
% \begin{syntax}
% left-hand side & ::=  & right-hand side & comment \\
%                & \alt & more right-hand side & more comment
% \end{syntax}
% Use \syntaxclass{Foo} to insert a title line above a syntax definition.
% Do not put \\ before \end{syntax} or \syntaxclass{...}
% Use \begin{syntaxleft} ... \end{syntaxleft} to insert the title
% lines to the left of the definitions.

% \def\syntax{
% \par\medskip\goodbreak\noindent
% \bgroup
% \let\\=\cr
% \interlinepenalty=50            % discourage page breaks in a definition
% \global\let\syntaxclass=\firstsyntaxclass
% \if@twocolumn
% \halign\bgroup~~$##$&\hfil${}##{}$&$##$~\hfil&##\hfil\cr
% \else
% \halign\bgroup\qquad\qquad$##$&\hfil${}##{}$&$##$\quad\hfil&##\hfil\cr
% \fi
% }
% \def\endsyntax{\cr\egroup\egroup\par\medskip\noindent\ignorespaces}

% \def\firstsyntaxclass#1{
% \omit\hbox to 0pt{#1\hss}\cr
% \global\let\syntaxclass=\nextsyntaxclass
% }

% \def\nextsyntaxclass#1{
% \cr\noalign{\smallskip\penalty-100}\omit\hbox to 0pt{#1\hss}\cr
% }

% \def\syntaxleft{
% \par\medskip\goodbreak\noindent
% \bgroup
% \let\\=\cr
% \interlinepenalty=50            % discourage page breaks in a definition
% \global\let\syntaxclass=\firstsyntaxclassleft
% \if@twocolumn
% \halign\bgroup\hfil$##$&\hfil${}##{}$&$##$~\hfil&##\hfil\cr
% \else
% \halign\bgroup\hfil$##$&\hfil${}##{}$&$##$\quad\hfil&##\hfil\cr
% \fi
% }
% \let\endsyntaxleft=\endsyntax

% \def\firstsyntaxclassleft#1{
% $\hfilneg#1\quad\hfil$
% \global\let\syntaxclass=\nextsyntaxclassleft
% }

% \def\nextsyntaxclassleft#1{
% \cr\noalign{\smallskip\goodbreak}$\hfilneg#1\quad\hfil$
% }


%% Pour afficher les listings de mani�re plus sexy... ;)
\lstdefinelanguage{Focal}
  {morekeywords={property, let, signature, species, is, in, inherits,
      all, exists, rep, if, then, else, proof, of, by, definition,
      collection, implements},
    sensitive=false,
    morecomment=[n]{(*}{*)},  %% Autoriser les nested comments
    morecomment=[n]{(**}{*)},  %% Autoriser les nested comments
    morestring=[b]",
  }

\lstset{
  language=Focal, tabsize=2, frame=single, breaklines=true,
  basicstyle=\ttfamily, framexleftmargin=1mm, xleftmargin=1mm
}


\geometry{a4paper,twoside,body={5.5in,9in}}

\begin{document}

\chapter{Introduction}

\section{Motivations}
 The \focal\ project was launched in 1998 by T. Hardin and R. Rioboo
\cite{HardinRiobooTSI04} with the 
objective of helping all stages of development of critical software
within safety and security framework, at least when formal methods are
required or chosen. The idea was to elaborate a development
environment able to provide high-level and justified confidence to
users. One the other hand, this system had to remain easy to use by
well-trained engineers.

Currently, \focal\ can be seen as still a prototype of an Integrated
Development Environment (IDE), for a language providing high level
mechanisms such as inheritance, late binding, redefinition,
parametrization, etc.  Confidence in proofs submitted by developers
relies on formal proof verification.

This support language was formally described and designed to provide
means for formal specification by declaring names and
properties. Then, design and implementation can incrementally be done
by adding definitions of functions and proving that the implementation
meets the specification or design requirements. Thus, developing in
\focal\ is a kind of refinement process from  formal model to design
and code, completely done within \focal. Taking the global development
in consideration within a same environment brings some conciseness,
helps documentation and reviewing.

A \focal\ development is organised as a hierarchy that may have
several roots. The upper levels of the hierarchy are built along the
specification stage while the lower ones correspond to
implementation. Each node of the hierarchy corresponds to a progress
toward a complete implementation.  We call here {\em refinement} the
process of building a top-down hierarchy.


\section{Presentation and requirements}
This manual describes the main constructions of the \focal\ language,
as well as the main tools given by the \focal\ environment. Basic
expressions of the \focal\ language are quite similar to the core
expressions of \ocaml\
({\tt http://caml.inria.fr}).A prior knowledge of some functional
languages might be helpful to better understand the \focal's core
language concepts. Last a good knowledge of \coq\ is needed if you
intend to manuall write proofs in \focal\. However, this task may be
helped by the \zenon\ automated theorem prover, as long as the user
write his proofs using the \zenon\ Proof Language embedded in the
\focal\ language.



\section{Output}
As we will see further, a \focal\ development contains both
``computational code'' (i.e. code performing operations that lead to
an effect, a result) and logical properties. When compiled, two
outputs are generated:
\begin{itemize}
  \item The ``computational code'' is compiled into \ocaml\ source
    that can then be compiled with the \ocaml\ compiler to lead to an
    executable binary. In this pass, logical properties are discarded
    since they do not lead to executable code.
  \item Both the ``computational code'' and the logical properties are
    compiled into a \coq\ model. This model can then be sent to the
    \coq\ proof assistant who will verify the consistency of both the
    ``computational code'' and the logical properties (whose proofs
    must be obviously provided) of the \focal\ development. This means
    that the \coq\ code generated is not intended to be used to
    generate an \ocaml\ source code by automated extraction. As stated
    above, the executable generation is prefered using directly the
    generated \ocaml\ code. In this idea, \coq\ acts as an assessor of
    the development instead of a code generator.
\end{itemize}



\section{Basic concepts}
As stated in the introduction, \focal\ language is designed to build
an application step by step, going from very abstract specifications
to the concrete implementation through a hierarchy of structures which
are quite similar to \emph{classes} in an Object-Oriented context.
However, as we will insist later, \focal\ is not Object-Oriented as
C++, Java are. It only owns features ``smelling objects''.

We will now present the basic concepts underlying a
\focal\ development, that is:
\begin{itemize}
  \item Species
  \item Collections
  \item Inheritance
  \item Late-binding
  \item Parameterisation
\end{itemize}

\subsection{Species}
{\bf Species} are the nodes of the \focal\ hierarchy. A species can be
roughly seen as a list of {\bf methods} or {\bf fields}. Hence, a
basic species looks like:
{\scriptsize
\begin{lstlisting}
  species Name =
    ... ;
    ... ;
  end ;;
\end{lstlisting}
}

Species names are always capitalized. Inside the species' body methods
are enumerated. These methods represent the internal datatype of the
species, functions to manipulate this datatype and properties that
must hold in the species.

A species can hence be seen as an abstract data type. On another hand,
it can be seen as a kind of object, with its internal state (private
variables) and its methods.

There are several kinds of methods:
\begin{itemize}
  \item The {\bf carrier}. This is a type definition that defines the
    type of the values embedded in the species. When ``instanciated'',
    a species will lead to \underline{a value having this type}. This
    especially means that an instanciation of a species is NOT, like
    in Object-Oriented languages an entity embedding data and
    code. The instanciation of a species will be a
    {\underline value}. The carrier is named {\tt rep}. For instance,
    we can define a species that implements couples of native integers
    data structure as:
    {\scriptsize
      \begin{lstlisting}
        species IntCouple =
          rep = (int * int) ;
        end ;;
      \end{lstlisting}
    }
    Each ``instance'' of this species will encapsulate \underline{1}
    value of type ``couple of integers''. This especially means that
    the species doesn't represent a ``set'' of couples but
    \underline{1} value that has type {\tt (int * int)}.

    The carrier may remain not specified in a species. However, in
    order to finally get a fully defined species that will be
    instanciated, the carrier must be defined \underline{once} at one
    moment (during inheritance).

    In the context of a species, the type denoted by the carrier is
    named {\tt Self}.

  \item {\bf Signatures}. They introduce the ``prototype'' of a
    function, that is provide their type. As a first introduction,
    types expressions are closely like ML-like type expressions.
    A signature specifies the name of the operation then its type. For
    instance:
    {\scriptsize
      \begin{lstlisting}
        species IntStack =
          signature push : int -> Self -> Self ;
        end ;;
      \end{lstlisting}
    }
    As we saw above, {\tt Self} represents the underlying carrier type
    of the current species. Hence an operation pushing an integer onto
    a stack will take as parameter the integer to push, the stack on
    which to push and give back a new configuration of stack (where
    the pushed element is on the top), that is of type {\tt Self}. 

  \item {\bf Functions}. They are operations that are allowed on the
    carrier's elements. They are implementations of signatures,
    providing effective code to execute. A function is introduced by
    the {\tt let} keyword. Recursive functions are introduced by
    {\tt let rec} to make explicit the recursivity.
    {\scriptsize
      \begin{lstlisting}
        species IntStack =
          rep = int list ;
          let push (v in int, s in Self) = v :: s ;
        end ;;
      \end{lstlisting}
    }

  \item {\bf Properties}. They are first order logic propositions
    describing requierements that functions must meet. When stating a
    property, the proof that it holds is not yet provided. However it
    will have to finally to get a species that can be
    instanciated.
    {\scriptsize
      \begin{lstlisting}
        species IntStack =
          ...
          property push\_returns\_non\_empty :
            all v in int, all s in Self, push (v, s) -> ~ is\_empty (s) ;
        end ;;
      \end{lstlisting}
    }

    Proofs of properties can be done afterwards using a {\tt proof}
    field in a species.The way to give proofs will be seen further.
    {\scriptsize
      \begin{lstlisting}
        species IntStack2 inherits IntStack =
          proof of push\_returns\_non\_empty = ... ;            
        end ;;
      \end{lstlisting}
    }

  \item {\bf Theorems}. They are properties with their proofs. In
    fact, when defining a property, we only give the statement of a
    theorem, leaving its proof for later.
    {\scriptsize
      \begin{lstlisting}
        species IntStack =
          ...
          theorem push\_returns\_non\_empty :
            all v in int, all s in Self, push (v, s) -> ~ is\_empty
            (s)
          proof : ... ;
        end ;;
      \end{lstlisting}
    }

\end{itemize}

\subsection{Collections}
\subsection{Inheritance}
\subsection{Late-binding}
\subsection{Parameterisation}

\end{document}
