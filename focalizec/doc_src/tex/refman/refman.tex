% $Id: refman.tex,v 1.11 2008-10-17 15:11:44 pessaux Exp $

\documentclass{report}
\usepackage[francais]{babel}
\usepackage{times}
\usepackage[latin1]{inputenc}
\usepackage{color}
\usepackage{listings}
\usepackage{index}


\def\month{August}
\def\year{2009}
\def{\focalversion}{{\sf 0.1.5}}

\def{\focalversiondate}{{\sf {\month} {\year}}}

\newenvironment{citemize}{\begin{list}{$\bullet$}{\itemsep 0pt \topsep 5pt}}{\end{list}}

\def\bbbn{{\rm I\!N}}

\def\url#1{{#1}}
\def\email#1{{\tt#1}}
\def\ocaml{{\sf OCaml}}
\def\moca{{\sf Moca}}
\def\focal{{\sf FoCaLiZe}}
\def\coq{{\sf Coq}}
\def\cime{{\sf CiME}}
\def\zenon{{\sf Zenon}}
\def\zvtov{{\sf zvtov}}
\def\focdoc{{\sf FoCaLiZeDoc}}
\def\dotty{{\sf dotty}}
\def\graphviz{{\sf graphviz}}
\def\emacs{{\sf emacs}}
\def\cygwin{{\sf Cygwin}}
\def\xsltproc{{\sf xsltproc}}
\def\terminal#1{\mbox{\tt\color{red}#1}}
\def\foc{{\sf Foc}}
\def\oldfocal{{\sf Focal}}
\def\focalizec{{\sf focalizec}}

\def\backslash{{\char92}}
\def\chapeau{{\char94}}
\def\vertical{{\char124}}
\def\tilde{{\char126}}

%% Pour afficher les listings de mani�re plus sexy... ;)
\lstdefinelanguage{FoCaLiZe}
  {morekeywords={alias, all, and, as, assume, assumed, begin, by, caml,
      collection, coq, coq_require, definition, else, end,
      ex, external, false, function, hypothesis, if, in,
      inherit, internal, implement, is, let, lexicographic,
      local, logical, match, measure, not, notation, of, open,
      on, or, order, proof, prop, property, prove, qed, rec,
      representation, Self, signature, species, step,
      structural, termination, then, theorem, true, type, use,
      with, conclude},
    otherkeywords={->, :, /\\, \\/, =>, \~, \#, ;, \&, \|},
    sensitive=false,
    morecomment=[n]{(*}{*)},  %% Autoriser les nested comments
    morecomment=[n]{(**}{*)}, %% Autoriser les nested comments
    morestring=[b]",
  }

\lstset{
  language=FoCaLiZe, tabsize=2, frame=single, breaklines=true,
  basicstyle=\ttfamily, framexleftmargin=1mm, xleftmargin=1mm
}

% For syntax definitions
% \begin{syntax}
% left-hand side & ::=  & right-hand side & comment \\
%                & \alt & more right-hand side & more comment
% \end{syntax}
% Use \syntaxclass{Foo} to insert a title line above a syntax definition.
% Do not put \\ before \end{syntax} or \syntaxclass{...}
% Use \begin{syntaxleft} ... \end{syntaxleft} to insert the title
% lines to the left of the definitions.

% \def\syntax{
% \par\medskip\goodbreak\noindent
% \bgroup
% \let\\=\cr
% \interlinepenalty=50            % discourage page breaks in a definition
% \global\let\syntaxclass=\firstsyntaxclass
% \if@twocolumn
% \halign\bgroup~~$##$&\hfil${}##{}$&$##$~\hfil&##\hfil\cr
% \else
% \halign\bgroup\qquad\qquad$##$&\hfil${}##{}$&$##$\quad\hfil&##\hfil\cr
% \fi
% }
% \def\endsyntax{\cr\egroup\egroup\par\medskip\noindent\ignorespaces}

% \def\firstsyntaxclass#1{
% \omit\hbox to 0pt{#1\hss}\cr
% \global\let\syntaxclass=\nextsyntaxclass
% }

% \def\nextsyntaxclass#1{
% \cr\noalign{\smallskip\penalty-100}\omit\hbox to 0pt{#1\hss}\cr
% }

% \def\syntaxleft{
% \par\medskip\goodbreak\noindent
% \bgroup
% \let\\=\cr
% \interlinepenalty=50            % discourage page breaks in a definition
% \global\let\syntaxclass=\firstsyntaxclassleft
% \if@twocolumn
% \halign\bgroup\hfil$##$&\hfil${}##{}$&$##$~\hfil&##\hfil\cr
% \else
% \halign\bgroup\hfil$##$&\hfil${}##{}$&$##$\quad\hfil&##\hfil\cr
% \fi
% }
% \let\endsyntaxleft=\endsyntax

% \def\firstsyntaxclassleft#1{
% $\hfilneg#1\quad\hfil$
% \global\let\syntaxclass=\nextsyntaxclassleft
% }

% \def\nextsyntaxclassleft#1{
% \cr\noalign{\smallskip\goodbreak}$\hfilneg#1\quad\hfil$
% }


%% Pour afficher les listings de mani�re plus sexy... ;)
\lstdefinelanguage{Focal}
  {morekeywords={property, let, signature, species, is, in, inherits,
      all, exists, rep, if, then, else, proof, of, by, definition,
      collection, implements, Self, match, with, end},
    sensitive=false,
    morecomment=[n]{(*}{*)},  %% Autoriser les nested comments
    morecomment=[n]{(**}{*)},  %% Autoriser les nested comments
    morestring=[b]",
  }

\lstset{
  language=Focal, tabsize=2, frame=single, breaklines=true,
  basicstyle=\ttfamily, framexleftmargin=1mm, xleftmargin=1mm
}

\makeindex

\begin{document}
%%%%%%%%%%%%%%%%%%%%%%%%%%%%%%%%%%%%%%%%%%%%%%%%%%%%%%%%%%%%%%%%%%%%%%%
%%%%%%%%%%%%%%%%%%%%%%%%%%%%%%%%%%%%%%%%%%%%%%%%%%%%%%%%%%%%%%%%%%%%%%%
%%%%%%%%%%%%%%%%%%%%%%%%%%%%%%%%%%%%%%%%%%%%%%%%%%%%%%%%%%%%%%%%%%%%%%%
%%%%%%%%%%%%%%%%%%%%%%%%%%%%%%%%%%%%%%%%%%%%%%%%%%%%%%%%%%%%%%%%%%%%%%%
\chapter{Introduction}


%%%%%%%%%%%%%%%%%%%%%%%%%%%%%%%%%%%%%%%%%%%%%%%%%%%%%%%%%%%%%%%%%%%%%%%
%%%%%%%%%%%%%%%%%%%%%%%%%%%%%%%%%%%%%%%%%%%%%%%%%%%%%%%%%%%%%%%%%%%%%%%
%%%%%%%%%%%%%%%%%%%%%%%%%%%%%%%%%%%%%%%%%%%%%%%%%%%%%%%%%%%%%%%%%%%%%%%
\section{Motivations}
 The \focal\ project was launched in 1998 by T. Hardin and R. Rioboo
\cite{HardinRiobooTSI04} with the 
objective of helping all stages of development of critical software
within safety and security framework, at least when formal methods are
required or chosen. The idea was to elaborate a development
environment able to provide high-level and justified confidence to
users. One the other hand, this system had to remain easy to use by
well-trained engineers.

Currently, \focal\ can be seen as still a prototype of an Integrated
Development Environment (IDE), for a language providing high level
mechanisms such as inheritance, late binding, redefinition,
parametrization, etc.  Confidence in proofs submitted by developers
relies on formal proof verification.

This support language was formally described and designed to provide
means for formal specification by declaring names and
properties. Then, design and implementation can incrementally be done
by adding definitions of functions and proving that the implementation
meets the specification or design requirements. Thus, developing in
\focal\ is a kind of refinement process from  formal model to design
and code, completely done within \focal. Taking the global development
in consideration within a same environment brings some conciseness,
helps documentation and reviewing.

A \focal\ development is organised as a hierarchy that may have
several roots. The upper levels of the hierarchy are built along the
specification stage while the lower ones correspond to
implementation. Each node of the hierarchy corresponds to a progress
toward a complete implementation.  We call here {\em refinement} the
process of building a top-down hierarchy.



%%%%%%%%%%%%%%%%%%%%%%%%%%%%%%%%%%%%%%%%%%%%%%%%%%%%%%%%%%%%%%%%%%%%%%%
%%%%%%%%%%%%%%%%%%%%%%%%%%%%%%%%%%%%%%%%%%%%%%%%%%%%%%%%%%%%%%%%%%%%%%%
%%%%%%%%%%%%%%%%%%%%%%%%%%%%%%%%%%%%%%%%%%%%%%%%%%%%%%%%%%%%%%%%%%%%%%%
\section{Presentation and requirements}
This manual describes the main constructions of the \focal\ language,
as well as the main tools given by the \focal\ environment. Basic
expressions of the \focal\ language are quite similar to the core
expressions of \ocaml\
({\tt http://caml.inria.fr}).A prior knowledge of some functional
languages might be helpful to better understand the \focal's core
language concepts. Last a good knowledge of \coq\ is needed if you
intend to manually write proofs in \focal\. However, this task may be
helped by the \zenon\ automated theorem prover, as long as the user
write his proofs using the \zenon\ Proof Language embedded in the
\focal\ language.



%%%%%%%%%%%%%%%%%%%%%%%%%%%%%%%%%%%%%%%%%%%%%%%%%%%%%%%%%%%%%%%%%%%%%%%
%%%%%%%%%%%%%%%%%%%%%%%%%%%%%%%%%%%%%%%%%%%%%%%%%%%%%%%%%%%%%%%%%%%%%%%
%%%%%%%%%%%%%%%%%%%%%%%%%%%%%%%%%%%%%%%%%%%%%%%%%%%%%%%%%%%%%%%%%%%%%%%
\section{Output}
As we will see further, a \focal\ development contains both
``computational code'' (i.e. code performing operations that lead to
an effect, a result) and logical properties.

\smallskip
When compiled, two outputs are generated:
\begin{itemize}
  \item The ``computational code'' is compiled into \ocaml\ source
    that can then be compiled with the \ocaml\ compiler to lead to an
    executable binary. In this pass, logical properties are discarded
    since they do not lead to executable code.
  \item Both the ``computational code'' and the logical properties are
    compiled into a \coq\ model. This model can then be sent to the
    \coq\ proof assistant who will verify the consistency of both the
    ``computational code'' and the logical properties (whose
    proofs\index{proof} must be obviously provided) of the
    \focal\ development. This means that the \coq\ code generated is
    not intended to be used to generate an \ocaml\ source code by
    automated extraction. As stated above, the executable generation
    is preferred using directly the generated \ocaml\ code. In this
    idea, \coq\ acts as an assessor of the development instead of a
    code generator.
\end{itemize}



%%%%%%%%%%%%%%%%%%%%%%%%%%%%%%%%%%%%%%%%%%%%%%%%%%%%%%%%%%%%%%%%%%%%%%%
%%%%%%%%%%%%%%%%%%%%%%%%%%%%%%%%%%%%%%%%%%%%%%%%%%%%%%%%%%%%%%%%%%%%%%%
%%%%%%%%%%%%%%%%%%%%%%%%%%%%%%%%%%%%%%%%%%%%%%%%%%%%%%%%%%%%%%%%%%%%%%%
\section{Basic concepts}
As stated in the introduction, \focal\ language is designed to build
an application step by step, going from very abstract specifications
to the concrete implementation through a hierarchy of structures which
are quite similar to \emph{classes} in an Object-Oriented context.
However, as we will insist later, \focal\ is not Object-Oriented as
C++, Java are. It only owns features ``smelling objects''.

\smallskip
We will now present the basic concepts underlying a
\focal\ development, that is:
\begin{itemize}
  \item Species
  \item Collections
  \item Parametrisation
  \item Inheritance
  \item Late-binding
\end{itemize}



%%%%%%%%%%%%%%%%%%%%%%%%%%%%%%%%%%%%%%%%%%%%%%%%%%%%%%%%%%%%%%%%%%%%%%%
%%%%%%%%%%%%%%%%%%%%%%%%%%%%%%%%%%%%%%%%%%%%%%%%%%%%%%%%%%%%%%%%%%%%%%%
\subsection{Species}
{\bf Species} \index{species} are the nodes of the
\focal\ hierarchy. A species can be roughly seen as a list of
{\bf methods} \index{method} or {\bf fields}\index{field}. Hence, a
basic species looks like:
{\scriptsize
\begin{lstlisting}
  species Name =
    ... ;
    ... ;
  end ;;
\end{lstlisting}
}

Species names are always {\bf capitalized}. Inside the species' body
methods are enumerated. These methods represent the internal datatype
of the species, functions\index{function} to manipulate this datatype
and properties\index{property} that must hold in the species.

\smallskip
\label{no-polymorphism-for-methods}
One important restriction on the type of the methods is that it cannot
be polymorphic\index{polymorphism}. However, \focal\ provides another
mechanism to circumvent this restriction, the
parametrisation\index{parametrisation} as explained further
(c.f. \ref{parametrisation}).

\smallskip
A species can hence be seen as an abstract data type. On another hand,
it can be seen as a kind of object, with its internal state (private
variables) and its methods. There are several kinds of methods:
\smallskip
\begin{itemize}
  \medskip
  \label{rep-is-method}
  \item The {\bf carrier}\index{carrier}. This is a type definition
    that defines\index{carrier!defined} the type of the values
    embedded in the species. When ``instantiated'', a species will
    lead to {\bf a value having this type}. This especially means that
    an instantiation of a species is {\bf NOT}, like in
    Object-Oriented languages an entity embedding data and code. The
    instantiation of a species will be a {\bf value}. The carrier is
    named {\tt rep}\index{rep}. For instance, we can define a species
    that implements couples of native integers data structure as:
    {\scriptsize
      \begin{lstlisting}
species IntCouple =
  rep = (int * int) ;
end ;;
      \end{lstlisting}
    }
    Each ``instance'' of this species will encapsulate {\bf 1}
    value of type ``couple of integers''. This especially means that
    the species doesn't represent a ``set'' of couples but
    {\bf 1} value that has type {\tt (int * int)}.

    \smallskip
    The carrier may remain not {\bf defined} in a species. However, in
    order to finally get a fully defined (or {\em complete}) species
    that will be instantiated, the carrier must be defined
    {\bf once} at one moment (during inheritance as described later in
    section \ref{inheritance}). \index{carrier!declared}However, the
    carrier is {\bf always implicitly declared}. This means that any
    species has a carrier that will remain declared until it is {\bf
    defined} ! 

    \smallskip
    In the context of a species, the type denoted by the carrier is
    named {\tt Self}.

  \medskip
  \item {\bf Signatures}\index{signature}. They introduce the
    ``prototype'' of a function, that is provide their type. As a
    first introduction, types expressions are closely like ML-like
    type expressions. A signature can be viewed as a declaration. It
    specifies the name of the operation then its type. For instance:
    {\scriptsize
      \begin{lstlisting}
 species IntStack =
   signature push : int -> Self -> Self ;
 end ;;
      \end{lstlisting}
    }
    As we saw above, {\tt Self} represents the underlying carrier type
    of the current species. Hence an operation pushing an integer onto
    a stack will take as parameter the integer to push, the stack on
    which to push and give back a new configuration of stack (where
    the pushed element is on the top), that is of type {\tt Self}. 

  \medskip
  \item {\bf Functions}\index{function}. They are operations that are
    allowed on the carrier's elements. They are implementations of
    signatures, providing effective code to execute. A function is
    introduced by the {\tt let} keyword. Recursive functions are
    introduced by {\tt let rec} to make explicit the recursivity. A
    function can be viewed as a definition.
    {\scriptsize
      \begin{lstlisting}
species IntStack =
  rep = int list ;
  let push (v in int, s in Self) = v :: s ;
end ;;
      \end{lstlisting}
    }
    Functions can use in their body other methods of the species,
    toplevel function definitions, methods of collections
    (described further in \ref{collection}), or methods of
    collections parameters (see \ref{collection-parameter}).

    \label{idea-fun-using-sig}
    When we say ``other methods of the species'', this includes
    signatures. This means that it is possible to use something only
    declared, without yet effective implementation. We will address
    this point later in detail in section \label{late-binding}.

    \smallskip
    Although \focal\ is a functional language, function application
    must always be total. This means that any function call must be
    provided all the effective arguments of the function. It is still
    possible to have partial application by eta-expansion. As
    described further in the core syntax (c.f \ref{core-syntax}),
    function application is ``� la C'', that is with arguments comma
    separated and enclosed by parentheses.

  \medskip
  \item {\bf Properties}\index{property}. They are first order logic
    propositions describing requirements that functions must
    meet. When stating a property, the proof that it holds is not yet
    provided. However it will have to finally to get a species that
    can be instantiated. A property can be viewed as a declaration.
    {\scriptsize
      \begin{lstlisting}
species IntStack =
  ...
  property push_returns_non_empty :
    all v in int, all s in Self, push (v, s) -> ~ is_empty (s) ;
end ;;
      \end{lstlisting}
    }

    Proofs\index{proof!delayed} of properties can be done afterwards
    using a {\tt proof} field in a species.The way to give proofs will
    be seen further.
    {\scriptsize
      \begin{lstlisting}
species IntStack2 inherits IntStack =
  proof of push_returns_non_empty = ... ;            
end ;;
      \end{lstlisting}
    }

  \medskip
  \item {\bf Theorems}\index{theorem}. They are properties with their
    proofs. In fact, when defining a property, we only give the
    statement of a theorem, leaving its proof for later. A theorem can
    be viewed as a definition.
    {\scriptsize
      \begin{lstlisting}
species IntStack =
  ...
  theorem push_returns_non_empty :
    all v in int, all s in Self, push (v, s) -> ~ is_empty
    (s)
  proof : ... ;
end ;;
      \end{lstlisting}
    }

\end{itemize}



%%%%%%%%%%%%%%%%%%%%%%%%%%%%%%%%%%%%%%%%%%%%%%%%%%%%%%%%%%%%%%%%%%%%%%%
%%%%%%%%%%%%%%%%%%%%%%%%%%%%%%%%%%%%%%%%%%%%%%%%%%%%%%%%%%%%%%%%%%%%%%%
\subsection{Collections}
\label{collection}
\index{collection}
A {\bf collection} is a kind of ``instance'' of a {\em complete}
\index{species!complete} species. A species is said complete if all
its methods are {\em defined}, i.e. have an implementation. In other
words this means that there is no more methods only {\em
  declared}. This notion implies that:
\begin{itemize}
  \item The carrier is defined by providing a type definition for
    {\tt rep}.
  \item For signature an effective function exists.
  \item For each property, a proof is given.
\end{itemize}

\smallskip
Obviously, it is possible to build a species without signatures and
properties, only providing functions and theorems directly. In this
case, if the carrier is also defined, then the obtained species is
trivially complete.

The important point for a species to be complete is that it can be
turned into effective executable code. In effect, since it is
complete, all its components are known then can be translated into a
target code.

The process of getting this effective executable code is by making a
collection that {\em implements} the species. As a result, one get an
entity whose internal representation has the type of the carrier of
the {\em implemented} species and that can be manipulated by the
methods present in this {\em implemented} species. However, the
manipulations will be done via the collection and not anymore via the
species.

{\scriptsize
\begin{lstlisting}
species Full =
  rep = int ;
  let create_random in Self = random_foc#random_int (42) ;
  let double (x in Self) = x + x ;
  let print (x in Self) = print_int (x) ;
end ;;

collection MyFull_Instance implements Full ;;

let v = Full.create_random ;;
Full.print (v) ;;
let dv = Full.double () ;;
Full.print (dv) ;;
\end{lstlisting}
}

In the above example, we defined a complete species {\tt Full}. Then
we ``take an instance'' of such a species by creating the collection
{\tt MyFull\_Instance}. We can then use methods of this collection on
values of this collection.

\smallskip
A very important point is that when creating a collection, the
representation of the carrier contained in the ``implemented'' species
gets opaque. In other words, the collection becomes an abstract
datatype. This especially means that it will be impossible to
manipulate values of the carrier without the collection's
methods. Moreover, two collections created from a same species will
not be type-compatible since their carriers got abstracted making now
impossible to ensure a type equivalence.

\smallskip
As a conclusion, collections are the only way to get something that
can be executed since they are the terminal items of a
\focal\ development hierarchy. Since they are ``terminal'', this also
means that no method can be added to a collection. Moreover, a
collection may not be used to create a new species by inheritance (as
explained in the next section).



%%%%%%%%%%%%%%%%%%%%%%%%%%%%%%%%%%%%%%%%%%%%%%%%%%%%%%%%%%%%%%%%%%%%%%%
%%%%%%%%%%%%%%%%%%%%%%%%%%%%%%%%%%%%%%%%%%%%%%%%%%%%%%%%%%%%%%%%%%%%%%%
\subsection{Interfaces}
\label{interface}
\index{interface}
The {\bf interface} of a species is the list of the declarations of
its methods. It corresponds to the end-user point of view, who wants
to know which functions he can use, and which properties these
functions have, but doesn't care about the details of the
implementation.

\smallskip
To get the interface of a species, we keep the methods only declared
(as previously stated, signatures and properties) and we forget the
bodies of the defined methods (carrier, functions and theorems).
Discarding the body of the carrier means we consider it was not given
at all, for a function we only keep its type and for a theorem we only
keep its statement. Hence, getting the interface of a species can
roughly be seen as erasing the carrier, turning the functions into
signatures and the theorems into properties.

\smallskip
While this abstraction is easy within programming languages, it is not
always possible when dealing with proofs and properties. Such
problematic species are rejected by \focal\ and will be described
later in \ref{species-constraints}.



%%%%%%%%%%%%%%%%%%%%%%%%%%%%%%%%%%%%%%%%%%%%%%%%%%%%%%%%%%%%%%%%%%%%%%%
%%%%%%%%%%%%%%%%%%%%%%%%%%%%%%%%%%%%%%%%%%%%%%%%%%%%%%%%%%%%%%%%%%%%%%%
%%%%%%%%%%%%%%%%%%%%%%%%%%%%%%%%%%%%%%%%%%%%%%%%%%%%%%%%%%%%%%%%%%%%%%%
\section{Building species}
In this chapter, we describe the relations between species to build
incrementally more complex species from previously existing ones. Two
mechanisms are available for this purpose: parametrisation and
inheritance.



%%%%%%%%%%%%%%%%%%%%%%%%%%%%%%%%%%%%%%%%%%%%%%%%%%%%%%%%%%%%%%%%%%%%%%%
%%%%%%%%%%%%%%%%%%%%%%%%%%%%%%%%%%%%%%%%%%%%%%%%%%%%%%%%%%%%%%%%%%%%%%%
\subsection{Parametrisation}
\index{parametrisation}
\label{parametrisation}


%%%%%%%%%%%%%%%%%%%%%%%%%%%%%%%%%%%%%%%%%%%%%%%%%%%%%%%%%%%%%%%%%%%%%%%
\subsubsection{Collection parameters}
\index{parameter!collection}
\index{collection!parameter}
\label{collection-parameter}
Remember that methods cannot be polymorphic\index{polymorphism}
(c.f. \ref{no-polymorphism-for-methods}). So it seems difficult to
create a species whose methods work on a carrier having a type
structure in which we would like to have some parts unconstrained. For
example, who to implement the well-known polymorphic type of the
lists ? A list is a structure grouping elements but independently of
the type of these elements. The only constraint is that all elements
have the same type. Hence, a ML-like representation of lists would be
like:
{\scriptsize
\lstset{language=Caml}
\begin{lstlisting}
type 'a list =
  | Nil
  | Cons of ('a * 'a list)
\end{lstlisting}
}

The {\tt 'a} is then a parameter of the type and is polymorphic.
In \focal\ we would like to create a species looking like:
{\scriptsize
\begin{lstlisting}
species List =
  signature nil : Self ;
  signature cons : 'a -> Self -> Self ;
end ;;
\end{lstlisting}
}

Instead of abstracting the type parameter and leaving it free in the
context of the species, in \focal\ we {\em parametrise} the species
by another species:
{\scriptsize
\begin{lstlisting}
species List (Elem is Basic_object) =
  signature nil : Self ;
  signature cons : Elem -> Self -> Self ;
end ;;
\end{lstlisting}
}

The {\tt Elem} is called a {\bf collection parameter} and is expected
to be a species having at least the methods of a species called
{\tt Basic\_object}\footnote{{\tt Basic\_object} is a basic and poor
species from the standard library, containing only few methods.} with
the same types. Collection parameters are introduced by the {\tt is}
keyword. When a parametrised species will be used, \focal\ expects it
to be applied to effective collection parameters having
{\em compatible}\index{interface!compatibility}
species interfaces (c.f \ref{interface}). In other words the effective
parameter will have to present an interface with at least the
functions of {\tt Basic\_object} and each method will have to have the
same type than the corresponding one in {\tt Basic\_object}.

\smallskip
In the example, we use this parameter in order to build the signature
of our method {\tt cons}. You may note that species names can be used
in type expressions. They denote the carrier abstracted of the species
whose name is used. By ``abstracted'', it is meant that the
representation of this carrier is not visible, but we can refer to it
as an abstract datatype. In other words, {\tt Elem -> Self -> Self}
stands for the type of a function:
\begin{itemize}
  \item taking a value whose type is the carrier type of a species
    having a compatible interface with the species
    {\tt Basic\_object}. (This especially means that such a value will
    have be created using methods of the compatible species),
  \item taking a value whose type is the carrier type of the current
    species,
  \item and returning a value whose type is the carrier type of the
    current species.
\end{itemize}

\smallskip
Although the parameters looks like a species, why is it called a
{\bf collection parameter} ? The answer to this question is especially
important to understand the programming model in \focal. It is called
a collection because finally, at the terminal nodes of the
development, this parameter will have to be instantiated by a
collection, that is an entity where everything is defined ! Imagine
how to build a code we could execute if a parameter was instantiated
with an entity with methods only declared\ldots Moreover, the carrier
of a collection parameter is abstract for the hosting, exactly like
the carrier of a collection is (c.f \ref{collection}).

So, even if the collection parameter ``{\tt is}'' a species name,
{\bf it will not be a species}. It will be {\bf a collection having an
interface compatible} with the interface of the species. Hence,
declaring a collection parameter for a parametrised species means
providing two things: the name of the parameter and the signature that
the instantiation of this parameter must satisfy.

\smallskip
It is important to note that we deal with dependent types
\index{type!dependent} here, and
therefore that the order of the parameters is important. To define the
type of a parameter, one can use the preceding parameters. For
instance, if we assume that there exists a parametrised species
{\tt List} which declares the basic operations over lists, one can
specify a new species working on couples of value and lists of values
like:
{\scriptsize
\begin{lstlisting}
species MyCouple (E is Basic_object, L is List (E)) =
  rep = (E * L) ;
  ... ;
end ;;
\end{lstlisting}
}

The carrier of this species will represent the type
{\tt ('a * ('a list))}. This means that the type of the values in the
first component of the couple is the same than the type of the
elements of the list in the second component of the couple.

\smallskip
If a collection parameter implements a parametrized species (like in
the example for the species {\tt List}), it must provide an
instantiation for {\bf all} its parameters. Once again, the preceding
parameters can be used to achieve this purpose (like we did to create
our parameter {\tt L}, instantiating the {\tt List}'s parameter by our
first collection parameter {\tt E}).

\smallskip
\label{method-qualification}
\index{method!qualification}
At the beginning of the presentation of collection parameters, we used
the parameter to build the carrier of our species. But obviously,
collection parameters can also be used via their other methods,
i.e. signatures, functions, properties and theorems. We want to create
a generic couple species. It will then have two collection parameters,
one for each component of the couple. If we want to have a printing
(i.e. returning a string, not making side effect in our example)
method, we will require that each collection parameter has one. Hence
our printing method will only have to add parentheses and comma
around and between what is printed by each parameter's printing
routine. Hence we now need to know how to call a collection parameter
method. The syntax used is qualifying the method name by the
parameter's name, separating them by the ``!''\index{bang character}
character.
{\scriptsize
\begin{lstlisting}
(* Minimal species requirement : having a print routine. *)
species Base_obj =
  signature print : Self -> string ;
end ;;

species Couple (C1 is Base_obj, c2 is Base_obj) =
  rep = (C1 * C2) ;
  let print (c in Self) =
    match (c) with
     | (component1, component2) ->
       "(" ^ C1!print (component1) ^
       ", " ^
       C2!print (component2) ^")" ;
end ;;
\end{lstlisting}
}

Hence, {\tt C1!print (component1)} means ``call the collection
{\tt C1}'s method {\tt print} with the argument {\tt component1}. The
qualification mechanism using ``!'' is general and can be used to
denote the method of any available species/collection, even those of
ourselves (i.e. {\tt Self}). Hence, in a species instead of calling:
{\scriptsize
\begin{lstlisting}
species Foo ... =
  let m1 (...) = ... ;
  let m2 (...) = if ... then ... else m1 (...) ;
end ;;
\end{lstlisting}
}
it is allowed to explicitly qualify the call to {\tt m1} by ``!''
with no species name, hence impolitely telling ``from myself'':
{\scriptsize
\begin{lstlisting}
species Foo ... =
  let m1 (...) = ... ;
  let m2 (...) = if ... then ... else !m1 (...) ;
end ;;
\end{lstlisting}
}
In fact, without explicit ``!'', the \focal\ compiler performs the
name resolution itself, allowing a lighter way of writing programs
instead of always needing a ``!'' character before each method call.



%%%%%%%%%%%%%%%%%%%%%%%%%%%%%%%%%%%%%%%%%%%%%%%%%%%%%%%%%%%%%%%%%%%%%%%
\subsubsection{Entity parameters}
\index{parameter!entity}
\label{entity-parameter}
The other way to parametrise a species is to pass it a
{\bf value of a certain species carrier}. For instance, in the
previous section, we made our {\tt List} species parametrised by the
``type'' of the elements in the list. We then used a collection
parameter. If we now want a species addition modulo a value, we need
to parametrise our species by this {\bf value}, or at least by a
value of a collection implementing the integers and giving a way to
have a value representing the one we whish. Such a parameter is called
an {\bf entity parameter} and is introduced by the keyword {\tt in}.
{\scriptsize
\begin{lstlisting}
species AddModN (Number is ASpeciesImplentingInts, val_mod in Number) =
  rep = Number ;
  let add (x in Self, y in Self) =
    Number!modulo (Number!add (x, y), val_mod) ;
end ;;

species
\end{lstlisting}
}

Hence, any collection created from {\tt AddModN} will embed it modulo
value. And it will be possible to create various collections with each
a specific module value. For instance, assuming that the species
{\tt AddModN} is complete and have a method {\tt from\_int} able to
create a value of the carrier from an integer, we can create a
collection implementing addition modulo 42. We also assume that we
have a collection {\tt ACollImplentingInts} implementing the
integers.
{\scriptsize
\begin{lstlisting}
collection AddMod42 implements AddModN
  (ACollImplentingInts, ACollImplentingInts!from_int (42)) ;;
\end{lstlisting}
}


\smallskip
Currently, entity parameters must live ``{\tt in}'' a collection. It
is not allowed to specify an entity parameter living in a basic type
like {\tt int}, {\tt string}, {\tt bool}\ldots This especially means
that one must have a collection embedding, implementing the type of
values we want to use as entity parameters.



%%%%%%%%%%%%%%%%%%%%%%%%%%%%%%%%%%%%%%%%%%%%%%%%%%%%%%%%%%%%%%%%%%%%%%%
%%%%%%%%%%%%%%%%%%%%%%%%%%%%%%%%%%%%%%%%%%%%%%%%%%%%%%%%%%%%%%%%%%%%%%%
\subsection{Inheritance and its mechanisms}
In this section, we address the second mean to build complex species
based on existing ones. It will cover the notion of {\em inheritance}
and its related feature the {\em late-binding}.



%%%%%%%%%%%%%%%%%%%%%%%%%%%%%%%%%%%%%%%%%%%%%%%%%%%%%%%%%%%%%%%%%%%%%%%
\subsubsection{Inheritance}
\label{inheritance}
\index{inheritance}
Like in Object-Oriented models, in \focal\ {\em inheritance} is the
ability to create a species, not from scratch, but integrating methods
of other species. For instance, assuming we have a species
{\tt IntCouple} that represent couples of integers, we want to create
a species {\tt OrderedIntCouple} in which we ensure that the first
component of the couple is lower or equal to the second. Instead of
inventing again all the species, we will take advantage of the
existing {\tt IntCouple} and ``import'' its printing, equality, etc
functions. However, we will have to change the creation function since
it must ensure at creation-time of the couple that it is really
ordered. We also may add new methods in the newly built species. The
inheritance mechanism also allows to redefine methods already existing
as long as they keep the same type. To have the same type means for
functions to have the same ML-like type, and for theorems to have the
same statement (but not the same proof).
{\scriptsize
\begin{lstlisting}
species IntCouple =
  rep = (int * int) ;
  let print (x in Self) = ... ;
  let create (x in int, y in int) = (x, y) ;
  let equal (c1, c2) =
    match (c1, c2) with
     | ((c11, c12), (c21, c22)) -> c11 = c21 && c12 = c22 ;
  ...
end ;;

species OrderedIntCouple inherits (IntCouple) =
  let create (x in int, y in int) =
    if x < y then (x, y) else (y, x) ;

  property is_ordered : all c in Self, first (c) <= scnd (c) ;
end ;;
\end{lstlisting}
}

In the example above, {\tt OrderedIntCouple} will have all the methods
of {\tt IntCouple}, except the redefined ones (here {\tt create}, plus
the methods exclusively defined in {\tt OrderedIntCouple} (here, the
property {\tt is\_ordered}) stating that the couple is really
ordered). During inheritance, it is also possible to redefine a
signature, replacing it by an effective definition, to redefine a
property by a theorem and in the same idea, to add a {\tt proof of} to
a property in order to conceptually redefine is as a theorem. Since
inherited methods are now owned by the species that inherits, they we
be called exactly like if they were defined ``from scratch'' in the
species.

\smallskip
\index{inheritance!multiple}
Multiple inheritance, i.e. inheriting from several species is
allowed by specifying several species separated by comma in the
{\tt inherits} clause. In case of methods appearing in several
parents, the kept one is the one coming from the rightmost parent in
the {\tt inherits} clause. For instance below, if species {\tt A} and
{\tt C} provide a method {\tt m}, {\tt C's} one will be kept.
{\scriptsize
\begin{lstlisting}
species Foo inherits A, B, C, D =
  ... m (...) ... ;
end ;;
\end{lstlisting}
}

\smallskip
\index{inheritance!parametrised species}
If a species {\tt S1} inherits from a parametrised species {\tt S0},
it must instantiate all the parameters of {\tt S0}. Because of our
dependent types\index{type!dependent} framework, if {\tt S1} is itself
parametrised, it can use its own parameters to do that. Assuming we
have a species {\tt List} parametrised by a collection parameter
representing the kind of elements of the list. We now want to derive
a species {\tt ListUnique} in which elements are present at most
once. We then want {\tt ListUnique} to inherit from {\tt List} with
the elements begin the same between {\tt List} and {\tt ListUnique}.
{\scriptsize
\begin{lstlisting}
species List (Elem is ...) =
  let empty = ... ;
  let add (e in Elem, l in Self) = ... ;
  let concat (l1 in Self, l2 in Self) = ... ;
end ;;

species ListUnique (UElem is ...) inherits List (UElem) =
  let add (e in Elem, l in Self) =
    ... (* Ensure the element e is not already present. *) ;
  let concat (l1 in Self, l2 in Self) =
    ... (* Ensure elements of l1 present in l2 are not added. *) ;
end ;;
\end{lstlisting}
}

\index{inheritance!parametrised by {\tt Self}}
A species can also inherit of a species parametrised by itself
(i.e. by {\tt Self}). Although this is rather tricky programming, the
standard library of \focal\ shows such an example in the file
{\em weak\_structures.foc} in the species
{\tt Commutative\_semi\_ring}. In such a case, this implies that the
current species must finally (when inheritance is resolved) have an
interface compatible with the interface required by the collection
parameter of the species we inherit. The \focal\ compiler will then
collect all the interfaces {\tt Self} must be compatible with due to
parametrised inheritance and will ensure afterwards that once built,
{\tt Self} has a really compatible interface.



%%%%%%%%%%%%%%%%%%%%%%%%%%%%%%%%%%%%%%%%%%%%%%%%%%%%%%%%%%%%%%%%%%%%%%%
\subsubsection{Conclusion on species expressions}
\index{species!expression}
At the beginning of this presentation, when dealing with collection
parameters (section \ref{parametrisation}) we explained that
collection parameters ``expressions'' were, first, species names. Like
in the example:
{\scriptsize
\begin{lstlisting}
species List (Elem is Basic_object) = ... ;
\end{lstlisting}
}

Then, we learnt about parametrised species and we saw that we can have
a species parametrised by a parametrised species, like in the example:
{\scriptsize
\begin{lstlisting}
species MyCouple (E is Basic_object, L is List (E)) = ... ;;
\end{lstlisting}
}

Going on, we got interested in inheritance and saw we could inherit
from species that were referenced only by their name, like in:
{\scriptsize
\begin{lstlisting}
species OrderedIntCouple inherits (IntCouple) = ... ;;
\end{lstlisting}
}

And finally, we also saw species inheriting from parametrised
species, like in:
{\scriptsize
\begin{lstlisting}
species ListUnique (UElem is ...) inherits List (UElem) = ... ;;
\end{lstlisting}
}

Hence, we can now defined more accurately the notion of {\bf species
exression} used for both inheritance and parametrisation. It is either
a simple species name or the application of a parametrised species to
as many species expressions as the parametrised species has
parameters.


%%%%%%%%%%%%%%%%%%%%%%%%%%%%%%%%%%%%%%%%%%%%%%%%%%%%%%%%%%%%%%%%%%%%%%%
\subsubsection{Late-binding}
\label{late-binding}
\index{late-binding}
We stated just above (c.f. \ref{inheritance}) that it was possible
during inheritance to replace a signature by a function or a property
by a theorem. On another hand, we saw that it was possible to define
function using a signature (c.f \ref{idea-fun-using-sig}), that is
something only declared, with no implementation yet. In the same
order, it is possible to redefine a method already used by an existing
method.

\smallskip
\focal\ proposes a mechanism known as {\em late-binding} to solve this
issue. During compilation, the selected method will be the {\bf most
recently defined} along the inheritance tree. This especially means
that as long as a method is a signature, in the children the effective
implementation of the method will remain undefined (that is not a
problem since in this case the species is not complete, hence cannot
lead to a collection, i.e. code that can really be executed
yet). Moreover, if a method {\tt m} previously defined in the
inheritance tree uses a method {\tt n} freshly {\bf re}defined, then
this {\bf fresh redefinition} of {\tt n} will be used in the method
{\tt m}.

\smallskip
This mechanism enables two programming features:
\begin{itemize}
  \item The mean to use a method known by its type (i.e. its interface
    in term of Software Engineering), but for which we do not know, or
    we don't need or we don't want yet to provide an implementation.

  \item To provide a new implementation of a method, for example more
    efficient because algorithms exist due to the refinements
    introduced by the inheriting species, while already having and
    keeping the initial implementation for the inherited species.
\end{itemize}



%%%%%%%%%%%%%%%%%%%%%%%%%%%%%%%%%%%%%%%%%%%%%%%%%%%%%%%%%%%%%%%%%%%%%%%
\subsubsection{Dependencies and erasing}
We previously saw that methods of a species can use other methods of
this species and methods from its collection parameters. This induce
what we call {\bf dependencies}\index{dependency}. There are two kinds
of dependencies, depending on their nature:
\begin{itemize}
  \item {\bf Decl-dependencies}
  \item {\bf Def-dependencies}
\end{itemize}
In order to understand the difference between, we must inspect further
the notion of carrier, function, and theorem.



\paragraph{Decl-dependencies}
\index{dependency!decl}
When defining a function, a property or a theorem it is possible to
use another function or signature. For instance:
{\scriptsize
\begin{lstlisting}
species Bla =
  signature test : Self -> bool ;
  let f1 (x in string) = ... ;
  let f2 (y in Self) = ... f1 ("Eat at Joe's") ... ;
  property p1 : all x in Self, test (f2 (x)) <-> test (f1 ("So what")) ;
  theorem t1 : all x in Self, p1 <->  test (f1 ("Bar"))
  proof: ... ;
end ;;
\end{lstlisting}
}

In this cases, knowing the type of the used methods is sufficient to
ensure that the using method is well-formed. We remind that the type
of a function and a signature is the ML-like type. And for a property
and a theorem, this type is their logical statement. The type of a
method being provided by its {\bf declaration}, we will call induced
dependencies {\bf decl-dependencies}.

Such dependencies also arise on the carrier as soon as the type of a
method makes reference to the type {\tt Self}. Hence we can have
dependencies on the carrier as well as on other methods (as
previously stated in \ref{rep-is-method}, {\tt rep} is a method).

Hence, in our example, {\tt test}, {\tt f2}, {\tt f1} (since it is
used in {\tt p1} and {\t1} as the argument of {\tt test} which expects
an argument of type {\tt Self}), {\tt p1} and {\tt t1} have a
decl-dependency on the carrier. Moreover, {\tt f2} has one on
{\tt f1}. The property {\tt p1} has decl-dependencies on {\tt test},
{\tt f1} and {\tt f2}. And finally {\tt t1} decl-depends on {\tt p1},
{\tt test} and {\tt f1}.



\paragraph{Def-dependencies}
\index{dependency!def}
It seems that when {\bf defining} a function, only decl-dependencies
may appear since the type system of \focal\ only needs the type of the
functions used by this function. This is right except about the
carrier. We must remind (c.f. \ref{rep-is-method}) that if {\tt rep}
is given, then it is {\bf defined} ! If it is not given, then it is
like if it was only {\bf declared}. Hence a function making usage
in its body of the known carrier representation (i.e. definition) will
have a {\bf def-dependency} on the carrier. Such dependencies means
that to ensure that the using method is well-formed the system need to
know the {\bf definition} of the entity we depend on.


\smallskip
In the same order, when {\bf using} a signature in another method,
since signature only contain types, no decl-dependencies can arise.

\smallskip
\index{dependency!def!on carrier}
Sligtly differently, when {\bf using} a property in another method,
since it only contains a ``type'' (i.e. a logical statement), only 
decl-dependencies appear except on the carrier. For consistency
reasons going beyond this manuel, the {\bf \focal\ system rejects
properties having def-dependencies on the carrier}.

\smallskip
There remains the case of {\bf defining} a theorem. This case is the
more complex since it can lead to decl-dependencies via its
proofs. These dependencies are introduce by the statement of the
proof. We won't explain here the way to introduce them, this will be
done further in section \ref{zenon-an-dependencies}. First of all, for
the same reasons than for properties, the {\bf \focal\ system rejects
theorems having def-dependencies on the carrier}.

Now, what does mean for a theorem to def-depend on a method ? This
basiclly means that to make the proof of it statement, one must use
not only the declaration of a method, but also it definition, its
body.

\paragraph{Erasing during inheritance}
As a consequence of def-dependencies



%%%%%%%%%%%%%%%%%%%%%%%%%%%%%%%%%%%%%%%%%%%%%%%%%%%%%%%%%%%%%%%%%%%%%%%
\printindex
\end{document}
