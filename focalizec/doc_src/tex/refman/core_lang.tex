%%%%%%%%%%%%%%%%%%%%%%%%%%%%%%%%%%%%%%%%%%%%%%%%%%%%%%%%%%%%%%%%%%%%%%%
%%%%%%%%%%%%%%%%%%%%%%%%%%%%%%%%%%%%%%%%%%%%%%%%%%%%%%%%%%%%%%%%%%%%%%%
%%%%%%%%%%%%%%%%%%%%%%%%%%%%%%%%%%%%%%%%%%%%%%%%%%%%%%%%%%%%%%%%%%%%%%%
\section{Lexical conventions}
\index{lexical conventions}



%%%%%%%%%%%%%%%%%%%%%%%%%%%%%%%%%%%%%%%%%%%%%%%%%%%%%%%%%%%%%%%%%%%%%%%
%%%%%%%%%%%%%%%%%%%%%%%%%%%%%%%%%%%%%%%%%%%%%%%%%%%%%%%%%%%%%%%%%%%%%%%
\subsection{Blanks}
\index{blank}
The following characters are considered as blanks: space, newline,
horizontal tabulation, carriage return, line feed and form
feed. Blanks are ignored, but they separate adjacent identifiers,
literals and keywords that would otherwise be confused as one single
identifier, literal or keyword.



%%%%%%%%%%%%%%%%%%%%%%%%%%%%%%%%%%%%%%%%%%%%%%%%%%%%%%%%%%%%%%%%%%%%%%%
%%%%%%%%%%%%%%%%%%%%%%%%%%%%%%%%%%%%%%%%%%%%%%%%%%%%%%%%%%%%%%%%%%%%%%%
\subsection{Comments}
\index{comment}
Comments (possibly spanning) on several lines are introduced by the
two characters {\tt (*}, with no intervening blanks, and terminated by
the characters {\tt *)}, with no intervening blanks. Comments are
treated as blanks. Comments can occur inside string or character
literals and can be nested. They are discarding during the compilation
process. Example:
{\scriptsize
\begin{lstlisting}
(* Discarded comment. *)
species S =
  ...
  let m (x in Self) = (* Another discarded comment. *)
  ...
end ;;
(* Another discarded comment for the end of file. *)
\end{lstlisting}
}


Comments spanning on only one line start by the two characters
{\tt --} and are implicitely ended by the end-of-line character.
Example:
{\scriptsize
\begin{lstlisting}
-- Discarded uniline comment.
species S =
  let m (x in Self) = -- Another discarded comment.
  ...
end ;;
\end{lstlisting}
}



%%%%%%%%%%%%%%%%%%%%%%%%%%%%%%%%%%%%%%%%%%%%%%%%%%%%%%%%%%%%%%%%%%%%%%%
%%%%%%%%%%%%%%%%%%%%%%%%%%%%%%%%%%%%%%%%%%%%%%%%%%%%%%%%%%%%%%%%%%%%%%%
\subsection{Documentations}
\index{documentation}
Documentations are introduced by the three characters {\tt (**},
with no intervening blanks, and terminated by the two characters
{\tt *)}, with no intervening blanks.
Documentation cannot occur inside string or character literals and
cannot be nested. They must preceed entities they document. This
especially means that a
{\bf source file cannot ends by a documentation} otherwise a syntax
error will be issued.

Conversely to comments, they are kept during the compilation process
and recorded in the compilation information (``.fo'' files). They can
be processed by the external documentation tool \focdoc\ provided with
the \focal\ development environment. Several documentations can be put
in sequence for the same construct. This allow to embeed inside them
tags that can be used by third-party tools to discriminate on
meaningful documentation for them. For more information, consult
\ref{documentation-generation}.
Example:
{\scriptsize
\begin{lstlisting}
(** This a documentation for species S. *)
species S =
  ...
  let m (x in Self) =
    (** [MY_TAG_TEST] Documentation for testers. *)
    (** [MY_TAG_MAINT] Documentation for maintainers. *)
    ... ;
end ;;
\end{lstlisting}
}



%%%%%%%%%%%%%%%%%%%%%%%%%%%%%%%%%%%%%%%%%%%%%%%%%%%%%%%%%%%%%%%%%%%%%%%
%%%%%%%%%%%%%%%%%%%%%%%%%%%%%%%%%%%%%%%%%%%%%%%%%%%%%%%%%%%%%%%%%%%%%%%
\subsection{Identifiers}
\index{identifier}
Identifiers are sequences of letters, digits, and {\tt \_} (the
underscore character), starting with a letter or an underscore.
Letters contain at least the 52 lowercase and uppercase
letters from the standard ASCII set. In an identifier, all characters
are meaningful.
Examples: {\tt foo}, {\tt bar}, {\tt \_20}, {\tt \_\_\_gee\_42}.



%%%%%%%%%%%%%%%%%%%%%%%%%%%%%%%%%%%%%%%%%%%%%%%%%%%%%%%%%%%%%%%%%%%%%%%
%%%%%%%%%%%%%%%%%%%%%%%%%%%%%%%%%%%%%%%%%%%%%%%%%%%%%%%%%%%%%%%%%%%%%%%
\subsection{Extended identifiers}
\index{identifier!extended}
Moreover, \focal\ has special forms of identifiers to allow using
spaces inside or extend the notion of operator identifiers.
\begin{itemize}
  \item {\bf Quoted identifiers}.
    \index{identifier!quoted}
    They start and end by the two characters {\tt ''} (quote) and may
    contain regular identifiers characters plus spaces. For example:
    {\tt ``equal is reflexive''}, {\tt ``fermat conjecture''}.
  \item {\bf Infix/prefix operators}.
    \index{identifier!operator}\index{operator}
    \focal allows infix and prefix operators built from a
    ``starting operator character'' and followed by a sequence of
    regular identifier or operators characters. For example, all the
    following are legal operators: 
    {\tt +}, {\tt ++}, {\tt $\sim$+zero}, {\tt =\_mod\_5}.

    The position in which to use the operator (i.e. infix or prefix)
    is determined by the position of the first operator character
    according to the following table:
    \begin{center}
    \begin{tabular}{|c|c|}
    \hline
    Prefix & Infix \\
    \hline
    ` $\sim$ ? \$ ! \#                            &
    , + - * / \% \& $|$ : ; $<$ = $>$ @ \^\ $\setminus$ \\
    \hline
    \end{tabular}
    \end{center}

    Hence, in the above examples, {\tt +}, {\tt ++} and
    {\tt =\_mod\_5} will be infix operators and {\tt $\sim$+zero} will
    be a prefix one.

    This notion of infix/prefix operator does not mean that
    \focal\ defines all these operators: it means that the programmer
    is free to lexically define them and use them as regular
    prefix/infix operators instead of using the regular function
    application. For instance, instead of creating a function to check
    if and integer {\tt x} is equal to an integer {\tt y} minus 1
{\scriptsize
\begin{lstlisting}
let is_eq_minus_1 (x, y) = ... ;
... is_eq_minus_1 (5, 7) ... ;
\end{lstlisting}
}
    it is possible to directly define
{\scriptsize
\begin{lstlisting}
let ( =-one ) (x, y) = ... ;
... 5 =-one 7 ... ;
\end{lstlisting}
}
    As shown in the example, at definition-time, the syntax requires
    the operator to be embraced by parentheses. {\bf Attention}: the
    common error while defining the {\tt *} operator is to embrace it
    between parentheses but with no spaces around, leading to the
    lexical entity {\tt (*)} which is a start (or end) of comment,
    hence to a lexical error. For this reason, we always advice to
    get the habit of clearly put spaces around the operators names.

    When a prefix or infox operator is defined, it is still possible
    to use it as a regular function application with the syntax:
{\scriptsize
\begin{lstlisting}
... ( =-one ) (5, 7) ... ;
\end{lstlisting}
}

    {\bf Attention} : since the comma character is a possible infix
    starting character, when using a comma to separate tuples
    components, be sure to plut a blank behind the comma otherwise,
    {\tt (0, 1,n)} will be understood as a couple starting with the
    integer 0 as first component, then gpiong on with the second
    commponent being the infix application of the operator {\tt ,n} to
    the integer 1 and nothing, hence leading to a syntax error.

\end{itemize}



%%%%%%%%%%%%%%%%%%%%%%%%%%%%%%%%%%%%%%%%%%%%%%%%%%%%%%%%%%%%%%%%%%%%%%%
%%%%%%%%%%%%%%%%%%%%%%%%%%%%%%%%%%%%%%%%%%%%%%%%%%%%%%%%%%%%%%%%%%%%%%%
\subsection{Species, collections names}
\index{species!name}
As previously introduced, species and collections names are
capitalized (non-extended) identifiers.



%%%%%%%%%%%%%%%%%%%%%%%%%%%%%%%%%%%%%%%%%%%%%%%%%%%%%%%%%%%%%%%%%%%%%%%
%%%%%%%%%%%%%%%%%%%%%%%%%%%%%%%%%%%%%%%%%%%%%%%%%%%%%%%%%%%%%%%%%%%%%%%
\subsection{Integer literals}
\label{integer literals}
\label{octal}
\label{hexadecimal}
\label{binary}
An integer literal is a sequence of one or more digits, optionally
preceded by a minus or plus sign and/or a base prefix. By default,
i.e. without base prefix, indegers are in decimal. For instance:
{\tt 0}, {\tt -42}, {\tt +36}. \focal\ syntax allows to also specify
integers in other bases by preceding the digits by the following
prefixes:
\begin{itemize}
  \item {\bf Binary}: base 2. Prefix {\tt 0b} or {\tt 0B}.
    Digits: [0-1].
  \item {\bf Octal}: base 8. Prefix {\tt 0o} or {\tt 00}.
    Digits: [0-7].
  \item {\bf Hexadecimal}: base 16. Prefix {\tt 0x} or {\tt 0X}.
    Digits: [0-9] [A-F] [a-F]
\end{itemize}
Here are various examples of integers in various bases:
{\tt -0x1Ff}, {\tt 0B01001}, {\tt +Oo347}.



%%%%%%%%%%%%%%%%%%%%%%%%%%%%%%%%%%%%%%%%%%%%%%%%%%%%%%%%%%%%%%%%%%%%%%%
%%%%%%%%%%%%%%%%%%%%%%%%%%%%%%%%%%%%%%%%%%%%%%%%%%%%%%%%%%%%%%%%%%%%%%%
\subsection{String literals}
\label{string literal}
String literals are sequences of any characters delimited by {\tt "}
(double quote) characters ({\em ipso facto} with no intervening
{\tt"}). 
Escape sequences (meta code to insert characters that can't appears
simply in a string) available in string literals are summarised in the
table below:

\medskip
\noindent
\begin{tabular}{|c|c|l|}
  \hline
  Sequence & Character & Comment \\
  \hline
  $\setminus$n & '$\setminus$010' & Line feed. \\
  \hline
  $\setminus$r & '$\setminus$013' & Carriage return. \\
  \hline
  $\setminus$b & '$\setminus$008' & Backspace. \\
  \hline
  $\setminus$t & '$\setminus$009' & Tabulation. \\
  \hline
  $\setminus$" & '"'              & Double quote. \\
  \hline
  $\setminus$* & '*'              & Since comments cannot appear
                                    inside string, \\
               &                  & to insert the
                                    sequence ``(*'' or ``*)'', use this \\
               &                  & escape sequence combined with the
                                    two following \\
               &                  & ones. \\
  \hline
  $\setminus$( & '('              & See comment above for $\setminus$*. \\
  \hline
  $\setminus$) & ')'              & See comment above for $\setminus$*. \\
  \hline
  $\setminus\setminus$ & '\'      & Backslash character. \\
  \hline
  $\setminus$- & '-' & Minus (dash) character. Like for multiline
                                    comments, \\
               &                  & uniline comments can't
                                    appear in strings. Hence, to \\
               &                  & insert the sequence ``--'', use
                                    twice this escape sequence. \\
  \hline
  $\setminus$[0-9][0-9][0-9] & & The character whose ASCII code in
                                {\bf decimal} is given \\
               &                  & by the 3 digits following the
                                  $\setminus$. This sequence is valid \\
               &                  & for any valid character
      ASCII code.\\
  \hline
\end{tabular}



%%%%%%%%%%%%%%%%%%%%%%%%%%%%%%%%%%%%%%%%%%%%%%%%%%%%%%%%%%%%%%%%%%%%%%%
%%%%%%%%%%%%%%%%%%%%%%%%%%%%%%%%%%%%%%%%%%%%%%%%%%%%%%%%%%%%%%%%%%%%%%%
\subsection{Character literals}
\label{character literals}
\label{hexadecimal}
Characters literals are one character enclosed between 2 ``{\tt '}''
(quote) characters. Example: {\tt 'a'}, {\tt '?'}.
Escape sequences (meta code to insert characters that can't appears
simply in a character literal) must also be enclosed by
quotes. Available escape sequences are summarised in the table below:

\medskip
\noindent
\begin{tabular}{|c|c|l|}
  \hline
  Sequence & Character & Comment \\
  \hline
  $\setminus\setminus$ & '$\setminus$' & Backslash. \\
  \hline
  $\setminus$'         & '{\tt '}'  & Quote. \\
  \hline
  $\setminus$"         & '"'        & Double quote. \\
  \hline
  $\setminus$n         &            & Carriage return. \\
  \hline
  $\setminus$t         &            & Tabulation. \\
  \hline
  $\setminus$b         &            & Backspace. \\
  \hline
  $\setminus$r         &            & Carriage return. \\
  \hline
  $\setminus$[0-9][0-9][0-9] & & The character whose ASCII code in
                                {\bf decimal} is given \\
               &                  & by the 3 digits following the
                                  $\setminus$. This sequence is valid \\
               &                  & for any valid character
      ASCII code.\\
  \hline
  $\setminus$x[0-9a-fA-F][0-9a-fA-F] & & The character whose ASCII code in
                                {\bf hexadecimal} is given \\
               &                  & by the 2 digits following the
                                  $\setminus$. This sequence is valid \\
               &                  & for any valid character
      ASCII code.\\
  \hline
\end{tabular}



%%%%%%%%%%%%%%%%%%%%%%%%%%%%%%%%%%%%%%%%%%%%%%%%%%%%%%%%%%%%%%%%%%%%%%%
%%%%%%%%%%%%%%%%%%%%%%%%%%%%%%%%%%%%%%%%%%%%%%%%%%%%%%%%%%%%%%%%%%%%%%%
\subsection{Floating-point numbers literals}
Floating-point numbers literals are made of an optional sign ('+' or
'-') followed by a non-empty sequence of digits followed by a dot
('.') followed by a possibly empty sequence of digits and finally an
optional scientific notation ('e' or 'E' followed an optionnal sign
then by a non-empty sequence of digits. \focal\ allows floats to be
written in decimal or in hexadecimal. In the first case, digits are
[0-9]. Example: {\tt 0.}, {\tt -0.1}, {\tt 1.e-10}, {\tt +5E7}.
In the second case, they are [0-9 a-f A-F] and the number must be
prefixed by ``0x'' or ``0X''. Example {\tt 0x-F2.E4}, {\tt 0X4.3A},
{\tt Ox5a.a3eef}, {\tt Ox5a.a3e-ef}.



%%%%%%%%%%%%%%%%%%%%%%%%%%%%%%%%%%%%%%%%%%%%%%%%%%%%%%%%%%%%%%%%%%%%%%%
%%%%%%%%%%%%%%%%%%%%%%%%%%%%%%%%%%%%%%%%%%%%%%%%%%%%%%%%%%%%%%%%%%%%%%%
\subsection{Proof step bullets}
\index{proof!step bullet}
A proof step bullet is a non-negative non-signed integer literal
(i.e. a non empty sequence of [0-9] characters) delimited by the
characters {\tt <} and {\tt >}, followed by a non-empty sequence of
alphanumeric characters (i.e. [A-Z a-z 0-9]). 
The first part of the bullet (i.e. the integer literal) stands for the
depth of the bullet and the second part stands for its name. Example:
{\scriptsize
\begin{lstlisting}
 <1>1 assume ...
      ...
      prove ...
   <2>1 prove ... by ...
   <2>9 qed by step <2>1 property ...
 <1>2 qed.
\end{lstlisting}
}



%%%%%%%%%%%%%%%%%%%%%%%%%%%%%%%%%%%%%%%%%%%%%%%%%%%%%%%%%%%%%%%%%%%%%%%
%%%%%%%%%%%%%%%%%%%%%%%%%%%%%%%%%%%%%%%%%%%%%%%%%%%%%%%%%%%%%%%%%%%%%%%
\subsection{Reserved keywords}
The identifiers below are reserved as keywords, and cannot be employed
otherwise:
\begin{verbatim}
   alias        all          and             as
   assume       assumed      begin           by
   caml         collection   coq             coq_require
   definition   else         end             ex
   external     false        function        hypothesis
   if           in           inherits        internal
   implements   is           let             lexicographic
   local        logical      match           measure
   not          notation     of              open
   on           or           order           proof
   prop         property     prove           qed
   rec          rep          representation  Self
   signature    species      step            structural
   termination  then         theorem         true
   type         use          with
\end{verbatim}



%%%%%%%%%%%%%%%%%%%%%%%%%%%%%%%%%%%%%%%%%%%%%%%%%%%%%%%%%%%%%%%%%%%%%%%
%%%%%%%%%%%%%%%%%%%%%%%%%%%%%%%%%%%%%%%%%%%%%%%%%%%%%%%%%%%%%%%%%%%%%%%
\subsection{Language constructs and syntax}
