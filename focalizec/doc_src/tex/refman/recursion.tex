\index{recursion}

When defining a recursive function {\focal}, as {\coq}, expects a proof that
this function terminates. A termination proof is stated immediately after the
body of its related recursive function definition:

\noindent
{\scriptsize
\begin{lstlisting}
let rec f (...) = ...
termination proof = ...
\end{lstlisting}
}

Termination proofs apply to recursive methods of species as well as to toplevel
recursive functions with the limitations presented in the next section.
Recursive functions or methods can be used as {\zenon} hints in proofs like
any other ones (\lstinline"by definition of").

As usually, proving that a function terminates implies showing that each
recursive call is made on an argument which is ``smaller'' than the initial
one. The notion of ``smaller'' can be stated by different means.


\section{Limitations}
Termination proofs are not supported for some forms of recursion. In such
cases, it may be possible to omit the proof of the related functions, letting
the compiler generating an assumed proof. The following shapes of recursion
or recursive functions do not currently support termination proofs:

\begin{compact-itemize}
\item Local recursive functions (nested in functions or more generally
      expressions).
\item Nested recursive calls, i.e. of the form \lstinline"f (f (e))".
\item Mutually recursive functions other than those structural on their first
   argument.
\item Termination proofs using a {\em measure} can only refer to {\bf one}
   parameter of the function among all the ones it has.
\item In this early state, {\bf polymorphic} toplevel recursive functions
  may be incorrectly generated, leading to a {\coq} error.
\end{compact-itemize}


\section{Kinds of termination proofs}
{\focal} proposes 3 ways to prove the termination of a function: by a
structural decreasing of an argument, using a well-founded relation or
using a measure. Each possibility is detailed in the following sections.


From a technical point of view, the difference in the generated code only
affects the {\coq} code. Recursive structural functions are compiled to the
construct {\tt Fixpoint} of {\coq}. Functions that are not recursive
structural or that do not have termination proofs are compiled to the
{\tt Function} construct of {\coq}.


\subsection{Structural termination}
A structural recursive function is characterized by recursive calls performed
on a subterm of the initial decreasing argument. Hence this argument must be of
an inductive type. The termination proof is stated by:
{\tt termination proof = structural on $arg$}.

\subsubsection{Example}
This function is a simple computation of the length of a list. The recursive
call is performed on the tail of the list thanks to the pattern-matching.
We have {\tt l = \_ :: q}, hence {\tt q} is structurally smaller than
{\tt l}.

\noindent
{\scriptsize
\begin{lstlisting}
open "basics" ;;

let rec length (l : list (int)) =
  match l with
  | [] -> 0
  | _ :: q ->  1 + (length (q))
termination proof = structural l ;;
\end{lstlisting}
}

\subsubsection{Example}
This function tests if a value {\tt x} belongs to the list {\tt l}. The
decreasing is the same than in the previous example. This example illustrates
the possibility to have a function with several arguments and state the
chosen one as decreasing.

{\scriptsize
\begin{lstlisting}
open "basics" ;;

let rec mem (x, l) =
  match l with
  | [] -> false
  | h :: q ->  if x = x then true else mem (x, q)
termination proof = structural l ;;
\end{lstlisting}
}

\subsubsection{{\bf Wrong} Example}
\noindent
{\scriptsize
\begin{lstlisting}
open "basics" ;;

let rec zero (x) =
  match x with
  | 0 -> 0
  | n -> zero (n - 1)
termination proof = structural x ;;
\end{lstlisting}
}

In this case $(n - 1)$ is not a subterm of $n$. This latter has type
{\tt int} which is not an inductive type. Currently, the compiler does not
ensure that the type is inductive, leading to a {\coq} error. This is a
known limitation and weakness of the compiler which should be fixed in a
future release.

\subsubsection{Example}

\noindent
{\scriptsize
\begin{lstlisting}
open "basics" ;;

type pint_t =
 | Z
 | S (pint_t)
;;

let rec zero (x) =
  match x with
  | Z -> 0
  | S (n) -> zero (n)
termination proof = structural x ;;
\end{lstlisting}
}

This new function uses a representation of the integers being an inductive
type (Peano's integer indeed). The structural decreasing is now ensured and
this definition is correct.


\subsection{Termination by a well-founded relation}
{\bf Note}: some explanations of this section come from
\cite{Dubois-Pessaux-termproofs-TFP2015}.

\medskip
The core of terminating recursion is that there are no infinite
chains of nested recursive calls. This intuition is commonly mapped to
the mathematical idea of a well-founded relation.

\medskip
We illustrate this section with a function {\tt div} computing the quotient
in the Euclidien division of two integers. We made this function total by
returning the (wrong) value 0 in order to only focus on termination.

\noindent
{\scriptsize
\begin{lstlisting}
open "basics" ;;
open "wellfoundation" ;;

let rec div (a, b) =
  if a <= 0 || b <= 0 then 0
  else ( if (a < b) then 0 else 1 + div ((a - b), b) )
termination proof = order pos_int_order on a ... ;
\end{lstlisting}
}

From the user's point of view, despite {\tt div} has two arguments,
only the first one {\tt a} is of interest for termination. The
well-founded relation used here, {\tt pos\_int\_order}
(of type {\tt int} $\rightarrow$ {\tt int}  $\rightarrow$ {\tt bool}) is the
usual ordering on positive integers provided by the standard library.
The well-foundedness obligation of this relation, is stated by
\lstinline"is_well_founded (pos_int_order)", which also provided by the
\focal\ standard library (in the module {\tt wellfoundation.fcl}).

The function {\tt div} having only one recursive call, the only
decreasing proof obligation will be:

$\forall a : int, \forall b : int,
      \lnot (a \leq 0 \lor b \leq 0) \rightarrow
      ~\lnot(a < b)  \rightarrow pos\_int\_order (a - b, a)
$

\noindent where the conditions on the execution path leading to the
recursion must be accumulated as hypotheses.

The termination proof consists in as many steps as there are recursive
calls, each one proving the ordering (according to the relation) of the
decreasing argument and the
initial one, then one step proving that the termination relation
is well-founded and
an immutable concluding step (\lstinline"<1>e qed coq proof {*wf_qed*}")
telling to the compiler to assemble the previous steps, generate some stub
code using a built-in \coq\ script to close the proof.
The complete termination proof for {\tt div} is:

\noindent
{\scriptsize
\begin{lstlisting}
open "basics" ;;
open "wellfoundation" ;;

let rec div (a, b) = ...
  termination proof = order pos_int_order on a
    <1>1 prove all a : int, all b : int,
      ~ (a <= 0 || b <= 0) ->
      ~ (a < b) -> pos_int_order (a - b, a)
      <2>1 assume  a : int, b : int,
           hypothesis H1: ~ (a <= 0 || b <= 0),
           hypothesis H2: ~ (a < b),
           prove pos_int_order (a - b, a)
           <3>1 prove b <= a
                by property int_not_lt_ge, int_ge_le_swap hypothesis H2
           <3>2 prove 0 <= a
                by property int_not_le_gt, int_ge_le_swap, int_gt_implies_ge
                hypothesis H1
           <3>3 prove 0 < b
                by property int_not_le_gt, int_gt_lt_swap hypothesis H1
           <3>4 prove (a - b) < a
                by step <3>1, <3>2, <3>3 property int_diff_lt
           <3>e qed by step <3>4, <3>2 definition of pos_int_order
      <2>e conclude
    <1>2 prove is_well_founded (pos_int_order)
       by  property pos_int_order_wf
   <1>e qed coq proof {*wf_qed*} ;;
\end{lstlisting}
}

Proof obligations can be printed by the compiler to prevent the user from
guessing them. For this, the termination proof must be stated as
{\tt order on ...} with an {\tt assumed} {\zenon} proof:

{\scriptsize
\begin{lstlisting}
open "basics" ;;
open "wellfoundation" ;;

let rec div (a, b) = ...
  termination proof = order pos_int_order on a
  assumed ;;
\end{lstlisting}
}

Then invoke the {\focalizec} compiler with the option {\tt -show-term-obls}.
During the compilation, the obligations get printed on the standard output
like:

\noindent
{\scriptsize
\begin{verbatim}
---------------------------------------------------------------
Termination proof obligations for the recursive function 'div':
<1>1 prove all a : basics#int, all b : basics#int,
  (basics#|| (basics#<= (a, 0), basics#<= (b, 0)) = false) ->
  ((basics#< (a, b)) = false) -> wellfoundation#pos_int_order
  ((basics#- (a, b)), a)
<1>2 prove is_well_founded (wellfoundation#pos_int_order)
<1>e qed coq proof {*wf_qed*}
---------------------------------------------------------------
\end{verbatim}
}

To summarize, a recursive function whose termination relies on a well-founded
relation is given by the four following points:

\begin{compact-enumerate}
\item the relation,
\item the theorem stating that this relation is well-founded,
\item a theorem for each recursive call, stating that the arguments
  of the recursive calls are smaller than the initial arguments
  according to the given relation,
\item the recursive function with a termination proof of the shape
  (the order of the obligations does not matter):
  \noindent
  {\scriptsize
  \begin{lstlisting}[mathescape=true,frame=none]
<1>$x$ proofs of right ordering for each recursive call
     (the same statements than corresponding theorems in point 3,
      even if it is possible to directly inline the proofs instead)
<1>$x+1$ proof of the relation being well-founded
     (same statement than in point 2, same remark than in steps <1>$x$)
<1>$x+2$ qed coq proof {*wf_qed*}
  \end{lstlisting}
  }
\end{compact-enumerate}



\subsection{Termination by a measure}
{\bf Note}: some explanations of this section come from
\cite{Dubois-Pessaux-termproofs}.

We consider here the particular case when the termination relies on a
\emph{measure} -- a  function that returns a natural number --
which must decrease at each recursive call.
We want to ease such termination proofs even if it would
be possible for the user to use the previous approach, by constructing
himself a well-founded relation from the measure. Precisely, the compiler
does this job for him.

A measure has to be positive, which will be a proof obligation.
However, the relation built from the  measure being internalized, its
well-foundedness is no more asked
to the user. From the user's point of view, the proof obligation for each
recursive call must now show that the measure decreases on the
argument of interest between each call.

\medskip
We illustrate this section with a function {\tt mem} checking if an element
belongs to a list. Although a structural argument could also be used, we
chose to rely on the decreasing {\em length} of the list where the element
(whose ``type'' is a parameter of the hosting species) is recursively
searched.  We first write the method {\tt length} whose termination is simply
structural on its argument {\tt l}. Then we write the method {\tt mem},
stating a termination proof using the measure {\tt length} on its argument
{\tt l}.

\noindent
{\scriptsize
\begin{lstlisting}
species Ex_mes (A is Basic_object) =
  let rec length (l : list(A)) =
    match l with
    | [] -> 0
    | h :: q ->  1 + length (q)
  termination proof = structural l ;

let rec mem (l, x: A) =
    match l with
    | [] -> false
    | h :: q ->  h = x || mem (q, x)
  termination proof =
    measure length on l ... ;
end ;;
\end{lstlisting}
}

From the user's point of view, despite {\tt mem} has two arguments,
only {\tt l} is of interest for the termination. The measure being
{\tt length}, the first proof obligation is:

$\forall l: list (A), 0 \leq length (l)$

\noindent
Then, the method {\tt mem} having only one recursive call, the only
decreasing proof obligation is:

$\forall l : list(A), \forall q : list(A), \forall h : A,
  \ l = h :: q \rightarrow length (q) < length (l)$

\noindent where variables bound on the execution path leading to the
recursion must be accumulated as hypotheses. Here, the recursion being
in a pattern-matching case, {\tt h} and {\tt q} must be related to the
matched value {\tt l}. The core of the decreasing fact is the $<$
relation between the argument of the recursive call and the one in the
current call.

For readability, instead of inlining the proofs of these obligations,
the user can state two related properties or theorems before the
function {\tt mem} itself.

\noindent
{\scriptsize
\begin{lstlisting}[frame=none]
  property length_pos : all l: list (A), 0 <= length (l) ;

  theorem mes_decr : all l : list(A) , all q : list(A), all h : A,
    l = h :: q -> length (q) < length (l)
  proof = ... ;
\end{lstlisting}
}

Note that {\tt length\_pos} is a {\em property}, not a theorem: it is not
yet proved.
Now the termination proof consists in as many steps as there are
recursive calls, each one proving the strict decreasing of the measure,
then one step proving that the measure is positive and an immutable
concluding step (\lstinline"<1>$x+2$ qed coq proof {*mf_qed*}" telling to the
compiler to assemble the previous steps and generate some stub code to close
the proof. Note that {\tt mf\_qed} and {\tt wf\_qed} are equivalent. Both
keywords are provided for users willing to make a difference between
termination by a well-founded relation and a measure in their source code.

\noindent
{\scriptsize
\begin{lstlisting}
  let rec mem (l, x: A) = ...
  termination proof = measure length on l
    <1>1 prove all l : list(A) , all q : list(A), all h : A,
           l = h :: q -> length (q) < length (l)
         by property mes_decr
    <1>2 prove all l: list (A), 0 <= length (l)
         by property length_pos
    <1>e qed coq proof {*mf_qed*} ;
\end{lstlisting}
}

To summarize, from the user's point of view, a recursive function
whose termination relies on a measure is given by the four following
points:
\begin{compact-enumerate}
\item The measure function returning a regular integer (which
  raises the issue that $<$ is well-founded on naturals, not integers).
\item The theorem stating that the measure is always positive or null.
\item A theorem for each recursive call, stating that the measure on the
  argument of interest decreases.
\item The recursive function with a termination proof of the shape:
  \noindent
  {\scriptsize
  \begin{lstlisting}[mathescape=true,frame=none]
<1>$x$ proofs of  decreasing for each recursive call
     (the same statements than corresponding theorems in point 3,
      even if it is possible to directly inline the proofs instead)
<1>$x+1$ proof of the measure being always positive or null
     (same statement than in point 2, same remark than in steps <1>$x$)
<1>$x+2$ qed coq proof {*mf_qed*}
  \end{lstlisting}
  }
\end{compact-enumerate}
