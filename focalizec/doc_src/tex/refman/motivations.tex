% $Id: motivations.tex,v 1.3 2008-12-11 10:48:24 hardin Exp $



The \focal\ project was launched in 1998 by T. Hardin and R. Rioboo
\cite{HardinRiobooTSI04} \footnote{They were members of the SPI (Semantics, Proofs,
Implementations) team of the LIP6 (Lab. Informatique de Paris 6)
at Universit\'e Pierre et Marie Curie (UMPC), Paris}
 with the 
objective of helping all stages of  development of critical software
within safety and security domains. The  methods used in these domains are evolving,
ad-hoc and empirical approaches  being replaced by more formal
methods. For example, for high levels of safety, formal models of the
requirement/specification phase are more and more considered as they
allow mechanized proofs, test or static analysis of the required
properties.  In the same way, high level assurance in system security asks for
the use of true formal methods along the process of software
development and is often required for the specification level.
Thus  the project was to elaborate an Integrated
Development Environment (IDE) able to provide high-level and justified confidence to
users, but remaining easy to use by well-trained engineers.

To ease developing high integrity systems with numerous software
components, an Integrated Development Environment (IDE) should provide
tools to formally express specifications, to describe design and
coding and to ensure that specification requirements are met by the
corresponding code. This is not
enough. First, standards of critical systems ask for pertinent
documentation which has to be maintained along all the revisions
during the system life cycle. Second, the evaluation conformance
process of software is by nature a sceptical analysis. Thus, any proof
of code correctness must be easily redone at request and traceability
must be eased. Third, design
and coding are difficult tasks. Research in software engineering has
demonstrated the help provided by some object-oriented 
features as inheritance, late binding and early research works on
programming languages have pointed out the importance of abstraction
mechanism such as modularity to help invariant maintaining. There are
a lot of other points which should also be considered when designing
an IDE for safe and/or secure systems to ensure conformance with high
Evaluation Assurance or Safety Integrity Levels (EAL-5,7 or SIL 3,4)
and to ease the evaluation process according to various standards
(e.g. IEC61508, CC, ...): handling of non-functional contents of
specification, handling of dysfunctional behaviors and vulnerabilities
from the true beginning of development and fault avoidance, fault
detection by validation testing, vulnerability and safety analysis.

Currently, \focal\  can be seen as an IDE still in development, which attempts
to give a positive solution to the three requirements identified above,
the other items being currently under consideration.
 It provides
means for the developers to formally express their specifications and
to go step by step (in an incremental approach) to design and
implementation while proving that
such an implementation meets its specification or design
requirements. Its language offers  high level
mechanisms such as inheritance, late binding, redefinition,
parametrization, etc.  Confidence in proofs submitted by developers or
automatically done 
relies on formal proof verification.  It also provides some automation of documentation
production and management. 

A formal specification can be built by declaring names of functions
and values and introducing
properties. Then, design and implementation can incrementally be done
by adding definitions of functions and proving that the implementation
meets the specification or design requirements. Thus, developing in
\focal\ is a kind of refinement process from  formal model to design
and code, completely done within \focal. Taking the global development
in consideration within a same environment brings some conciseness,
helps documentation and reviewing.

A \focal\ development is organised as a hierarchy that may have
several roots. The upper levels of the hierarchy are built along the
specification stage while the lower ones correspond to
implementation. Each node of the hierarchy corresponds to a progress
toward a complete implementation.  We call here {\em refinement} the
process of building a top-down hierarchy.

\focal\ semantics was initially specified in \coq, which brings a
satisfactory confidence in the language's correctness\cite{BoulmePhD00}. On the other
side, the correction of the compiler against \focal's semantics is
proved (by hand) \cite{TPHOL2002,TLCA2005,PrevostoJAR02}.


\focal\ has already been used to develop huge examples. In fact, when
the project was launched, only one specific domain was considered, the
one of Computer Algebra. Algorithms used in this domain can be rather
intricated and difficult to test and this is not rare that computer
algebra systems issue a bad result, due to semantical flaws, compiler
anomalies, etc.  Thus, Rioboo~\cite{ } developed a huge a computer algebra
library, which  offers  full specification
and implementation of usual algebraic structures up to multivariate
polynomial rings with complex algorithms.  The point was to measure
how \focal\ can help to render mathematical specifications and also to
measure efficiency of the produced code. Such a library is very useful
when formalising the algebra of access control models (another huge
example) using
implementations of orderings, lattices and boolean
algebras\cite{fcsarspa06,MorissetPhd}. \focal{} was also very
successfully used to formalize airport security
regulations~\cite{EDEMOI-All}.



