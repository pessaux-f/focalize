% $Id: motivations.tex,v 1.8 2009-06-04 15:11:07 jaeger Exp $


\subsection*{Motivations}

The {\foc} project was launched in 1998 by T. Hardin and R. Rioboo
\cite{HardinRiobooTSI04} \footnote{They were members of the SPI (Semantics, Proofs,
Implementations) team of the LIP6 (Lab. Informatique de Paris 6)
at Universit\'e Pierre et Marie Curie (UMPC), Paris}
with the objective of helping all stages of  development of critical software
within safety and security domains. The  methods used in these domains
are evolving, ad-hoc and empirical approaches  being replaced by more formal
methods. For example, for high levels of safety, formal models of the
requirement/specification phase are more and more considered as they
allow mechanized proofs, test or static analysis of the required
properties.  In the same way, high level assurance in system security asks for
the use of true formal methods along the process of software
development and is often required for the specification level.
Thus  the project was to elaborate an Integrated
Development Environment (IDE) able to provide high-level and justified
confidence to users, but remaining easy to use by well-trained
engineers.

To ease developing high integrity systems with numerous software
components, an IDE should provide
tools to formally express specifications, to describe design and
coding and to ensure that specification requirements are met by the
corresponding code. But this is not enough. First, standards of critical systems
ask for pertinent documentation which has to be maintained along all the revisions
during the system life cycle. Second, the evaluation conformance
process of software is by nature a skeptical analysis. Thus, any proof
of code correctness must be easily redone at request and traceability
must be eased. Third, design
and coding are difficult tasks. Research in software engineering has
demonstrated the help provided by some object-oriented
features as inheritance, late binding and early research works on
programming languages have pointed out the importance of abstraction
mechanisms such as modularity to help invariant preservation. There are
a lot of other points which should also be considered when designing
an IDE for safe and/or secure systems to ensure conformance with high
Evaluation Assurance or Safety Integrity Levels (EAL-5 to 7 or SIL 3 and 4)
and to ease the evaluation process according to various standards
(e.g. IEC61508, CC, ...): handling of non-functional contents of
specification, handling of dysfunctional behaviors and vulnerabilities
from the true beginning of development as well as fault avoidance and fault
detection by validation testing, vulnerability and safety analysis.


\subsection*{Initial application testbed}

 When the {\foc} project was launched by T. Hardin and R. Rioboo,
only the specific domain of Computer Algebra was initially considered.
Algorithms used in this domain can be rather intricated and
difficult to test and this is not rare that computer algebra systems
issue a bad result, due to semantical flaws, compiler anomalies,
etc. Thus the idea was to design a language allowing to specify the
mathematics underlying these algorithms and to go step by step to
different kinds of implementations according to the specifities of the
problem under consideration\footnote{For example Computer Algebra
Libraries use different representations of polynomials
according to the treatment to be done}. The first step was to design
the semantics of such a language, trying to fit to several
requirements: easing the expression of mathematical statements, clear
distinction between the mathematical structure (semi-ring, polynomial,
..)  and its different implementations, easing the development
(modularity, inheritance, parametrisation, abstraction, ..), runtime
efficiency and confidence in the whole development (mechanised proofs,
..).  After an initial phase of conceptual design, the {\foc}
semantics was submitted to a double test. On one hand, this semantics
was specified in {\coq} and in a categorical model of type theories by
S. Boulm\'e (see his thesis \cite{BoulmePhD00}), a point which
enlightened the borders of this approach, regarding the logical % [EJ] pas compris "border" -> "limits" ?
background. On the other hand, as a preliminary step before designing the syntax,
a strudy of the typical development style was conducted.
R. Rioboo~\cite{ThRRCalculemus,HardinRiobooTSI04} used the
{\ocaml} language to try different solutions, recorded in
~\cite{HardinRiobooTSI04}.

\subsection*{Initial {\oldfocal} design}

Then the time came to design the syntax of the language and the compiler. To
overcome inconsistencies risks, an original dependency analysis was
incorporated into the compiler (V. Prevosto
thesis \cite{PrevostoPhD03,TPHOL2002,PrevostoJAR02}) and the correction of the
compiler (mostly written by V. Prevosto) against {\oldfocal}'s semantics is
proved (by hand) \cite{TLCA2005}, a point which
brings a satisfactory confidence in the language's correctness. Then
R. Rioboo \cite{ } began the development of a huge Computer Algebra % [EJ] Citation manquante
library, offering full specification
and implementation of usual algebraic structures up to multivariate
polynomial rings with complex algorithms, to
extensively test the language and the
efficiency  of the produced code, aswell as to provide a standard library
of mathematical backgrounds. D. Doligez \cite{ZenonBDD} started the
development of {\zenon}, an automatic prover based on tableaux method,
which takes a {\oldfocal} statement and tries to build a proof of it and,
when succeeds,  issues a {\coq} term. More  recently, M. Carlier and
C. Dubois\cite{CarlierDuboisLNCS2008} began the development of a
test tool for {\oldfocal}.

{\oldfocal} has already been used to develop huge examples such as  the
standard library and the computer algebra library. The library
dedicated to the algebra of access control models, developed by
M. Jaume and C. Morisset \cite{jias06,fcsarspa06,MorissetPhd},
is another huge example, which borrows implementations of orderings,
lattices and boolean algebras from the computer algebra library.
{\oldfocal} was also very successfully used to formalize airport security
regulations, a work by D. Delahaye, J.-F. Etienne, C.  Dubois,
V. Donzeau-Gouge ~\cite{EDEMOI-All,EDEMOI-Model,EDEMOI-Proof}. This
last work led to the development of a translator \cite{Focal-UML}
from {\oldfocal} to UML for documentation purposes.

\subsection*{The {\focalize} system}

The {\focalize} development started in 2006, as a
continuation of the {\foc} and {\oldfocal} efforts. The new system was rewritten
from scratch. A new language and syntax was designed and carefully
implemented, with in mind ease of use, expressivity, and programmer
friendlyness. The addition of powerful data structure definitions -- together
with the corresponding pattern matching facilities -- leads to new expressive
power.

The {\zenon} automatic theorem prover was also integrated in the
compiler and natively interfaced within the {\focalize} language. New
developments for a better support of recursive functions is on the way (in particular for
termination proofs).

% [EJ] Suite sortie de l'historique et replacee plus bas


\subsection*{The {\focalize} system in short}

The {\focalize} system provides means for the developers to formally express
their specifications and to go step by step (in an incremental approach) to
design and implementation while proving that such an implementation
meets its specification or design requirements. The {\focalize} language offers
high level mechanisms such as multiple inheritance, late binding, redefinition,
parametrization, etc.  Confidence in proofs submitted by developers or
automatically done relies on formal proof verification. {\focalize} also
provides some automation of documentation production and management.

A formal specification can be built by declaring names of functions
and values and introducing
properties. Then, design and implementation can incrementally be done
by adding definitions of functions and proving that the implementation
meets the specification or design requirements. Thus, developing in
{\focalize} is a kind of refinement process from  formal model to design
and code, completely done within {\focalize}. Taking the global development
in consideration within the same environment brings some conciseness,
helps documentation and reviewing.

A {\focalize} development is organised as a hierarchy that may have
several roots. The upper levels of the hierarchy are built along the
specification stage while the lower ones correspond to
implementation and each  node of the hierarchy corresponds to a progress
toward a complete implementation.

We would like to mention several works about safety and/or security
concerns within {\focalize} and specially the definition of a safety life
cycle by P. Ayrault, T. Hardin and F. Pessaux \cite{TTSS08} and the
study of some traps within formal methods by E. Jaeger and
T. Hardin\cite{traps}.

{\focalize} can be seen as an IDE still in development, which
gives a positive solution to the three requirements identified above:

\begin{enumerate}
\item pertinent documentation is maintained within the system being written,
      and its extraction is an automatic part of the compilation process,
\item proofs are produced using an automated proved which can be guided using a high level proof language, so that proofs
      are easier to write and their verification is automatic and reliable,
\item the framework provides powerful abstraction mechanisms to facilitate
      design and development; however, these mechanisms are carefully ruled:
      the compiler performs numerous validity checks to ensure that no
      further development can inadvertantly break the invariants or
      invalidate the proofs; indeed, the compiler ensures that if a theorem
      was based on assumptions that are now violated by the new development,
      then the theorem is out of reach of the programmer and the properties have to be proven again.
\end{enumerate}




