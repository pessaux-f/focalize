% $Id: glimpse.tex,v 1.4 2008-12-19 10:13:53 pessaux Exp $
\def\bbbn{{\rm I\!N}} 
\label{glimpse}
Before entering the precise description of \focal\ we give an
informal 
presentation of near all its features, to help further reading of the
reference manual. Every construction or feature of \focal\ will be
entirely described in the following chapters. 

\section{The Basic Brick}

The primitive entity of a \focal\ development is the
{\em species}. It can be viewed as a record grouping ``things'' related
to a same concept. Like in most modular design systems (i.e. objected 
oriented, algebraic abstract types) the idea is to group a data
structure with the operations to process it. Since in \focal\ we don't
only address data type and operations, among these ``things'' we also
find the declaration (specification)  of these operations,  the
properties (which may represent 
requirements) and their proofs. 

We now describe each of these ``things'', called {\em methods}.

\begin{itemize}
  \item The {\em method} introduced by the keyword
    {\tt representation} gives the data representation of entities
    manipulated by the {\em species}. It is a type called the
    {\em representation} (or the representation type when emphasising
    on the fact that it is a type) and defined by a type expression. 
    The {\em representation} may be not-yet-defined in a {\em species}, 
    meaning that the real structure of the data-type the {\em species}
    embeds does not  need to be known at this point. In this case, it is
    simply a type variable.  However, to obtain an
    implementation,  the {\em representation}
     has to be defined later either by setting {\tt representation = exp} where
     {\tt  exp} is a type expression or by inheritance (see below). 
     Type expressions in \focal\ are roughly ML-like types (variables,
     basic types, inductive types, record types) plus
     {\em species representation types}, denoted by keyword {\tt Self}
     inside the species and by  the name of their {\em species}
     outside of them.

    Each {\em species} has a unique method {\em representation}.  This
    is not a restriction compared to other languages where
    programs/objects/modules can own several private variables
    representing the internal state, hence the data structure of the
    manipulated entities by the program/object/module. In such a case,
    the {\em representation} can simply be the tuple grouping all these
    variables that were disseminated all along the
    program/object/module.


  \item Declarations are composed of the keyword {\tt signature}
    followed by a name and a type. It serves to announce a {\em method} to
    be defined later, i.e. to only specify its type, without
    implementation yet. Such {\em methods} are especially dedicated
    for specification or design purposes since declared names may be
    used  to define others {\em methods} while
    delaying their definition. The type provided by
    the {\em signature} allows \focal\ to ensure via type-checking
    that the method is used in contexts compatibles with this
    type. The late-binding and the collection mechanisms, further
    introduced, ensure that the definition of the method will be
    effectively known when needed.

  \item Definitions are composed of the keyword {\tt let}, followed
    by a name, a type and an expression. They serve to introduce
    constants or functions, i.e. computational operations. The core
    language used to implement them is roughly ML-like expressions
    (let-binding, pattern matching, conditional, higher order
    functions, \ldots) with the addition of a construction to call a
    {\em method} from a given {\em species}.  Mutually recursive
    definitions are introduced by {\tt let rec}.



  \item Statements are composed of the keyword {\tt property}
    followed by a name and a first-order
    formula. A {\em property} may serve to express requirements
    (i.e. facts that the system must hold to conform to the Statement
    of Work) and then can be viewed as a
    specification purpose {\em method}, like {\em signature}s were for
    {\tt let}-{\em method}s. It will lead to a proof obligation later
    in the development. A {\em property} may also be used to express
    some ``quality'' information of the system (soundness,
    correctness, ..) also submitted to a proof obligation.  Formulae
    are written with usual logical connectors, universal and
    existential quantifications over a \focal\ type, and names of {\em
      methods} known within the {\em species}'s context. For instance,
    a {\em property} telling that if the speed is non-null, then doors
    can't be opened could look like:
    \begin{center}
    {\tt all v in Speed, v <> Speed!zero -> $\sim$ doors\_open}
    \end{center}
   In the same way as {\em signatures}, even if no proof is
    yet given, the name of the {\em property} can be used to express 
    other ones and its statement can be used  as an hypothesis in
    proofs.  \focal\ late binding and collection mechanisms  ensure
    that the proof of a {\em property} will be ultimately done.

  \item Theorems ({\tt theorem}) made of a  name, a statement and a
    proof are  {\em properties} together  with the formal proof that their
    statement holds in the context of the {\em species}. The proof
    accompanying the statement will be processed by \focal\ and
    ultimately checked with the theorem prover \coq.

    Like in any formal development, one severe difficulty before
    proving is obviously to state a true interesting and meaningful
    statement. For instance, claiming that a piece of software is
    ``formally proved'' as respecting the safety requirements {\tt
      system\_ok} ``since {\bf its} property is demonstrated'' is a
    lie if this property was, for instance, {\tt 1 = 1 ->
      system\_ok}. This is obviously a non-sense since the text of the
    property is trivial and does not link {\tt system\_ok} with the
    rest of the software (see \cite{traps} for less trivial examples).



\end{itemize}

We now make concrete these notions on an example  we will incrementally
extend. We want to model some simple algebraic structures. Let's start
with the description of a ``setoid'' representing the data structure of
``things'' belonging to a set, which can be submitted to an
equality test and
exhibited (i.e. one can get a witness of existence of one of these
``things'').

{\scriptsize
\begin{lstlisting}
species Setoid =
  signature ( = ) : Self -> Self -> bool ;
  signature element : Self ;

  property refl : all x in Self, x = x ;
  property symm : all x y in Self, x = y -> y = x ;
  property trans: all x y z in Self, x=y and y=z -> x=z ;
  let different (x, y) = basics#not_b (x = y) ;

end ;;
\end{lstlisting}
}

In this {\em species},  the {\em representation} is not explicitly
given (no keyword {\tt representation}) , since we don't need to set
it to be able to express functions and properties our ``setoid''
requires. However, we can refer to it via {\tt Self} and it is in fact
a type variable. In the same way, we specify a {\em signature} for the
equality (operator {\tt =}). We introduce the three properties  that
an equality (equivalence relation) must conform to. 

We complete the example by the definition of
the function {\tt different} which use the name {\tt =}
(here {\tt basics\#not\_b} stands for the
function {\tt not\_b}, the boolean {\tt and} coming from the \focal\
source file {\tt basics.fcl}). It is possible right now to prove that
{\tt different} is irreflexive, under the hypothesis that {\tt =} is an
equivalence relation (i.e. that each implementation of {\tt =} given
further will satisfy these properties). 

\bigskip

It is possible to use {\em methods} only declared before they get a
real {\em definition} thanks to the {\em late-binding} feature
provided by \focal. In the same idea,  redefining a {\em method} is
allowed in \focal\ and, it is 
always the last version  which is  kept as the effective {\em definition}
inside the species.

\section{Type of Species, Interfaces and Collections}
\label{species-type}
\label{species-interface}
The {\em type} of a {\em species} is obtained by removing definitions
and proofs. Thus, it is a kind of record type, made of all the method
types of the species. If the {\tt representation} is still a type
variable say $\alpha$, then the {\em species} type is prefixed with an
existential binder $\exists \alpha$. This binder will be eliminated as
soon as the {\tt representation} will be instantiated (defined) and
must be eliminated to obtain runnable code. 

\bigskip

The {\em interface} of a species is obtained by abstracting the {\em
  representation} type in the {\em species type} and this abstraction
is permanent. 

\bigskip

{\bf Beware!} {\em  No special construction
is given to denote interfaces in the concrete syntax, they are
simply denoted by the name of the species underlying them.} Do not
confuse a species and its interface. 

\bigskip

 The {\em
  species type} remain totally implicit in the concrete syntax, being
just used as a step to build {\em species interface}. It is used
during inheritance resolution.  

 Interfaces can be ordered by inclusion, a point providing a very
simple notion of subtyping. This point will be further commented. 


A species is said to be {\em complete} if all declarations have
received definitions and all properties have received proofs.

When {\em complete}, a species can be submitted to an abstraction
process of its representation to create a {\em collection}. Thus the
{\em interface} of the collection is just the {\em interface} of the
complete species underlying it. A collection can hence be seen as an
abstract data type, only usable through the methods of its interface,
but having the guarantee that all methods/theorems are defined/proved. 


\section{Combining Bricks by Inheritance}

A \focal\ development is organised as a hierarchy which may have
several roots. Usually the upper levels of the hierarchy are built
during the specification stage while the lower ones correspond to
implementations. Each node of the hierarchy, i.e. each {\em species},
is a progress to a complete implementation. On the previous  
example, forgetting {\tt different},  we typically presented a kind of
{\em species} for ``specification'' since it expressed only
{\em signatures} of functions to be later implemented and properties
to which, later, give {\em proofs}.

We can now create a new {\em species}, may be more complex, by
{\bf inheritance} of a previously defined. We say here ``may be more
complex'' because it can add new operations and properties, but it can
also only bring real definitions to {\em signatures} and {\em proofs}
to {\em properties}, adding no new {\em method}.

Hence, in \focal\ inheritance serves two kinds of evolutions. In the
first case the evolution aims making a {\em species} with more
operations but keeping those of its parents (or redefining some of
them). In the second case, the {\em species} only tends to be closer
to a ``run-able'' implementation, providing explicit definitions to
{\em methods} that were previously only declared.


 

Continuing our example, we want to extend our model to represent
``things'' with a multiplication and a neutral element for this
operation.

{\scriptsize
\begin{lstlisting}
species Monoid inherits Setoid =
  signature ( * ) : Self -> Self -> Self ;
  signature one : Self ;
  let element = one * one ;
end ;;
\end{lstlisting}
}

We see here that we added new {\em methods} but also gave a definition
to {\tt element}, saying it is the application of the method {\tt *}
to {\tt one} twice, both of them being only {\em declared}. Here, we
used the inheritance in both the presented ways: making a more complex
entity by adding {\em methods} and getting closer to the
implementation by explicitly defining {\tt element}.

Multiple inheritance is available in \focal. For sake of simplicity,
the above example uses simple inheritance. In case of inheriting a
{\em method} from several parents, the order of parents in the
{\tt inherits} clause serves to determine the chosen {\em method}.

The {\em type} of a {\em  species} built using inheritance is defined
like for other {\em species}, the {\em methods} types retained inside
it being those of the {\em methods} present in the {\em species} after
inheritance is resolved.

A strong constraint in inheritance is that the type of inherited,
and/or redefined {\em methods} must not change.  This is required to
ensure consistence of the \focal\ model, hence of the developed
software. More precisely, if the representation is given by a type
expression containing some type variables, then it can be more defined
by instanciation of these variables. In the same way, two signatures
have compatible types if they have a common unifier, thus, roughly
speaking if they are compatible Ml-like types. For example, if the
representation was not yet defined, thus being still a type variable,
it can be defined by {\tt int}. And if a species $S$ inherits from
$S_1$ and $S2$ a method called $m$, there is no type clash if $S_1 !m$
and $S_2!m$ can be unified, then the method $S!m$ has the most general
unifier of these two types as its own type.  



\section{Combining Bricks by Parametrisation}

Until now we are only able to enrich {\em species}. 
However, we sometimes need to use a {\em species}, not to take over
its {\em methods}, but rather to use it as an ``ingredient'' to build
a new structure. For instance, a pair of setoids is a
new structure, using the previous {\em species} as the ``ingredient''
to create the structure of the pair. Indeed, the structure of a pair is
independent of the structure of each component it is made of. A pair
can be seen as {\em parametrised} by its two components.
Following this idea, \focal\ allows two flavors of parametrisation.

\subsection{Parametrisation by Collection Parameters}

We first introduce the {\em collection parameters}. They are
{\em collections} that the hosting species may use through their
{\em methods} to define its own ones.


 A {\em collection parameter} is given a name $C$ and an interface
$I$.  The name $C$ serves to call the {\em methods} of $C$ which figure in
$I$. $C$ can be instantiated by an effective parameter $CE$ of
interface $IE$. $CE$ is a collection and its interface $IE$ must
contain $I$.  Moreover, the collection and late-binding mechanisms
ensure that all methods appearing in $I$ are indeed implemented
(defined for functions, proved for properties) in $CE$. Thus, no
runtime error, due to linkage of libraries, can occur and any {\em
properties} stated in $I$ can be safely used as an hypothesis.

 Calling a {\em species}'s {\em method} is
done via the ``bang'' notation:
{\tt !meth} or\break{}{\tt Self!meth} for a {\em method} of the current
{\em species} (and in this case, even simpler: {\tt meth}, since the
\focal\ compiler will resolve scoping issues). To call
{\em collection parameters}'s {\em method}, the same notation is used:
{\tt A!element} stands for the {\em method} {\tt element} of the
{\em collection parameter} {\tt A}.


To go on with our example, a pair of setoids has two components, hence a
{\em species} for pairs of setoids will have two
{\em collection parameters}. It is itself a setoid, a fact which is
simply recorded via the inheritance mechanism:
{\tt inherits Setoid} gives to {\tt Setoid\_product} all the methods
of {\tt Setoid}.

{\scriptsize
\begin{lstlisting}
species Setoid_product (A is Setoid, B is Setoid) inherits Setoid =
  representation = (A * B) ;

  let ( = ) (x, y) =
     and_b
       (A!( = ) (first (x), first (y)),
        B!( = ) (scnd (x), scnd (y))) ;
  let create (x, y) in Self = basics#crp (x, y) ;
  let element = Self!create (A!element, B!element) ;

  proof of refl = by definition of ( = ) ;
end ;;
\end{lstlisting}
}

We express the {\em representation} of the product of two setoids as the
Cartesian product of  the {\em representation} of the two parameters. In
{\tt A * B},  {\tt *} is the \focal\ type �constructor of pairs, {\tt
  A} denotes indeed 
the representation of the  first {\em collection parameter}, and {\tt B}
the one of  of the second {\em collection parameter}.

Next, we add a definition for {\tt =} of {\tt Setoid\_product},
relying on the methods {\tt =} of {\tt A} ({\tt A!( = )}) and {\tt B}
(which are not yet defined).  Similarly, we introduce a definition for
{\tt element} by building a pair, using
the function  {\tt create} (which calls the predefined function {\tt
  basics\#crp}) and the methods {\tt element} of respectively {\tt A}
and {\tt B}. And we can prove that {\tt =} of {\tt Setoid\_product} is
indeed reflexive, upon the hypothesis made on {\tt A!( = )}
and {\tt B!( = )}.  The part of \focal\ used to write proofs will be
shortly presented later, in section \ref{focal-proof-language}.


 This way, the {\em species} {\tt Setoid\_product} builds its {\em
methods} relying on those of its {\em collection parameters}. Note the
two different uses of {\tt Setoid} in our {\em species} {\tt
Setoid\_product}, which inherits of {\tt Setoid} and is parametrised
by {\tt Setoid}.

\bigskip

Why such {\em collection parameters} and not simply {\em species
  parameters}? There are two reasons. First, effective parameters must
  provide definitions/proofs for all the methods of the required
  interface: this is the contract. Thus, effective parameters must 
  be {\em complete} species. Then,  we do not want the parametrisation
  to introduce dependencies
on the parameters' {\em representation} definitions. For example, it is
  impossible to express `` if {\tt A!representation} is {\tt int} and {\tt B!representation}
  is {\tt bool} then {\tt A*B} is a list of boolean values''. This would
dramatically restrict possibilities to instantiate parameters since
assumptions on the {\em representation},  possibly used in the
parametrised {\em species} to write its own {\em methods}, 
could prevent {\em collections} having the right set of {\em methods} but
a different  representation  to be used as
effective parameters. Such a behaviour would make parametrisation too
weak to be usable. We choose to always hide  the {\em representation} of a
{\em collection parameter}  to the parametrised
hosting {\em species}. Hence the introduction of the  notion of
{\em collection}, obtained by abstracting the representation from a
complete species.

\subsection{Parametrisation by Entity Parameters}

Let us imagine we want to make a {\em species} working on natural numbers
modulo a certain value.  In the expression
$5 {\tt\ modulo\ } 2\ is\ 1$, both $5$ and $2$
are natural numbers. To be sure that the {\em species} will
consistently work with the same modulo, this last one must be embedded
in the {\em species}. However, the {\em species} itself doesn't rely
on a particular value of the modulo. Hence this value is clearly a
{\bf parameter} of the species, but a parameter in which we are
interested by its {\bf value}, not only by its {\em representation} and the
methods acting on it. We call
such parameters {\em entity parameters}, their introduction rests upon
the introduction  of a {\em collection parameter} and they  denote a
{\em value} having the type of the {\em  representation} of this
{\em collection  parameter}.  

Let us first have a {\em species} representing natural numbers:

{\scriptsize
\begin{lstlisting}
species IntModel =
  signature one : Self ;
  signature modulo : Self -> Self -> Self ;
end ;;
\end{lstlisting}
}
Note that {\tt IntModel} can be later implemented in various ways,
using Peano's integers, machine integers, arbitrary-precision
arithmetic \ldots

We now build our {\em species} ``working modulo \ldots'', embedding
the value of this modulo:
{\scriptsize
\begin{lstlisting}
species Modulo_work (Naturals is IntModel, n in Naturals) =
  let job1 (x in Naturals) in ... =
    ... Naturals!modulo (x, n) ... ;
  let job2 (x in Naturals, ...) in ... =
    ... ... Naturals!modulo (x, n) ... ... ;
end ;;
\end{lstlisting}
}
Using the {\em entity parameter} {\tt n}, we ensure that the
{\em species} {\tt Modulo\_work} works for {\em any} value of the
modulo, but will always use the {\em same} value {\tt n} of the modulo
everywhere inside the {\em species}.


\section{The Final Brick}

As briefly introduced in \ref{species-interface}, a {\em species}
needs to be fully defined to lead to executable code for its functions
and checkable proofs for its theorems. When a {\em species} is fully
defined, it can be turned into a {\em collection}. Hence, a {\em collection}
represents the final stage of the inheritance tree of a {\em species}
and leads to an effective data representation  with
executable functions processing it.




For instance, providing that the previous
{\em species} {\tt IntModel} turned into a fully-defined species
{\tt MachineNativeInt} through inheritances steps, with a {\em method}
{\tt from\_string} allowing to create the natural representation of a
string, we could get a related collection by:

{\scriptsize
\begin{lstlisting}
collection MachineNativeIntColl implements MachineNativeInt ;;
\end{lstlisting}
}

Next, to get a {\em collection} implementing arithmetic modulo 8, we
could extract from the {\em species} {\tt Modulo\_work} the following
{\em collection}:

{\scriptsize
\begin{lstlisting}
collection Modulo_8_work implements Modulo_work
   (MachineNativeIntColl, MachineNativeIntColl!from_string (``8'') ;;
\end{lstlisting}
}

As seen by this example, a species can be applied to effective
parameters by giving their values with the usual syntax of parameter
passing. 

As said before, to ensure modularity and abstraction, the
{\em representation} of a 
{\em collection} turns hidden. This means that any software component
dealing with a {\em collection} will only be able to manipulate it
through the operations ({\em methods}) its interface provides. This
point is especially important since it prevents other software
components from possibly breaking invariants required by the internals
of the {\em collection}.



\section{Properties,  Theorems and  Proofs}
\label{focal-proof-language}

\focal\ aims not only to write programs, it intends to encompass both
the executable model (i.e. program) and properties this model must
satisfy. For this reason, ``special'' {\em methods} deal with logic
instead of purely behavioural aspects of the system: {\em theorems},
{\em properties} and {\em proofs}.

Stating a {\em property} expects that a {\em proof} that it
{\bf holds} will finally be given. For {\em theorems}, the {\em proof} is
directly embedded in the {\em theorem}. Such proofs must be done by
the developer and will finally be sent to the formal proof assistant
\coq\ who will automatically check that the demonstration of the
{\em property} is consistent. Writing a proof can be done in several ways.

It can be written in ``\focal's proof language'', a hierarchical proof
language that allows to give hints and directions for a proof. This
language will be sent to an external theorem prover,
\zenon~\cite{Zenon,zenon0.4.1} developed by D. Doligez. This prover is
a first order theorem prover based on the tableau method incorporating
implementation novelties such as sharing.  \zenon\ will attempt, from
these hints to automatically generate the proof and exhibit a \coq\
term suitable for verification by \coq. Basic hints given by the
developer to \zenon\ are: ``prove by definition of a {\em method}''
(i.e. looking inside its body) and ``prove by {\em property}''
(i.e. using the logical body of a {\em theorem} or {\em property}''.
Surrounding this hints mechanism, the language allows to build the
proof by stating assumptions (that must obviously be demonstrated
next) that can be used to prove lemmas or parts for the whole
property. We show below an example of such demonstration.

{\scriptsize
\begin{lstlisting}
  theorem order_inf_is_infimum: all x y i in Self,
    !order_inf(i, x) -> !order_inf(i, y) ->
      !order_inf(i, !inf(x, y))
    proof:
      <1>1 assume x in Self, assume y in Self,
           assume i in Self, assume H1: !order_inf(i, x),
           assume H2: !order_inf(i, y),
           prove !order_inf(i, !inf(x, y))
        <2>1 prove !equal(i, !inf(!inf(i, x), y))
          by hypothesis H1, H2
             property inf_left_substitution_rule,
               equal_symmetric, equal_transitive
             definition of order_inf
        <2>9 qed
          by step <2>1
             property inf_is_associative, equal_transitive
             definition of order_inf
      <1>2 qed.
    ;
\end{lstlisting}
}

The important point is that \zenon\ works for the
developer: {\bf it searches the proof itself}, the developer does not
have to elaborate it formally ``from scratch''.

Like any automatic theorem prover, \zenon\ may fail finding a
demonstration. In this case, \focal\ allows to write verbatim
\coq\ proofs. In this case, the proof is not anymore automated, but
this leaves the full power of expression of \coq\ to the developer.

Finally, the {\tt assumed} keyword is the ultimate proof backdoor,
telling that the proof is not given but that the property must be
admitted. Obviously, a really safe development should not make usage of
such ``proofs'' since they bypass the formal verification of
software's model. However, such a functionality remains needed since
some of ``well-known'' properties can never be proved for a computer.
For instance, $\forall x \in \bbbn, x+1 >n$ does not hold in a
computer with native integers. However, in a mathematical
framework, this property holds and is needed to carry out other
proofs. Thus the developer may prove either that all manipulated values
remain in an interval where this property holds or may admit this
property or may add code to detect overflow ... 
 On another side, a development may be linked with external
code, trusted or not, but for which properties cannot be proved inside
the \focal\ part since it does not belong to it. Expressing properties
of the \focal\ part may need to express properties on the imported
code, that cannot be formally proved, then must be ``assumed''.

\section{Around the Language}

In the previous sections, we presented \focal\ through its programming
model and shortly its syntax. We especially investigated the various
entities making a \focal\ program. We now address what becomes a
\focal\ program once compiled. We recall that \focal\ supports the
redefinition of functions, which permits for example  to specialise
code to a specific representation (for example, there
exists a generic implementation of integer addition modulo {\tt n} but
it can be redefined in arithmetics modulo {\tt 2} if boolean values
are used to represent the two values). It is also a very convenient
tool to maintain software. 

\subsection{Consistency of the Software}

All along the development cycle of a \focal\ program, the compiler
keeps trace of dependencies between {\em species}, their
{\em methods}, the {\em proofs}, \ldots to ensure that modifications
of one of them will be detected  those depending of it.

\focal\ considers two types of dependencies:
\begin{itemize}
\item The {\bf decl}-dependency: a {\em method} $A$ decl-depends on a
  {\em method} $B$, if the {\bf declaration} of $B$ is required to
  state $A$.
\item The {\bf def}-dependency: a {\em method} (and more especially, a
  {\em theorem}) $A$ def-depends on a {\em method} $B$, if the
  {\bf definition} of $B$ is required to state $A$ (and more
  especially, to prove the property stated by the {\em theorem}
  $A$).
\end{itemize}

The redefinition of a function may invalidate the proofs that use
properties of the body of the redefined function.  All the proofs
which truly depend of the definition are then erased by the compiler
and must be done again in the context updated with the new
definition. Thus the main difficulty is to choose the best level in
the hierarchy to do a proof. In \cite{PrevostoJaume2003}, Prevosto and
Jaume propose a \emph{coding style} to minimise the number of proofs
to be redone in the case of a redefinition, by a certain kind of
modularisation of the proofs.

\subsection{Code Generation}

\focal\ currently compiles programs toward two languages, \ocaml\ to
get an executable piece of software, and \coq\ to have a formal model
of the program, with theorems and proofs.

In \ocaml\ code generation, all
the logical aspects are discarded since they do not lead to executable
code.

Conversely, in \coq, all the {\em methods} are compiled,
i.e. ``computational'' {\em methods} and logical {\em methods} with
their proofs. This allows \coq\ to check the entire consistence of
the system developed in \focal.


\def\focaltest{FocalTest}

\subsection{Tests}
\focal\ incorporates a tool named {\em \focaltest}
\cite{CarlierDuboisLNCS2008} for Integration/Validation testing. It
allows to confront automatically a property of the specification with
an implementation. It generates automatically test cases, executes
them and produces a test report as an XML document. The property under
test is used to generate the test cases, it also serves as an
oracle. When a test case fails, it means a counterexample of the
property has been found: the implantation does not match the property;
it can also indicate an error in the specification.
 
The tool {\em \focaltest} automatically produces the test environment and
the drivers to conduct the tests.  We benefit from the inheritance
mechanism to isolate the testing harness from the components written
by the programmer.

The testable properties are required to be broken down into a
precondition and a conclusion, both executable.
{\em \focaltest} proposes a pure random test cases generation: it
generates test cases until the precondition is satisfied, the verdict
of the test case is given by executing the post-condition. It can be
an expensive process for some kind of preconditions. To overcome this
drawback, a constraint based generation is under development: it
allows to produce directly test cases for which the precondition is
satisfied.



\subsection{Documentation}

The tool called \focdoc\ \cite{MaarekCalculemus03} automatically
generates documentation, thus the documentation of a component is
always coherent with respect to its implementation.

This tool uses its own XML format that contains information coming not
only from structured comments (that are parsed and kept in the
program's abstract syntax tree) and \focal\ concrete syntax but also
from type inference and dependence analysis. From this XML
representation and thanks to some XSLT stylesheets, it is possible to
generate HTML files or \LaTeX{} files. Although this documentation is
not the complete safety case, it can helpfully contribute to its
elaboration. In the same way, it is possible to produce UML
models~\cite{Focal-UML} as means to provide a graphical documentation
for \focal{} specifications. The use of graphical notations appears
quite useful when interacting with end-users, as these tend to be more
intuitive and are easier to grasp than their formal (or textual)
counterparts. This transformation is based on a formal schema and
captures every aspect of the \focal{} language, so that it has been
possible to prove the soundness of this transformation (semantic
preservation).

\focal's architecture is designed to easily plug third-parties
analyses that can use the internal structures elaborated by the
compiler from the source code. This allows, for example, to make
dedicated documentation tools for custom purposes, just exploiting
information stored in the \focal\ program's abstract syntax tree, or
extra information possibly added by extra processes, analyses.
