% $Id: more_on_meths.tex,v 1.13 2009-08-25 12:38:07 weis Exp $

%%%%%%%%%%%
\subsection{More about methods definition}
We will now examine more technical points in methods definitions.
\subsubsection{Well-formation}
\label{well-formation}
\index{well-formation}

{\focal} providing late-binding, it is possible to {\bf declare} a
method {\tt m0} and use it in another {\bf defined method} {\tt m1}.

{\scriptsize
\begin{lstlisting}
species S0 =
  signature m0 : Self ;
  let m1 = m0 ;
end ;;
\end{lstlisting}
}

In another species {\tt S1}, it is also possible to {\bf declare} a
method {\tt m1} and use it in another {\bf defined method} {\tt m0}.
{\scriptsize
\begin{lstlisting}
species S1 =
  inherit S0;
  signature m1 : Self ;
  let m0 = x ;
end ;;
\end{lstlisting}
}

As long as these two species have no interactions no problem can arise. Now, we
consider a third species {\tt S2} inheriting from both {\tt S0} and
{\tt S1}.
{\scriptsize
\begin{lstlisting}
species S2 =
  inherit S0, S1;
  ...
end ;;
\end{lstlisting}
}

The inheritance mechanism will take each method {\bf definition} from
its hosting species: from {\tt S0} for {\tt m1} and from {\tt S1} for
{\tt m2}. We have hence a configuration where {\tt m0} calls {\tt m1}
and {\tt m1} calls {\tt m0}, i.e. the two methods are now mutually
recursive although it was not the case where each of them was
{\bf defined}.

To avoid this situation, we will say that a species is well-formed if
and only if, once inheritance is resolved, no method initially not
recursive turns to become recursive. The {\focal} compiler performs this
analysis and rejects any species that is not compliant to this
criterion. In the above example, an error would be raised, explaining
how the mutual recursion (the cycle of dependencies) appears,
i.e. from {\tt m1} to {\tt m0} (and implicitly back to {\tt m1} from
{\tt m0}).

\noindent
{\scriptsize
{\tt
Species 'S2' is not well-formed. Field
'm1' involves a non-declared recursion\\
for the following dependent fields: m1 -> m0.
}
}

%%%%%%%%%%%%%%
\subsubsection{Def-dependencies on the representation}
\label{def-dep-on-representation}
\index{dependency!def!on representation}
As we previously said (cf. \ref{def-dependency}) def-dependencies on
the representation are not allowed in properties and theorems. The reason
comes from the need to create consistent species interfaces. Let's
consider the following species with the definitions:
{\scriptsize
\begin{lstlisting}
species Counter =
  representation = int ;
  let inc (x in Self) = x + 1 ;
  theorem inc_spec : all x in Self, inc (x) >= x + 1
    proof = ... ;
end ;;
\end{lstlisting}
}

The statement of {\tt inc\_spec} contains a def-dependency on the
representation since to type-check this statement, one need to know
that the representation is  {\tt int}. To create the species'
interface, we must make the representation abstract, hence hiding the
fact that it is {\tt int}. Without this information it it now
impossible to type-check  {\tt inc\_spec}  body since it
makes explicit reference to {\tt +}, {\tt <=}, {\tt 1} that are
operations about {\tt int}.

In practice, such an error is reported as a typechecking error telling
that {\tt representation} ``is not compatible with type'' {\tt t}
where {\tt t} is the type expression that was assigned to the
representation (i.e. {\tt int} in our example).
