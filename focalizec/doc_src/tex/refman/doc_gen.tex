% $Id: doc_gen.tex,v 1.1 2009-01-14 15:00:12 pessaux Exp $


When invoked with the {\tt $-$focalize-doc} option, the command
{\tt focalizec} generates an extra file (with the ``.fcd'' suffix)
containing ``documentation'' information extracted from the compiled
source file.

This information describes the different elements found in the source
file (species, collections, methods, toplevel definitions, type
definitions) with various annotations like type,
definition/inheritance locations. It also contains the comments
previously called ``documentations'' (c.f \ref{documentation}) and that
were kept during the compilation process. Moreover, these comments can
contain special markups used by the documentation generator of \focal.



%%%%%%%%%%%%%%%%%%%%%%%%%%%%%%%%%%%%%%%%%%%%%%%%%%%%%%%%%%%%%%%%%%%%%%%
%%%%%%%%%%%%%%%%%%%%%%%%%%%%%%%%%%%%%%%%%%%%%%%%%%%%%%%%%%%%%%%%%%%%%%%
\subsection{Special markups}




%%%%%%%%%%%%%%%%%%%%%%%%%%%%%%%%%%%%%%%%%%%%%%%%%%%%%%%%%%%%%%%%%%%%%%%
%%%%%%%%%%%%%%%%%%%%%%%%%%%%%%%%%%%%%%%%%%%%%%%%%%%%%%%%%%%%%%%%%%%%%%%
\subsection{Transforming the generated documentation file}
The generated documentation file is a plain ASCII text containing some
XML compliant with \focal's DTD
({\tt focalize/focalizec/src/docgen/focdoc.dtd}). Like for any XML
files processing is performed thank to the command \xsltproc\ with 
XSL stylesheets (``.xsl'' files).

You may write custom XSL stylesheets to process this XML but the
distribution already provides 2 stylesheets to format this
information.



%%%%%%%%%%%%%%%%%%%%%%%%%%%%%%%%%%%%%%%%%%%%%%%%%%%%%%%%%%%%%%%%%%%%%%%
\subsubsection{XML to HTML}
Transformation from ``.fcd'' to a format that can be read by a WEB
browser is performed in two passes.
\begin{enumerate}
  \item Convert the ``.fcl'' file to HTML with MathML annotations.
  This is done applying the stylesheet
  {\tt focalize/focalizec/src/docgen/focdoc2html.xsl} with the command
  \xsltproc.

  For example:
  {\scriptsize
  \begin{verbatim}
  xsltproc ''directory to the stylesheet''/focdoc2html.xsl mysrc.fcd > tmp
  \end{verbatim}
  }

  \item Convert the HTML+MathML temporary file into HTML.
  This is done applying the stylesheet
  {\tt focalize/focalizec/src/docgen/focdoc2html.xsl} with the command
  \xsltproc.

  For example:
  {\scriptsize
  \begin{verbatim}
  xsltproc ''directory to the stylesheet''/mmlctop2_0.xsl mysrc.fcd > mysrc.xml
  \end{verbatim}
  }
  You may note that the final result file name must be ended by the
  suffix ``.xml'' otherwise your browser won't be able to interpret it
  correctly and won't display symbols ($\Rightarrow, \in, \exists,
  \rightarrow, \ldots$) correctly.
\end{enumerate}

\subsection{XML to LaTeX}
