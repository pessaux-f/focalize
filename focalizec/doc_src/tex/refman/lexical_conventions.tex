% $Id: lexical_conventions.tex,v 1.27 2012-10-30 12:27:54 pessaux Exp $

%%%%%%%%%%%%%%%%%%%%%%%%%%%%%%%%%%%%%%%%%%%%%%%%%%%%%%%%%%%%%%%%%%%%%%%
\section{Lexical conventions}
\index{lexical conventions}

%%%%%%%%%%%
\subsection{Blanks}
\index{blank}
The following characters are considered as blanks: space, newline,
horizontal tabulation, carriage return, line feed and form
feed. Blanks are ignored, but they separate adjacent identifiers,
literals and keywords that would otherwise be confused as one single
identifier, literal or keyword.

%%%%%%%%%%%
\subsection{Escaped characters}
\index{escaped character}



%%%%%%%%%%%
\subsection{Comments}
\index{comment}

Comments are treated as blanks and discarded during the compilation process.
{\focal} features two kinds of comments, {\em uni-line comments} introduced by
{\tt --} and {\em general comments} enclosed between {\tt (*} and {\tt *)}.

Note: The three character sequences {\tt (*}, {\tt *)}, and {\tt --} are
named {\em comment delimiters}.
Due to their particular lexical role the comment delimiters must be escaped in
other regular tokens, in particular string and character literals.

\subsubsection{General comments}

The two characters {\tt (*}, with no intervening blanks, start a {\em
general comment}; the two characters {\tt *)}, with no intervening blanks,
close a general comment. General comments may span on any number of lines and
may be arbitrarily nested. In addition, any legal {\focal} program may be
commented out via a general comment.

Note: Almost arbitrary text can be written inside a general comment, therefore
general comments are used to add explanations in the code to help the reader.

Example:

{\scriptsize
\begin{lstlisting}
(* The main species of the development: S.
   S contains only one method, m, since m is general enough to perform the
   entire work under any circumstance. *)
species S =
  ...
  let m (x in Self) = (* Another useful comment *)
  ...
end
;;
(* Another discarded comment at end of file *)
\end{lstlisting}
}

\subsubsection{Uni-line comments}

The two characters {\tt --}, with no intervening blanks, start a {\em uni-line
comment}; a uni-line comment always spreads to the end of the line.

Note: No general comment marker may appear in a uni-line comment.

Example:

{\scriptsize
\begin{lstlisting}
-- Discarded uni-line comment
species S =
  let m (x in Self) = -- The powerful m method.
  ...
end
;;
\end{lstlisting}
}

Note that double quotes (symbol $"$) should not appear in comments, and that
a spanning comment should not start with uni-line comment mark. A uni-line
comment should also always be terminated by a carriage return (an unclosed
uni-line comment cannot end a file).

%%%%%%%%%%%
\subsection{Annotations}
\index{annotation}
\index{documentation}
\label{annotation}
{\bf Annotations} are introduced by the three characters {\tt (**},
with no intervening blanks, and terminated by the two characters
{\tt *)}, with no intervening blanks.
Annotations cannot occur inside string or character literals and
cannot be nested. They must precede the construct they document.
In particular, a {\bf source file cannot end by an annotation}.

Unlike comments, annotations are kept during the compilation process
and recorded in the compilation information (``{\tt .fo}'' files). Annotations can
be processed later on by external tools that could analyse them to
produce a new {\focal} source code accordingly.
For instance, the {\focal} development environment provides the {\focdoc}
automatic production tool that uses annotations to automatically generate
documentation.
Several annotations can be put in sequence for the same construct. We call
such a sequence an {\bf annotation block}\index{annotation!block}.
Using embedded tags in annotations allows third-party tools to easily find
out annotations that are meaningful to them, and safely ignore others.
For more information, consult
\ref{documentation-generation}.
Example:

{\scriptsize
\begin{lstlisting}
(** Annotation for the automatic documentation processor.
    Documentation for species S. *)
species S =
  ...
  let m (x in Self) =
    (** {@TEST} Annotation for the test generator. *)
    (** {@MY_TAG_MAINTAIN} Annotation for maintainers. *)
    ... ;
end ;;
\end{lstlisting}
}

%%%%%%%%%%%
\subsection{Identifiers}
\index{identifier}
\vspace{0.2cm}

{\focal} features a rich class of identifiers with sophisticated lexical
rules that provide fine distinction between the kind of notion a given
identifier can designate.

\subsubsection{Introduction}

Sorting words to find out which kind of meaning they may have is a very common
conceptual categorisation of names that we use when we write or read ordinary
English texts. We routinely distinguish between:
\begin{citemize}
\item a word only made of lowercase characters, that is supposed to be an
  ordinary noun, such as ``table'', ``ball'', or a verb as in ``is'', or an
  adjective as in ``green'',
\item a word starting with an uppercase letter, that is supposed to be a name,
  maybe a family or Christian name, as in ``Kennedy'' or ``David'', or a location
  name as in ``London''.
\end{citemize}

We use this distinctive look of words as a useful hint to help understanding
phrases. For instance, we accept the phrase "my ball is green" as meaningful,
whereas "my Paris is green" is considered a nonsense. This is simply because
"ball" is a regular noun and "Paris" is a name. The word "ball" as the right
lexical classification in the phrase, but "Paris" has not. This is also clear
that you can replace "ball" by another ordinary noun and get something
meaningful: "my table is green"; the same nonsense arises as well if you
replace "Paris" by another name: "my Kennedy is green".

Natural languages are far more complicated than computer languages, but
{\focal} uses the same kind of tricks: the ``look'' of words helps a lot to
understand what the words are designating and how they can be used.

\subsubsection{Conceptual properties of names}

{\focal} distinguishes 4 concepts for each name:

\begin{citemize}
\item the {\em fixity} assigns the place where an identifier must be written,
\item the {\em precedence} decides the order of operations when
  identifiers are combined together,
\item the {\em categorisation} fixes which concept the identifier designates.
\item the {\em nature} of a name can either be symbolic or alphanumeric.
\end{citemize}

Those concepts are compositional, i.e. all these concepts are independent
from one another. Put is another way: for any fixity, precedence, category
and nature, there exist identifiers with this exact set of properties.

We further explain those concepts below.

\subsubsection{Fixity of identifiers}
\index{fixity of identifiers}
\index{infix identifier}
\index{prefix identifier}

The fixity of an identifier answers to the question ``where this identifier
must be written ?''.

\begin{citemize}
\item a {\em prefix} is written {\em before} its argument, as $sin$ in
 $sin\; x$ or $-$ in $- y$,
\item an {\em infix} is written {\em between} its arguments, as $+$ in
 $x\; +\; y$ or $mod$ in $x\; mod \;3$.
\item a {\em mix-fix} is written {\em among} its arguments, as
  {\em if ... then ... else ...} in
  {\em if\ c then 1 else 2}.
\end{citemize}

In {\focal}, all the ordinary identifiers are prefix and all the binary
arithmetics operators are infix as they are in mathematics.

\subsubsection{Precedence of identifiers}
\index{precedence of identifiers}

The precedence rules out where implicit parentheses take place in a
complex combination of symbols. For instance, according to the usual mathematical
conventions:
\begin{citemize}
\item $1\; +\; 2\; *\; 3$  means $1\; +\; (2\; *\; 3)$ hence $7$,
      it does not mean $(1\; +\; 2)\; *\; 3$ which is $9$,
\item $2\; *\; 3\; ^4\; +\; 5$ means
      $(2\; *\; (3\; ^4))\; +\; 5$ hence $167$, it does not mean
      $((2\; *\; 3)\; ^4)\; +\; 5$ which is $1301$,
      nor $2\; *\; (3\; ^{(4\; +\; 5)})$ which is $39366$.
\end{citemize}

In {\focal}, all the binary infix operators have the precedence they have in mathematics.

\subsubsection{Categorisation of identifiers}

\index{category of identifiers}
The category of an identifier answers to the question ``is this identifier a
possible name for this kind of concept ?''.
In programming languages categories are often strict, meaning that the category
exactly states which concept attaches to the identifier.

For {\focal} there are two categories of identifiers, the {\em lowercase} and
the {\em uppercase} identifiers.

\begin{citemize}

\item a {\em lowercase identifier} designates a simple entity of the
  language. It may name some of the language expressions, a function name, a
  function parameter or bound variable name, a method name, a type name, or a
  record field label name.

\item an {\em uppercase identifier} designates a more complex entity in the
  language. It may name a sum type constructor name, a module name, a species
  or a collection name.

\end{citemize}

Roughly speaking, the first letter of an identifier fixes its category:
if the first letter is lowercase the identifier is lowercase, and conversely
if an identifier starts with an uppercase letter it is indeed an uppercase identifier.

More precisely, we classify identifiers by looking at their
{\em starter character}: the starter character is the first character of the identifier
that is not an underscore. Considering underscores as some sort of spacing
marks into names, the starter character is the first ``meaningful'' character
of the identifier,

Recall that the category of identifiers is orthogonal to the rest of their
properties. For instance, the fixity of an identifier does not tell if the
identifier is lowercase or not. Put it another way: as stated above, {\tt +}
is infix but this does not say if {\tt +} is a lowercase or uppercase
identifier! (In fact {\tt +} is lowercase, since as stated below its {\tt
  '+'} starter character is a lowercase symbolic character.)

\subsubsection{Nature of identifiers}
\index{nature of identifiers}

In {\focal} identifiers are either:

\begin{citemize}
\item {\em symbolic}: the identifier contains characters that are not
  letters. {\tt +}, {\tt :=}, {\tt ->}, {\tt +float} are symbolic.

\item {\em alphanumeric}: the identifier only contains letters, digits and
  underscores. {\tt x}, {\tt \_1}, {\tt Some}, {\tt Basic\_object} are
  alphanumeric.
\end{citemize}

\subsubsection{Alphanumeric identifiers}
\index{regular identifier}
\index{alphanumeric identifier}

An alphanumeric identifier is a sequence of letters, digits, and
{\tt \_} (the underscore character).
Letters contain at least the 52 lowercase and uppercase letters from the
standard ASCII set. In an identifier, all characters are meaningful.

{\bf Alphanumeric lowercase identifiers} designate the names of variables, functions,
types. and labels of records.

{\bf Alphanumeric uppercase identifiers} designate the names of constructors, species,
and collections.

\begin{syn}
\nt{digit} \is \tok{0} \ldots \tok{9}
\sep
\nt{lower} \is \tok{a} \ldots \tok{z}
\sep
\nt{upper} \is \tok{A} \ldots \tok{Z}
\sep
\nt{letter} \is \nt{lower} \orelse \nt{upper}
\sep
\nt{lident} \is \rep{ \tok{\_}} \nt{lower}
                \rep{ \nt{letter} \orelse \nt{digit} \orelse \tok{\_} }
\sep
\nt{uident} \is \rep{ \tok{\_}} \nt{upper}
                \rep{ \nt{letter} \orelse \nt{digit} \orelse \tok{\_} }
\sep
\nt{ident} \is \nt{lident} \orelse \nt{uident}
\end{syn}
\vspace{0.2cm}

Roughly speaking, an alphanumeric lowercase identifier is a sequence of
letters, digits, and {\tt \_} (the underscore character), starting with a
lowercase alphanumeric letter (a lowercase letter, a digit, or an
underscore).

More precisely, an alphanumeric identifier is lowercase if its starter
character (or first ``meaningful'' letter) is lowercase.

Examples: {\tt foo}, {\tt bar}, {\tt \_20},
{\tt \_\_\_gee\_42} are lowercase alphanumeric identifiers;
{\tt foo}, {\tt bar}, {\tt \_20},

Roughly speaking, an alphanumeric uppercase identifier is a sequence of
letters, digits, and {\tt \_} (the underscore character), starting with an
uppercase letter.

More precisely, an alphanumeric identifier is uppercase if its starter
character (or first ``meaningful'' letter) is uppercase.

Examples: {\tt Some}, {\tt None}, {\tt \_One\_}, {\tt Basic\_object},
{\tt \_\_\_GEE\_42} are uppercase alphanumeric identifiers.

\subsubsection{Infix/prefix operators}
\index{identifier!operator}\index{operator}

{\focal} allows infix and prefix operators built from a
``starting operator character'' and followed by a sequence of
regular identifiers or operator characters. For example, all the
following are legal operators:
{\tt +}, {\tt ++}, {\tt \tilde+zero}, {\tt =\_mod\_5}.

The position in which to use the operator (i.e. infix or prefix)
is determined by the position of the first operator character
according to the following table:
\begin{center}
\begin{tabular}{|c|c|}
\hline
Prefix & Infix \\
\hline
` \tilde{} ? \$ ! \# &
, + - * / \% \& $\vertical$ : ; $<$ = $>$ @ \chapeau{} $\backslash$ \\
\hline
\end{tabular}
\end{center}

\begin{syn}
\nt{prefix-char} \is
  \tok{`} \orelse \tok{\tilde} \orelse \tok{?} \orelse \tok{\$} \orelse
  \tok{!} \orelse \tok{\#}
\sep
\nt{infix-char} \is
  \tok{,} \orelse \tok{+} \orelse \tok{-} \orelse \tok{*} \orelse
  \tok{/} \orelse \tok{\%} \orelse \tok{\&} \orelse \tok{\vertical} \orelse
  \tok{:} \orelse \tok{;} \orelse \tok{<} \orelse \tok{=} \orelse
  \tok{>} \orelse \tok{@} \orelse \tok{\chapeau} \orelse
  \tok{\backslash}
\sep
\nt{prefix-op} \is
  \nt{prefix-char} \rep{\nt{letter} \orelse \nt{prefix-char} \orelse
  \nt{infix-char} \orelse \nt{digit} \orelse \tok{\_} }
\sep
\nt{infix-op} \is
  \nt{infix-char} \rep{\nt{letter} \orelse \nt{prefix-char} \orelse
  \nt{infix-char} \orelse \nt{digit} \orelse \tok{\_} }
\sep
\nt{operator} \is
  \nt{infix-op} \orelse \nt{prefix-op}
\end{syn}

% \begin{syntax}
% \syntaxclass{\hspace{1cm} Infix/prefix operators:}
% prefix\_char & ::= & \terminal{` \ \sim\ ?\ \$\ !\ \#\ }& \\
%  infix\_char & ::= &
%     \terminal{ ,\ +\ -\ *\ /\ \%\ \&\ \mid\ :\ ;\ <\ =\ >\ @ \wedge
%               \ \setminus} & \\
%   prefix\_op & ::= & prefix\_char
%     \ \{ letter \mid prefix\_char \mid infix\_char
%     \mid digit \mid \terminal{\_} \}* & \\
%    infix\_op & ::= & infix\_char
%     \ \{ letter \mid prefix\_char \mid infix\_char
%     \mid digit \mid \terminal{\_} \}* & \\
%           op & ::= & infix\_op \mid prefix\_op
% \end{syntax}

Hence, {\tt +}, {\tt ++} and {\tt =\_mod\_5} are infix symbolic operators and
{\tt \tilde+zero} is a prefix symbolic one.

Note that symbolic character starters are classified into disjoint sets of uppercase
and lowercase characters.
\begin{verbatim}
  | '*'
  | '+' | '-'
  | '/' | '%' | '&' | '|' | '<' | '=' | '>' | '@' | '^' | '\\'
\end{verbatim}
are lowercase infix starters;
{\tt ~, ?, \$}
are lowercase prefix starters;
{\tt :, `}
are uppercase infix starters;
{\tt (, [, \{}
are uppercase prefix starters.

For instance, {\tt +} and {\tt @++} are lowercase infix identifiers.
Since its starter is {\tt :}, {\tt ::} is an uppercase infix identifier; hence,
as any other uppercase identifier, {\tt ::} could designate a
value constructor name for a union type. Similarly {\tt []} is an uppercase prefix identifier that also
could designate a value constructor.
On the contrary, no notation starting with a {\tt :} can designate a method name.

\subsubsection{Defining an infix operator}
\index{defining operators}
\index{defining an infix operator}
\index{defining a prefix operator}

The notion of infix/prefix operator does not mean that {\focal} defines all
these operators: it means that the programmer may freely define and use them
as ordinary prefix/infix operators instead of only writing prefix function
names and regular function application. For instance, if you do not like the
{\focal} predefined \verb"^" operator to concatenate strings, you can define your
own infix synonym for \verb"^", say {\tt ++}, using:

{\scriptsize
\begin{lstlisting}
let ( ++ ) (s1, s2) = s1 ^ s2 ;
\end{lstlisting}
}
Then you can use the {\tt ++} operator in the usual way

{\scriptsize
\begin{lstlisting}
let hw = "Hello" ++ " world!" ;
\end{lstlisting}
}

As shown in the example, at definition-time, the syntax requires
the operator to be embraced by parentheses. More precisely, you must
enclose the operator between {\bf spaces} and parentheses.
You must write {\tt ( + )} with spaces, not simply {\tt (+)} (which leads
to a syntax error anyway).

\subsubsection{Prefix form notation}
\index{prefix form notation}
\index{infix in prefix position}

The notation {\tt ( op )} is named the {\em prefix form notation} for
operator {\tt op}.

Since you can only define prefix identifiers in {\focal}, you must use the
prefix form notation to define an infix or prefix operator.

When a prefix or infix operator has been defined, it is still possible
to use it as a regular identifier using its prefix form notation.
For instance, you can use the prefix form of operator {\tt ++}
to apply it in a prefix position as a simple regular function: % (with a
%strange name admittedly!):

{\scriptsize
\begin{lstlisting}
( ++ ) ("Hello", " world!") ;
\end{lstlisting}
}

{\bf Warning!} A common error while defining an operator is to forget
the blanks around the operator. This is particularly confusing, if you
type the {\tt *} operator without blanks around the operator: you
write the lexical entity {\tt (*)} which is the beginning (or the end)
of a comment!

The {\focal} notion of symbolic identifiers goes largely beyond simple
infix operators. Symbolic identifiers let you assign sophisticated names
to your functions and operators.
For instance, instead of creating a function to check if integer {\tt x}
is equal to the predecessor of integer {\tt y}, as in

{\scriptsize
\begin{lstlisting}
let is_eq_to_predecessor (x, y) = ... ;
... if is_eq_to_predecessor (5, 7) ... ;
\end{lstlisting}
}
it is possible to directly define

{\scriptsize
\begin{lstlisting}
let ( =pred ) (x, y) = ... ;
... if 5 =pred 7 ... ;
\end{lstlisting}
}

{\bf Attention} : since a comma can start an infix symbol, be careful
when using commas to add a space after each comma to prevent confusion.
In particular, when using commas to separate tuple components, always type
a space after each comma. For instance, if you write {\tt (1,n)}
then the lexical analyser finds only two words: the integer {\tt 1} as
desired, then the infix operator {\tt ,n} which is certainly not the
intended meaning. Hence, following usual typography rules, always type a
space after a comma (unless you have define a special operator starting
by a comma).

{\bf Rule of thumb}: The prefix version of symbolic identifiers is obtained
by enclosing the symbol between spaces and parens.

%%%%%%%%%%%
\subsection{Extended identifiers}
\label{extended-identifiers}
\index{identifier!extended}

Moreover, {\focal} has special forms of identifiers to allow using
spaces inside or to extend the notion of operator identifiers.
\begin{compact-itemize}
  \item {\bf Delimited alphanumerical identifiers}.
    \index{identifier!delimited}
    They start by two {\tt `} (backquote) characters and end by two
    {\tt '} (quote) characters. In addition to usual alpha-numerical
    characters, the delimited identifiers can have spaces. For example:
    {\tt ``equal is reflexive''}, {\tt ``fermat conjecture''}.
  \item {\bf Delimited symbolic identifiers}.
    They are delimited by the same delimiter characters and contain
    symbolic characters.
\end{compact-itemize}
% [EJ] Certes, mais ``equal is reflexive'' c'est uppercase ou lowercase ?

As usual, the first meaningful character at the beginning of a delimited
identifier rules out its conceptual properties. For instance,
{\tt ``equal is reflexive''} has {\tt e} as its first meaningful character; hence
{\tt ``equal is reflexive''} is alphanumeric, prefix, lowercase, and has the
same precedence as any other regular function.
Similarly, {\tt ``+ for matrices''} starting with the {\tt +} symbol is
symbolic, infix, lowercase, and has the same precedence as the {\tt +} operator.

%%%%%%%%%%%
\subsection{Species and collection names}
\index{species!name} Species, collection
names and collection parameters are uppercase identifiers.

%%%%%%%%%%%
\subsection{Integer literals}
\label{integer literals}
\label{octal}
\label{hexadecimal}
\label{binary}

\begin{syn}
\nt{binary-digit} \is \tok{0} \orelse \tok{1}
\sep
\nt{octal-digit} \is \tok{0} \ldots \tok{7}
\sep
\nt{hexadecimal-digit} \is
  \tok{0} \ldots \tok{9}
  \orelse \tok{A} \ldots \tok{F}
  \orelse \tok{a} \ldots \tok{f}
\sep
\nt{sign} \is \tok{+} \orelse \tok{-}
\sep
\nt{unsigned-binary-literal} \is
  \tok{0} \paren{ \tok{b} \orelse \tok{B} }
  \nt{binary-digit} \rep{ \nt{binary-digit} \orelse \tok{\_} }
\sep
\nt{unsigned-octal-literal} \is
  \tok{0} \paren{ \tok{o} \orelse \tok{O} }
  \nt{octal-digit} \rep{ \nt{octal-digit} \orelse \tok{\_} }
\sep
\nt{unsigned-decimal-literal} \is
  \nt{digit} \rep{ \nt{digit} \orelse \tok{\_} }
\sep
\nt{unsigned-hexadecimal-literal} \is
  \tok{0} \paren{ \tok{x} \orelse \tok{X} }
  \nt{hexadecimal-digit} \rep{ \nt{hexadecimal-digit} \orelse \tok{\_} }
\sep
\nt{unsigned-integer-literal} \is
     \nt{unsigned-binary-literal}
\alt \nt{unsigned-octal-literal}
\alt \nt{unsigned-decimal-literal}
\alt \nt{unsigned-hexadecimal-literal}
\sep
\nt{integer-literal} \is
  \opt{\nt{sign}} \nt{unsigned-integer-literal}
\end{syn}

% \begin{syntax}
% \syntaxclass{Integer literals:}
%                  binary\_digit & ::= & \terminal{0} \mid \terminal{1} & \\
%                   octal\_digit & ::= & \terminal{0} \ldots \terminal{7} & \\
%                 decimal\_digit & ::= & \terminal{0} \ldots \terminal{9} & \\
%             hexadecimal\_digit & ::= & \terminal{0} \ldots \terminal{9}
%                                        \mid \terminal{A} \ldots \terminal{F}
%                                        \mid \terminal{a} \ldots \terminal{f} & \\
%                           sign & ::= & \terminal{+} \mid \terminal{-} & \\
%      unsigned\_binary\_literal & ::= & \terminal{0} \{ \terminal{b}
%                                        \mid \terminal{B} \} \ binary\_digit\ \{ binary\_digit
%                                        \mid \terminal{\_} \}* & \\
%       unsigned\_octal\_literal & ::= & \terminal{0} \{ \terminal{o}
%                                        \mid \terminal{O} \}
%                                        \ octal\_digit\ \{ octal\_digit
%                                        \mid \terminal{\_} \}* & \\
%     unsigned\_decimal\_literal & ::= & decimal\_digit \{ decimal\_digit
%                                        \mid \terminal{\_} \} * & \\
% unsigned\_hexadecimal\_literal & ::= &
%   \terminal{0} \{ \terminal{x} \mid \terminal{X} \}
%   \ hexadecimal\_digit\ \{ hexadecimal\_digit \mid \terminal{\_} \}* & \\
%     unsigned\_integer\_literal & ::= & unsigned\_binary\_literal \\
%                                &     & \mid unsigned\_octal\_literal & \\
%                                &     & \mid unsigned\_decimal\_literal \\
%                                &     & \mid unsigned\_hexadecimal\_literal & \\
%               integer\_literal & ::= & sign?\ unsigned\_integer\_literal
% \end{syntax}

An integer literal is a sequence of one or more digits, optionally
preceded by a minus or plus sign and/or a base prefix. By default,
i.e. without a base prefix, integers are in decimal. For instance:
{\tt 0}, {\tt -42}, {\tt +36}. {\focal} syntax allows to also specify
integers in other bases by preceding the digits by the following
prefixes:
\begin{compact-itemize}
  \item {\bf Binary}: base 2. Prefix is {\tt 0b} or {\tt 0B}.
    Digits are [0-1].
  \item {\bf Octal}: base 8. Prefix is {\tt 0o} or {\tt 00}.
    Digits are [0-7].
  \item {\bf Hexadecimal}: base 16. Prefix is {\tt 0x} or {\tt 0X}.
    Digits are [0-9] [A-F] [a-f].
\end{compact-itemize}
Here are various examples of integers in various bases:
{\tt -0x1Ff}, {\tt 0B01001}, {\tt +Oo347}, {\tt -OxFF\_FF}.

%%%%%%%%%%%%
\subsection{String literals}
\label{string literal}

\begin{syn}
\nt{string-literal} \is
  \tok{"}
  \rep{ \nt{plain-char} \orelse \tok{\backslash} \nt{char-escape} }
  \tok{"}
\sep
\nt{plain-char} \is \mbox{\em any printable character except backslash
  ($\backslash$) and double quote ({\tt"})}
\sep
\nt{char-escape} \is
     \tok{b} \orelse \tok{n} \orelse \tok{r} \orelse \tok{t}
\alt \tok{\textvisiblespace} \orelse \tok{"} \orelse \tok{'} \orelse \tok{*}
     \orelse \tok{\backslash} \orelse \tok{`} \orelse \tok{-}
\alt \tok{(} \orelse \tok{)} \orelse \tok{[} \orelse \tok{]}
     \orelse \tok{\{} \orelse \tok{\}} \orelse \tok{\%}
\alt \nt{digit} \nt{digit} \nt{digit}
\alt \nt{hexadecimal-digit} \nt{hexadecimal-digit}
\end{syn}

String literals are sequences of any characters delimited by {\tt "}
(double quote) characters ({\em ipso facto} with no intervening
{\tt"}).
Escape sequences (meta code to insert characters that can't appear
simply in a string) available in string literals are summarised in the
table below:

\medskip
\noindent
\begin{tabular}{|c|c|p{7cm}|}
  \hline
  Sequence & Character & Comment \\
  \hline
  \terminal{\backslash{}b} & $\backslash008$ & Backspace. \\
  \hline
  \terminal{\backslash{}n} & $\backslash010$ & Line feed. \\
  \hline
  \terminal{\backslash{}r} & $\backslash013$ & Carriage return. \\
  \hline
  \terminal{\backslash{}t} & $\backslash009$ & Tabulation. \\
  \hline
  \terminal{\backslash\textvisiblespace}
                           & \textvisiblespace      & Space character. \\
  \hline
  \terminal{\backslash"} & {\tt"}              & Double quote. \\
  \hline
  \terminal{\backslash'} & {\tt'}              & Single quote. \\
  \hline
  \terminal{\backslash*} & *              & Allows e.g. for insertion of ``(*'' in a string \\
  \hline
  \terminal{\backslash(} & ( & \multirow{6}{*}{See comment above for \terminal{\backslash*}} \\
  \cline{1-2}
  \terminal{\backslash)} & ) & \\
  \cline{1-2}
  \terminal{\backslash[} & [ & \\
  \cline{1-2}
  \terminal{\backslash]} & ] & \\
  \cline{1-2}
  \terminal{\backslash\{} & \{ & \\
  \cline{1-2}
  \terminal{\backslash\}} & \} & \\
  \hline
  \terminal{\backslash\backslash} & $\backslash$      & Backslash character. \\
  \hline
  \terminal{\backslash`} & `              & Backquote character. \\
  \hline
  \terminal{\backslash-} & {\tt-}  & Minus (dash) character. As for multi-line
                                  comments, uni-line comments can't appear in
                                  strings. Hence, to insert the sequence
                                  ``{\tt-{}-}'' use this escape
                                  sequence twice. \\
  \hline
  \terminal{\backslash}{\em digit digit digit}
                            &  & The character whose ASCII code in
                                  {\bf decimal} is given by the 3 digits
                                  following the $\backslash$. This
                                  sequence is valid for all
                                  ASCII codes. \\
  \hline
  \terminal{\backslash0x}{\em\ hex hex} & & The character whose ASCII code in
                                  {\bf hexadecimal} is given by the 2
                                  characters following the $\backslash$. This
                                  sequence is valid for all
                                  ASCII codes.\\
  \hline
\end{tabular}

%%%%%%%%%%%
\subsection{Character literals}
\label{character literals}

\begin{syn}
\nt{character-literal} \is
  \tok{'}
  \paren{ \nt{plain-char} \orelse \tok{\backslash} \nt{char-escape} }
  \tok{'}
\end{syn}

Characters literals are composed of one character enclosed between two
``{\tt '}'' (quote) characters. Example: {\tt 'a'}, {\tt '?'}.
Escape sequences (meta code to insert characters that can't appear
simply in a character literal) must also be enclosed by
quotes. Available escape sequences are summarised in the table above
(see section~\ref{string literal}).

%%%%%%%%%%%
\subsection{Floating-point number literals}

\begin{syn}
\nt{decimal-literal} \is
  \opt{\nt{sign}} \nt{unsigned-decimal-literal}
\sep
\nt{hexadecimal-literal} \is
  \opt{\nt{sign}} \nt{unsigned-hexadecimal-literal}
\sep
\nt{scientific-notation} \is \tok{e} \orelse \tok{E}
\sep
\nt{unsigned-decimal-float-literal} \is
  \nt{unsigned-decimal-literal}
  \opt{\tok{.} \rep{\nt{unsigned-decimal-literal}}} \cutline
  \opt{\nt{scientific-notation} \nt{decimal-literal}}
\sep
\nt{unsigned-hexadecimal-float-literal} \is
  \nt{unsigned-hexadecimal-literal}
  \opt{\tok{.} \rep{\nt{unsigned-hexadecimal-literal}}} \cutline
  \opt{\nt{scientific-notation} \nt{hexadecimal-literal}}
\sep
\nt{unsigned-float-literal} \is
     \nt{unsigned-decimal-float-literal}
\alt \nt{unsigned-hexadecimal-float-literal}
\sep
\nt{float-literal} \is \opt{\nt{sign}} \nt{unsigned-float-literal}
\end{syn}

% \begin{syntax}
% \syntaxclass{Float literals:}
%                  decimal\_literal & ::= & sign?\ unsigned\_decimal\_literal & \\
%              hexadecimal\_literal & ::= & sign?\ unsigned\_hexadecimal\_literal & \\
%              scientific\_notation & ::= & \terminal{e} \mid \terminal{E} & \\
% unsigned\_decimal\_float\_literal & ::= & unsigned\_decimal\_literal & \\
%                                   & &  \{ \terminal{.}\ unsigned\_decimal\_literal* \} ? & \\
%                                   & &  \{ scientific\_notation\ decimal\_literal \}? & \\
% unsigned\_hexadecimal\_float\_literal & ::= & unsigned\_hexadecimal\_literal &\\
%                                   & & \{ \terminal{.}\ unsigned\_hexadecimal\_literal* \} ? \\
%                                   & & \{ scientific\_notation\ hexadecimal\_literal \} ? & \\
%          unsigned\_float\_literal & ::= & unsigned\_decimal\_float\_literal & \\
%                                   & & \mid unsigned\_hexadecimal\_float\_literal & \\
%                    float\_literal & ::= & sign?\ unsigned\_float\_literal
% \end{syntax}

Floating-point numbers literals are made of an optional sign ('+' or
'-') followed by a non-empty sequence of digits followed by a dot
('.') followed by a possibly empty sequence of digits and finally an
optional scientific notation ('e' or 'E' followed an optional sign
then by a non-empty sequence of digits. {\focal} allows floats to be
written in decimal or in hexadecimal. In the first case, digits are
[0-9]. Example: {\tt 0.}, {\tt -0.1}, {\tt 1.e-10}, {\tt +5E7}.
In the second case, they are [0-9 a-f A-F] and the number must be
prefixed by ``0x'' or ``0X''. Example {\tt 0xF2.E4}, {\tt 0X4.3A},
{\tt Ox5a.a3eef}, {\tt Ox5a.a3e-ef}.

%%%%%%%%%%%
\subsection{Proof step bullets}
\index{proof!step bullet}
\begin{syn}
\nt{proof-step-bullet} \is
  \tok{<} \reps{\nt{digit}} \tok{>}
  \reps{ \nt{letter} \orelse \nt{digit} }
\end{syn}
% \begin{syntax}
% \syntaxclass{Proof step bullets:}
% proof\_step\_bullet & ::= &
%    \terminal{<} \{ \terminal{0} \ldots \terminal{9} \}+ \terminal{>}
%    \ \{letter \mid digit \}+
% \end{syntax}

A proof step bullet is a non-negative non-signed integer literal
(i.e. a non empty sequence of [0-9] characters) delimited by the
characters {\tt <} and {\tt >}, followed by a non-empty sequence of
alphanumeric characters (i.e. [A-Z a-z 0-9]).
The first part of the bullet (i.e. the integer literal) stands for the
depth of the bullet and the second part stands for its name. Example:

{\scriptsize
\begin{lstlisting}
 <1>1 assume ...
      ...
      prove ...
   <2>1 prove ... by ...
   <2>f qed by step <2>1 property ...
 <1>2 conclude
\end{lstlisting}
}

%%%%%%%%%%%
\subsection{Name qualification}
\label{qualified-name}
\index{qualified name} \index{name!resolution}
\index{name!qualification}

Name qualification is done according to the
compilation unit status.

As precisely described in section (\ref{toplevel-def}),
toplevel-definitions include species, collections, type definitions
(and their constitutive elements like constructors, record fields),
toplevel-theorems and toplevel-functions.  Any toplevel-definition (thus
outside species and collections) is visible all along the compilation unit
after its apparition.
If a toplevel-definition is required by another compilation unit, you can
reference it by referencing the external compilation unit
(with a {\tt use} or a {\tt open} command) and then
{\bf qualifying} its name, i.e. making explicit the
compilation unit's name before the definition's name using the '\#'
character as delimiter. Examples:

\begin{compact-itemize}
  \item {\tt basics\#string} stands for the type definition of
    {\tt string} coming from the source file ``basics.fcl''.
  \item {\tt  basics\#Basic\_object} stands for the species
    {\tt Basic\_object} defined in the source file ``basics.fcl''.
  \item {\tt db\#My\_db\_coll!create} stands for the method
    {\tt create} of a collection {\tt My\_db\_coll} hosted in the
    source file ``db.fcl''.
\end{compact-itemize}

\index{directive!open}
The qualification can be omitted by using the {\tt open} directive
that loads the interface of the argument compilation unit and make it
directly visible in the scope of the current compilation unit. For
instance:

{\scriptsize
\begin{lstlisting}
use "basics";;
species S = inherit basics#Basic_object; ... end ;;
\end{lstlisting}
}
can be transformed with no explicit qualification into:

{\scriptsize
\begin{lstlisting}
open "basics";;
species S = inherit Basic_object; ... end ;;
\end{lstlisting}
}

After an {\tt open} directive, the definitions of loaded (object files
of) compilation units are added in head of the current scope and mask
existing definitions wearing the same names. For example, in
the following program:

{\scriptsize
\begin{lstlisting}
(* Redefine my basic object, containing nothing. *)
species Basic_object = end ;;
open "basics";;
species S = inherit Basic_object; ... end ;;
\end{lstlisting}
}
the species {\tt S} inherits from the last {\tt Basic\_object} in the
scope, that is the one loaded by the {\tt open} directive and not from
the one defined at the beginning of the program. It is still possible
to recover the first definition by using the ``empty'' qualification
{\tt \#Basic\_object} in the definition of {\tt S}:

{\scriptsize
\begin{lstlisting}
(* Redefine my basic object, containing nothing. *)
species Basic_object = end ;;
open "basics";;
species S = inherit #Basic_object; ... end ;;
\end{lstlisting}
}

The qualification starting by a '\#' character without compilation
unit name before stands for ``the definition at toplevel of the
current compilation unit''.

%%%%%%%%%%%
\subsection{Reserved keywords}
The identifiers below are reserved keywords that cannot be employed
otherwise:
\begin{verbatim}
   alias all and as assume assumed
   begin by
   caml collection conclude coq coq_require
   definition
   else end ex external
   false final function
   hypothesis
   if in inherit internal implement is
   let lexicographic local logical
   match measure
   not notation
   of on open or order
   proof prop property prove
   qed
   rec representation
   Self signature species step structural
   termination then theorem true type
   use
   with
\end{verbatim}

% [EJ] Attention, certains de ces mots cles ne sont pas documentes... local par exemple

Keywords of {\coq} or {\ocaml} (for example {\tt Set}, {\tt class}),
can be safely used in a \focal\ source code.

Some symbols (such as {\tt :}) are also reserved, and cannot be used to name methods.
It is still possible to use such symbols as first character of a symbolic identifier.
