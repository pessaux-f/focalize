% $Id: lexical_conventions.tex,v 1.11 2009-01-25 23:58:44 hardin Exp $


%%%%%%%%%%%%%%%%%%%%%%%%%%%%%%%%%%%%%%%%%%%%%%%%%%%%%%%%%%%%%%%%%%%%%%%
\section{Lexical conventions}
\index{lexical conventions}

%%%%%%%%%%%
\subsection{Blanks}
\index{blank}
The following characters are considered as blanks: space, newline,
horizontal tabulation, carriage return, line feed and form
feed. Blanks are ignored, but they separate adjacent identifiers,
literals and keywords that would otherwise be confused as one single
identifier, literal or keyword.

%%%%%%%%%%%
\subsection{Comments}
\index{comment}
Comments (possibly spanning) on several lines are introduced by the
two characters {\tt (*}, with no intervening blanks, and terminated by
the characters {\tt *)}, with no intervening blanks. Comments are
treated as blanks. Comments can occur inside string or character
literals (provided the {\tt *} character is escaped) and can be nested. They
are discarded during the compilation process. Example:
{\scriptsize
\begin{lstlisting}
(* Discarded comment. *)
species S =
  ...
  let m (x in Self) = (* Another discarded comment. *)
  ...
end ;;
(* Another discarded comment at end of file. *)
\end{lstlisting}
}

Comments spanning on a single line start by the two characters
{\tt --} and end with the end-of-line character.
Example:
{\scriptsize
\begin{lstlisting}
-- Discarded uni-line comment.
species S =
  let m (x in Self) = -- Another uni-line comment.
  ...
end ;;
\end{lstlisting}
}

%%%%%%%%%%%
\subsection{Annotations}
\index{annotation}
\index{documentation}
\label{annotation}
{\bf Annotations} are introduced by the three characters {\tt (**},
with no intervening blanks, and terminated by the two characters
{\tt *)}, with no intervening blanks.
Annotations cannot occur inside string or character literals and
cannot be nested. They must precede the construct they document.
In particular, a {\bf source file cannot end by an annotation}.

Unlike comments, annotations are kept during the compilation process
and recorded in the compilation information (``{\tt .fo}'' files). Annotations can
be processed later on by external tools that could analyze them to
produce a new {\focal} source code accordingly.
For instance, the {\focal} development environment provides the {\focdoc}
automatic production tool that uses annotations to automatically generate
documentation.
Several annotations can be put in sequence for the same construct. We call
such a sequence an {\bf annotations block}\index{annotation!block}.
Using embedded tags in annotations allows third-party tools to easily find
out annotations that are meaningful to them, and safely ignore others.
For more information, consult
\ref{documentation-generation}.
Example:
{\scriptsize
\begin{lstlisting}
(** This a documentation for species S. *)
species S =
  ...
  let m (x in Self) =
    (** [MY_TAG_TEST] Documentation for testers. *)
    (** [MY_TAG_MAINT] Documentation for maintainers. *)
    ... ;
end ;;
\end{lstlisting}
}

%%%%%%%%%%%
\subsection{Identifiers}
\index{identifier}
\vspace{0.2cm}

{\focal} features a rich class of identifiers with sophisticated lexical
rules that provide fine distinction between the kind of notion a given
identifier can designate.

\subsubsection{Introduction}

Sorting words to find out which kind of meaning they may have is a very common
conceptual categorization of names that we use when we write or read ordinary
english texts. We routinely distinguish between:
\begin{citemize}
\item a word only made of lowercase characters, that is supposed to be an
  ordinary noune, such as "table", "ball", or a verb as in "is", or an
  adjective as in "green",
\item a word starting with an uppercase letter, that is supposed to be a name,
  maybe a family or christian name, as in "Kennedy" or "David", or a location
  name as in "London".
\end{citemize}

We use this distinctive look of words as a useful hint to help understanding
phrases. For instance, we accept the phrase "my ball is green" as meaningful,
whereas "my Paris is green" is considered a nonsense. This is simply because
"ball" is a regular noune and "Paris" is a name. The word "ball" as the right
lexical classification in the phrase, but "Paris" has not. This is also clear
that you can replace "ball" by another ordinary noune and get something
meaningful: "the table is green"; the same nonsense arrises as well if you
replace "Paris" by another name: "Kennedy is green".

Natural languages are far more complicated than computer languages, but
{\focalize} uses the same kind of tricks: the ``look'' of words helps a lot to
understand what the words are designating and how they can be used.

\subsubsection{Conceptual properties of names}

{\focal} distinguishes 4 concepts for each name:

\begin{citemize}
\item the {\em fixity}: assigns the place where an identifier must be written,
\item the {\em precedence}: decides the order of operations when
  identifiers are combined together,
\item the {\em categorisation}: fixes which concept the identifier designates.
\item the {\em nature} of a name can be symbolic of alphanumeric.
\end{citemize}

Those concepts are compositional: these concepts are independant form one
another. Put is another way: for any fixity, precedence, category and nature,
there exist an identifier with this exact properties.

We explain those concepts below.

\subsubsection{Fixity of identifiers}
\index{fixity of identifiers}
\index{infix identifier}
\index{prefix identifier}

The fixity of an identifier answers to the question ``where this identifier
must be written ?''.
\begin{citemize}
\item a {\em prefix} is written {\em before} its argument, as $sin$ in $sin\;
  x$ or $-$ in $- y$,
\item an {\em infix} is written {\em between} its arguments, as $+$ in $x +
  y$ or $mod$ in $x\; mod \;3$.
\end{citemize}

In {\focal}, as in maths, ordinary identifiers are always prefix and binary operators are
always infix.

\subsubsection{Precedence of identifiers}
\index{precedence of identifiers}

The precedence rules out where implicit parentheses should be written in a
complex combinaison of symbols. For instance, according to usual mathematical
conventions:
\begin{citemize}
\item $1 + 2 * 3$  means $1 + (2 * 3)$ hence $7$, it does not not mean $(1 +
  2) * 3$ which is $9$,
\item $2 * 3 ^ 4 + 5$ means $(2 * (3 ^ 4)) + 5$ hence $167$, it does not mean
  $((2 * 3) ^ 4) + 5$ which is $1301$, nor $2 * (3 ^ (4 + 5))$ which is $39366$.
\end{citemize}

In {\focal}, all the binary operators have the precedence they have in maths.

\subsubsection{Categorization of identifiers}

\index{category of identifiers}
The category of an identifier answers to the question is this identifier a
``possible name for this kind of concept'' in the language.
Very often, this is more strict and the category exactly states which concept
must attach to the identifier.

For {\focal} these categories are
\begin{citemize}
\item {\em lowercase}: the identifier starts with a lowercase letter and
  designates a simple entity of the language. It may names some of
  the language expressions, a function name, a function parameter or bound
  variable name, a method name; a type name, or a record field label name.

\item {\em uppercase}: the identifier starts with an uppercase letter and
  designates a more complex entity in the langage. It may name a sum type
  constructor name, a module name, a species or a collection name.
\end{citemize}

We distinguish identifiers using their first ``meaningful'' character:

\subsubsection{Nature of identifiers}
\index{nature of identifiers}

In {\focal} identifiers are either:

\begin{citemize}
\item {\em symbolic}: the identifier contains characters that are not
  letters. {\tt +}, {\tt :=}, {\tt ->}, {\tt +float} are symbolic

\item {\em alphanumeric}: the identifier contains letters, digits and
  underscores. {\tt x}, {\tt \_1}, {\tt Some}, {Basic\_object} are alphanumeric.
\end{citemize}

\subsubsection{Regular identifiers}
\index{regular identifiers}

Regular lower case identifiers are used to designate the names of variables, functions,
and labels of records.

\begin{syntax}
\syntaxclass{Basic identifiers:}
digit & ::= & \terminal{0} \ldots \terminal{9} & \\
lower & ::= & \terminal{a} \ldots \terminal{z} & \\
upper & ::= & \terminal{A} \ldots \terminal{Z} & \\
letter & ::= & lower \mid upper & \\
lident & ::= & \{\ lower \mid \terminal{\_} \}
            \{ letter \mid digit \mid \terminal{\_} \}* & \\
uident & ::= & upper\ \{\ letter \mid digit \mid \terminal{\_} \}* & \\
ident & ::= & lident \mid uident
\end{syntax}
\vspace{0.2cm}

A regular identifier is a sequence of letters, digits, and {\tt \_} (the
underscore character), starting with a letter or an underscore.

The identifier is lowercase if its first letter is lowercase.

The identifier is uppercase if its first letter is uppercase.

Letters contain at least the 52 lowercase and uppercase
letters from the standard ASCII set. In an identifier, all characters
are meaningful.
Examples: {\tt foo}, {\tt bar}, {\tt \_20}, {\tt \_\_\_gee\_42}.

\subsubsection{Infix/prefix operators}
\index{identifier!operator}\index{operator}

{\focal} allows infix and prefix operators built from a
``starting operator character'' and followed by a sequence of
regular identifiers or operator characters. For example, all the
following are legal operators:
{\tt +}, {\tt ++}, {\tt $\sim$+zero}, {\tt =\_mod\_5}.

The position in which to use the operator (i.e. infix or prefix)
is determined by the position of the first operator character
according to the following table:
\begin{center}
\begin{tabular}{|c|c|}
\hline
Prefix & Infix \\
\hline
` $\sim$ ? \$ ! \#                            &
, + - * / \% \& $|$ : ; $<$ = $>$ @ \^\ $\setminus$ \\
\hline
\end{tabular}
\end{center}

\begin{syntax}
\syntaxclass{\hspace{1cm} Infix/prefix operators:}
prefix\_char & ::= & \terminal{` \ \sim\ ?\ \$\ !\ \#\ }& \\
 infix\_char & ::= &
    \terminal{ ,\ +\ -\ *\ /\ \%\ \&\ \mid\ :\ ;\ <\ =\ >\ @ \wedge
              \ \setminus} & \\
  prefix\_op & ::= & prefix\_char
    \ \{ letter \mid prefix\_char \mid infix\_char
    \mid digit \mid \terminal{\_} \}* & \\
   infix\_op & ::= & infix\_char
    \ \{ letter \mid prefix\_char \mid infix\_char
    \mid digit \mid \terminal{\_} \}* & \\
    op & ::= & infix\_op \mid prefix\_op
\end{syntax}

Hence, in the above examples, {\tt +}, {\tt ++} and {\tt =\_mod\_5} will be
infix operators and {\tt $\sim$+zero} will be a prefix one.

\subsubsection{Defining an infix operator}
\index{defining operators}
\index{defining an infix operator}
\index{defining a prefix operator}

The notion of infix/prefix operator does not mean that
{\focal} defines all these operators: it means that the programmer
may freely define and use them as ordinary
prefix/infix operators instead of only writing prefix function names and regular function
application. For instance, if you do not like the {\focal} predefined \verb"^"
operator to catenate strings, you can define a new infix synonym for \verb"^",
say {tt ++}, using:
{\scriptsize
\begin{lstlisting}
let ( ++ ) (s1, s2) = s1 ^ s2 ;
\end{lstlisting}
}
Then you can use the {\tt ++} operator in the usual way
{\scriptsize
\begin{lstlisting}
let hw = "Hello" ++ " world!" ;
\end{lstlisting}
}

As shown in the example, at definition-time, the syntax requires
the operator to be embraced by parentheses. More precisely, you must
enclose the operator between {\bf spaces} and parentheses.
You must write {\tt ( + )} with spaces, not simply {\tt (+)} (which leads
to a syntax error anyway).

\subsubsection{Prefix form notation}
\index{prefix form notation}
\index{infix in prefix position}

The notation {\tt ( op )} is named the {\em prefix form notation} for
operator {\tt op}.

Since you can only define prefix identifiers in {\focal}, you must use the
prefix form notation to define an infix or prefix operator.

When a prefix or infix operator has been defined, it is still possible
to use it as a regular identifier using the prefix form notation.
For instance, you can use the prefix form of operator {\tt ++}
to apply it in a prefix position as a simple regular function (with a
strange name admittedly!):

{\scriptsize
\begin{lstlisting}
... ( ++ ) ("Hello", " world!") ;
\end{lstlisting}
}

{\bf Attention}: a common error while defining an operator is to forget
the spaces around the operator. This is particularly confusing, if you
type the {\tt *} operator without spaces around the operator: you
write the lexical entity {\tt (*)} which is the beginning (or the end)
of a comment!

The {\focal} notion of symbolic identifiers go largely beyond simple
infix operators. Symbolic identifiers let you assign sophisticated names
to your functions and operators.
For instance, instead of creating a function to check if integer {\tt x}
is equal to the predecessor of integer {\tt y}, as in
{\scriptsize
\begin{lstlisting}
let is_eq_to_predecessor (x, y) = ... ;
... if is_eq_to_predecessor (5, 7) ... ;
\end{lstlisting}
}
it is possible to directly define
{\scriptsize
\begin{lstlisting}
let ( =pred ) (x, y) = ... ;
... if 5 =pred 7 ... ;
\end{lstlisting}
}

{\bf Attention} : since a comma can start an infix symbol, be careful
when using commas to add a space after each comma to prevent confusion.
In particular when using commas to separate tuple components, always type
a space after each comma. For instance, if you write {\tt (1,n)}
then the lexical analyser finds only two words: the integer {\tt 1} as
desired, then the infix operator {\tt ,n} which is certainly not the
intended meaning. Hence, following usual typography rules, always type a
space after a comma (unless you have define a special operator starting
by a comma).

{\bf Rule of thumb}: The prefix version of symbolic identifiers is obtained by enclosing the
symbol between spaces and parens.

%%%%%%%%%%%
\subsection{Extended identifiers}
\label{extended-identifiers}
\index{identifier!extended}

Moreover, {\focal} has special forms of identifiers to allow using
spaces inside or extend the notion of operator identifiers.
\begin{itemize}
  \item {\bf Delimited alphanumerical identifiers}.
    \index{identifier!delimited}
    They start by two characters {\tt `} (backquote) and end by two
    characters {\tt '} (quote). In addition to usual alpha-numerical
    characters, the delimited identifiers can have spaces. For example:
    {\tt ``equal is reflexive''}, {\tt ``fermat conjecture''}.
  \item {\bf Delimited symbolic identifiers}.
    They are delimited by the same characters and may contain symbolic characters.
\end{itemize}

The first meaningful character at the beginning of a delimited
ident/symbol is used to find its associated token.

%%%%%%%%%%%
\subsection{Species and collection names}
\index{species!name} Species, collection
names and collection parameters are uppercase identifiers.

%%%%%%%%%%%
\subsection{Integer literals}
\label{integer literals}
\label{octal}
\label{hexadecimal}
\label{binary}

\begin{syntax}
\syntaxclass{Integer literals:}
binary\_digit & ::= & \terminal{0} \mid \terminal{1} & \\
octal\_digit & ::= & \terminal{0} \ldots \terminal{7} & \\
decimal\_digit & ::= & \terminal{0} \ldots \terminal{9} & \\
hexadecimal\_digit & ::= & \terminal{0} \ldots \terminal{9}
       \mid \terminal{A} \ldots \terminal{F}
       \mid \terminal{a} \ldots \terminal{f} & \\
sign & ::= & \terminal{+} \mid \terminal{-} & \\
unsigned\_binary\_literal & ::= &
  \terminal{0} \{ \terminal{b} \mid \terminal{B} \}
  \ binary\_digit\ \{ binary\_digit \mid \terminal{\_} \}* & \\
unsigned\_octal\_literal & ::= &
  \terminal{0} \{ \terminal{o} \mid \terminal{O} \}
  \ octal\_digit\ \{ octal\_digit \mid \terminal{\_} \}* & \\
unsigned\_decimal\_literal & ::= &
  decimal\_digit \{ decimal\_digit \mid \terminal{\_} \} * & \\
unsigned\_hexadecimal\_literal & ::= &
  \terminal{0} \{ \terminal{x} \mid \terminal{X} \}
  \ hexadecimal\_digit\ \{ hexadecimal\_digit \mid \terminal{\_} \}* & \\
unsigned\_integer\_literal & ::= & unsigned\_binary\_literal \\
& & \mid unsigned\_octal\_literal & \\
& & \mid unsigned\_decimal\_literal \\
& & \mid unsigned\_hexadecimal\_literal & \\
integer\_literal & ::= & sign?\ unsigned\_integer\_literal
\end{syntax}

An integer literal is a sequence of one or more digits, optionally
preceded by a minus or plus sign and/or a base prefix. By default,
i.e. without a base prefix, integers are in decimal. For instance:
{\tt 0}, {\tt -42}, {\tt +36}. {\focal} syntax allows to also specify
integers in other bases by preceding the digits by the following
prefixes:
\begin{itemize}
  \item {\bf Binary}: base 2. Prefix {\tt 0b} or {\tt 0B}.
    Digits: [0-1].
  \item {\bf Octal}: base 8. Prefix {\tt 0o} or {\tt 00}.
    Digits: [0-7].
  \item {\bf Hexadecimal}: base 16. Prefix {\tt 0x} or {\tt 0X}.
    Digits: [0-9] [A-F] [a-F]
\end{itemize}
Here are various examples of integers in various bases:
{\tt -0x1Ff}, {\tt 0B01001}, {\tt +Oo347}.

%%%%%%%%%%%%
\subsection{String literals}
\label{string literal}
String literals are sequences of any characters delimited by {\tt "}
(double quote) characters ({\em ipso facto} with no intervening
{\tt"}).
Escape sequences (meta code to insert characters that can't appear
simply in a string) available in string literals are summarised in the
table below:

\medskip
\noindent
\begin{tabular}{|c|c|p{7cm}|}
  \hline
  Sequence & Character & Comment \\
  \hline
  $\setminus$b & $\setminus$008 & Backspace. \\
  \hline
  $\setminus$t & $\setminus$009 & Tabulation. \\
  \hline
  $\setminus$n & $\setminus$010 & Line feed. \\
  \hline
  $\setminus$r & $\setminus$013 & Carriage return. \\
  \hline
  $\setminus$\textvisiblespace & \textvisiblespace      & Space character. \\
  \hline
  $\setminus$" & "              & Double quote. \\
  \hline
  $\setminus$' & '              & Single quote. \\
  \hline
  $\setminus$* & *              & Since comments cannot appear inside
                                  strings, to insert one of the
                                  sequence ``(*'', ``*)'', ``\{*'', or
                                  ``*\}'', use this escape sequence
                                  combined with the four following
                                  ones. \\
  \hline
  $\setminus$( & (              & See comment above for $\setminus$*. \\
  \hline
  $\setminus$) & )              & See comment above for $\setminus$*. \\
  \hline
  $\setminus$[ & [              & See comment above for $\setminus$*. \\
  \hline
  $\setminus$] & ]              & See comment above for $\setminus$*. \\
  \hline
  $\setminus$\{ & \{              & See comment above for $\setminus$*. \\
  \hline
  $\setminus$\} & \}              & See comment above for $\setminus$*. \\
  \hline
  $\setminus\setminus$ & $\setminus$      & Backslash character. \\
  \hline
  $\setminus$` & `      & Backquote character. \\
  \hline
  $\setminus$\-- & \-- & Minus (dash) character. Like for multi-line
                           comments, uni-line comments can't appear in
                           strings. Hence, to insert the sequence
                           ``\--\--'', use this escape sequence twice. \\
  \hline
  $\setminus$[0-9][0-9][0-9] & & The character whose ASCII code in
                                {\bf decimal} is given by the 3 digits
                                following the $\setminus$. This
                                sequence is valid for all
                                ASCII codes. \\
  \hline
  $\setminus$x[0-9a-fA-F][0-9a-fA-F] & & The character whose ASCII code in
                                {\bf hexadecimal} is given by the 2
                                digits following the $\setminus$. This
                                sequence is valid for all
                                ASCII codes.\\
  \hline
\end{tabular}

%%%%%%%%%%%
\subsection{Character literals}
\label{character literals}
\label{hexadecimal}
Characters literals are composed of one character enclosed between two
``{\tt '}'' (quote) characters. Example: {\tt 'a'}, {\tt '?'}.
Escape sequences (meta code to insert characters that can't appear
simply in a character literal) must also be enclosed by
quotes. Available escape sequences are summarised in the table above
(see section~\ref{string literal}).

%%%%%%%%%%%
\subsection{Floating-point number literals}
\begin{syntax}
\syntaxclass{Float literals:}
decimal\_literal & ::= & sign?\ unsigned\_decimal\_literal & \\
hexadecimal\_literal & ::= & sign?\ unsigned\_hexadecimal\_literal & \\
scientific\_notation & ::= & \terminal{e} \mid \terminal{E} & \\
unsigned\_decimal\_float\_literal & ::= & unsigned\_decimal\_literal & \\
& &  \{ \terminal{.}\ unsigned\_decimal\_literal* \} ? & \\
& &  \{ scientific\_notation\ decimal\_literal \}? & \\
unsigned\_hexadecimal\_float\_literal & ::= & unsigned\_hexadecimal\_literal &\\
& & \{ \terminal{.}\ unsigned\_hexadecimal\_literal* \} ? \\
& & \{ scientific\_notation\ hexadecimal\_literal \} ? & \\
unsigned\_float\_literal & ::= & unsigned\_decimal\_float\_literal & \\
& & \mid unsigned\_hexadecimal\_float\_literal & \\
float\_literal & ::= & sign?\ unsigned\_float\_literal
\end{syntax}

Floating-point numbers literals are made of an optional sign ('+' or
'-') followed by a non-empty sequence of digits followed by a dot
('.') followed by a possibly empty sequence of digits and finally an
optional scientific notation ('e' or 'E' followed an optional sign
then by a non-empty sequence of digits. {\focal} allows floats to be
written in decimal or in hexadecimal. In the first case, digits are
[0-9]. Example: {\tt 0.}, {\tt -0.1}, {\tt 1.e-10}, {\tt +5E7}.
In the second case, they are [0-9 a-f A-F] and the number must be
prefixed by ``0x'' or ``0X''. Example {\tt 0x-F2.E4}, {\tt 0X4.3A},
{\tt Ox5a.a3eef}, {\tt Ox5a.a3e-ef}.

%%%%%%%%%%%
\subsection{Proof step bullets}
\index{proof!step bullet}
\begin{syntax}
\syntaxclass{Proof step bullets:}
proof\_step\_bullet & ::= &
   \terminal{<} \{ \terminal{0} \ldots \terminal{9} \}+ \terminal{>}
   \ \{letter \mid digit \}+
\end{syntax}

A proof step bullet is a non-negative non-signed integer literal
(i.e. a non empty sequence of [0-9] characters) delimited by the
characters {\tt <} and {\tt >}, followed by a non-empty sequence of
alphanumeric characters (i.e. [A-Z a-z 0-9]).
The first part of the bullet (i.e. the integer literal) stands for the
depth of the bullet and the second part stands for its name. Example:
{\scriptsize
\begin{lstlisting}
 <1>1 assume ...
      ...
      prove ...
   <2>1 prove ... by ...
   <2>9 qed by step <2>1 property ...
 <1>2 conclude
\end{lstlisting}
}

%%%%%%%%%%%
\subsection{Name qualification}
\label{qualified-name}
\index{qualified name} \index{name!resolution}
\index{name!qualification}

Name qualification is done according to the
compilation unit status.

As precisely described in section (\ref{toplevel-def}),
toplevel-definitions include species, collections, type definitions
(and their constitutive elements like constructors, record fields),
toplevel-theorems and toplevel-functions.  Any toplevel-definition (thus outside species
and collections) is visible all along the compilation unit after its
apparition.
If a toplevel-definition is required by another compilation unit, you can
reference it by {\bf qualifying} its name, i.e. making explicit the
compilation unit's name before the definition's name using the '\#'
character as delimiter. Examples:

\begin{itemize}
  \item {\tt basics\#string} stands for the type definition of
    {\tt string} coming from the source file ``basics.fcl''.
  \item {\tt  basics\#Basic\_object} stands for the species
    {\tt Basic\_object} defined in the source file ``basics.fcl''.
  \item {\tt db\#My\_db\_coll!create} stands for the method
    {\tt create} of a collection {\tt My\_db\_coll} hosted in the
    source file ``db.fcl''.
\end{itemize}

\index{directive!open}
The qualification can be omitted by using the {\tt open} directive
that loads the interface of the argument compilation unit and make it
directly visible in the scope of the current compilation unit. For
instance:

{\scriptsize
\begin{lstlisting}
species S inherits basics#Basic_object = ... end ;;
\end{lstlisting}
}
can be transformed with no explicit qualification into:
{\scriptsize
\begin{lstlisting}
open "basics";;
species S inherits Basic_object = ... end ;;
\end{lstlisting}
}

After an {\tt open} directive, the definitions of loaded (object files
of) compilation units are added in head of the current scope and mask
existing definitions wearing the same names. For example, in
the following program: {\scriptsize
\begin{lstlisting}
(* Redefine my basic object, containing nothing. *)
species Basic_object = end ;;
open "basics";;
species S inherits Basic_object = ... end ;;
\end{lstlisting}
}
the species {\tt S} inherits from the last {\tt Basic\_object} in the
scope, that is the one loaded by the {\tt open} directive and not from
the one defined at the beginning of the program. It is still possible
to recover the first definition by using the ``empty'' qualification
{\tt \#Basic\_object} in the definition of {\tt S}:
{\scriptsize
\begin{lstlisting}
(* Redefine my basic object, containing nothing. *)
species Basic_object = end ;;
open "basics";;
species S inherits #Basic_object = ... end ;;
\end{lstlisting}
}

The qualification starting by a '\#' character without compilation
unit name before stands for ``the definition at toplevel of the
current compilation unit''.

%%%%%%%%%%%
\subsection{Reserved keywords}
The identifiers below are reserved keywords that cannot be employed
otherwise:
\begin{verbatim}
   alias all and as assume assumed
   begin by
   caml collection conclude coq coq_require
   definition
   else end ex external
   false function
   hypothesis
   if in inherits internal implements is
   let lexicographic local logical
   match measure
   not notation
   of on open or order
   proof prop property prove
   qed
   rec representation
   Self signature species step structural
   termination then theorem true type
   use
   with
\end{verbatim}
