% $Id: building_species.tex,v 1.7 2008-12-05 12:54:00 pessaux Exp $



In this chapter, we describe the relations between species to build
incrementally more complex species from previously existing ones. Two
mechanisms are available for this purpose: parametrisation and
inheritance.



%%%%%%%%%%%%%%%%%%%%%%%%%%%%%%%%%%%%%%%%%%%%%%%%%%%%%%%%%%%%%%%%%%%%%%%
%%%%%%%%%%%%%%%%%%%%%%%%%%%%%%%%%%%%%%%%%%%%%%%%%%%%%%%%%%%%%%%%%%%%%%%
\subsection{Parametrisation}
\index{parametrisation}
\label{parametrisation}


%%%%%%%%%%%%%%%%%%%%%%%%%%%%%%%%%%%%%%%%%%%%%%%%%%%%%%%%%%%%%%%%%%%%%%%
\subsubsection{Collection parameters}
\index{parameter!collection}
\index{collection!parameter}
\label{collection-parameter}
Remember that methods cannot be polymorphic\index{polymorphism}
(c.f. \ref{no-polymorphism-for-methods}). So it seems difficult to
create a species whose methods work on a carrier having a type
structure in which we would like to have some parts unconstrained. For
example, who to implement the well-known polymorphic type of the
lists ? A list is a structure grouping elements but independently of
the type of these elements. The only constraint is that all elements
have the same type. Hence, a ML-like representation of lists would be
like:
{\scriptsize
\lstset{language=Caml}
\begin{lstlisting}
type 'a list =
  | Nil
  | Cons of ('a * 'a list)
\end{lstlisting}
}

The {\tt 'a} is then a parameter of the type and is polymorphic.
In \focal\ we would like to create a species looking like:
{\scriptsize
\begin{lstlisting}
species List =
  signature nil : Self ;
  signature cons : 'a -> Self -> Self ;
end ;;
\end{lstlisting}
}

Instead of abstracting the type parameter and leaving it free in the
context of the species, in \focal\ we {\em parametrise} the species
by another species:
{\scriptsize
\begin{lstlisting}
species List (Elem is Basic_object) =
  signature nil : Self ;
  signature cons : Elem -> Self -> Self ;
end ;;
\end{lstlisting}
}

The {\tt Elem} is called a {\bf collection parameter} and is expected
to be a species having at least the methods of a species called
{\tt Basic\_object}\footnote{{\tt Basic\_object} is a basic and poor
species from the standard library, containing only few methods.} with
the same types. Collection parameters are introduced by the {\tt is}
keyword. When a parametrised species will be used, \focal\ expects it
to be applied to effective collection parameters having
{\em compatible}\index{interface!compatibility}
species interfaces (c.f \ref{interface}). In other words the effective
parameter will have to present an interface with at least the
functions of {\tt Basic\_object} and each method will have to have the
same type than the corresponding one in {\tt Basic\_object}.

\smallskip
In the example, we use this parameter in order to build the signature
of our method {\tt cons}. You may note that species names can be used
in type expressions. They denote the carrier abstracted of the species
whose name is used. By ``abstracted'', it is meant that the
representation of this carrier is not visible, but we can refer to it
as an abstract data-type. In other words, {\tt Elem -> Self -> Self}
stands for the type of a function:
\begin{itemize}
  \item taking a value whose type is the carrier type of a species
    having a compatible interface with the species
    {\tt Basic\_object}. (This especially means that such a value will
    have be created using methods of the compatible species),
  \item taking a value whose type is the carrier type of the current
    species,
  \item and returning a value whose type is the carrier type of the
    current species.
\end{itemize}

\smallskip
Although the parameters looks like a species, why is it called a
{\bf collection parameter} ? The answer to this question is especially
important to understand the programming model in \focal. It is called
a collection because finally, at the terminal nodes of the
development, this parameter will have to be instantiated by a
collection, that is an entity where everything is defined ! Imagine
how to build a code we could execute if a parameter was instantiated
with an entity with methods only declared\ldots Moreover, the carrier
of a collection parameter is abstract for the hosting, exactly like
the carrier of a collection is (c.f \ref{collection}).

So, even if the collection parameter ``{\tt is}'' a species name,
{\bf it will not be a species}. It will be {\bf a collection having an
interface compatible} with the interface of the species. Hence,
declaring a collection parameter for a parametrised species means
providing two things: the name of the parameter and the signature that
the instantiation of this parameter must satisfy.

\smallskip
It is important to note that we deal with dependent types
\index{type!dependent} here, and
therefore that the order of the parameters is important. To define the
type of a parameter, one can use the preceding parameters. For
instance, if we assume that there exists a parametrised species
{\tt List} which declares the basic operations over lists, one can
specify a new species working on couples of value and lists of values
like:
{\scriptsize
\begin{lstlisting}
species MyCouple (E is Basic_object, L is List (E)) =
  representation = (E * L) ;
  ... ;
end ;;
\end{lstlisting}
}

The carrier of this species will represent the type
{\tt ('a * ('a list))}. This means that the type of the values in the
first component of the couple is the same than the type of the
elements of the list in the second component of the couple.

\smallskip
If a collection parameter implements a parametrized species (like in
the example for the species {\tt List}), it must provide an
instantiation for {\bf all} its parameters. Once again, the preceding
parameters can be used to achieve this purpose (like we did to create
our parameter {\tt L}, instantiating the {\tt List}'s parameter by our
first collection parameter {\tt E}).

\smallskip
\label{method-qualification}
\index{method!qualification}
At the beginning of the presentation of collection parameters, we used
the parameter to build the carrier of our species. But obviously,
collection parameters can also be used via their other methods,
i.e. signatures, functions, properties and theorems. We want to create
a generic couple species. It will then have two collection parameters,
one for each component of the couple. If we want to have a printing
(i.e. returning a string, not making side effect in our example)
method, we will require that each collection parameter has one. Hence
our printing method will only have to add parentheses and comma
around and between what is printed by each parameter's printing
routine. Hence we now need to know how to call a collection parameter
method. The syntax used is qualifying the method name by the
parameter's name, separating them by the ``!''\index{bang character}
character.
{\scriptsize
\begin{lstlisting}
(* Minimal species requirement : having a print routine. *)
species Base_obj =
  signature print : Self -> string ;
end ;;

species Couple (C1 is Base_obj, c2 is Base_obj) =
  representation = (C1 * C2) ;
  let print (c in Self) =
    match (c) with
     | (component1, component2) ->
       "(" ^ C1!print (component1) ^
       ", " ^
       C2!print (component2) ^")" ;
end ;;
\end{lstlisting}
}

Hence, {\tt C1!print (component1)} means ``call the collection
{\tt C1}'s method {\tt print} with the argument {\tt component1}. The
qualification mechanism using ``!'' is general and can be used to
denote the method of any available species/collection, even those of
ourselves (i.e. {\tt Self}). Hence, in a species instead of calling:
{\scriptsize
\begin{lstlisting}
species Foo ... =
  let m1 (...) = ... ;
  let m2 (...) = if ... then ... else m1 (...) ;
end ;;
\end{lstlisting}
}
it is allowed to explicitly qualify the call to {\tt m1} by ``!''
with no species name, hence implicitly telling ``from myself'':
{\scriptsize
\begin{lstlisting}
species Foo ... =
  let m1 (...) = ... ;
  let m2 (...) = if ... then ... else !m1 (...) ;
end ;;
\end{lstlisting}
}
\index{name!resolution}
\index{scoping}
In fact, without explicit ``!'', the \focal\ compiler performs the
name resolution itself, allowing a lighter way of writing programs
instead of always needing a ``!'' character before each method call.



%%%%%%%%%%%%%%%%%%%%%%%%%%%%%%%%%%%%%%%%%%%%%%%%%%%%%%%%%%%%%%%%%%%%%%%
\subsubsection{Entity parameters}
\index{parameter!entity}
\label{entity-parameter}
The other way to parametrise a species is to pass it a
{\bf value of a certain species carrier}. For instance, in the
previous section, we made our {\tt List} species parametrised by the
``type'' of the elements in the list. We then used a collection
parameter. If we now want a species addition modulo a value, we need
to parametrise our species by this {\bf value}, or at least by a
value of a collection implementing the integers and giving a way to
have a value representing the one we wish. Such a parameter is called
an {\bf entity parameter} and is introduced by the keyword {\tt in}.
{\scriptsize
\begin{lstlisting}
species AddModN (Number is ASpeciesImplentingInts, val_mod in Number) =
  representation = Number ;
  let add (x in Self, y in Self) =
    Number!modulo (Number!add (x, y), val_mod) ;
end ;;

species
\end{lstlisting}
}

Hence, any collection created from {\tt AddModN} will embed it modulo
value. And it will be possible to create various collections with each
a specific module value. For instance, assuming that the species
{\tt AddModN} is complete and have a method {\tt from\_int} able to
create a value of the carrier from an integer, we can create a
collection implementing addition modulo 42. We also assume that we
have a collection {\tt ACollImplentingInts} implementing the
integers.
{\scriptsize
\begin{lstlisting}
collection AddMod42 implements AddModN
  (ACollImplentingInts, ACollImplentingInts!from_int (42)) ;;
\end{lstlisting}
}


\smallskip
Currently, entity parameters must live ``{\tt in}'' a collection. It
is not allowed to specify an entity parameter living in a basic type
like {\tt int}, {\tt string}, {\tt bool}\ldots This especially means
that one must have a collection embedding, implementing the type of
values we want to use as entity parameters.



%%%%%%%%%%%%%%%%%%%%%%%%%%%%%%%%%%%%%%%%%%%%%%%%%%%%%%%%%%%%%%%%%%%%%%%
%%%%%%%%%%%%%%%%%%%%%%%%%%%%%%%%%%%%%%%%%%%%%%%%%%%%%%%%%%%%%%%%%%%%%%%
\subsection{Inheritance and its mechanisms}
In this section, we address the second mean to build complex species
based on existing ones. It will cover the notion of {\em inheritance}
and its related feature the {\em late-binding}.



%%%%%%%%%%%%%%%%%%%%%%%%%%%%%%%%%%%%%%%%%%%%%%%%%%%%%%%%%%%%%%%%%%%%%%%
\subsubsection{Inheritance}
\label{inheritance}
\index{inheritance}
Like in Object-Oriented models, in \focal\ {\em inheritance} is the
ability to create a species, not from scratch, but integrating methods
of other species. For instance, assuming we have a species
{\tt IntCouple} that represent couples of integers, we want to create
a species {\tt OrderedIntCouple} in which we ensure that the first
component of the couple is lower or equal to the second. Instead of
inventing again all the species, we will take advantage of the
existing {\tt IntCouple} and ``import'' its printing, equality, etc
functions. However, we will have to change the creation function since
it must ensure at creation-time of the couple that it is really
ordered. We also may add new methods in the newly built species. The
inheritance mechanism also allows to redefine methods already existing
as long as they keep the same type. To have the same type means for
functions to have the same ML-like type, and for theorems to have the
same statement (but not the same proof).
{\scriptsize
\begin{lstlisting}
species IntCouple =
  representation = (int * int) ;
  let print (x in Self) = ... ;
  let create (x in int, y in int) = (x, y) ;
  let equal (c1, c2) =
    match (c1, c2) with
     | ((c11, c12), (c21, c22)) -> c11 = c21 && c12 = c22 ;
  ...
end ;;

species OrderedIntCouple inherits (IntCouple) =
  let create (x in int, y in int) =
    if x < y then (x, y) else (y, x) ;

  property is_ordered : all c in Self, first (c) <= scnd (c) ;
end ;;
\end{lstlisting}
}

In the example above, {\tt OrderedIntCouple} will have all the methods
of {\tt IntCouple}, except the redefined ones (here {\tt create}, plus
the methods exclusively defined in {\tt OrderedIntCouple} (here, the
property {\tt is\_ordered}) stating that the couple is really
ordered). During inheritance, it is also possible to redefine a
signature, replacing it by an effective definition, to redefine a
property by a theorem and in the same idea, to add a {\tt proof of} to
a property in order to conceptually redefine is as a theorem. Since
inherited methods are now owned by the species that inherits, they we
be called exactly like if they were defined ``from scratch'' in the
species.

\smallskip
\index{inheritance!multiple}
Multiple inheritance, i.e. inheriting from several species is
allowed by specifying several species separated by comma in the
{\tt inherits} clause. In case of methods appearing in several
parents, the kept one is the one coming from the rightmost parent in
the {\tt inherits} clause. For instance below, if species {\tt A} and
{\tt C} provide a method {\tt m}, {\tt C's} one will be kept.
{\scriptsize
\begin{lstlisting}
species Foo inherits A, B, C, D =
  ... m (...) ... ;
end ;;
\end{lstlisting}
}

\smallskip
\index{inheritance!parametrised species}
If a species {\tt S1} inherits from a parametrised species {\tt S0},
it must instantiate all the parameters of {\tt S0}. Because of our
dependent types\index{type!dependent} framework, if {\tt S1} is itself
parametrised, it can use its own parameters to do that. Assuming we
have a species {\tt List} parametrised by a collection parameter
representing the kind of elements of the list. We now want to derive
a species {\tt ListUnique} in which elements are present at most
once. We then want {\tt ListUnique} to inherit from {\tt List} with
the elements begin the same between {\tt List} and {\tt ListUnique}.
{\scriptsize
\begin{lstlisting}
species List (Elem is ...) =
  let empty = ... ;
  let add (e in Elem, l in Self) = ... ;
  let concat (l1 in Self, l2 in Self) = ... ;
end ;;

species ListUnique (UElem is ...) inherits List (UElem) =
  let add (e in Elem, l in Self) =
    ... (* Ensure the element e is not already present. *) ;
  let concat (l1 in Self, l2 in Self) =
    ... (* Ensure elements of l1 present in l2 are not added. *) ;
end ;;
\end{lstlisting}
}

\index{inheritance!parametrised by {\tt Self}}
A species can also inherit from a species parametrised by itself
(i.e. by {\tt Self}). Although this is rather tricky programming, the
standard library of \focal\ shows such an example in the file
{\em weak\_structures.foc} in the species
{\tt Commutative\_semi\_ring}. In such a case, this implies that the
current species must finally (when inheritance is resolved) have an
interface compatible with the interface required by the collection
parameter of the species we inherit. The \focal\ compiler will then
collect all the interfaces {\tt Self} must be compatible with due to
parametrised inheritance and will ensure afterwards that once built,
{\tt Self} has a really compatible interface.



%%%%%%%%%%%%%%%%%%%%%%%%%%%%%%%%%%%%%%%%%%%%%%%%%%%%%%%%%%%%%%%%%%%%%%%
\subsubsection{Conclusion on species expressions}
\index{species!expression}
At the beginning of this presentation, when dealing with collection
parameters (section \ref{parametrisation}) we explained that
collection parameters ``expressions'' were, first, species names. Like
in the example:
{\scriptsize
\begin{lstlisting}
species List (Elem is Basic_object) = ... ;
\end{lstlisting}
}

Then, we learnt about parametrised species and we saw that we can have
a species parametrised by a parametrised species, like in the example:
{\scriptsize
\begin{lstlisting}
species MyCouple (E is Basic_object, L is List (E)) = ... ;;
\end{lstlisting}
}

Going on, we got interested in inheritance and saw we could inherit
from species that were referenced only by their name, like in:
{\scriptsize
\begin{lstlisting}
species OrderedIntCouple inherits (IntCouple) = ... ;;
\end{lstlisting}
}

And finally, we also saw species inheriting from parametrised
species, like in:
{\scriptsize
\begin{lstlisting}
species ListUnique (UElem is ...) inherits List (UElem) = ... ;;
\end{lstlisting}
}

Hence, we can now defined more accurately the notion of {\bf species
expression} used for both inheritance and parametrisation. It is either
a simple species name or the application of a parametrised species to
as many species expressions as the parametrised species has
parameters.


%%%%%%%%%%%%%%%%%%%%%%%%%%%%%%%%%%%%%%%%%%%%%%%%%%%%%%%%%%%%%%%%%%%%%%%
\subsubsection{Late-binding}
\label{late-binding}
\index{late-binding}
We stated just above (c.f. \ref{inheritance}) that it was possible
during inheritance to replace a signature by a function or a property
by a theorem. On another hand, we saw that it was possible to define
function using a signature (c.f \ref{idea-fun-using-sig}), that is
something only declared, with no implementation yet. In the same
order, it is possible to redefine a method already used by an existing
method.

\smallskip
\focal\ proposes a mechanism known as {\em late-binding} to solve this
issue. During compilation, the selected method will be the {\bf most
recently defined} along the inheritance tree. This especially means
that as long as a method is a signature, in the children the effective
implementation of the method will remain undefined (that is not a
problem since in this case the species is not complete, hence cannot
lead to a collection, i.e. code that can really be executed
yet). Moreover, if a method {\tt m} previously defined in the
inheritance tree uses a method {\tt n} freshly {\bf re}defined, then
this {\bf fresh redefinition} of {\tt n} will be used in the method
{\tt m}.

\smallskip
This mechanism enables two programming features:
\begin{itemize}
  \item The mean to use a method known by its type (i.e. its interface
    in term of Software Engineering), but for which we do not know, or
    we don't need or we don't want yet to provide an implementation.

  \item To provide a new implementation of a method, for example more
    efficient because algorithms exist due to the refinements
    introduced by the inheriting species, while already having and
    keeping the initial implementation for the inherited species.
\end{itemize}



%%%%%%%%%%%%%%%%%%%%%%%%%%%%%%%%%%%%%%%%%%%%%%%%%%%%%%%%%%%%%%%%%%%%%%%
\subsubsection{Dependencies and erasing}
We previously saw that methods of a species can use other methods of
this species and methods from its collection parameters. This induce
what we call {\bf dependencies}\index{dependency}. There are two kinds
of dependencies, depending on their nature:
\begin{itemize}
  \item {\bf Decl-dependencies}
  \item {\bf Def-dependencies}
\end{itemize}
In order to understand the difference between, we must inspect further
the notion of carrier, function, and theorem.



\paragraph{Decl-dependencies}
\index{dependency!decl}
When defining a function, a property or a theorem it is possible to
use another function or signature. For instance:
{\scriptsize
\begin{lstlisting}
species Bla =
  signature test : Self -> bool ;
  let f1 (x in string) = ... ;
  let f2 (y in Self) = ... f1 ("Eat at Joe's") ... ;
  property p1 : all x in Self, test (f2 (x)) <-> test (f1 ("So what")) ;
  theorem t1 : all x in Self, p1 <->  test (f1 ("Bar"))
  proof = ... ;
end ;;
\end{lstlisting}
}

In this cases, knowing the type of the used methods is sufficient to
ensure that the using method is well-formed. We remind that the type
of a function and a signature is the ML-like type. And for a property
and a theorem, this type is their logical statement. The type of a
method being provided by its {\bf declaration}, we will call induced
dependencies {\bf decl-dependencies}.

Such dependencies also arise on the carrier as soon as the type of a
method makes reference to the type {\tt Self}. Hence we can have
dependencies on the carrier as well as on other methods (as
previously stated in \ref{rep-is-method}, {\tt representation} is a method).

Hence, in our example, {\tt test}, {\tt f2}, {\tt f1} (since it is
used in {\tt p1} and {\t1} as the argument of {\tt test} which expects
an argument of type {\tt Self}), {\tt p1} and {\tt t1} have a
decl-dependency on the carrier. Moreover, {\tt f2} has one on
{\tt f1}. The property {\tt p1} has decl-dependencies on {\tt test},
{\tt f1} and {\tt f2}. And finally {\tt t1} decl-depends on {\tt p1},
{\tt test} and {\tt f1}.



\paragraph{Def-dependencies}
\label{def-dependency}
\index{dependency!def}
It seems that when {\bf defining} a function, only decl-dependencies
may appear since the type system of \focal\ only needs the type of the
functions used by this function. This is right except about the
carrier. We must remind (c.f. \ref{rep-is-method}) that if {\tt rep}
is given, then it is {\bf defined} ! If it is not given, then it is
like if it was only {\bf declared}. Hence a function making usage
in its body of the known carrier representation (i.e. definition) will
have a {\bf def-dependency} on the carrier. Such dependencies means
that to ensure that the using method is well-formed the system need to
know the {\bf definition} of the entity we depend on.


\smallskip
In the same order, when {\bf using} a signature in another method,
since signature only contain types, no decl-dependencies can arise.

\smallskip
\index{dependency!def!on carrier}
Slightly differently, when {\bf using} a property in another method,
since it only contains a ``type'' (i.e. a logical statement), only 
decl-dependencies appear except on the carrier. For consistency
reasons going beyond this manual but that will be shortly presented
below in \ref{def-dep-on-carrier}, the {\bf \focal\ system rejects
properties having def-dependencies on the carrier}.

\smallskip
There remains the case of {\bf defining} a theorem. This case is the
more complex since it can lead to decl-dependencies via its
proofs. These dependencies are introduce by the statement of the
proof. We won't explain here the way to introduce them, this will be
done further in section \ref{zenon-an-dependencies}. First of all, for
the same reasons than for properties, the {\bf \focal\ system rejects
theorems having def-dependencies on the carrier}.

Now, what does mean for a theorem to def-depend on a method ? This
basically means that to make the proof of its statement, one must use
not only the declaration of a method, but also it definition, its
body. For instance, a theorem needs to have visibility on the code of
a function to have its proof possible.



\paragraph{Erasing during inheritance}
\index{erasing}
\label{erasing}
As a consequence of def-dependencies and late-binding, if a method is
redefined, all the proofs of theorems having def-dependencies on these
methods are erased. This means that since the body of the method
changed, may be the proof is not correct anymore and must be done
again. In practice, it can happen that the proof still holds, but the
compiler can't ensure this, hence will turn the theorem into a property
in the species where the redefinition occurred. The developer will
then have to provide a new proof of the inherited theorem thanks the
{\tt proof of} field.



\paragraph{Dependencies on collection parameters}
Since collection parameters always have their carrier abstracted,
hidden, only {\bf decl-dependencies} can appear in the parametrised
method using them. Hence they can never lead to erasing. These
dependencies are only used internally by the \focal\ compiler in order
to generate it target code. For this reason, we will not focus anymore
on them.
