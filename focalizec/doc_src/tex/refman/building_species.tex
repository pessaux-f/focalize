% $Id: building_species.tex,v 1.18 2012-10-30 12:27:54 pessaux Exp $

%%%%%%%%%%%%%%%%%%%%%%%%%%%%%%%%%%%%%%%%%%%%%%%%%%%%%%%%%%%%%%%%%%%%%%%
%%%%%%%%%%%%%%%%%%%%%%%%%%%%%%%%%%%%%%%%%%%%%%%%%%%%%%%%%%%%%%%%%%%%%%%
\section{Parametrisation}
\index{parametrisation}
\label{parametrisation}

This section describes a first mechanism to incrementally build new
species from existing ones: the parametrisation.
%%%%%%%%%%%%%%%%%%%%%%%%%%%%%%%%%%%%%%%%%%%%%%%%%%%%%%%%%%%%%%%%%%%%%%%
\subsection{Collection parameters}
\index{parameter!collection}
\index{collection!parameter}
\label{collection-parameter}
Remember that methods cannot be polymorphic\index{polymorphism}
(cf. \ref{no-polymorphism-for-methods}). For
example, how to implement the well-known polymorphic type of
lists ?  Grouping elements in a list does not depend  of
the type of these elements. The only constraint is that all elements
must have the same type. Hence, a ML-like representation of lists would be
like:

{\scriptsize
\lstset{language=Caml}
\begin{lstlisting}
type 'a list =
  | Nil
  | Cons of ('a * 'a list)
\end{lstlisting}}

The {\tt 'a} is a parameter of the constructor type {\tt list}, which
is indeed a polymorphic ML type.

In {\focal} we would like to create a species looking like:

{\scriptsize
\begin{lstlisting}
species List =
  signature nil : Self ;
  signature cons : 'a -> Self -> Self ;
end ;;
\end{lstlisting}}

Instead of abstracting the type parameter and leaving it free in the
context of the species, in {\focal} we {\em parametrise} the species
by a {\bf collection parameter}, the parameter named {\tt Elem} in the
example:

{\scriptsize
\begin{lstlisting}
species List (Elem is Basic_object) =
  signature nil : Self ;
  signature cons : Elem -> Self -> Self ;
end ;;
\end{lstlisting}}

The collection parameters are introduced by their name, followed by the {\tt is}
keyword, followed by an {\bf interface name} (remember that an interface has
the same name as its underlying species). In the example, {\tt Basic\_object}
is a pre-defined species from the standard library, containing only few
methods and this name is used here to denote the interface of this species.
A collection parameter can be instantiated by any collection which interface
is {\em compatible}\index{interface!compatibility} with the one required by
the parametrised species (cf. \ref{interface}). In the example, any effective
parameter instantiating {\tt Elem} is a collection which interface contains
at least the methods listed in the interface of {\tt Basic\_object}.

\smallskip
In the example, we use the parameter {\tt Elem} to build the signature
of the method {\tt cons}. Note that collection names can be used
in type expressions to denote the ``abstracted'' representation of the
collection. Here ``abstracted'' means that the
representation is not visible but we can refer to it
as an abstract type. In other words, {\tt Elem -> Self -> Self}
stands for the type of a function:
\begin{compact-itemize}
\item taking a first argument whose type is the representation  of a
  collection having a compatible interface with the interface {\tt
    Basic\_object}. (This especially means that such an argument is
  created using methods of the compatible collection),
  \item taking a second argument whose type is the representation of
    the current species,
  \item and returning a value whose type is the representation  of the
    current species.
\end{compact-itemize}

\noindent{\bf Why a collection parameter and not a species parameter?}

The answer to this question is especially important to
understand the programming model in {\focal}. It is a {\bf collection
parameter} because ultimately, at the terminal nodes of the
development, this parameter will have to be instantiated by an entity
where everything is defined, so at least a complete species. Imagine
how to build an executable code if a parameter can be instantiated by
a species with some methods only declared\ldots This is the first
reason.

Remember that properties mentioned in the collection interface have
been proved in the underlying complete species. Indeed in the hosting
species, these theorems can be used as lemmas to do current proofs. If
the collection representation was not abstracted, then some methods of
the hosting species would have the ability to directly manipulate
entities of the collection parameter, with the risk of breaking some
invariants of the collection parameter.  This is the second reason.
Thus the representation of a collection parameter is abstract for the
hosting, exactly as is the representation of a collection (cf.
\ref{collection}).

To summarize, declaring a collection parameter for a parametrised
species means providing two things: the (capitalized) name of the
parameter and the interface (denoted by a species name) that the
instantiation of this parameter must satisfy.


\smallskip It is important at this point to note that {\focal} deals with
dependent types\index{type!dependent}, and therefore that {\em
  the order of the parameters is important}. To define the type of a
parameter, one can use the preceding parameters. For instance,
assuming that a parametrised species {\tt List}
declares the basic operations over lists, one can specify a new
species working on couples of respectively values and lists of values
like:

{\scriptsize
\begin{lstlisting}
species MyCouple (E is Basic_object, L is List (E)) =
  representation = (E * L) ;
  ... ;
end ;;
\end{lstlisting}}

The representation of this species represents the type
{\tt ('a * ('a list))}. This means that the type of the values in the
first component of the couple is the same than the type of the
elements of the list in the second component of the couple.


A parametrized species (like {\tt MyCouple} in the example) cannot be only
partially instantiated. An instantiation for {\bf all} its parameters is
required.

% Once again, the preceding
% parameters can be used to achieve this purpose (like we did to create
% our parameter {\tt L}, instantiating the {\tt List}'s parameter by our
% first collection parameter {\tt E}).

\medskip
\label{method-qualification}
\index{method!qualification}
The previous example  used a parameter to build the representation of the
species.
Collection parameters can also be used via their other methods,
i.e. signatures, functions, properties and theorems, denoted by the
parameter's name followed by  the ``!''\index{bang character}
character followed by the method name.

To create a species describing a notion of generic couple, it suffices
to use two collection parameters, one for each component of the
couple. To define a printing (i.e. returning a string, not making side
effect in our example) method, it suffices to require each collection
parameter to provide one. Now the printing method has only to
add parentheses and comma around and between what is printed by each
parameter's printing routine.

{\scriptsize
\begin{lstlisting}
(* Minimal species requirement : having a print routine. *)
species Base_obj =
  signature print : Self -> string ;
end ;;

species Couple (C1 is Base_obj, c2 is Base_obj) =
  representation = (C1 * C2) ;
  let print (c in Self) =
    match (c) with
     | (component1, component2) ->
       "(" ^ C1!print (component1) ^
       ", " ^
       C2!print (component2) ^")" ;
end ;;
\end{lstlisting}}

Hence, {\tt C1!print (component1)} means ``call the collection
{\tt C1}'s method {\tt print} with the argument {\tt component1}''.

The qualification mechanism using ``!'' is general and can be used to
denote the method of any available species/collection, even those of
species being defined (i.e. {\tt Self}). Hence, in a species instead of calling:

{\scriptsize
\begin{lstlisting}
species Foo ... =
  let m1 (...) = ... ;
  let m2 (...) = if ... then ... else m1 (...) ;
end ;;
\end{lstlisting}}

it is allowed to explicitly qualify the call to {\tt m1} by ``!''
with no species name, hence implicitly telling ``from myself'':

{\scriptsize
\begin{lstlisting}
species Foo ... =
  let m1 (...) = ... ;
  let m2 (...) = if ... then ... else !m1 (...) ;
end ;;
\end{lstlisting}}
\index{name!resolution}
\index{scoping}

In fact, without explicit ``!'', the {\focal} compiler performs the
name resolution itself, allowing a lighter way of writing programs
instead of always needing a ``!'' character before each method call.



%%%%%%%%%%%%%%%%%%%%%%%%%%%%%%%%%%%%%%%%%%%%%%%%%%%%%%%%%%%%%%%%%%%%%%%
\subsection{Entity parameters}
\index{parameter!entity}
\label{entity-parameter}
There is a second kind of parameter: the {\bf entity-parameter}. Such
a parameter can be instantiated by an {\bf entity of a certain
collection}.

For example, to obtain a species offering addition modulo an
integer value, we need to parametrise it by an entity of a collection
implementing the integers and to give a way to build an entity
representing the value of the modulo. Such a parameter is called an
{\bf entity parameter} and is introduced by the keyword {\tt in}.

{\scriptsize
\begin{lstlisting}

species AddModN (Number is InterfaceForInts, val_mod in Number) =
  representation = Number ;
  let add (x in Self, y in Self) =
    Number!modulo (Number!add (x, y), val_mod) ;
end ;;

species
\end{lstlisting}}

Hence, any collection created from {\tt AddModN} embeds the addition
modulo the effective value instantiating {\tt val\_mod}. It is then
possible to create various collections with each a specific modulo
value. For instance, assuming that the species {\tt AddModN} is
complete and have a method {\tt from\_int} able to create a value of
the representation from an integer, we can create a collection
implementing addition modulo 42. We also assume that we have a
collection {\tt ACollImplentingInts} having at least {\tt
InterfaceForInts} as interface.

 {\scriptsize
\begin{lstlisting}
collection AddMod42 =
  implement
    AddModN
      (ACollImplentingInts,
       ACollImplentingInts!from_int (42));
end ;;
\end{lstlisting}}


\smallskip

Currently, entity parameters must live ``{\tt in}'' a collection. It is not
allowed to specify an entity parameter living in a basic type like {\tt int},
{\tt string}, {\tt bool}\ldots This especially means that these basic types
must be embedded in a collection if we want to use their values as entity
parameters.



%%%%%%%%%%%%%%%%%%%%%%%%%%%%%%%%%%%%%%%%%%%%%%%%%%%%%%%%%%%%%%%%%%%%%%%
%%%%%%%%%%%%%%%%%%%%%%%%%%%%%%%%%%%%%%%%%%%%%%%%%%%%%%%%%%%%%%%%%%%%%%%
\section{Inheritance and its mechanisms}

We now address the second mechanism to build complex species
based on existing ones. It will cover the notion of {\em inheritance}
and its related feature, the {\em late-binding}.



%%%%%%%%%%%%%%%%%%%%%%%%%%%%%%%%%%%%%%%%%%%%%%%%%%%%%%%%%%%%%%%%%%%%%%%
\subsection{Inheritance}
\label{inheritance}
\index{inheritance} {\focal} {\em inheritance} is the ability to create
a species, not from scratch, but by integrating methods of other
species.  The inheritance mechanism also allows to redefine methods
already existing as long as they keep the same type expression.  For
theorems to have the same type is simply to have the same statement
(but proofs can differ).

During inheritance, it is also possible to replace a signature by an
effective definition, to redefine a property by a theorem and in the
same idea, to add a {\tt proof of} to a property in order to
conceptually redefine it as a theorem. Moreover new methods can be
added to the inheriting species.

Since inherited methods are owned by the species that inherits, they are
called exactly like if they were defined ``from scratch'' in the species.

For instance, assuming we have a species {\tt IntCouple} that
represent couples of integers, we want to create a species {\tt
OrderedIntCouple} in which we ensure that the first component of the
couple is lower or equal to the second. Instead of inventing again all
the species, we will take advantage of the existing {\tt IntCouple}
and ``import'' all its methods. However, we will have to change the
creation function since it must ensure at creation-time of a couple
(so at run-time) that it is indeed properly ordered.
{\tt OrderedIntCouple} has all the methods of {\tt IntCouple}, except
{\tt create} which is redefined and the property {\tt is\_ordered} that
states that the couple is really ordered.

{\scriptsize
\begin{lstlisting}
species IntCouple =
  representation = (int * int) ;
  let print (x in Self) = ... ;
  let create (x in int, y in int) = (x, y) ;
  let first (c1, c2) = c1 ;
  ...
end ;;

species OrderedIntCouple =
  inherit IntCouple;
  let create (x in int, y in int) =
    if x < y then (x, y) else (y, x) ;

  property is_ordered : all c in Self, first (c) <= scnd (c) ;
end ;;
\end{lstlisting}}




\smallskip
\index{inheritance!multiple}
{\bf Multiple inheritance}, i.e. inheriting from several species is
allowed by specifying several species separated by comma in the
{\tt inherit} clause. The inheriting species inherits of all the
methods of inherited species. When a name appears more than once
in the inherited species, the compiler proceeds as follows.

If all the inherited species have only declared representations, then
the representation of the inheriting species is only declared, unless
it is defined in this inheriting species. If some representations are
declared, the other ones being defined, then the totally defined
representations of inherited species must be the same and this is also
the one of the inheriting species. In the following example, species
{\tt S3} will be rejected while species {\tt S4} has {\tt int} as
representation.

{\scriptsize
\begin{lstlisting}
species S0; -- no defined representation
end;;
species S1 =
representation = int ; .. end ;;
species S2 =
representation = bool; ... end;;
species S3 = inherit S1, S2; ... end;;
species S4 = inherit S0, S1; ... end;;
\end{lstlisting}}

If some methods of inherited species have the same name, if they are all
signatures or properties, if these species have no parameters, then
signatures and properties must be identical. If some of these methods have
already received a definition and if they have the same type, then the
definition which is retained for the inheriting species is the one coming
from the rightmost defining parent in the {\tt inherit} clause. For instance
below, if species {\tt A}, {\tt B} and {\tt C} provide a method {\tt m} which
is defined in {\tt A} and {\tt B} but only declared in {\tt C}, then
{\tt B!m} is the method inherited in {\tt Foo}.

{\scriptsize
\begin{lstlisting}
species Foo = inherit A, B, C, D;
  ... m (...) ... ;
end ;;
\end{lstlisting}}

\smallskip
\index{inheritance!parametrised species}
\noindent{\bf Inheritance and parametrisation}
If a species {\tt S1} inherits from a parametrised species {\tt S0},
it must instantiate all the parameters of {\tt S0}. Due to the
dependent types\index{type!dependent} framework, if {\tt S1} is itself
parametrised, it can use its own parameters to do that.

Assume we have a species {\tt List} parametrised by a collection parameter
representing the kind of elements of the list. We want to derive
a species {\tt ListUnique} in which elements are present at most
once. We build {\tt ListUnique} by inheriting from {\tt List}.

{\scriptsize
\begin{lstlisting}
species List (Elem is ...) =
  representation = Elem list;
  let empty = ... ;
  let add (e in Elem, l in Self) = ... ;
  let concat (l1 in Self, l2 in Self) = ... ;
end ;;

species ListUnique (UElem is ...) =
  inherit List (UElem);
  let add (e in UElem, l in Self) =
    ... (* Ensure the element e is not already present. *) ;
  let concat (l1 in Self, l2 in Self) =
    ... (* Ensure elements of l1 present in l2 are not added. *) ;
end ;;
\end{lstlisting}}

{\tt UElem} is a formal collection parameter of {\tt ListUnique} which acts
as an effective collection parameter in the expression {\tt ListUnique}. The
representation of {\tt ListUnique} is {\tt UElem list}. The representation of
{\tt UElem} is hidden: it denotes a collection. But, the value constructors
of the type {\tt list} are available, for instance, for pattern-matching.

As a consequence, if two methods in inherited species have the same
name and if at least one of them is itself a parametrised one, then
the signatures of these methods are no longer required to be identical
but their type must have a common instance after instanciation of the
collection parameters.

\index{inheritance!parametrised by {\tt Self}}
\noindent{\bf Species inheriting species parametrised by {\tt Self}}
A species can also inherit from a species parametrised by itself
(i.e. by {\tt Self}). Although this is rather tricky programming, the
standard library of {\focal} shows such an example in the file {\em
  weak\_structures.fcl} in the species {\tt
  Commutative\_semi\_ring}. Indeed this species specifies the fact
that a commutative semi-ring is a semi-ring on itself (as a semi-ring
of scalars).  In such a case, this implies that the current species
must finally (when inheritance is resolved) have an interface
compatible with the interface required by the collection parameter of
the inherited species. The {\focal} compiler collects the parts of the
interface of {\tt Self} obtained either by inheritance or directly in
the species body. Then it checks that the obtained interface is indeed
compatible with the required interfaces of the parametrised inherited
species. if so, the compiler is able to build the new species. Thus
the compiler tries to build a kind of fix-point but  this process is
always terminating, issuing either the new species or rejecting it in
case of interface non-compliance.



%%%%%%%%%%%%%%%%%%%%%%%%%%%%%%%%%%%%%%%%%%%%%%%%%%%%%%%%%%%%%%%%%%%%%%%
\subsection{Species expressions}
\index{species!expression}
% At the beginning of this presentation, when dealing with collection
% parameters (section \ref{parametrisation}) we explained that
% collection parameters ``expressions'' were, first, species names. Like
% in the example:
We summarize the different ways of building species. The first way is
to introduce a simple collection parameter, requiring that the
effective parameter can offer all the methods listed in the associated
interface.

{\scriptsize
\begin{lstlisting}
species List (Elem is Basic_object) = ... ;
\end{lstlisting}}

Then, we can iterate the process and build
a species parametrised by a parametrised species, like in the example:

{\scriptsize
\begin{lstlisting}
species MyCouple (E is Basic_object, L is List (E)) = ... ;;
\end{lstlisting}}

Going on, we can  inherit
from species that are referenced only by their name, like in:

{\scriptsize
\begin{lstlisting}
species OrderedIntCouple = inherit IntCouple; ... ;;
\end{lstlisting}}

And finally, we mix the two possibilities, building a species by
inheritance of a parametrised species, like in:

{\scriptsize
\begin{lstlisting}
species ListUnique (UElem is ...) = inherit List (UElem); ... ;;
\end{lstlisting}}

Hence, we can now define more accurately the notion of {\bf species
expression} used for both inheritance and parametrisation. It is either
a simple species name or the application of a parametrised species to
as many collection expressions as the parametrised species has
parameters.


%%%%%%%%%%%%%%%%%%%%%%%%%%%%%%%%%%%%%%%%%%%%%%%%%%%%%%%%%%%%%%%%%%%%%%%
\section{Late-binding and dependencies}

\subsection{Late-binding}
\label{late-binding}
\index{late-binding} When building by multiple inheritance
(cf. \ref{inheritance}) some signatures can be replaced by functions
and properties by theorems. It is also possible to associate a
definition of function to a signature (cf. \ref{idea-fun-using-sig})or
a proof to a property. In the same order, it is possible to redefine a
method even if it is  already used by an existing method.
All these features are relevant of a mechanism known as {\em late-binding}.


During compilation, the selected method is always the {\bf most
recently defined} along the inheritance tree. This especially means
that as long as a method is a signature, in the children the effective
implementation of the method will remain undefined (that is not a
problem since in this case the species is not complete, hence cannot
lead to a collection, i.e. code that can really be executed
yet). Moreover, if a method {\tt m} previously defined in the
inheritance tree uses a method {\tt n} freshly {\bf re}defined, then
this {\bf fresh redefinition} of {\tt n} will be used in the method
{\tt m}.

\smallskip
This mechanism enables two programming features:
\begin{compact-itemize}
  \item The mean to use a method known by its type (i.e. its prototype
    in term of Software Engineering), but for which we do not know, or
    we don't need or we don't want yet to provide an implementation.

  \item To provide a new implementation of a method while  keeping the
    initial implementation for the inherited species. For example, the
    inheriting species can provide some new information
    (representation, functions, ..) which allow a more efficient
    implementation of a given function.
\end{compact-itemize}



%%%%%%%%%%%%%%%%%%%%%%%%%%%%%%%%%%%%%%%%%%%%%%%%%%%%%%%%%%%%%%%%%%%%%%%
\subsection{Dependencies and erasing}\label{dependencies}
We previously saw that methods of a species can use other methods of
this species and methods from its collection parameters. This induce
what we call {\bf dependencies}\index{dependency}. There are two kinds
of dependencies, depending on their nature:
\begin{compact-itemize}
  \item {\bf Decl-dependencies}
  \item {\bf Def-dependencies}
\end{compact-itemize}
In order to understand the difference between, we must inspect further
the notion of representation, function, and theorem.



\subsubsection{Decl-dependencies}
\index{dependency!decl}
When defining a function, a property or a theorem it is possible to
use another functions or signatures. For instance:

{\scriptsize
\begin{lstlisting}
species Bla =
  signature test : Self -> bool ;
  let f1 (x in string) = ... ;
  let f2 (y in Self) = ... f1 ("Eat at Joe's") ... ;
  property p1 : all x in Self, test (f2 (x)) <-> test (f1 ("So what")) ;
  theorem t1 : all x in Self, p1 <->  test (f1 ("Bar"))
  proof = ... ;
end ;;
\end{lstlisting}}

In this cases, knowing the type (or the logical statement) of the used
methods is sufficient to ensure that the using method is
well-formed. The type of a method being provided by its
{\bf declaration}, we will call these induced dependencies
{\bf decl-dependencies}.

Such dependencies also arise on the representation as soon as the type
of a method makes reference to the type {\tt Self}. Hence we can have
dependencies on the representation as well as on other methods.

Hence, in our example, {\tt test}, {\tt f2}, {\tt f1} (since it is
used in {\tt p1} and {\tt t1} as the argument of {\tt test} which expects
an argument of type {\tt Self}), {\tt p1} and {\tt t1} have a
decl-dependency on the representation. Moreover, {\tt f2} has one on
{\tt f1}. The property {\tt p1} has decl-dependencies on {\tt test},
{\tt f1} and {\tt f2} and {\tt Self}. And finally {\tt t1}
decl-depends on {\tt p1}, {\tt test}, {\tt f1} and {\tt Self}.



\subsubsection{Def-dependencies}
\label{def-dependency}
\index{dependency!def}
A method $m$ has a {\bf def-dependency} over another one $p$ if the system
needs to know the {\bf definition} of $p$ to ensure that $m$ is well-formed.

A definition of function can create only decl-dependencies on
methods differing from the representation since the type system of
{\focal} only needs the types of the names present in the body of this
function. Note also that when {\bf using} a signature in another
method, since signature only contain types, no def-dependencies can
arise.

Now remember that {\tt representation} is also a method and there is no
syntactical way to forbid constructions like {\tt if representation = int ..}
in function or properties. Such definitions would have a {\bf def-dependency}
on the representation\index{dependency!def!on representation}.  For
consistency reasons going beyond the scope of this manual but that will be
shortly presented below in \ref{def-dep-on-representation}, the {\focal}
system rejects functions and properties having def-dependencies on the
representation.

\smallskip There remains the case of theorems. This case is the most
complex since it can lead to def-dependencies in proofs. For the same
reasons as for properties, the {\focal} system rejects theorems
whose statements have def-dependencies on the representation. Other
def-dependencies are accepted.  These dependencies must be introduced
by the statement of the proof (with a syntax given in section
\ref{zenon-and-dependencies}).  Now, what does mean for a theorem to
def-depend on a method ? This basically means that to make the proof
of the theorem statement, one must use not only the declaration of a
method, but also its definition, its body. This is a needed and powerful
feature.



\subsubsection{Erasing during inheritance}
\index{erasing}
\label{erasing}
As a consequence of def-dependencies and late-binding, if a method is
redefined, all the proofs of theorems having def-dependencies on these
methods are erased. This means that since the body of the method
changed, may be the proof is not correct anymore and must be done
again. In practice, it can happen that the proof still holds, but the
compiler can't ensure this, hence will turn the theorem into a
property in the species where the redefinition occurred. The developer
will then have to provide a new proof of the inherited theorem thanks
to the {\tt proof of} field. For example, any sorting list algorithm
must satisfy the invariant that its result is a sorted list with the
same elements as its effective argument  but the
proof that indeed this requirement is satisfied depends on the
different possible implementations of sort. It is perhaps possible to
decompose this proof into different lemmas to minimize erasing by
redefinition, some lemmas needing only decl-dependencies over the
redefined method.



\subsubsection{Dependencies on collection parameters}
Since collection parameters always have their representation abstracted,
hidden, only {\bf decl-dependencies} can appear in the parametrised
species using them. Hence they can never lead to erasing. These
dependencies are only used internally by the {\focal} compiler in order
to generate the target code. For this reason, we will not focus anymore
on them.
