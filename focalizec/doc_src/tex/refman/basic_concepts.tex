As stated in the introduction, \focal\ language is designed to build
an application step by step, going from very abstract specifications
to the concrete implementation through a hierarchy of structures which
are quite similar to \emph{classes} in an Object-Oriented context.
However, as we will insist later, \focal\ is not Object-Oriented as
C++, Java are. It only owns features ``smelling objects''.

\smallskip
We will now present the basic concepts underlying a
\focal\ development, that is:
\begin{itemize}
  \item Species
  \item Collections
  \item Parametrisation
  \item Inheritance
  \item Late-binding
\end{itemize}



%%%%%%%%%%%%%%%%%%%%%%%%%%%%%%%%%%%%%%%%%%%%%%%%%%%%%%%%%%%%%%%%%%%%%%%
%%%%%%%%%%%%%%%%%%%%%%%%%%%%%%%%%%%%%%%%%%%%%%%%%%%%%%%%%%%%%%%%%%%%%%%
\subsection{Species}
{\bf Species} \index{species} are the nodes of the
\focal\ hierarchy. A species can be roughly seen as a list of
{\bf methods} \index{method} or {\bf fields}\index{field}, each ended
by a semi character (``;'')\index{;;}. Hence, a
basic species looks like:
{\scriptsize
\begin{lstlisting}
  species Name =
    ... ;
    ... ;
  end ;;
\end{lstlisting}
}

Species names are always {\bf capitalized}. Inside the species' body
methods are enumerated. These methods represent the internal data-type
of the species, functions\index{function} to manipulate this data-type
and properties\index{property} that must hold in the species. As
any \focal\ toplevel definition, a species is ended by a double semi
character (``;;'').\index{;;}

\smallskip
\label{no-polymorphism-for-methods}
One important restriction on the type of the methods is that it cannot
be polymorphic\index{polymorphism}. However, \focal\ provides another
mechanism to circumvent this restriction, the
parametrisation\index{parametrisation} as explained further
(c.f. \ref{parametrisation}).

\smallskip
A species can hence be seen as an abstract data type. On another hand,
it can be seen as a kind of object, with its internal state (private
variables) and its methods. There are several kinds of methods:
\smallskip
\begin{itemize}
  \medskip
  \label{rep-is-method}
  \item The {\bf carrier}\index{carrier}. This is a type definition
    that defines\index{carrier!defined} the type of the values
    embedded in the species. When ``instantiated'', a species will
    lead to {\bf a value having this type}. This especially means that
    an instantiation of a species is {\bf NOT}, like in
    Object-Oriented languages an entity embedding data and code. The
    instantiation of a species will be a {\bf value}. The carrier is
    named {\tt rep}\index{rep}. For instance, we can define a species
    that implements couples of native integers data structure as:
    {\scriptsize
      \begin{lstlisting}
species IntCouple =
  rep = (int * int) ;
end ;;
      \end{lstlisting}
    }
    Each ``instance'' of this species will encapsulate {\bf 1}
    value of type ``couple of integers''. This especially means that
    the species doesn't represent a ``set'' of couples but
    {\bf 1} value that has type {\tt (int * int)}.

    \smallskip
    The carrier may remain not {\bf defined} in a species. However, in
    order to finally get a fully defined (or {\em complete}) species
    that will be instantiated, the carrier must be defined
    {\bf once} at one moment (during inheritance as described later in
    section \ref{inheritance}). \index{carrier!declared}However, the
    carrier is {\bf always implicitly declared}. This means that any
    species has a carrier that will remain declared until it is {\bf
    defined} ! 

    \smallskip
    In the context of a species, the type denoted by the carrier is
    named {\tt Self}.

  \medskip
  \item {\bf Signatures}\index{signature}. They introduce the
    ``prototype'' of a function, that is provide their type. As a
    first introduction, types expressions are closely like ML-like
    type expressions. A signature can be viewed as a declaration. It
    specifies the name of the operation then its type. For instance:
    {\scriptsize
      \begin{lstlisting}
 species IntStack =
   signature push : int -> Self -> Self ;
 end ;;
      \end{lstlisting}
    }
    As we saw above, {\tt Self} represents the underlying carrier type
    of the current species. Hence an operation pushing an integer onto
    a stack will take as parameter the integer to push, the stack on
    which to push and give back a new configuration of stack (where
    the pushed element is on the top), that is of type {\tt Self}. 

  \medskip
  \item {\bf Functions}\index{function}. They are operations that are
    allowed on the carrier's elements. They are implementations of
    signatures, providing effective code to execute. A function is
    introduced by the {\tt let} keyword. Recursive functions are
    introduced by {\tt let rec} to make explicit the recursivity. A
    function can be viewed as a definition.
    {\scriptsize
      \begin{lstlisting}
species IntStack =
  rep = int list ;
  let push (v in int, s in Self) = v :: s ;
end ;;
      \end{lstlisting}
    }
    Functions can use in their body other methods of the species,
    toplevel function definitions, methods of collections
    (described further in \ref{collection}), or methods of
    collections parameters (see \ref{collection-parameter}).

    \label{idea-fun-using-sig}
    When we say ``other methods of the species'', this includes
    signatures. This means that it is possible to use something only
    declared, without yet effective implementation. We will address
    this point later in detail in section \label{late-binding}.

    \smallskip
    Although \focal\ is a functional language, function application
    must always be total. This means that any function call must be
    provided all the effective arguments of the function. As
    described further in the core syntax (c.f \ref{core-syntax}),
    function application is ``� la C'', that is with arguments comma
    separated and enclosed by parentheses.

  \medskip
  \item {\bf Properties}\index{property}. They are first order logic
    propositions describing requirements that functions must
    meet. When stating a property, the proof that it holds is not yet
    provided. However it will have to finally to get a species that
    can be instantiated. A property can be viewed as a declaration.
    {\scriptsize
      \begin{lstlisting}
species IntStack =
  ...
  property push_returns_non_empty :
    all v in int, all s in Self, push (v, s) -> ~ is_empty (s) ;
end ;;
      \end{lstlisting}
    }

    Proofs\index{proof!delayed} of properties can be done afterwards
    using a {\tt proof} field in a species.The way to give proofs will
    be seen further.
    {\scriptsize
      \begin{lstlisting}
species IntStack2 inherits IntStack =
  proof of push_returns_non_empty = ... ;            
end ;;
      \end{lstlisting}
    }

  \medskip
  \item {\bf Theorems}\index{theorem}. They are properties with their
    proofs. In fact, when defining a property, we only give the
    statement of a theorem, leaving its proof for later. A theorem can
    be viewed as a definition.
    {\scriptsize
      \begin{lstlisting}
species IntStack =
  ...
  theorem push_returns_non_empty :
    all v in int, all s in Self, push (v, s) -> ~ is_empty
    (s)
  proof : ... ;
end ;;
      \end{lstlisting}
    }

\end{itemize}



%%%%%%%%%%%%%%%%%%%%%%%%%%%%%%%%%%%%%%%%%%%%%%%%%%%%%%%%%%%%%%%%%%%%%%%
%%%%%%%%%%%%%%%%%%%%%%%%%%%%%%%%%%%%%%%%%%%%%%%%%%%%%%%%%%%%%%%%%%%%%%%
\subsection{Collections}
\label{collection}
\index{collection}
A {\bf collection} is a kind of ``instance'' of a {\em complete}
\index{species!complete} species. A species is said complete if all
its methods are {\em defined}, i.e. have an implementation. In other
words this means that there is no more methods only {\em
  declared}. This notion implies that:
\begin{itemize}
  \item The carrier is defined by providing a type definition for
    {\tt rep}.
  \item For signature an effective function exists.
  \item For each property, a proof is given.
\end{itemize}

\smallskip
Obviously, it is possible to build a species without signatures and
properties, only providing functions and theorems directly. In this
case, if the carrier is also defined, then the obtained species is
trivially complete.

The important point for a species to be complete is that it can be
turned into effective executable code. In effect, since it is
complete, all its components are known then can be translated into a
target code.

The process of getting this effective executable code is by making a
collection that {\em implements} the species. As a result, one get an
entity whose internal representation has the type of the carrier of
the {\em implemented} species and that can be manipulated by the
methods present in this {\em implemented} species. However, the
manipulations will be done via the collection and not anymore via the
species.

{\scriptsize
\begin{lstlisting}
species Full =
  rep = int ;
  let create_random in Self = random_foc#random_int (42) ;
  let double (x in Self) = x + x ;
  let print (x in Self) = print_int (x) ;
end ;;

collection MyFull_Instance implements Full ;;

let v = Full.create_random ;;
Full.print (v) ;;
let dv = Full.double (v) ;;
Full.print (dv) ;;
\end{lstlisting}
}

In the above example, we defined a complete species {\tt Full}. Then
we ``take an instance'' of such a species by creating the collection
{\tt MyFull\_Instance}. We can then use methods of this collection on
values of this collection.

\smallskip
A very important point is that when creating a collection, the
representation of the carrier contained in the ``implemented'' species
gets opaque. In other words, the collection becomes an abstract
data-type. This especially means that it will be impossible to
manipulate values of the carrier without the collection's
methods. Moreover, two collections created from a same species will
not be type-compatible since their carriers got abstracted making now
impossible to ensure a type equivalence.

\smallskip
As a conclusion, collections are the only way to get something that
can be executed since they are the terminal items of a
\focal\ development hierarchy. Since they are ``terminal'', this also
means that no method can be added to a collection. Moreover, a
collection may not be used to create a new species by inheritance (as
explained in the next section).



%%%%%%%%%%%%%%%%%%%%%%%%%%%%%%%%%%%%%%%%%%%%%%%%%%%%%%%%%%%%%%%%%%%%%%%
%%%%%%%%%%%%%%%%%%%%%%%%%%%%%%%%%%%%%%%%%%%%%%%%%%%%%%%%%%%%%%%%%%%%%%%
\subsection{Interfaces}
\label{interface}
\index{interface}
The {\bf interface} of a species is the list of the declarations of
its methods. It corresponds to the end-user point of view, who wants
to know which functions he can use, and which properties these
functions have, but doesn't care about the details of the
implementation.

\smallskip
To get the interface of a species, we keep the methods only declared
(as previously stated, signatures and properties) and we forget the
bodies of the defined methods (carrier, functions and theorems).
Discarding the body of the carrier means we consider it was not given
at all, for a function we only keep its type and for a theorem we only
keep its statement. Hence, getting the interface of a species can
roughly be seen as erasing the carrier, turning the functions into
signatures and the theorems into properties.

\smallskip
While this abstraction is easy within programming languages, it is not
always possible when dealing with proofs and properties. Such
problematic species are rejected by \focal\ and will be described
later in \ref{species-constraints}.
