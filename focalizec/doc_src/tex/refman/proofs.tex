%%%%%%%%%%%%%%%%%%%%%%%%%%%%%%%%%%%%%%%%%%%%%%%%%%%%%%%%%%%%%%%%%%%%%%%
%%%%%%%%%%%%%%%%%%%%%%%%%%%%%%%%%%%%%%%%%%%%%%%%%%%%%%%%%%%%%%%%%%%%%%%
%%%%%%%%%%%%%%%%%%%%%%%%%%%%%%%%%%%%%%%%%%%%%%%%%%%%%%%%%%%%%%%%%%%%%%%
\section{Proofs of theorems}
As presented in \ref{proof-short-intro}, \focal\ proposes 3 ways to
make proof of properties. We will only deal here with proofs written
with the \focal Proof Languages. As a reminder, those written by a
direct \coq\ script will be addressed in \ref{coq-proofs}. And the
last kind of proof, by {\tt assumed} doesn't need anymore description
since it consists in bypassing the formal proof mechanism.

Conceptually, a proof is a tree......
