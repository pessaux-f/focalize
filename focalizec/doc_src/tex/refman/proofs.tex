% $Id: proofs.tex,v 1.7 2009-01-15 17:06:54 doligez Exp $



%%%%%%%%%%%%%%%%%%%%%%%%%%%%%%%%%%%%%%%%%%%%%%%%%%%%%%%%%%%%%%%%%%%%%%%
%%%%%%%%%%%%%%%%%%%%%%%%%%%%%%%%%%%%%%%%%%%%%%%%%%%%%%%%%%%%%%%%%%%%%%%
%%%%%%%%%%%%%%%%%%%%%%%%%%%%%%%%%%%%%%%%%%%%%%%%%%%%%%%%%%%%%%%%%%%%%%%
\section{Proofs of  theorems}
As presented in \ref{proof-short-intro}, \focal\ proposes 3 ways to
make proof of properties. We will only deal here with proofs written
with the \focal\ Proof Languages. As a reminder, those written by a
direct \coq\ script will be addressed in \ref{coq-proofs}. And the
last kind of proof, by {\tt assumed} doesn't need anymore description
since it consists in bypassing the formal proof mechanism.

The syntax of proofs is as follows.
\begin{syntax}
\syntaxclass{Proofs:}
proof & ::=  & proof\_step* qed\_step \\
      & \mid & \terminal{by}\ fact* \\
      & \mid & \terminal{conclude}
\end{syntax}

\begin{syntax}
\syntaxclass{Proof steps:}
proof\_step & ::=  & proof\_bullet\ statement\ proof
\end{syntax}

\begin{syntax}
\syntaxclass{Statements:}
statement & ::=  & [\terminal{assume}\ logical\_expr\ \terminal{,}]*
                   [\terminal{prove}\ logical\_expr]
\end{syntax}

\begin{syntax}
\syntaxclass{QED steps:}
qed\_step & ::=  & proof\_bullet\ \terminal{qed}\ proof \\
          & \mid & proof\_bullet\ \terminal{conclude}
\end{syntax}
