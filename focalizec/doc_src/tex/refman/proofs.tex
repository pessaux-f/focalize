% $Id: proofs.tex,v 1.15 2009-03-13 12:21:14 doligez Exp $

%%%%%%%%%%%%%%%%%%%%%%%%%%%%%%%%%%%%%%%%%%%%%%%%%%%%%%%%%%%%%%%%%%%%%%%
\section{Proofs of  theorems}
As presented in \ref{proof-short-intro}, {\focal} proposes 3 ways to
make proof of properties. We will only deal here with proofs written
in the {\focal} Proof Language. As a reminder, proofs written as
direct {\coq} scripts will be addressed in \ref{coq-proofs}. And the
last kind of proof, by \terminal{assumed} doesn't need anymore description
since it consists in bypassing the formal proof mechanism.

The syntax of proofs is as follows.
\begin{syn}
\nt{proof} \is
     \rep{\nt{proof\_step}} \nt{qed\_step}
\alt \tok{by} \reps{\nt{fact}}
\alt \tok{conclude}
\end{syn}

%\begin{syntax}
%\syntaxclass{Proofs:}
%proof & ::=  & proof\_step*\ qed\_step \\
%      & \mid & \terminal{by}\ fact+{} \\
%      & \mid & \terminal{conclude}
%\end{syntax}

A proof is either a leaf proof or a compound proof. A leaf proof
(introduced with the \terminal{by} or \terminal{conclude} keywords)
invokes \zenon{} with the
assumptions being the given facts and the goal being the goal of the
proof itself (i.e. the statement that is proved by this leaf proof).
See below for the kinds of facts that can be given.

The \terminal{conclude} keyword is used to invoke \zenon{} without
assumptions.

A compound proof is a sequence of steps that ends with a \terminal{qed}
step.  The goal of each step is stated in the step itself, except for
the \terminal{qed} step, which has the same goal as the enclosing
proof.

\begin{syn}
\nt{proof\_step} \is
     \nt{proof\_bullet} \nt{statement} \nt{proof}
\end{syn}

%\begin{syntax}
%\syntaxclass{Proof steps:}
%proof\_step & ::=  & proof\_bullet~~statement~~proof
%\end{syntax}

A proof step starts with a proof bullet, which gives its level of
nesting.  The top level of a proof is 0.  In a compound proof, the
steps are at level one plus the level of the proof itself.

\goodbreak
For example, consider the following proof.

\begin{verbatim}
  theorem foo : A -> (B -> A)
  proof =
    <1>1 assume h1: A,
         prove B -> A
      <2>1 assume h2: B,
           prove A
        by hypothesis h1
      <2>2 qed
        by step <2>1
    <1>2 qed
      conclude
\end{verbatim}

In this proof, the steps \verb"<1>1" and \verb"<1>2" are at level 1
and form a compound proof of the top-level theorem.  Step \verb"<1>1"
also has a compound proof, composed of steps \verb"<2>1"
and \verb"<2>2".  These are at level 2 (one more than the level of
their enclosing step).

After the proof bullet comes the statement of the step.  This is the
statement that is asserted and proved by this step.  At the end of
this step's proof, it becomes available as a fact for the next steps
of this proof.  In our example, step \verb"<2>1" is available in the
proof of \verb"<2>2", and \verb"<1>1" is available in the proof of
\verb"<1>2".  Note that \verb"<2>1" is not available in the proof of
\verb"<1>2": see section~\ref{sec:scoping} for the scoping rules.

After the statement is the proof of the step.  See below (under
Statements) for a description of what is the current goal for this
proof.

\begin{syn}
\nt{qed\_step} \is
    \nt{proof\_bullet} \tok{ qed} \nt{proof}
\alt\nt{proof\_bullet} \tok{ conclude}
\end{syn}

%\begin{syntax}
%\syntaxclass{QED steps:}
%qed\_step & ::=  & proof\_bullet\ \terminal{qed}\ proof \\
%          & \mid & proof\_bullet\ \terminal{conclude}
%\end{syntax}

A \terminal{qed} step is similar to a normal step, except that its
statement is the goal of the enclosing proof.  It may be reduced to
the word \terminal{conclude} when its proof is reduced to
\terminal{conclude}.  In our example, we could have replaced
\verb"<1>2" with:
\begin{verbatim}
    <1>2 conclude
\end{verbatim}

\begin{syn}
\nt{statement} \is
     \rep{\tok{assume} \nt{assumption} \tok{,}}
     \opt{\tok{prove} \nt{logical\_expr}}
\end{syn}

%\begin{syntax}
%\syntaxclass{Statements:}
%statement & ::=  & {\{\terminal{assume}\ assumption\ \terminal{,}\}*}
%                   \ \{\terminal{prove}\ logical\_expr\}?
%\end{syntax}
A statement must be non-empty: at least one \terminal{assume} or the
\terminal{prove} part must be present.

A statement appearing in a step has two readings: internal and
external.  The external reading is for the rest of the
proof: the current step proves that the assumptions imply the
conclusion (i.e. the {\em logical\_expr} that appears after
\terminal{prove}).  The internal reading is for the proof of the step:
the current goal is the \terminal{prove} expression, and the
assumptions are available as facts.

\begin{syn}
\nt{assumption} \is
    \nt{ident} \tok{in} \nt{type\_expr}
\alt\nt{ident} \tok{ :} \nt{logical\_expr}
\end{syn}

%\begin{syntax}
%\syntaxclass{Assumptions:}
%assumption & ::= & ident\ \terminal{in}\ type\_expr \\
%           & \mid & ident\ \terminal{:}\ logical\_expr
%\end{syntax}

An assumption can either introduce a new (universally quantified)
variable with its type (first form), or a new named hypothesis (second
form).

\begin{syn}
\nt{fact} \is
     \tok{definition~of}
        \repsep{\opt{ \opt{\nt{ident}} \tok{\#} } \nt{ident}}{\tok{,}}
\alt \tok{hypothesis}
        \repsep{\nt{ident}}{\tok{,}}
\alt \paren{\tok{property} \orelse \tok{theorem}}
        \repsep{\opt{\opt{\opt{\opt{\nt{ident}} \tok{\#}} \nt{ident}}
                     \tok{!}} \nt{ident}}{\tok{,}}
\alt \tok{step}
        \repsep{\nt{proof\_bullet}}{\tok{,}}
\end{syn}

%\begin{syntax}
%\syntaxclass{Facts:}
%fact & ::= & \terminal{definition~of}
%                     \ \{ident?\terminal{\#}\}?ident
%     \ \{\{\terminal{,}\ ident?\terminal{\#}\}?ident\}*
%\\
%     & \mid & \terminal{hypothesis}
%                     \ ident
%     \ \{\terminal{,}\ ident\}*
%\\
%     & \mid & \terminal{property}
%          \ \{\{\{ident?\terminal{\#}\}?ident\}?\terminal{!}\}?ident
%\ \{\{\{\{\terminal{,}\ ident?\terminal{\#}\}?ident\}?\terminal{!}\}?ident\}*
%\\
%     & \mid & \terminal{theorem}
%          \ \{\{\{ident?\terminal{\#}\}?ident\}?\terminal{!}\}?ident
%\ \{\{\{\{\terminal{,}\ ident?\terminal{\#}\}?ident\}?\terminal{!}\}?ident\}*
%\\
%     & \mid & \terminal{step}
%                 \ proof\_bullet
%  \ \{\terminal{,}\ proof\_bullet\}*
%\end{syntax}

A fact used in a leaf proof can be a definition, a hypothesis, a
property, a theorem, or a step.

Giving a definition as a fact allows \zenon{} to unfold this
definition in the goal and in the other facts.

Giving a hypothesis/property/theorem as a fact allows \zenon{} to use
this hypothesis/property/theorem to prove the goal.

Giving a {\em proof\_bullet} as a fact allows \zenon{} to use the
(external reading of the) corresponding step as an assumption to prove
the goal.  Note that even if several steps are labelled with this
proof bullet, only one of them is in scope at any point, so there is
no ambiguity (see section~\ref{sec:scoping}).

\subsection{Scoping rules}\label{sec:scoping}

The scope of a step bullet extends from the end of the proof of that
step to the end of the proof of the enclosing step (i.e. the end of
the proof of the \terminal{qed} step that has the same level as this
step).  This means that proof bullets can be reused in other branches
of the proof to name different steps.

The scope of an assumption is the proof of the step where this
assumption appears.
