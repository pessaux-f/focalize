%%%%%%%%%%%%%%%%%%%%%%%%%%%%%%%%%%%%%%%%%%%%%%%%%%%%%%%%%%%%%%%%%%%%%%%
\section*{Unable to find file 'name' in the search path.}

{\em Description}: The source file made reference to a \focal\ module
$name$ (by the {\tt open} or {\tt use} directives, or by explicit
qualification with the ``\#'' notation but the related
\focal\ interface file was not found in the current libraries search
path.

{\em Hints}: Locate in which directory the missing interface file is
and add this directory to the libraries search path with the {\tt -I}
compiler option.



%%%%%%%%%%%%%%%%%%%%%%%%%%%%%%%%%%%%%%%%%%%%%%%%%%%%%%%%%%%%%%%%%%%%%%%
\section*{Invalid or corrupted compiled interface '$name$'.}
{\em Description}: The source file made reference to a \focal\ module
$name$ (by the {\tt open} or {\tt use} directives, or by explicit
qualification with the ``\#'' notation but the related
\focal\ interface file was found with an incorrect format.

{\em Hints}: May be the interface was compiled with another version
of \focal\ or was mangled and you must compile it again with your
current version.



%%%%%%%%%%%%%%%%%%%%%%%%%%%%%%%%%%%%%%%%%%%%%%%%%%%%%%%%%%%%%%%%%%%%%%%
\section*{Invalid file extension for '$name$'.}
{\em Description}: The \focal\ compiler expects input files to be
ended by the suffix ``.foc''. If the submitted input file doesn't end
by this suffix, this error message arises with the name, $name$ of the
involved file.

{\em Hints}: Change the extension of the input file name or ensure the
submitted input file name is the correct one.



%%%%%%%%%%%%%%%%%%%%%%%%%%%%%%%%%%%%%%%%%%%%%%%%%%%%%%%%%%%%%%%%%%%%%%%
\section*{System error - $sysmsg$.}
{\em Description}: During the compilation process an error related to
the operating system occurred (I/O error, permission error, file-system
error, \ldots). The original message $sysmsg$ of the system explaining
the problem follows the \focal's message.

{\em Hints}: Consult the original message of the system and get an
appropriate solution depending on this message.



%%%%%%%%%%%%%%%%%%%%%%%%%%%%%%%%%%%%%%%%%%%%%%%%%%%%%%%%%%%%%%%%%%%%%%%
\section*{Input file name is already set.}
{\em Description}: The \focal\ compiler only compiles one file at
once. If several files are specified on the command line this message
arises.

{\em Hints}: If several input files are specified, invoke the
compiler once for each separately. Check if among what is considered
as an input file, you didn't forget to put a dash (``-'') character,
hence turning an option into a regular string considered as another
input file name.



%%%%%%%%%%%%%%%%%%%%%%%%%%%%%%%%%%%%%%%%%%%%%%%%%%%%%%%%%%%%%%%%%%%%%%%
\section*{No input file. FoCaL is cowardly and gives up...}
{\em Description}: The \focal\ compiler needs one input file to
compile. If none is supplied, this error message arises.

{\em Hints}: Add the input source file to compile on the command
line.



%%%%%%%%%%%%%%%%%%%%%%%%%%%%%%%%%%%%%%%%%%%%%%%%%%%%%%%%%%%%%%%%%%%%%%%
\section*{Lexical error $str$}
{\em Description}: In the currently submitted source file, a sequence
of characters is not recognised as legal according to the
\focal\ programming language legal words structure. The involved
character $str$ follows in the error message.

{\em Hints}: Change the source code at the indicated location.



%%%%%%%%%%%%%%%%%%%%%%%%%%%%%%%%%%%%%%%%%%%%%%%%%%%%%%%%%%%%%%%%%%%%%%%
\section*{Syntax error.}
{\em Description}: In the currently submitted source file, a phrase of
the program doesn't follow \focal's syntax.

{\em Hints}: Change the source code at the indicated location. It
sometimes happen that the location gets fuzzy due to the parsing
process. If the error is not immediate to you, think to have a look at
around the specified location. If you still can't find out the error,
have the following emergency process: comment your code and
incrementally uncomment it to find the point where the error appears
without having to search in the whole file. Once the error appears,
have a look at the part of code you uncommented since the previous
successful compilation and try to guess the syntactic cause.



%%%%%%%%%%%%%%%%%%%%%%%%%%%%%%%%%%%%%%%%%%%%%%%%%%%%%%%%%%%%%%%%%%%%%%%
\section*{Unclear syntax error $msg$.}
{\em Description}: An error occurred during the syntactic analysis but
was not reported to be due to a syntax non-compliance. This error is
not clearly identified and this message is displayed as post-mortem
report with the exception $msg$ that caused the error.

{\em Hints}: None



%%%%%%%%%%%%%%%%%%%%%%%%%%%%%%%%%%%%%%%%%%%%%%%%%%%%%%%%%%%%%%%%%%%%%%%
\section*{Module '$m$' was "use" several times.}
{\em Description}: Several occurrences of a {\tt use} directive
specify the same module $m$.

{\em Hints}: Remove the spurious occurrences.



%%%%%%%%%%%%%%%%%%%%%%%%%%%%%%%%%%%%%%%%%%%%%%%%%%%%%%%%%%%%%%%%%%%%%%%
\section*{Module '$m$' was not declared as "use"}
{\em Description}: {\color{red}To to once ``use'' really works.}

{\em Hints}: {\color{red}To to once ``use'' really works.}



%%%%%%%%%%%%%%%%%%%%%%%%%%%%%%%%%%%%%%%%%%%%%%%%%%%%%%%%%%%%%%%%%%%%%%%
\section*{Species application expression expected $n_1$ arguments but
  was provided $n_2$.}
{\em Description}: A species expression (used in species parameter
expression or {\tt inherits} clause) applies a species with $n_1$
argument(s) although its definition declared it as using $n_2$
argument(s).

{\em Hints}: Source program to fix.



%%%%%%%%%%%%%%%%%%%%%%%%%%%%%%%%%%%%%%%%%%%%%%%%%%%%%%%%%%%%%%%%%%%%%%%
\section*{Non-logical let must not bind '$ident$' to a property.}
{\em Description}: A {\tt let} construct (not a {\tt logical let})
attempts to bind the identifier $ident$ to a logical expression
although it can only bind it to a computational expression.

{\em Hints}: Source program to fix. May be the {\tt let} should be
turned into a {\tt logical let} if the body of the binding is really a
logical expression.



%%%%%%%%%%%%%%%%%%%%%%%%%%%%%%%%%%%%%%%%%%%%%%%%%%%%%%%%%%%%%%%%%%%%%%%
\section*{Delayed termination proof refers to an unknown method
  '$ident$' of the species.}
{\em Description}: A {\tt proof of} clause was found in a species for
the property $ident$ but this property was not found in the species.

{\em Hints}: Source program to fix.



%%%%%%%%%%%%%%%%%%%%%%%%%%%%%%%%%%%%%%%%%%%%%%%%%%%%%%%%%%%%%%%%%%%%%%%
\section*{Ambiguous logical expression. Add explicit parentheses to
  associate the $side$ argument of the $/\setminus$ properly.}
{\em Description}: A logical expression contains a
{\tt $/\setminus$} (logical ``and'') with at least one argument being a
{\tt -> } (logical ``implication'') or a {\tt <->} (logical
``equivalence'') without parentheses around the $side$ argument (``left''
or ``right''). Since this is not clear of how to associate, we prefer
to asks the user to explicitly add parentheses.

{\em Hints}: Explicitly add the parentheses to make the association
non-ambiguous.



%%%%%%%%%%%%%%%%%%%%%%%%%%%%%%%%%%%%%%%%%%%%%%%%%%%%%%%%%%%%%%%%%%%%%%%
\section*{Ambiguous logical expression. Add explicit parentheses to
  associate the $side$ argument of the {\tt $\setminus/$} properly.}
{\em Description}: A logical expression contains a
{\tt $\setminus/$} (logical ``or'') with at least one argument being a
{\tt -> } (logical ``implication'') or a {\tt <->} (logical
``equivalence'') without parentheses around the $side$ argument (``left''
or ``right''). Since this is not clear of how to associate, we prefer
to asks the user to explicitly add parentheses.

{\em Hints}: Explicitly add the parentheses to make the association
non-ambiguous.



%%%%%%%%%%%%%%%%%%%%%%%%%%%%%%%%%%%%%%%%%%%%%%%%%%%%%%%%%%%%%%%%%%%%%%%
\section*{Unbound sum type constructor '$name$'.}
{\em Description}: An identifier representing a sum type constructor
was not found among the available sum type definitions.

{\em Hints}: Source program to fix. Since in core expressions
capitalized identifiers are considered as sum types constructors, may
be you tried to use a capitalized name for one of your variables. In
this case, as any variables, make it starting with a lowercase
letter. Otherwise, may be your type definition is missing or not
reachable in the current scope (missing explicit qualification with
the ``\#'' notation or {\tt open} directive if your type definition is
hosted in another source file).



%%%%%%%%%%%%%%%%%%%%%%%%%%%%%%%%%%%%%%%%%%%%%%%%%%%%%%%%%%%%%%%%%%%%%%%
\section*{Unbound record field label '$name$'.}
{\em Description}: An identifier representing a record type label
was not found among the available record type definitions.

{\em Hints}: Source program to fix. May be your type definition is
missing or not reachable in the current scope (missing explicit
qualification with the ``\#'' notation or {\tt open} directive if your
type definition is hosted in another source file).



%%%%%%%%%%%%%%%%%%%%%%%%%%%%%%%%%%%%%%%%%%%%%%%%%%%%%%%%%%%%%%%%%%%%%%%
\section*{Unbound identifier '$name$'.}
{\em Description}: An identifier (expected to be bound by a {\tt let},
a pattern of a function parameter declaration) was not found.

{\em Hints}: Source program to fix. May be your definition should be
toplevel and is missing or not reachable in the current scope (missing
explicit qualification with the ``\#'' notation or {\tt open}
directive if your definition is hosted in another source file).



%%%%%%%%%%%%%%%%%%%%%%%%%%%%%%%%%%%%%%%%%%%%%%%%%%%%%%%%%%%%%%%%%%%%%%%
\section*{Unbound type '$name$'.}
{\em Description}: The definition of an identifier expected to be a
type constructor was not found.

May be your type definition is missing or not reachable in the current
scope (missing explicit qualification with the ``\#'' notation or
{\tt open} directive if your type definition is hosted in another
source file).



%%%%%%%%%%%%%%%%%%%%%%%%%%%%%%%%%%%%%%%%%%%%%%%%%%%%%%%%%%%%%%%%%%%%%%%
\section*{Unbound module '$name$'.}
{\em Description}: A {\tt open} or {\tt use} directive or an explicit
qualification by the ``\#'' notation makes reference to a module that
(interface file) was not found in the current libraries search path.

{\em Hints}: Locate in which directory the missing interface file is
and add this directory to the libraries search path with the {\tt -I}
compiler option.



%%%%%%%%%%%%%%%%%%%%%%%%%%%%%%%%%%%%%%%%%%%%%%%%%%%%%%%%%%%%%%%%%%%%%%%
\section*{Unbound species '$name$'.}
{\em Description}: The definition of the species $name$ was not found
in the current scope.

{\em Hints}: May be your species definition is missing or not
reachable in the current scope (missing explicit qualification with
the ``\#'' notation or {\tt open} directive if your type definition is
hosted in another source file).




%%%%%%%%%%%%%%%%%%%%%%%%%%%%%%%%%%%%%%%%%%%%%%%%%%%%%%%%%%%%%%%%%%%%%%%
\section*{Type name '$name$' already bound in the current scope}.
{\em Description}: In a source file it is not allowed to redefine a
type definition. This means that each type name definition must be
unique inside a file. However, it is possible to have several type
definitions with the same names as long as they are in different
source files (even if they are used together via {\tt open} directives
of explicit qualification by the ``\#'' notation).

{\em Hints}: Source program to fix.



%%%%%%%%%%%%%%%%%%%%%%%%%%%%%%%%%%%%%%%%%%%%%%%%%%%%%%%%%%%%%%%%%%%%%%%
\section*{Species name '$name$' already bound in the current scope.}
{\em Description}: In a source file it is not allowed to redefine a
species definition. This means that each species name definition must
be unique inside a file. However, it is possible to have several species
definitions with the same names as long as they are in different
source files (even if they are used together via {\tt open} directives
of explicit qualification by the ``\#'' notation).

{\em Hints}: Source program to fix.



%%%%%%%%%%%%%%%%%%%%%%%%%%%%%%%%%%%%%%%%%%%%%%%%%%%%%%%%%%%%%%%%%%%%%%%
\section*{Types $t_1$ and $t_2$ are not compatible.}
{\em Description}: The typechecking system detected a type conflict
between two expressions $t_1$ and $t_2$ that were expected to be
type-compatible.

{\em Hints}: Source program to fix. This message sometimes involves
one the the types being {\tt Self}. This is mostly due to an attempt
to use the type structure of {\tt rep} although is turned abstracted
by the collection or parametrisation mechanisms. In this case, ensure
that you are not trying to make assumptions on the type {\tt Self} of
a species parameter or a collection.



%%%%%%%%%%%%%%%%%%%%%%%%%%%%%%%%%%%%%%%%%%%%%%%%%%%%%%%%%%%%%%%%%%%%%%%
\section*{Type $t_1$ occurs in $t_2$ and would lead to a cycle.}
{\em Description}: The \focal\ type system does not allow cyclic
types. This especially means that a type expression must not be a
sub-part of itself to prevent cycles.

{\em Hints}: Source program to fix.



%%%%%%%%%%%%%%%%%%%%%%%%%%%%%%%%%%%%%%%%%%%%%%%%%%%%%%%%%%%%%%%%%%%%%%%
\section*{Type constructor '$name$' used with the different arities
  $n_1$ and $n_2$.}
{\em Description}: A type expression applies a type constructor $name$
to $n_1$ argument(s) although its definition declared it as using $n_2$
argument(s) (or in the other order, depending on the way the error was
detected: in any way the definition and the usage of the type involve
2 different numbers of arguments).

{\em Hints}: Source program to fix.



%%%%%%%%%%%%%%%%%%%%%%%%%%%%%%%%%%%%%%%%%%%%%%%%%%%%%%%%%%%%%%%%%%%%%%%
\section*{No expected argument(s).}
{\em Description}: A type expression applies a type constructor to
arguments although this constructor needs none.

{\em Hints}: Source program to fix.



%%%%%%%%%%%%%%%%%%%%%%%%%%%%%%%%%%%%%%%%%%%%%%%%%%%%%%%%%%%%%%%%%%%%%%%
\section*{In field '$name$', type scheme $sch$ contains variables than
  cannot be generalized or is polymorphic.}

{\em Description}: As presented in \ref{no-polymorphism-for-methods},
species methods cannot be polymorphic. The method $name$ has a
polymorphic type scheme.

{\em Hints}: Source program to fix. You may explicitly add type
annotations (constraints) on the arguments or/and return type of your
method definition.
