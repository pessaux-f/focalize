\documentclass[10pt,a4paper]{article}
% No page numbers.
\renewcommand\thepage{}


\def\month{August}
\def\year{2009}
\def{\focalversion}{{\sf 0.1.5}}

\def{\focalversiondate}{{\sf {\month} {\year}}}

\newenvironment{citemize}{\begin{list}{$\bullet$}{\itemsep 0pt \topsep 5pt}}{\end{list}}

\def\bbbn{{\rm I\!N}}

\def\url#1{{#1}}
\def\email#1{{\tt#1}}
\def\ocaml{{\sf OCaml}}
\def\moca{{\sf Moca}}
\def\focal{{\sf FoCaLiZe}}
\def\coq{{\sf Coq}}
\def\cime{{\sf CiME}}
\def\zenon{{\sf Zenon}}
\def\zvtov{{\sf zvtov}}
\def\focdoc{{\sf FoCaLiZeDoc}}
\def\dotty{{\sf dotty}}
\def\graphviz{{\sf graphviz}}
\def\emacs{{\sf emacs}}
\def\cygwin{{\sf Cygwin}}
\def\xsltproc{{\sf xsltproc}}
\def\terminal#1{\mbox{\tt\color{red}#1}}
\def\foc{{\sf Foc}}
\def\oldfocal{{\sf Focal}}
\def\focalizec{{\sf focalizec}}

\def\backslash{{\char92}}
\def\chapeau{{\char94}}
\def\vertical{{\char124}}
\def\tilde{{\char126}}

%% Pour afficher les listings de mani�re plus sexy... ;)
\lstdefinelanguage{FoCaLiZe}
  {morekeywords={alias, all, and, as, assume, assumed, begin, by, caml,
      collection, coq, coq_require, definition, else, end,
      ex, external, false, function, hypothesis, if, in,
      inherit, internal, implement, is, let, lexicographic,
      local, logical, match, measure, not, notation, of, open,
      on, or, order, proof, prop, property, prove, qed, rec,
      representation, Self, signature, species, step,
      structural, termination, then, theorem, true, type, use,
      with, conclude},
    otherkeywords={->, :, /\\, \\/, =>, \~, \#, ;, \&, \|},
    sensitive=false,
    morecomment=[n]{(*}{*)},  %% Autoriser les nested comments
    morecomment=[n]{(**}{*)}, %% Autoriser les nested comments
    morestring=[b]",
  }

\lstset{
  language=FoCaLiZe, tabsize=2, frame=single, breaklines=true,
  basicstyle=\ttfamily, framexleftmargin=1mm, xleftmargin=1mm
}

% For syntax definitions
% \begin{syntax}
% left-hand side & ::=  & right-hand side & comment \\
%                & \alt & more right-hand side & more comment
% \end{syntax}
% Use \syntaxclass{Foo} to insert a title line above a syntax definition.
% Do not put \\ before \end{syntax} or \syntaxclass{...}
% Use \begin{syntaxleft} ... \end{syntaxleft} to insert the title
% lines to the left of the definitions.

% \def\syntax{
% \par\medskip\goodbreak\noindent
% \bgroup
% \let\\=\cr
% \interlinepenalty=50            % discourage page breaks in a definition
% \global\let\syntaxclass=\firstsyntaxclass
% \if@twocolumn
% \halign\bgroup~~$##$&\hfil${}##{}$&$##$~\hfil&##\hfil\cr
% \else
% \halign\bgroup\qquad\qquad$##$&\hfil${}##{}$&$##$\quad\hfil&##\hfil\cr
% \fi
% }
% \def\endsyntax{\cr\egroup\egroup\par\medskip\noindent\ignorespaces}

% \def\firstsyntaxclass#1{
% \omit\hbox to 0pt{#1\hss}\cr
% \global\let\syntaxclass=\nextsyntaxclass
% }

% \def\nextsyntaxclass#1{
% \cr\noalign{\smallskip\penalty-100}\omit\hbox to 0pt{#1\hss}\cr
% }

% \def\syntaxleft{
% \par\medskip\goodbreak\noindent
% \bgroup
% \let\\=\cr
% \interlinepenalty=50            % discourage page breaks in a definition
% \global\let\syntaxclass=\firstsyntaxclassleft
% \if@twocolumn
% \halign\bgroup\hfil$##$&\hfil${}##{}$&$##$~\hfil&##\hfil\cr
% \else
% \halign\bgroup\hfil$##$&\hfil${}##{}$&$##$\quad\hfil&##\hfil\cr
% \fi
% }
% \let\endsyntaxleft=\endsyntax

% \def\firstsyntaxclassleft#1{
% $\hfilneg#1\quad\hfil$
% \global\let\syntaxclass=\nextsyntaxclassleft
% }

% \def\nextsyntaxclassleft#1{
% \cr\noalign{\smallskip\goodbreak}$\hfilneg#1\quad\hfil$
% }



\begin{document}

\begin{center}
{\Huge \textbf{\focal} FAQ}
\end{center}


%%%%%%%%%%%%%%%%%%%%%%%%%%%%%%%%%%%%%%%%%%%%%%%%%%%%%%%%%%%%%%%%%%%%%%%%%%%%%%%
\begin{faqitem}
\noindent {\bf $\bullet$ Q:}
I get the error message:\\
  \verb+Error: The reference basics.int__t was not found in the current+\\
  \verb+environment+\\
when \coq\ compiles.

%%%%%%%%%%%%%%%
\medskip
\noindent {\bf $\bullet$ A:}
You probably forgot to open the module {\tt basics} in you \focal\ program.
Add the directive \lstinline{open "basics" ;;} at the top of your source
file.
\end{faqitem}


%%%%%%%%%%%%%%%%%%%%%%%%%%%%%%%%%%%%%%%%%%%%%%%%%%%%%%%%%%%%%%%%%%%%%%%%%%%%%%%
\bigskip
\begin{faqitem}
\noindent {\bf $\bullet$ Q:}
I get the error message:\\
  \verb+Error: Types Self and ... are not compatible.+\\
when \focalizec\ compiles.

%%%%%%%%%%%%%%%
\medskip
\noindent {\bf $\bullet$ A:}
You probably have created a def-dedendency on the
representation in the statement of a property or a theorem, like in:
{\small
\begin{lstlisting}
species Bug =
  representation = int ;
  property wrong : all x : Self, x = x + 0 ;
end ;;
\end{lstlisting}}


This statement reveals that the representation is indeed {\tt int} since to
have \lstinline"x + 0" well-typed, {\tt Self} must exactly be {\tt int}. This
makes the interface of the species impossible to be typed as a collection
since the representation will be abstracted. You may need to add extra methods
hidding the dependency on representation.
\end{faqitem}



%%%%%%%%%%%%%%%%%%%%%%%%%%%%%%%%%%%%%%%%%%%%%%%%%%%%%%%%%%%%%%%%%%%%%%%%%%%%%%%
\bigskip
\begin{faqitem}
\noindent {\bf $\bullet$ Q:}
How to I make a function taking a tuple in argument ?

%%%%%%%%%%%%%%%
\medskip
\noindent {\bf $\bullet$ A:}
\lstinline"let f ( (x, y) ) = ... ;;"
\end{faqitem}



%%%%%%%%%%%%%%%%%%%%%%%%%%%%%%%%%%%%%%%%%%%%%%%%%%%%%%%%%%%%%%%%%%%%%%%%%%%%%%%
\bigskip
\begin{faqitem}
\noindent {\bf $\bullet$ Q:}
How to I make a function taking {\tt unit} in argument ?

%%%%%%%%%%%%%%%
\medskip
\noindent {\bf $\bullet$ A:}
\lstinline"let f (_x : unit) = ... ;;"
\end{faqitem}



%%%%%%%%%%%%%%%%%%%%%%%%%%%%%%%%%%%%%%%%%%%%%%%%%%%%%%%%%%%%%%%%%%%%%%%%%%%%%%%
\bigskip
\begin{faqitem}
\noindent {\bf $\bullet$ Q:}
What is the difference between a constructor having several
arguments and a constructor having one argument being a tuple ?

%%%%%%%%%%%%%%%
\medskip
\noindent {\bf $\bullet$ A:}
A constructor with one argument being a tuple is defined using the ``tupling''
type constructor \lstinline{*}:
{\small
\begin{lstlisting}
type with_1_tuple_arg = 
  | A (int * bool * string) ;;   (* Note the stars. *)
\end{lstlisting}}

A valid usage of this construtor is:
{\small
\begin{lstlisting}
let ok = A ( (1, false, "") ) ;;
\end{lstlisting}}

\noindent where it is important to note the double parentheses. This
constructor has 1 argument that is a tuple. The syntax for constructors with
arguments already requires parentheses, that's the reason for these double
parentheses.

Trying to use this constructor as:

{\small
\begin{lstlisting}
let ko = A (1, false, "") ;;
\end{lstlisting}}

\noindent would lead to an error telling that types {\tt int * bool * string}
and {\tt int} are not compatible. In effect in this case, {\tt A} is
considered to be applied to several arguments, the first one being {\tt 1}
that is of type {\tt int}. And {\tt int} is really incompatible with a tuple
type.

A similar constructor with several separate arguments is defined using
the ``comma'' construct:

{\small
\begin{lstlisting}
type with_several_args =
  | B (int, bool, string) ;;   (* Note the comas. *)
\end{lstlisting}}

A valid usage of this construtor is:
{\small
\begin{lstlisting}
let ok = B (1, false, "") ;;
\end{lstlisting}}

\noindent where it is important to note the unique pair of parentheses. This
constructor has 3 arguments.

Trying to use this constructor as:

{\small
\begin{lstlisting}
let ko = B ( (1, false, "") ) ;;
\end{lstlisting}}

\noindent would lead to an error telling that types {\tt int} and
{\tt int * bool * string} are not compatible. In effect, we try to pass to
{\tt B} one unique argument that is a tuple. And a tuple is incompatible with
the first expected argument of {\tt B}, that is {\tt int}.
\end{faqitem}



%%%%%%%%%%%%%%%%%%%%%%%%%%%%%%%%%%%%%%%%%%%%%%%%%%%%%%%%%%%%%%%%%%%%%%%%%%%%%%%
\bigskip
\begin{faqitem}
\noindent {\bf $\bullet$ Q:}
I get a syntax error on a sum type definition or a pattern-matching.

%%%%%%%%%%%%%%%
\medskip
\noindent {\bf $\bullet$ A:}
Beware that conversely to \ocaml, the first bar is not optional in \focal.

\begin{minipage}{6.2cm}
Wrong
{\small
\begin{lstlisting}
type t =
    Z
  | S (t) ;;

match x with
    Z -> ...
  | S (y) ...
\end{lstlisting}}
\end{minipage}\hskip1cm
\begin{minipage}{6.2cm}
Correct
{\small
\begin{lstlisting}
type t =
  | Z
  | S (t) ;;

match x with
  | Z -> ...
  | S (y) ...
\end{lstlisting}}
\end{minipage}
\end{faqitem}



%%%%%%%%%%%%%%%%%%%%%%%%%%%%%%%%%%%%%%%%%%%%%%%%%%%%%%%%%%%%%%%%%%%%%%%%%%%%%%%
\bigskip
\begin{faqitem}
\noindent {\bf $\bullet$ Q:}
I get the error message:\\
  \verb+Zenon error: uncaught exception File "coqterm.ml", line +\\
  \verb+325, characters 6-12: Assertion failed+

%%%%%%%%%%%%%%%
\medskip
\noindent {\bf $\bullet$ A:}
This is a current issue in \zenon. Compile your program
adding the option \verb+-zvtovopt -script+ to \focalizec. This asks
\zenon\ to output proofs as a \coq\ script instead of a \coq\ term. This should
be fixed in the future.
\end{faqitem}



%%%%%%%%%%%%%%%%%%%%%%%%%%%%%%%%%%%%%%%%%%%%%%%%%%%%%%%%%%%%%%%%%%%%%%%%%%%%%%%
\bigskip
\begin{faqitem}
\noindent {\bf $\bullet$ Q:}
I get the error message:\\
  \verb+Error: Types coq_builtins#prop and basics#bool are not compatible.+

%%%%%%%%%%%%%%%
\medskip
\noindent {\bf $\bullet$ A:}
You confused the ``not'' operators \lstinline"~" and \lstinline"~~" or
probably used a \lstinline"logical let" definition in a \lstinline"let"
definition like in the following example.

{\small
\begin{lstlisting}
species Bug =
  logical let not0 (x) = ~~ (x = 0) ;
  let g (y) = if not0 (y) then ... else ... ;
end ;;
\end{lstlisting}}

\lstinline"logical let" are definitions whose result type is {\tt prop}, i.e.
the type of logical {\bf statement}. They are intended to be used in theorems
or properties and are discarded in \ocaml\ code since this latter doesn't
deal with logical/proof aspects. The type of logical {\bf expressions} is
{\tt bool} and is automatically transformed into {\tt prop} in the context
of logical statements of theorems or properties. However, in the context of
computational definitions, {\tt prop} is always rejected. In the above example,
you may have defined the {\em computational} function {\tt not0} by:

{\small
\begin{lstlisting}
species Bug =
  let not0 (x) = ~ (x = 0) ;
  let g (y) = if not0 (y) then ... else ... ;
end ;;
\end{lstlisting}}

\noindent where \lstinline"~" is the ``not'' on booleans, whereas
\lstinline"~~" is the ``not'' on logical propositions.
\end{faqitem}
\end{document}


(*
* Attention, ordre des assumes important.
* Attention, ne pas couper le ET en 2 hypothèses séparées directement.

  theorem set_makes_on2:
    all state1 state2 : Self, all y : bool,
    (y = true /\ state2 = my_snd (one_step (state1, y))) -> is_on (state2)
  proof =
   <1>1 assume state1 state2 : Self,
        assume y : bool,
        hypothesis h1: (y = true /\ state2 = my_snd (one_step (state1, y))),
        prove is_on (state2)
        <2>1 prove state2 = my_snd (one_step (state1, y))
             by hypothesis h1
        <2>2 prove y = true
             by hypothesis h1
        <2>e qed by definition of one_step, is_on, my_snd
                 property AA!set_makes_on
                 step <2>1, <2>2 hypothesis h1
   <1>e conclude ;

*)










* Forme des buts pour l'induction de Zenon.


* Tuples et patterns imbriqués pas encore gérés par Zenon --> les éclater.
