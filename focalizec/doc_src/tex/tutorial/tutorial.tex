%----------------------------------------------------------------------------------------------
% FoCaLize Tutorials : Sets - LaTeX, standard article style
%----------------------------------------------------------------------------------------------
\documentclass[10pt,a4paper,twoside,onecolumn,fleqn]{article}
\pagenumbering{arabic}
\pagestyle{headings}

%----------------------------------------------------------------------------------------------
% Packages selection
%----------------------------------------------------------------------------------------------

\usepackage{amsmath}
\usepackage{amsfonts}
\usepackage{xspace}
\usepackage{listings}
\lstset{basicstyle=\footnotesize\mdseries\itshape\ttfamily,
        frame=single} %numbers=left, numberstyle=\tiny,
\usepackage[all]{xy}

%----------------------------------------------------------------------------------------------
% Document header
%----------------------------------------------------------------------------------------------

\author{The {F}o{C}a{L}ize Team}
\title{A Short Tutorial for {F}o{C}a{L}ize:\\Implementing Sets}
\date{July 2009}

%----------------------------------------------------------------------------------------------
% Contexts, languages, presentation
%----------------------------------------------------------------------------------------------

\newtheorem{tip}{Tip}

\begin{document}

\maketitle

%----------------------------------------------------------------------------------------------
% Macros
%----------------------------------------------------------------------------------------------
\def\month{August}
\def\year{2009}
\def{\focalversion}{{\sf 0.1.5}}

\def{\focalversiondate}{{\sf {\month} {\year}}}

\newenvironment{citemize}{\begin{list}{$\bullet$}{\itemsep 0pt \topsep 5pt}}{\end{list}}

\def\bbbn{{\rm I\!N}}

\def\url#1{{#1}}
\def\email#1{{\tt#1}}
\def\ocaml{{\sf OCaml}}
\def\moca{{\sf Moca}}
\def\focal{{\sf FoCaLiZe}}
\def\coq{{\sf Coq}}
\def\cime{{\sf CiME}}
\def\zenon{{\sf Zenon}}
\def\zvtov{{\sf zvtov}}
\def\focdoc{{\sf FoCaLiZeDoc}}
\def\dotty{{\sf dotty}}
\def\graphviz{{\sf graphviz}}
\def\emacs{{\sf emacs}}
\def\cygwin{{\sf Cygwin}}
\def\xsltproc{{\sf xsltproc}}
\def\terminal#1{\mbox{\tt\color{red}#1}}
\def\foc{{\sf Foc}}
\def\oldfocal{{\sf Focal}}
\def\focalizec{{\sf focalizec}}

\def\backslash{{\char92}}
\def\chapeau{{\char94}}
\def\vertical{{\char124}}
\def\tilde{{\char126}}

%% Pour afficher les listings de mani�re plus sexy... ;)
\lstdefinelanguage{FoCaLiZe}
  {morekeywords={alias, all, and, as, assume, assumed, begin, by, caml,
      collection, coq, coq_require, definition, else, end,
      ex, external, false, function, hypothesis, if, in,
      inherit, internal, implement, is, let, lexicographic,
      local, logical, match, measure, not, notation, of, open,
      on, or, order, proof, prop, property, prove, qed, rec,
      representation, Self, signature, species, step,
      structural, termination, then, theorem, true, type, use,
      with, conclude},
    otherkeywords={->, :, /\\, \\/, =>, \~, \#, ;, \&, \|},
    sensitive=false,
    morecomment=[n]{(*}{*)},  %% Autoriser les nested comments
    morecomment=[n]{(**}{*)}, %% Autoriser les nested comments
    morestring=[b]",
  }

\lstset{
  language=FoCaLiZe, tabsize=2, frame=single, breaklines=true,
  basicstyle=\ttfamily, framexleftmargin=1mm, xleftmargin=1mm
}

% For syntax definitions
% \begin{syntax}
% left-hand side & ::=  & right-hand side & comment \\
%                & \alt & more right-hand side & more comment
% \end{syntax}
% Use \syntaxclass{Foo} to insert a title line above a syntax definition.
% Do not put \\ before \end{syntax} or \syntaxclass{...}
% Use \begin{syntaxleft} ... \end{syntaxleft} to insert the title
% lines to the left of the definitions.

% \def\syntax{
% \par\medskip\goodbreak\noindent
% \bgroup
% \let\\=\cr
% \interlinepenalty=50            % discourage page breaks in a definition
% \global\let\syntaxclass=\firstsyntaxclass
% \if@twocolumn
% \halign\bgroup~~$##$&\hfil${}##{}$&$##$~\hfil&##\hfil\cr
% \else
% \halign\bgroup\qquad\qquad$##$&\hfil${}##{}$&$##$\quad\hfil&##\hfil\cr
% \fi
% }
% \def\endsyntax{\cr\egroup\egroup\par\medskip\noindent\ignorespaces}

% \def\firstsyntaxclass#1{
% \omit\hbox to 0pt{#1\hss}\cr
% \global\let\syntaxclass=\nextsyntaxclass
% }

% \def\nextsyntaxclass#1{
% \cr\noalign{\smallskip\penalty-100}\omit\hbox to 0pt{#1\hss}\cr
% }

% \def\syntaxleft{
% \par\medskip\goodbreak\noindent
% \bgroup
% \let\\=\cr
% \interlinepenalty=50            % discourage page breaks in a definition
% \global\let\syntaxclass=\firstsyntaxclassleft
% \if@twocolumn
% \halign\bgroup\hfil$##$&\hfil${}##{}$&$##$~\hfil&##\hfil\cr
% \else
% \halign\bgroup\hfil$##$&\hfil${}##{}$&$##$\quad\hfil&##\hfil\cr
% \fi
% }
% \let\endsyntaxleft=\endsyntax

% \def\firstsyntaxclassleft#1{
% $\hfilneg#1\quad\hfil$
% \global\let\syntaxclass=\nextsyntaxclassleft
% }

% \def\nextsyntaxclassleft#1{
% \cr\noalign{\smallskip\goodbreak}$\hfilneg#1\quad\hfil$
% }


%----------------------------------------------------------------------------------------------
% Table of Contents
%----------------------------------------------------------------------------------------------

\tableofcontents

%----------------------------------------------------------------------------------------------
 % Start of Document Body
 %----------------------------------------------------------------------------------------------

 \section{Forewords}
 \label{foreword}

 %----------------------------------------------------------------------------------------------
 \subsection{Content}
 \label{foreword_content}

 This document is a short tutorial for \foc, version {\focalizeversion} released in \month
 \year\footnote{\foc can be downloaded from $\fid{http://focalize.inria.fr}$.}.
 It describes a full but simple
 development using the main features of the \foc language, from the specification to the
 implementation, including proofs of correction.

 The reader is expected to have some basic knowledge of the \ocaml language, of object-oriented
 programming as well as an understanding of the standard logical operators ($\forall$,
 $\Rightarrow$, etc.) and notions about proofs.

 This tutorial is not intended to be complete with regard to the \foc features, nor does it
 recommend a specific design philosophy. In particular, It does not make use of the \foc
 library (providing numerous mathematical structures) but on the contrary builds from scratch
 such a structure, for the sake of illustration.

 %----------------------------------------------------------------------------------------------
 \subsection{Notations and Recommandations}
 \label{foreword_notation}

 In the rest of this tutorial, pieces of \foc code will be presented in frames, as in this
 example:
 \begin{lstlisting}
 use "basics" ;;
 species Superset =
 \end{lstlisting}
 This type of code has to be inserted in \unix ASCII files with a $\fid{fcl}$ extension. \foc
 keywords, types or identifiers can also be inserted directly in the text, for example
 \code{Superset}.

 The \foc syntax is case sensitive, uppercase and lowercase identifier being associated to
 different sorts of entities. Delimiters (blank spaces, tabulations, line feeds) are often
 required to separate syntactical constructs. Finally, the use in \foc sources of \ocaml or
 \coq keywords (such as \code{Set}) or of standard library file names (such as \code{list.v})
 may result in name clashes. For these reasons, the reader is advised to respect as much as
 possible the form and the names used in this tutorial.

 The \foc syntax includes several forms of comments. Single line comments start with a
 \code{--} and go to the end of the line. Block comments are delimited by \code{(*} and
 \code{*)}, and can be nested, that is, it is possible to comment a block of code which
 includes comments. However, for the sake of readability, we will not use comments in our
 examples; the reader is free to add comments were he feels appropriate. It is advised to avoid
 some forms of comments, such as \code{(**}, \code{(*--}, or \code{"} (double quote) in block
 comments; please refer to the \foc reference manual for detailed explanations.

 Commands, file names as well as outputs or error messages are in bold font, for example
 \textbf{focalizec superset.fcl}. Terms representing specific \foc concepts are introduced
 using an emphasised font, for example \emph{collection}. Finally, mathematical statements are
 presented using the standard notations, for example $\forall\:(x\!:\!T),\:x\!=\!x$ (for any
 $x$ belonging to the set/type $T$, $x$ is equal to $x$).

 %----------------------------------------------------------------------------------------------
 %----------------------------------------------------------------------------------------------
 \section{A Quick Overview of \foc}
 \label{overview}

 \foc is an integrated development environment (IDE) with formal features. Beyond the compiler,
 that from a \foc source file produces source code for \ocaml and proof scripts to be checked
 by \coq, \foc also provides an automated prover (\zenon), a test tool ({\sc{FoCTest}}), a
 documentation tool ({\sc{FoCDoc}}), and other utilities. We mainly focus in this tutorial on
 the use of the compiler (\textbf{focalizec}) and of the \zenon automated prover
 (\textbf{zvtov}). Note that \foc is the successor of the {\sc{FoC}} and {\sc{FoCaL}} tools,
 with a fully redeveloped compiler.

 \foc being a formal method, a typical development includes not only data structures and
 programs, but also logical properties. This allows for the description of specifications and
 of implementations, while providing a mathematical guarantee that implementations are indeed
 compliant with their specifications. A \emph{specification} consists of function signatures,
 but also of associated properties that have to be satisfied, as in the following example
 describing a commutative binary operation $f$ over a set $\mathbb{S}$ with a left neutral
 element:
 \begin{small}
 \[\begin{array}{lllll}
 f:\mathbb{S}\!\to\!\mathbb{S}\!\to\!\mathbb{S} & \quad &
 \forall\:x\:y,\:f(x,y)\!=\!f(y,x) & \quad &
 \forall\:x,\:f(Z,x)\!=\!x
 \end{array}\]
 \end{small}
 The left part is frequently encountered in classical, non-formal languages, under the names
 signature, interface, prototype or specification. The two properties on the right part, on the
 other hand, are less usual; in a formal method, they do not represent comments or assertions
 checked at runtime, but a true requirement constraining \emph{implementations} -- that is the
 programs with concrete datatypes and algorithms. The correctness of the implementations has to
 be statically ensured at compilation time by proving that they satisfy the properties, an
 activity relying on \emph{formal proofs} (in the mathematical sense), whose production is only
 partially automated but which are mechanically checked.

 The cost of develop ping formal specifications and proofs is compensated by the important gain
 w.r.t. the confidence in the development. Various studies furthermore indicate that the total
 development and ownership cost is likely to be reduced, due to disambiguation of the
 specifications, earlier detection of problems, lighter test obligations and reduced
 maintenance activities. Note also that it is possible to further assess the specification
 itself by deriving consequences, to check that the specification is both correct and
 sufficiently detailed; taking our previous example, it is possible to show that a valid
 implementation of $f$ is such that $f(x,Z)\!=\!x$ -- just combining required properties,
 without having to know what is this implementation.

 The \foc language therefore combines a pure functional programming language inspired by \ocaml
 and a logical language. It is also object-flavoured, developments being organised through a
 hierarchy of \emph{species}, build by \emph{inheritance} or \emph{parameterisation}. In
 essence, a \emph{species} is a kind of record combining function \emph{signatures} (that is
 their type), function \emph{definitions} (that is their code), \emph{logical properties} and
 \emph{proofs}.

 When a species $S_2$ \emph{inherits} from a species $S_1$, that means that $S_2$ includes at
 least all the features of $S_1$, and can be enriched with its own functions or properties --
 intuitively speaking, any correct implementation of $S_2$ is therefore also a correct
 implementation of $S_1$.

 A species can also use the implementation of other species to build its own definitions,
 properties and proofs, through a \emph{parameterisation mechanism}. The parameters are
 required to be effective and consistent, that is to be such that every function declaration
 has an associated definition, and every property a proof that it holds. Furthermore, the
 validity of the properties of these effective parameters is likely to depend upon the
 implementation choices done during their development, and it is therefore very important to
 preserve them. This is addressed in \foc by the notion of \emph{collection}: the
 implementation of a complete species (that is a species whose all functions are defined and
 all properties are proved) whose concrete data representation is hidden -- a form of abstract
 data type. A collection only exposes the declaration of its functions and the statement of its
 properties, through a structure called the \emph{interface} of the collection; collections can
 therefore be used safely as effective parameters.

 Practically, if a species $S$ has a formal parameter $C$ of interface $I$, then $S$ can use
 all the functions and properties named in $I$ in its body to build the definition of its
 functions and to prove its properties, but cannot modify these functions or even examine their
 definition. The compiler ensures that any collection $D$ given as an effective parameter for
 $C$ indeed offers all the functions and properties stated in $I$. Note that species can also
 be parameterised by entities, that is values of a collection. This is a powerful feature, e.g.
 to describe structures such as $\REL/n\REL$ where $n$ is such a parameter, but we will not use
% Note Didou: Euhhhh justement, ce point est un peu limite en FoCaL car comme Phil l'a tr�s
% justement fait remarquer... �a ne nous emp�che pas d'additionner des entiers modulo 42 et des
% entiers modulo 314159... Le probl�me vient du fait que l'on ne sait pas contraindre
% l'instanciation coh�rente de param�tres d'instance entre 2 instanciations de param�tres de 2
% esp�ces s�par�es.
it in this tutorial.

%----------------------------------------------------------------------------------------------
%----------------------------------------------------------------------------------------------
\section{The Development Case: Sets and Subsets}
\label{case}

We deal in this tutorial with the classical example of subsets. We want to describe the
implementation of subsets of values, ensuring that it is bug-free, at least w.r.t. a
specification that we will make as complete as possible.

It is important, in such a development, to learn to abstract the fundamental concepts. Indeed,
one can easily describe such mathematical structures having a precise idea about their
implementation in mind, and consequently describing this implementation rather that the
structure itself. For example, in the case of subsets, a classical choice is to use sorted
and injective\footnote{That is, lists in which a given value appears at most once.} lists;
whereas this is a valid implementation, there are other choices, and it would be a mistake to
capture too early some of the specificities of this encoding, such as for example the
existence of a comparison between values or of an order inside the concrete representation of
a subset.

As we will see, we adopt here a progressive approach, justified by various engineering
considerations regarding \foc developments, but also by the will to stick to the formal vision
presented just before. An interesting consequence of this vision is that a genuine freedom is
given to the developer, that can choose between various approaches for example to favour
genericity and reusability, or to optimize algorithms.

The rest of this paper describes the various stages of the development, distinguishing between
specification activities (the ``what to do'') and refinements, that is the activities ensuring
progress toward an implementation (the ``how to do''). We describe supersets, from which
values are obtained, and subsets of these supersets, before implementing these specifications.
Our main objective is to implement finite subsets as lists of values, but we will consider 
possible alternatives as well.

%----------------------------------------------------------------------------------------------
%----------------------------------------------------------------------------------------------
\section{Specifying Supersets}
\label{specsuperset}

When attempting to specify what is a subset of a superset -- for example subsets of $\NAT$ --
we need first to specify this superset. The simplest approach would be to decide to represent
subsets of a given superset, for example subsets of values from \code{int}. Yet of course that
would force us to fully redevelop such an implementation for any relevant type, a rather
inappropriate strategy.

We therefore favour the generic approach, through a form of polymorphic specification. In
\foc, this is possible by using parametrisation. Species can indeed be parametrised by an
\emph{interface}, which means that implementation of the species will have to instantiate
this parameter by a collection having the required interface. In practice, this means that
before defining a species for subset, parametrized by an interface describing what is a
superset, we need first to define the superset interface. This is done by creating first a
species for superset, whose only role is to be passed as a parameter.

%----------------------------------------------------------------------------------------------
\subsection{The Superset Species}
\label{specsuperset_species}

What do we expect from a superset? Not much, as we can consider subsets of nearly anything.
Furthermore we do not need to enforce supersets to have a rich structure with powerful
operations. For now, only one very generic operator appears useful, an equivalence relation --
representing a form of equality, for example to be able to test whether or not a subset
contains a value. Note that we do not require this equivalence to be a strict equality, for
example if we store strings that encodes natural values, we may decide that ``$1$'', ``$01$''
and ``$+001$'' are equivalent -- and checking if a given subset contains for example
``$+1$'' will return true even if the stored value is in fact ``$01$''.

We therefore edit a \foc source file, named \textbf{superset.fcl}, and type in the following
code:
\begin{lstlisting}
use "basics" ;;

species Superset =

 signature equal : Self -> Self -> basics#bool ;

end ;;
\end{lstlisting}
The first line of our code indicates that we
% Didou
intend to
use declarations and definitions from the file
\textbf{basics.fcl} (or more precisely its compiled version, \textbf{basics.fo}, which has to
be in the library path of the compiler). This standard file contains the definition of the
type \code{bool} which is needed here, as well as other utilities, and is generally required
for any \foc development.

\begin{tip}\small[Compilation Units] Access to declarations and definitions from other files
is possible with the \code{use} or \code{open} directives; the latter allows for shorter
notations (e.g. \code{#bool} or \code{bool} instead of \code{basics#bool}
% Didou
which is referred to ``file qualified notation''. We will see later a similar mechanism to
invoke species methods that we call ``species qualified notation''. In absence of
ambiguity between these two qualification mechanisms, we will generally use the shorter
term of ``qualified notation''
). Note that the
\code{use} and \code{open} directives are not transitive, that is for example if $B$ opens $A$
and $C$ opens $B$, $C$ does not have access to declaration and definitions of $A$.
\end{tip}

We then define a species \code{Superset} to represent the very abstract entity that we call a
superset. This species represents in fact any collection (any implementation) whose interface
contains a function \code{equal} with the required type; as we do not provide the definition
(the code) of this function, but just its type, we use the keyword \code{signature}. The type
of the function is $\fid{Self}\to\fid{Self}\to\fid{bool}$, which means that it takes two
parameters of the current species\footnote{More precisely, \code{equal} expects two values of
the concrete type used as the support for the collection implementing the species.} and
returns a boolean value.

\begin{tip}\small[Syntax] Species names have to start with an uppercase character.
\end{tip}

\begin{tip}\small[Syntax] Separators (that is blank spaces, tabulations, line feeds) are often
required between syntactical constructs.
\end{tip}

At this stage, we are not using any formal feature; indeed, we are just specifying here a form
of object with a given method. To have a meaningful specification, we further require the
equal function to represent an equivalence relation (that is to be reflexive, symmetric and
transitive). In addition, we also change the name of the method \code{equal}: to be easier
to read, we benefit from the \foc ability to manage symbols, and use the much more explicit
symbol \code{=} instead. We therefore modify \code{Superset} as follows:
\begin{lstlisting}
species Superset =

 signature ( = ) : Self -> Self -> basics#bool ;
 property eq_refl : all x : Self, x = x ;
 property eq_symm : all x y : Self, x = y -> y = x ;
 property eq_tran : all x y z : Self, x = y -> y = z -> x = z ;

end ;;
\end{lstlisting}
The symbol \code{=}, in \foc, is associated to infix notations (but can still be used as a
prefix operator, when put between parentheses, as in its signature). This allows for a very
user-friendly presentation of the three properties that we expect about an equivalence
relation, namely reflexivity, symmetry and transitivity. We use the keyword \code{property},
% Didou
% that is while we do not prove anything at this stage, we require any collection implementing
% this species to provide such a proof at some point of the development -- for example as soon
hence we do not prove anything at this stage. We just require any collection implementing
this species to provide such a proof at some point of the development -- for example as soon
as \code{=} is defined.

\begin{tip}\small[Using Symbols] Methods can be denoted by names or symbols; the choice of an
appropriate symbol may greatly improve the readability of a code.
\end{tip}

\code{Superset} now specifies any collection offering a function named \code{=} taking two
parameters in the collection and returning a boolean, that is a relation; but it also requires
this relation to be reflexive, transitive and symmetric. This is a pure specification: there
are no definitions, no code.  We explain what we expect, but the developer is free to
propose any compliant implementation. Note by the way that our specification does not even
enforce \code{Superset} to contain a value: it can be implemented by an empty type.

At this stage, it is possible to compile our development with the command
\textbf{focalizec superset.fcl} -- of course, there is not much to expect from this
compilation, except for checking the syntax. It invokes the \foc compiler, as well as \ocaml
compiler, the \zenon prover -- but there are no proof obligations at this stage, the species
only containing properties -- and the \coq proof checker, producing the following files:

\begin{tabular}{|l|c|c|l|}\hline
File name & Produced by & Used by & Description\\\hline
\textbf{superset.fcl} & Text editor & \textbf{focalizec} & \foc source file\\
\hline
\textbf{superset.fo} & \textbf{focalizec} & \textbf{focalizec} & \foc object file\\
\textbf{superset.zv} & \textbf{focalizec} & \textbf{zvtov} & Proof obligations\\
\textbf{superset.ml} & \textbf{focalizec} & \textbf{ocamlc} & Program source file\\
\hline
\textbf{superset.pfc} & \textbf{zvtov} & \textbf{zvtov} & \zenon proof cache\\
\textbf{superset.v} & \textbf{zvtov} & \textbf{coqc} & Proofs\\
\textbf{superset.vo} & \textbf{coqc} & \textbf{coqc} & \coq object file\\
\textbf{superset.cmi} & \textbf{ocamlc} & \textbf{ocamlc} & \ocaml interface file\\
\textbf{superset.cmo} & \textbf{ocamlc} & \textbf{ocamlc} & \ocaml object code\\
\hline
\end{tabular}

%----------------------------------------------------------------------------------------------
\subsection{A First Proof}
\label{specsuperset_proof}

Whereas \code{Superset} is a very short and simple specification, it can be enriched with
additional results without further requirements. For example, combining reflexivity and
transitivity, it is possible to prove the following property about \code{=} of the
\code{Superset} species as follows:
\begin{lstlisting}
 theorem eq_symmtran : all x y z : Self, x = y -> x = z -> y = z
  proof = by property eq_symm, eq_tran ;
\end{lstlisting}
We use the keyword \code{theorem} instead of \code{property} to indicate that the proof
follows -- it is provided after the keyword \code{proof}.
% Didou
Hence, a \code{property} accompanied by its \code{proof} is equivalent to a \code{theorem}.
In fact, we do not detail a real
proof, but rather tips that allow \zenon to automatically derive the proof\footnote{The
success of \zenon for complex proofs depends upon the computer architecture and the parameters
used for its invocation.}; we claim that \code{eq_symmtran} is a direct consequence of
\code{eq_symm} and \code{eq_tran}. At this stage, we can invoke the compiler to check that
indeed \zenon succeeds.

\begin{tip}\small[Derived properties] Once primitive properties have been introduced, other
results can be proved, for example user-relevant corollaries, to check the validity and the
completeness of the specification.
\end{tip}

Note that \code{eq_symm} and \code{eq_tran} are in fact used as hypothesis to prove
\code{eq_symmtran}, in other words we have proven
$\fid{eq\_symm}\!\Rightarrow\!\fid{eq\_tran}\!\Rightarrow\!\fid{eq\_symmtran}$. That means
that the validity of the latter depends upon the
% Didou: validity
consistence
of the formers, and that there is
therefore no real assurance until proofs are provided for them.

In the rest of this tutorial, we will not further mention standard compilation steps; the
reader is invited to compile when he considers it is relevant.

%----------------------------------------------------------------------------------------------
\subsection{A First Definition}
\label{specsuperset_definition}

To further illustrate the \emph{late binding} feature of \foc, that is in essence the ability
to describe and use entities which are not yet defined, we again enrich the species
\code{Superset}, this time with the \emph{definition} of a function:
\begin{lstlisting}
 let ( <> ) (x, y) = basics#( ~~ )(x = y) ;

 theorem diff_irrefl : all x : Self, ~(x <> x)
  proof = by definition of ( <> ) property eq_refl ;

 theorem diff_symm : all x y : Self, x <> y -> y <> x
  proof = by definition of ( <> ) property eq_symm ;
\end{lstlisting}
In this case, we provide the code for the function \code{<>}, introduced by the keyword
\code{let}. Without surprise, we indicate that \code{<>} is the negation of \code{=}, but one
should note that \code{=} itself is not yet defined. However, this is sufficient to prove
properties of \code{<>}, such as \code{diff_irrefl} and \code{diff_symm}. The proof of these
properties is again automatically derived by \zenon, the tips in this case also ensuring that
the definition of \code{<>} is visible from the proof context.

A definition of \code{<>} is provided in the species \code{Superset}, but it is possible to
override it later (in inheriting species or collection), for example to optimize the
implementation, once more information about the structure are provided.

It is fundamental to remember that \foc enforces encapsulation; this means that any definition
in a species, such as the one for \code{<>} here, is visible only in this species (or the
inheriting ones). As definitions are never visible outside a species, we recommend for such
early definition to associate dedicated properties capturing the essence of the definition in
a logical form, visible from other species, such as \code{diff_eq} that we add to our species
as follows:
\begin{lstlisting}
 theorem diff_eq : all x y : Self, x <> y <-> ~(x = y)
  proof = by definition of ( <> ) ;
\end{lstlisting}
\code{diff_eq} is here described as a theorem whose proof depends upon the definition of
\code{<>}. Any later redefinition of \code{<>} will delete this proof, and transform
% Didou
back
\code{diff_eq} into a proof obligation. The interest of a property like \code{diff_eq} is
that it can be used instead of the definition of \code{<>} to prove \code{diff_irrefl} and
\code{diff_symm}: if \code{<>} is indeed redefined later, only the proof of \code{diff_eq}
will have to be redone.

\begin{tip}\small[Early Definitions] Definitions can be introduced at any stage of a \foc
development; it is recommended to capture the meaning of early definitions in properties
(proved using the definition), and to use these properties in proofs, instead of the
definitions themselves, to limit the consequences of redefinitions.
\end{tip}

%----------------------------------------------------------------------------------------------
\subsection{Clearing of the Superset Species}
\label{specsuperset_clear}

We have enriched our \code{Superset} species just to illustrate some of the features of \foc.
As these artificial examples are not used later in this tutorial, we suggest to delete them.
The file \textbf{superset.fcl} should be modified as follows:
\begin{lstlisting}
use "basics" ;;

species Superset =

 signature ( = ) : Self -> Self -> basics#bool ;
 property eq_refl : all x : Self, x = x ;
 property eq_symm : all x y : Self, x = y -> y = x ;
 property eq_tran : all x y z : Self, x = y -> y = z -> x = z ;

end ;;
\end{lstlisting}

%----------------------------------------------------------------------------------------------
%----------------------------------------------------------------------------------------------
\section{Specifying Subsets}
\label{specsubset}

It is now time to specify subsets of values from a superset in a new species. It is possible
to append the new code in the previous file, but we prefer to create a new file named
\textbf{subset.fcl}.
% Didou
As previously stated, the directives \code{use} or \code{open} can be used to access previous
definitions. Remember that with \code{use} explicit and detailed names are required, for
example \code{superset#Superset!eq_refl}, whereas \code{Superset!eq_refl} is sufficient if we
use \code{open} instead.

\begin{lstlisting}
use "basics" ;;
open "superset" ;;
\end{lstlisting}

%----------------------------------------------------------------------------------------------
\subsection{Generic Subsets}
\label{specsubset_generic}

The species \code{Subset} is parametrised by a collection \code{Val} which implements the
interface defined by the species \code{Superset}. Regarding the method, we provide a
membership operation deciding whether or not a value belongs to a subset, denoted \code{<<};
practically, membership defines a subset, as any subset is characterised by the values it
contains\footnote{Note that we describe in fact subsets for which membership is decidable.}:
\begin{lstlisting}
species Subset(Val is Superset) =

 signature ( << ) : Val -> Self -> basics#bool ;

end;;
\end{lstlisting}
Of course using only this method we cannot do much: neither do we have a way to exhibit a
subset, nor to express any property about membership. So additional methods have to be
declared, in association with properties that sufficiently describe what we expect from these
methods. However, we also have to be cautious not to add too powerful methods, that would lead
to over-specify the species and reduce the acceptable implementations (or logical models).

It appears reasonable to add a method to build an empty subset. Indeed, it always exists such
a subset, even if for example \code{Val} is itself empty. Another seemingly harmless feature
(in the sense that it is likely to be implementable whatever the concrete representation we
choose) is the ability to derive subsets from other subsets by adding or removing a value.
Following these principles, our species is enriched as follows:
\begin{lstlisting}
 signature empty : Self ;
 property mem_empty : all v : Val, ~(v << empty) ;

 signature ( + ) : Self -> Val -> Self ;
 property mem_insert : all v1 v2 : Val, all s : Self,
                       v1 << s + v2 <->
                       (Val!( = )(v1, v2) \/ v1 << s) ;

 signature ( - ) : Self -> Val -> Self ;
 property mem_remove : all v1 v2 : Val, all s : Self,
                       v1 << s - v2 <->
                       (~(Val!( = )(v1, v2)) /\ v1 << s) ;
\end{lstlisting}
In this specification, each method producing a subset is associated to a property detailing
its behaviour with regard to membership. As membership indeed characterises a specific subset,
such properties are as precise as a definition. This is typical of \foc specifications, where
the concrete datatype stays hidden from outside the species: properties are generally
relations between methods, describing the structure of a species.

We have adopted a very dense presentation of these properties: \code{mem_insert} claims that a
value is a member of \code{s + v} if and only if this is \code{v} or if it was already a
member of \code{s}. It could have been split in three different properties, but we here
trust \zenon to exploit automatically all variants of this properties, and we favor this form
to reduce the number of tips to be provided for proofs.

Note that the notation \code{=}, in the context of the species \code{Subset}, would refer
either to a method defined in the current species or to a method defined at \emph{top level}
(that is outside a species or a collection), possibly in one of the included files. As we want
to use the equality of the parameter \code{Val}, we need the explicit notation
\code{Val!( = )}, which is a prefix notation.

Are additional methods desirable at this stage of the specification? We choose here not to go
further, yet this is a matter of taste and style. For example, it may be acceptable to specify
union and intersection, as it does not seem to restrict the possible concrete representations
for subsets. On the other hand, from a practical point of view, that means that any
implementation of the species \code{Subset}, in addition to providing the definition for
\code{<<}, \code{+}, \code{-} as well as the proofs of the associated properties, would also
have to provide a definition and the proofs for union and intersection, even if these last
operations are not useful for a given application. So it may be more appropriate to add such
operations in a later species and let developers to choose to use the \code{Subset}
specification or one of its extended versions.

%----------------------------------------------------------------------------------------------
\subsection{Extensional Subsets}
\label{specsubset_extensional}

Whereas our main objective is to implement finite subsets as lists, we choose here to
introduce very gradually various levels of specifications. Note that nothing enforces at this
stage the species \code{Subset} to only describe finite subsets, and we do not want to further
constrain it by requiring other methods and properties in \code{Subset}. We prefer to adopt a
smoother approach, adding another level of specification by creating a new species to
represent finite subsets, or more precisely extensional subsets.

Indeed, it is not trivial to specify finite subsets. One of the most classical approaches is
to define a cardinal operator in $Self\!\to\!\NAT$, returning the number of elements in the
subset. The simple existence of this operator, total over $Self$, combined with
properties indicating its semantics, is indeed sufficient. Yet that would require using
natural values, that is to have a collection parameter with a species representing $\NAT$.
Appropriate species exist in the \foc standard library, but we prefer to avoid in this
tutorial such dependencies for now.

We therefore adopt here a slightly different approach, by specifying extensional subsets, that
is subset whose content can be analysed systematically.

\subsubsection{Inclusion}
\label{specsubset_extensional_inclusion}

Extensional subsets being subsets, we use the inheritance mechanism: we create a new species
\code{ExtSubset} parametrised by a collection \code{Val}, which inherits of
\code{Subset(Val)}, that is of all its methods and properties. Of course, we intend to enrich
this specification with new methods and properties.

The new requirement in \code{ExtSubset} is to have a method checking whether or not a subset
is included in another, denoted \code{<:} in our code.
% Didou Benefitting
Taking benefit
from the specification of
the (yet undefined) inclusion method, it is already possible to prove that inclusion is
reflexive and transitive:
\begin{lstlisting}
species ExtSubset(Val is Superset) =

 inherit Subset(Val) ;

 signature ( <: ) : Self -> Self -> basics#bool ;
 property mem_incl : all s1 s2 : Self,
                     s1 <: s2 <-> all v : Val, v << s1 -> v << s2 ;
 theorem incl_refl : all s : Self, s <: s
  proof = by property mem_incl ;
 theorem incl_tran : all s1 s2 s3 : Self,
                     s1 <: s2 -> s2 <: s3 -> s1 <: s3
  proof = by property mem_incl ;

end;;
\end{lstlisting}
Any collection implementing \code{ExtSubset} will therefore have to provide code for the
method \code{<:} as well as a proof that this code indeed represents inclusion; provided this
proof reflexivity and transitivity will be ensured as well without further work.

\begin{tip}[Inheritance] The \code{inherit} clause has to be the first one in a species; it
indicates that the current species includes all the signatures, definitions, properties and
proofs on the inherited species.
\end{tip}

\subsubsection{Extensional Equality}
\label{specsubset_extensional_equal}

An evident enrichment is to define the equality of two finite subsets as the reciprocal
inclusion. This should however ring a bell: if the species \code{ExtSubset} offers an
equality, then it can probably also inherits from the species \code{Superset}. That's our
design choice here, using the multiple inheritance feature provided by \foc. We therefore
modify and extends the species \code{ExtSubset} as follows:
\begin{lstlisting}
species ExtSubset(Val is Superset) =

 inherit Superset, Subset(Val) ;

 signature ( <: ) : Self -> Self -> basics#bool ;
 property mem_incl : all s1 s2 in Self,
                     s1 <: s2 <-> all v in Val, v << s1 -> v << s2 ;
 theorem incl_refl : all s in Self, s <: s
  proof = by property mem_incl ;
 theorem incl_tran : all s1 s2 s3 in Self,
                     s1 <: s2 -> s2 <: s3 -> s1 <: s3
  proof = by property mem_incl ;

 let ( = ) (s1, s2) = if (s1 <: s2) then (s2 <: s1) else false;

end;;
\end{lstlisting}
\code{ExtSubset} is now a species combining the interface of \code{Superset} and
\code{Subset}: equality (of subsets), membership, insertion, removal, as well as the
associated properties\footnote{As \code{ExtSubset} inherits from \code{Superset}, it can
parametrise \code{Subset}, that is it is possible to consider subsets of a superset, but also
subsets of subsets of a superset, and so on.}. Equality is also defined using
inclusion\footnote{\code{=} is defined with an \code{if}, which is primitive in \foc; it is
also possible to use the boolean and, denoted \code{&&} and provided in \textbf{basics.fcl}.}.
Using these properties and the definition of \code{=}, we can discharge the proof obligations
inherited from \code{Superset}:
\begin{lstlisting}
 proof of eq_refl = by definition of ( = ) property incl_refl ;
 proof of eq_symm = by definition of ( = ) ;
 proof of eq_tran = by definition of ( = ) property incl_tran ;
\end{lstlisting}
Defining \code{=} using \code{<:} is apparently relevant, and allows us to discharge the
associated proof obligations. This is not constraining, as we know that thanks to
late binding it is possible to later change the definition of \code{=}, for example for
optimisation. But of course such a redefinition would erase the proofs depending upon the
current definition, here \code{eq_refl}, \code{eq_symm} and \code{eq_tran}.

Now, we can already consider that such a redefinition is more than likely to happen. For
example, if we implement subsets as lists, checking equality will require checking twice
inclusion between lists, while comparing directly the two lists is more efficient. So it is
indeed pleasant to be able to discharge so soon the proof obligations, but it is probably
useless -- at least in the current form of the species. Therefore, exactly as we have done
when dealing with the definition of \code{<>} of the species \code{Superset} in Sub.
\ref{specsuperset_definition}, instead of defining \code{=} using \code{<:}, we capture the
meaning of the definition in a property, which is used to discharge the proof obligations:
\begin{lstlisting}
 property eq_incl : all s1 s2 : Self,
                    s1 = s2 <-> s1 <: s2 /\ s2 <: s1 ;
 proof of eq_refl = by property eq_incl, incl_refl ;
 proof of eq_symm = by property eq_incl ;
 proof of eq_tran = by property eq_incl, incl_tran ;
 theorem mem_eq : all s1 s2 : Self, s1 = s2 <->
                  (all v : Val, v << s1 <-> v << s2)
  proof = by property eq_incl, mem_incl ;
\end{lstlisting}
Note that here we can delete the definition of \code{=}. It is trivial to show that
\code{s1<:s2 && s2<:s1} indeed satisfies \code{eq_incl}, if we decide to use such a
definition; on the other hand, if we provide a more efficient algorithm for \code{=}, we have
just to check its validity by proving \code{eq_incl}, never having to prove again
\code{eq_refl}, \code{eq_symm} and \code{eq_tran}.

\subsubsection{A Few Trivial Properties}
\label{specsubset_extensional_trivial}

Having proposed a definition of extensional subsets by enforcing a signature containing
inclusion with some standard properties, we now complete our species with additional facts
that we consider relevant and useful for later uses. We add the following properties and
proofs in the species \code{ExtSubset}:
\begin{lstlisting}
 theorem incl_empty : all s : Self, empty <: s
  proof = by property mem_incl, mem_empty ;
 theorem incl_insert : all s : Self, all v : Val, s <: s + v
  proof = by property mem_insert, mem_incl ;
 theorem incl_remove : all s : Self, all v : Val, s - v <: s
  proof = by property mem_remove, mem_incl ;
 theorem incl_insert_mem : all s : Self, all v : Val,
                           v << s -> s + v <: s
  proof = by property mem_insert, mem_incl ;
\end{lstlisting}
Compiling this new version, \zenon is able to derive a proof for \code{incl_empty},
\code{incl_insert} and \code{incl_remove}, but not for \code{incl_insert_mem}. Before blaming
\zenon, let's produce by hand a derivation tree of the property \code{incl_insert_mem}, giving
a hierarchical vision of the deductive proof. In such a tree the goal is at the bottom, and
using deduction rules we progress toward trivial subgoals at the top of the tree, possibly
branching (for example, to prove $A\!\land\!B$ one may provide a proof of $A$ and a proof of
$B$). We got something like:
\begin{small}
\[\begin{array}{c}
\begin{array}{c}
?\\\hline
s\!:\!Self, v\:w\!:\!\fid{Val}, v\!\in\!s, v\!=\!w\vdash w\!\in\!s
\end{array}
\quad\quad
\begin{array}{c}
\text{Trivial, by assumption}\\\hline
s\!:\!Self, v\:w\!:\!\fid{Val}, v\!\in\!s, w\!\in\!s\vdash w\!\in\!s
\end{array}
\\\hline
\begin{array}{c}
\begin{array}{c}
s\!:\!Self, v\:w\!:\!\fid{Val}, v\!\in\!s, w\!=\!v \lor w\!\in\!s\vdash w\!\in\!s
\\\hline
s\!:\!Self, v\:w\!:\!\fid{Val}, v\!\in\!s, w\!\in\!s\!+\!v\vdash w\!\in\!s
\end{array}
\\\hline
s\!:\!Self, v\!:\!\fid{Val}, v\!\in\!s\vdash
\forall\:w\!:\!\fid{Val},\:w\!\in\!s\!+\!v\Rightarrow w\!\in\!s
\\\hline
\vdash
\forall\:s\!:\!Self,\:v\!:\!\fid{Val},\:
v\!\in\!s\Rightarrow s\!+\!v\!\subseteq\!s
\end{array}
\end{array}\]
\end{small}
This identifies the difficulty preventing the derivation of the proof by \zenon, as something
is missing in the left branch, the subgoal denoted by the interrogation mark. Indeed, this
subgoal has no reason to be valid: \code{=} is an equivalence relation, but nothing enforces
it to be a congruence w.r.t. membership, that is we have never required
$v\!\in\!s, v\!=\!w\vdash w\!\in\!s$. To understand why this may not be the case even in
perfectly legitimate cases, remember the comment in Sub. \ref{specsuperset_species}, about the
encoding of natural values as strings.

Of course, using the symbol \code{=} it is clear that our intention was to capture a form of
equality; so we can emphasise this interpretation by adding the congruence property at the
beginning of the species \code{ExtSubset}:
\begin{lstlisting}
 property mem_congr : all v1 v2 in Val, Val!( = )(v1, v2) ->
                      (all s in Self, (v1 << s) <-> (v2 << s)) ;
\end{lstlisting}
Given this property as an additional tip, \zenon indeed succeeds:
\begin{lstlisting}
 theorem incl_insert_mem : all s : Self, all v : Val,
                           v << s -> s + v <: s
  proof = by property mem_congr, mem_insert, mem_incl ;
\end{lstlisting}

\subsubsection{Proving Non Trivial Properties}
\label{specsubset_extensional_nontrivial}

We can also try to prove the following property:
\begin{lstlisting}
 theorem incl_remove_mem : all s : Self, all v : Val,
                           ~(v << s) -> s <: s - v
  proof = by property mem_congr, mem_remove, mem_incl ;
\end{lstlisting}
\zenon does not succeeds\footnote{Remember that actual results depend upon the computer and
the parameters of \zenon.}, but in this case it is apparently not because of missing
information as a proof by hand appears possible:
\begin{small}
\[\begin{array}{c}
\begin{array}{c}
\text{Assuming }w\!=\!v\text{ is absurd}\\\hline
s\!:\!Self, v\:w\!:\!\fid{Val}, v\!\not\in\!s, w\!\in\!s\vdash w\!\not=\!v
\end{array}
\quad\quad
\begin{array}{c}
\text{Trivial, by assumption}\\\hline
s\!:\!Self, v\:w\!:\!\fid{Val}, v\!\not\in\!s, w\!\in\!s\vdash w\!\in\!s
\end{array}
\\\hline
\begin{array}{c}
\begin{array}{c}
s\!:\!Self, v\:w\!:\!\fid{Val}, v\!\not\in\!s, w\!\in\!s\vdash
w\!\not=\!v \land w\!\in\!s
\\\hline
s\!:\!Self, v\:w\!:\!\fid{Val}, v\!\not\in\!s, w\!\in\!s\vdash w\!\in\!s\!-\!v
\end{array}
\\\hline
s\!:\!Self, v\!:\!\fid{Val}, v\!\not\in\!s\vdash
\forall\:w\!:\!\fid{Val},\:w\!\in\!s\Rightarrow w\!\in\!s\!-\!v
\\\hline
\vdash
\forall\:s\!:\!Self,\:v\!:\!\fid{Val},\:
v\!\not\in\!s\Rightarrow s\!\subseteq\!s\!-\!v
\end{array}
\end{array}\]
\end{small}
It is likely to be a problem of complexity, and human guidance is required.

Two main approaches are possible in \foc to prove non-trivial properties using \zenon. The
first one consists into introducing progressively lemmas, easier to prove, then to use these
lemmas to derive the complex results. Unfortunately, this
% Didou: complexifies
makes the signature of the species more complex  with numerous uninteresting results; in other word, the documentation of the species
can quickly becomes clumsy. The second approach requires user-guided proof using the \foc
Proof Language (\fpl). We adopt the latter, detailing the important steps of the proof and
expecting \zenon to complete it, as follows:
\begin{lstlisting}
 theorem incl_remove_mem : all s : Self, all v : Val,
                           ~(v << s) -> s <: s - v
  proof = <1>1 assume s : Self, v : Val, hypothesis Hv : ~(v << s),
               prove s <: s - v
           <2>1 assume w : Val, hypothesis Hw : w << s,
                prove w << s - v
            <3>1 prove ~(Val!( = )(w, v)) /\ w << s
             <4>1 prove ~(Val!( = )(w, v))
                  by property mem_congr hypothesis Hv, Hw
             <4>2 prove w << s
                  by hypothesis Hw
             <4>f conclude
            <3>f qed by property mem_remove step <3>1
           <2>f qed by property mem_incl step <2>1
          <1>f conclude ;
\end{lstlisting}
This script details the proof structure to be followed by \zenon. The label \code{<x>y}
indicates the level \code{x} and the step \code{y} in this level; levels have to be managed
consistently, while steps are just names. In essence, to prove a result labelled \code{<x>y},
one can introduce a few lemmas labelled \code{<x+1>y1}, \ldots, \code{<x+1>yn} and conclude
with a command \code{<x+1>f qed by step <x+1>y1, ..., <x+1>yn} (and additional hypotheses
and properties if required). Note that the step \code{f} here is the user notation for the
final step for a given level, but that the compiler does not care about the value specifying
the step as long as it is a unique identifier for the current level. Similarly, the label
\code{<1>f} does not mark the proof of the statement \code{<1>1} but use this statement to
prove the theorem \code{incl_remove_mem}. Take care to use different step identifiers for a
given level, as repeating an identifier will not cause an error but will mask the associated
results and prevent \zenon to conclude.

A goal step is associated with assumptions (keyword \code{assume} or \code{hypothesis}), a
goal (keyword \code{prove}), and possibly the tips to solve the goal (keyword \code{by}
completed with \code{definition of}, \code{property}, \code{hypothesis}, \code{step} and
\code{type}). If the goal \code{<x>y} is not provided with tips, then the next steps (at level
\code{x+1}) provide a strategy to solve it, the last step of the level \code{x+1} being
expected to be a \code{qed} with tips (or a \code{conclude}, which is equivalent to a
\code{qed by step <x+1>*}).

Writing such a script is better done incrementally and interactively. The first attempt is
generally a fully automated proof by \zenon, levels being introduced gradually to give
additional tips and guidances for tricky parts. Let's develop such a proof for a new theorem
in our species \code{ExtSubset} that describes a way to split a subset, first attempting an
automated proof:
\begin{lstlisting}
 theorem remove_insert : all s : Self, all v : Val,
                         v << s -> s = (s - v) + v
  proof = by property eq_incl, mem_incl, mem_insert, mem_remove,
                      mem_congr, Val!diff_eq ;
\end{lstlisting}
\zenon is however unlikely to derive the proof, so we have to give more details:
\begin{lstlisting}
 theorem remove_insert : all s in Self, all v in Val,
                         v << s -> s = (s - v) + v
  proof = <1>1 assume s : Self, v : Val, hypothesis Hv : v << s,
               prove s = (s - v) + v
           <2>1 assume w : Val, hypothesis Hw : w << s,
                prove w << (s - v) + v
                by property mem_insert, mem_remove
           <2>2 assume w : Val, hypothesis Hw : w << (s - v) + v,
                prove w << s
                by property mem_insert, mem_remove, mem_congr,
                   Val!diff_eq
           <2>f qed by property eq_incl, mem_incl step <2>1, <2>2
          <1>f qed by step <1>1 ;
\end{lstlisting}
The level \code{<1>} is associated to the theorem we want to prove, and the level \code{<2>}
indicates that to prove the equality, we have to prove the mutual inclusion. Invoking the
compiler, we got a message indicating that no proof has been found for \code{<2>1} and
\code{<2>2}; on the other hand the absence of error messages for \code{<1>f} also indicates
that provided a proof for \code{<2>1} and \code{<2>2} \zenon will indeed conclude. So we
provide more details, but only for the failing steps, first \code{<2>1}:
\begin{lstlisting}
           <2>1 assume w : Val, hypothesis Hw : w << s,
                prove w << (s - v) + v
            <3>1 prove ~(Val!( = )(w, v)) -> w << s - v
                 by property mem_remove hypothesis Hw
            <3>f qed by property mem_insert step <3>1
\end{lstlisting}
The important indication given in step \code{<3>1} is to consider the case $w\!\not=\!v$;
\zenon is then able to conclude (by also considering the case $w\!=\!v$, for which the
property \code{mem_insert} apply) and prove the subgoal \code{<2>1}.

Given the same amount of details for \code{<2>2}, the proof is finally derived:
\begin{lstlisting}
           <2>2 assume w : Val, hypothesis Hw : w << (s - v) + v,
                prove w << s
            <3>1 prove Val!( = )(w, v) -> w << s
                 by property mem_congr hypothesis Hv
            <3>2 prove w << s - v -> w << s
                 by property mem_remove
            <3>f qed by property mem_insert hypothesis Hw
                        step <3>1, <3>2
\end{lstlisting}

\zenon provides various indications when failing or succeeding, for example:
\begin{itemize}
\item When succeeding, the message \textbf{unused hypothesis} (alternatively
\textbf{unused variable}) indicates that some of the tips or variables are useless; note
however that useless tips are not always be detected as there is no guarantee that \zenon will
find the most ``efficient'' proof\footnote{In exceptional cases, it may also happen that
\zenon indicates that a tip is useless, yet is not able to find a proof without it.}.
\item When failing, the message \textbf{exhausted search space without finding a proof}
indicates that it is impossible to prove the goal, at least with just the provided tips. The
goal is not derivable from the provided assumptions,
% Didou: and
or may be false.
\item When failing, the message \textbf{could not find a proof within the time limit}
(alternatively \textbf{within the memory size limit} or
\textbf{within the inference steps limit}) indicates that \zenon has reached the fixed limits
passed as parameters; there may be a proof or not. It is then possible either to modify the
parameters used when invoking \zenon, or to reduce the number of tips, or to use the \fpl to
split the proof in subproofs.
\end{itemize}

To avoid error messages, an alternative solution is to incrementally develop the proof using
the keyword \code{assumed}. For example, to prove a goal, you may consider 3 subgoals and
start your proof as follows:
\begin{lstlisting}
 proof = <1>1 assume ...
              prove subgoal1
              assumed (* TODO *)
         <1>2 assume ...
              prove subgoal2
              assumed (* TODO *)
         <1>3 assume ...
              prove subgoal3
              assumed (* TODO *)
         <1>f conclude ;
\end{lstlisting}
The compiler can be invoked to check that indeed the goal can be proved using the three
subgoals. It is then possible to focus for example on the proof of step \code{<1>2}, etc.

\begin{tip}[Proofs] Complex proofs are better developed incrementally, from the general levels
to the more detailed ones, and invoking regularly \zenon to check the validity of the
currently developed steps.
\end{tip}

%----------------------------------------------------------------------------------------------
\subsection{Finite Subsets}
\label{specsubset_finite}

We have used inheritance and parametrisation as composition operators between specifications,
describing a species \code{Superset}, a species \code{Subset} and a species \code{ExtSubset}.

Our aim is to define additional species inheriting from \code{ExtSubset}, to provide more and
more details, for example indicating that the representation (the concrete datatype of the
elements of the species) is based on lists and providing the associated algorithms for the
methods. In such a case the inheritance acts as a refinement operator -- that is progress
toward an implementation, with concrete datatypes and algorithms.

The most straightforward approach would be to embed the definition of the type and operations
for lists in the species inheriting from \code{ExtSubset}. Yet we choose here to first specify
and implement lists in an independent hierarchy of species, and then to use this hierarchy to
refine \code{ExtSubset}. Beyond the illustration provided in this tutorial, this approach is
also fully justifiable, ensuring reusability of lists in other contexts.

%----------------------------------------------------------------------------------------------
%----------------------------------------------------------------------------------------------
\section{Specifying Lists}
\label{speclist}

We edit a new file, named \textbf{mylist.fcl}\footnote{The file name \textbf{list.fcl} cannot
be used because of name clashes with \coq library.}, to describe lists, with two directives:
\begin{lstlisting}
use "basics" ;;
open "superset" ;;
\end{lstlisting}

%----------------------------------------------------------------------------------------------
\subsection{Co-Lists}
\label{speclist_co}

We start with a species \code{CoList} that represents a very abstract form of lists, possibly
infinite. As for \code{Subset}, \code{CoList} needs to be parametrised by a \code{Superset},
representing the values that are put in a list. We also specify various methods, as follows:
\begin{lstlisting}
species CoList(Val is Superset) =

 signature nil : Self ;
 signature cons : Val -> Self -> Self ;
 signature isnil : Self -> basics#bool ;
 signature head : Self -> Val ;
 signature tail : Self -> Self ;

 property isnil_nil : all l : Self,
                      isnil(l) <-> basics#( = )(l, nil) ;
 property isnil_cons : all v : Val, all l : Self,
                       ~ isnil(cons(v, l)) ;
 property head_cons : all v : Val, all l : Self,
                      basics#( = )(head(cons(v, l)), v) ;
 property tail_cons : all v : Val, all l : Self,
                      basics#( = )(tail(cons(v, l)), l) ;
 property list_dec : all l : Self,
                     ~ isnil(l) ->
                     basics#( = )(l, cons(head(l), tail(l))) ;

end ;;
\end{lstlisting}
The \code{nil} and \code{cons} methods are used to build
% Didou: colists,
\code{CoList}s,
while the \code{isnil},
\code{head} and \code{tail} methods are used to analyse and destruct them. In the properties,
\code{basics#( = )} represents the structural equality, that is the standard equality in both
\coq and \ocaml. The combination of the properties \code{isnil_nil} and \code{list_dec} is
very strong, indicating that the concrete implementation of any
%Didou: colist
\code{CoList}
is either \code{nil} or
\code{cons}, a form of surjectivity of these methods w.r.t. the species.

Note that whereas subsets are things with a decidable membership,
% Didou: colists
\code{CoList}s
are things defining
a succession of values: membership is not specified\footnote{It may not be possible to browse
all elements in a
% Didou: colist
\code{CoList}
to find a specific value.} but we have a function returning the head
element of a
% Didou: colist
\code{CoList}
. Another characteristic of
% Didou: colists
\code{CoList}s
is that values may appear several time
-- that is it is possible for example to have $\fid{head}(l)\!=\!head(tail(l))$. This
illustrates the difference of point of view: (finite) subsets and (finite) lists can appear
pretty similar once implemented, but the intentions are different
% Didou
(with the only difference being the number of occurrences of a same element).

The properties listed in \code{CoList} are sufficient to prove other expected results, such as
for example the injectivity of \code{cons}:
\begin{lstlisting}
 theorem cons_left : all l1 l2 : Self, all v1 v2 : Val,
                     basics#( = )(cons(v1, l1), cons(v2, l2)) ->
                     basics#( = )(v1, v2)
  proof = <1>1 assume t1 t2 : Self, h1 h2 : Val,
               hypothesis Heq : basics#( = )(cons(h1, t1),
                                             cons(h2, t2)),
               prove basics#( = )(h1, h2)
           <2>1 prove basics#( = )(head(cons(h1, t1)),
                                   head(cons(h2, t2)))
                by hypothesis Heq
           <2>2 prove basics#( = )(h1, head(cons(h2, t2)))
                by step <2>1 property head_cons
           <2>f qed by step <2>2 property head_cons
          <1>f qed by step <1>1 ;

 theorem cons_right : all l1 l2 : Self, all v1 v2 : Val,
                      basics#( = )(cons(v1, l1), cons(v2, l2)) ->
                      basics#( = )(l1, l2)
  proof = <1>1 assume l1 l2 : Self, v1 v2 : Val,
               hypothesis Heq : basics#( = )(cons(v1, l1),
                                             cons(v2, l2)),
               prove basics#( = )(l1, l2)
           <2>1 prove basics#( = )(tail(cons(v1, l1)),
                                   tail(cons(v2, l2)))
                by hypothesis Heq
           <2>2 prove basics#( = )(l1, tail(cons(v2, l2)))
                by step <2>1 property tail_cons
           <2>f qed by step <2>2 property tail_cons
          <1>f qed by step <1>1 ;
\end{lstlisting}

As a final point, one can remark that the \code{head} function is specified only for non-empty
list. For an empty list, it has to return a value from \code{Val}\footnote{Unless an
exception is raised.}, but which one? Remember that \code{Val}, which represents a collection
implementing the interface of \code{Superset}, can be empty; in such a case, the list species
still contains the empty list (and only the empty list), and implementing the \code{head}
function can be ``difficult''.

%----------------------------------------------------------------------------------------------
\subsection{Finite Lists}
\label{speclist_finite}

An abstract entity such as \code{CoList} is not very interesting. We aim at representing
standard lists, as they are defined in \ocaml and \coq for example, which are in particular
finite (well-founded), even if the species of all lists is infinite.

Finiteness is a tricky concept to capture. Similarly to what is mentioned in Sub.
\ref{specsubset_extensional}, a possible approach is for example to specify a function in
$\fid{Self}\!\to\!\NAT$ computing the length of the list provided as a parameter. The
existence of such a function, that has to be total, indeed ensures finiteness of any value of
the species. But how do we specify that natural values are themselves finite? We are just
pushing back the problem to another species.

Another possible approach is to provide a concrete representation which is itself finite and
encodes only finite values. This would correspond, in our case, into using the inductive
definition of lists of \coq or \ocaml, knowing that such a definition only describes finite
constructions.

But it is also possible indeed to specify finiteness, as illustrated here by introducing a
new species \code{FiniteList}:
\begin{lstlisting}
species FiniteList(Val is Superset) =

 inherit CoList(Val) ;

 property finite :
  all f : (Self -> basics#bool),
  f(nil) ->
  (all l : Self, f(l) -> all v : Val, f(cons(v, l))) ->
  all l : Self, f(l) ;

end ;;
\end{lstlisting}
The property \code{induction} requires the standard induction principle for lists to be valid:
to prove that a boolean function $f$ is true for any list, it is sufficient to prove that
it is true for the empty list and that if it is true for a list $t$ then it is also true for
the list $\fid{cons}(h, t)$.

It indeed provides a strategy to prove e.g.
$f(\fid{cons}(a,\fid{cons}(b,\fid{cons}(c,\fid{nil}))))$: from $f(nil)$ and
$f(t)\!\Rightarrow\!f(cons(h, t))$, one can prove $f(\fid{cons}(c, nil))$, using again
$f(t)\!\Rightarrow\!f(cons(h, t))$ he derives $f(\fid{cons}(b, \fid{cons}(c, nil)))$, etc.

The property \code{induction} is higher-order: it quantifies over all possible decidable
predicates that can be applied to \code{Self} -- that is in fact over all the methods
of the species acting over \code{Self}, as encapsulation prevents manipulations from outside
the species. Note that higher-order properties can be expressed in \foc, but are not
manageable by \zenon; in such a case, a proof in \coq is required, as illustrated later.

Why does the property \code{induction} prove that lists are finite? Let's first define the
finiteness predicate $\fid{finite}$ by two axioms:
\begin{small}
\[\begin{array}{c}
\fid{finite}(\fid{nil}) \quad \quad \quad \quad
\forall\:t,\:\fid{finite}(t)\Rightarrow \forall\:h,\:\fid{finite}(\fid{cons}(h,t))
\end{array}\]
\end{small}
The first claims that the empty list is finite, and the second that if a list is finite, then
it is still finite after having added an element to it. It is then possible to prove the
theorem $\forall\:l,\:\fid{finite}(l)$ by applying \code{induction}.

Having ensured that \code{FiniteList} only describes finite lists, we may also add the
specification of two methods, membership \code{mem} and deletion \code{del}, whose
implementation is likely to have to browse all the elements of a list:
\begin{lstlisting}
 signature mem : Val -> Self -> basics#bool ;
 property mem_nil : all v : Val, ~ mem(v, nil) ;
 property mem_cons : all v h : Val, all t : Self,
                     mem(v, cons(h, t)) <->
                     Val!( = )(v, h) \/ mem(v, t) ;

 signature del : Self -> Val -> Self ;
 property mem_del : all v w : Val, all l : Self,
                    mem(v, del(l, w)) <->
                    ~ Val!( = )(v, w) /\ mem(v, l) ;
\end{lstlisting}
Note that the property \code{mem_del} states that the method \code{del} has to remove all
occurrences of values which are equal to its second parameter.

%----------------------------------------------------------------------------------------------
%----------------------------------------------------------------------------------------------
\section{Refining Lists}
\label{reflist}

Intuitively, we have specified in the previous section what we need: \code{FiniteList}
describes structures to collect finitely many values, with methods to build and analyse such
structures.

In this section, we continue the development by inheritance, but having in mind a
\emph{refinement} approach: we do not specify anymore the structure itself, but instead we
provide an implementation. Still favouring a very progressive approach in this tutorial, we
\emph{refine} the species \code{FiniteList} by a species \code{EnumList}, with a few
algorithms, before providing a complete species \code{InductiveList} in which the
\emph{representation}, that is the concrete datatype used to encode the specified structures,
is defined.

%----------------------------------------------------------------------------------------------
\subsection{Enumerable Lists}
\label{reflist_enumerable}

We create first a new species \code{EnumList} describing intuitively lists whose elements can
be enumerated. Editing the file \textbf{mylist.fcl}, we start \code{EnumList} as follows:
\begin{lstlisting}
species EnumList(Val is Superset) =

 inherit FiniteList(Val) ;

end ;;
\end{lstlisting}

\subsubsection{Recursive Membership}
\label{reflist_enumerable_mem}

The first definition that we provide in \code{EnumList} is the membership. Knowing that any
list in $\fid{Self}$ is finite, we browse the structure using the dedicated methods to look
for a specific value:
\begin{lstlisting}
 let rec mem(v : Val, l : Self) =
  if isnil(l)
  then false
  else (if Val!( = )(v, head(l)) then true else mem(v, tail(l))) ;
 (* TODO : Termination Proof *)
\end{lstlisting}
The definition for the method \code{mem} is recursive, as computing \code{mem(v,l)} requires
in some cases to compute \code{mem(v, tail(l))}. In a formal development, one has to prove
that recursive functions always terminate to avoid logical inconsistency\footnote{Consider for
example the function \code{let rec f(n in Nat) = f(n)}: should it be a valid definition, it
could have any return type -- and in particular it could be used as a trick to build a value
from an empty type.}. In its current version, \foc offers two ways to deal with such a
function:
\begin{itemize}
\item The keyword \code{recstruct} indicates that the function is recursive, and that any
recursive call is done on a parameter which is structurally ``smaller'' than the initial one.
It is illustrated later in this tutorial.
\item The keyword \code{rec} indicates that the function is recursive. In this case a
termination proof is normally added after the definition, for example:

\code{let rec f(...)=... termination proof = by order ...}.
\end{itemize}
With regard to the second case, the full support in \foc of recursive functions, with proofs
of termination based on measures, orders and so on is still in development and is not further
detailed here. For this reason, the termination proof is for now admitted and is not required
in the code. The \foc compiler however warns that the species is potentially unsafe
% Didou
and will generate a pseudo-termination proof instead, in order to have the whole program
however accepted by \coq. In some sense, we afford lying to \coq, being optimistic about the
part of the consistency of our program involved by the termination of this function.

We adopt here the second version, noting that if the property \code{finite} is valid, the
function \code{mem} terminates. Indeed, a possible interpretation of \code{finite} is that it
enforces any list to be build-able (accessible) starting from the list \code{nil} and using
only the function \code{cons}; reciprocally, because of the properties stated in \code{CoList}
we know that \code{head} and \code{tail} act as decomposition operators and allow for a
systematic enumeration of the values of the list.

Having defined \code{mem} (and admitted termination), we can prove that this function has the
expected properties as follows:
\begin{lstlisting}
 proof of mem_nil = by definition of mem property isnil_nil ;

 proof of mem_cons =
  <1>1 assume v h : Val, t : Self, hypothesis H : Val!( = )(v, h),
       prove mem(v, cons(h, t))
   <2>1 prove Val!( = )(v, head(cons(h, t)))
        by property head_cons hypothesis H
   <2>f qed by definition of mem property isnil_cons step <2>1
  <1>2 assume v h : Val, t : Self, hypothesis H : mem(v, t),
       prove mem(v, cons(h, t))
   <2>1 prove mem(v, tail(cons(h, t)))
        by property tail_cons hypothesis H
   <2>f qed by step <2>1 definition of mem property isnil_cons
  <1>3 assume v h : Val, t : Self, hypothesis H : mem(v, cons(h, t)),
       prove Val!( = )(v, h) \/ mem(v, t)
   <2>1 hypothesis H2 : ~ Val!( = )(v, h),
        prove mem(v, tail(cons(h, t)))
        by hypothesis H, H2 definition of mem
           property isnil_cons, head_cons
   <2>f qed by step <2>1 property tail_cons
  <1>f conclude ;
\end{lstlisting}

\subsubsection{Recursive Deletion}
\label{list_enumerable_del}

We can add a second definition in our species, for the deletion method:
\begin{lstlisting}
 let rec del(l : Self, v : Val) =
  if isnil(l)
  then nil
  else if Val!( = )(v, head(l))
       then del(tail(l), v)
       else cons(head(l) , del(tail(l), v)) ;
 (* TODO : Termination Proof *)
\end{lstlisting}
Again, termination is admitted in this case.

This second method is very illustrative. Indeed, provided this recursive definition, it is
possible to prove the property \code{mem_del}, yet we need to use the induction principle.
That is, as indicated by the property \code{induction}\footnote{In fact, we need a slightly
more generic form as we need to quantify over all predicates.} to prove $\forall\:l,\:P(l)$
(any list satisfies a property $P$), it is sufficient to prove $P(\fid{nil})$ (the property is
true for the empty list) and that for any list $t$, assuming $P(t)$ (the property is true for
$t$), then for any $h$, $P(cons(h, t))$ is provable as well (the property is still true when
adding an element to $t$).

We can foresee a rather complex proof, as \code{mem_del} is a composite statement and because
we have to use \code{induction}. To avoid a very long proof we suggest here a decomposition of
\code{mem_del} into three subgoals, and further to use the \code{logical let} feature to
improve the readability of the proof. Let's start with a first subgoal:
\begin{lstlisting}
 logical let mdm_(v : Val, w : Val, l : Self) =
  mem(v, del(l, w)) -> mem(v, l) ;

 theorem mem_del_mem : all v w : Val, all l : Self, mdm_(v, w, l)
\end{lstlisting}
We first define a statement \code{mdm_}, using the keyword \code{logical let} to introduce a
parametrised statement. Note that we do not claim that such a statement is valid: we just
introduce a name for a complex
% Didou: logical expression.
{\bf logical expression}.
The theorem \code{mem_del_mem}, on the
contrary, claims that \code{mdm_} is always valid.

The first level of the proof of \code{mem_del_mem} is as follows:
\begin{lstlisting}
  proof = <1>1 assume v w : Val,
               prove mdm_(v, w, nil)
               assumed (* TODO *)
          <1>2 assume v w : Val, t : Self,
               hypothesis Hind : mdm_(v, w, t), assume h : Val,
               prove mdm_(v, w, cons(h, t))
               assumed (* TODO *)
          <1>3 assume v w : Val,
               prove mdm_(v, w, nil)->
                     (all t : Self, mdm_(v, w, t) ->
                      all h : Val, mdm_(v, w, cons(h, t))) ->
                     all l : Self, mdm_(v, w, l)
               assumed (* Standard induction principle *)
          <1>f conclude ;
\end{lstlisting}
Thanks to the use of the \code{mdm_} name, we can see clearly that the step \code{<1>1} is of
the form $P(\fid{nil})$, and the step \code{<1>2} of the the form
$P(t)\!\Rightarrow\!P(\fid{cons}(h,t))$. They have to be proved but for the current stage we
just assume these results.

The label \code{<1>3} states very clearly the induction principle, instantiated for the
property we are trying to prove, that is \code{mdm_}. We also assume this result, but in this
case because the current version of \zenon is not able to manage higher-order declarations
such as \code{induction} -- a \coq proof is still possible, but here we decide to just accept
this result as a fact
% Didou
to prevent running in too complex notions out of the scope of this tutorial
. The step \code{<1>f} concludes.

We can check that this proof structure is valid by invoking the compiler, before pursuing the
development. For the step \code{<1>1}, it is straightforward:
\begin{lstlisting}
  proof = <1>1 assume v w : Val,
               prove mdm_(v, w, nil)
               by definition of mdm_, del property isnil_nil
\end{lstlisting}
Note that we need to make visible the definition of the statement \code{mdm_}.

For the step \code{<1>2}, it is a little longer. The idea is that the definition of the
\code{del} function is visible, and can be analysed by case reasoning -- for example, an
\code{if} construct is dealt with by assuming that the condition is true in a first case, and
false in a second case. The resulting proof is as follows:
\begin{lstlisting}
          <1>2 assume v w : Val, t : Self,
               hypothesis Hind : mdm_(v, w, t), assume h : Val,
               prove mdm_(v, w, cons(h, t))
           <2>1 hypothesis H1 : mem(v, del(cons(h, t), w)),
                prove mem(v, cons(h, t))
            <3>1 hypothesis H2 : Val!( = )(v, h),
                 prove mem(v, cons(h, t))
                 by property mem_cons hypothesis H2
            <3>2 hypothesis H2 : ~ Val!( = )(v, h),
                            H3 : Val!( = )(w, h),
                 prove mem(v, cons(h, t))
             <4>1 prove ~ isnil(cons(h, t))
                  by property isnil_cons
             <4>2 prove Val!( = )(w, head(cons(h, t)))
                  by property head_cons hypothesis H3
             <4>3 prove mem(v, del(tail(cons(h, t)), w))
                  by definition of del step <4>1, <4>2 hypothesis H1
             <4>4 prove mem(v, del(t, w))
                  by step <4>3 property tail_cons
             <4>f qed by step <4>4 hypothesis Hind definition of mdm_
                         property mem_cons
            <3>3 hypothesis H2 : ~ Val!( = )(v, h),
                            H3 : ~ Val!( = )(w, h),
                 prove mem(v, cons(h, t))
             <4>1 prove ~ isnil(cons(h, t))
                  by property isnil_cons
             <4>2 prove ~ Val!( = )(w, head(cons(h, t)))
                  by property head_cons hypothesis H3
             <4>3 prove mem(v, cons(head(cons(h, t)),
                                    del(tail(cons(h, t)), w)))
                  by definition of del step <4>1, <4>2 hypothesis H1
             <4>4 prove mem(v, cons(h, del(t, w)))
                  by step <4>3 property head_cons, tail_cons
             <4>5 prove mem(v, del(t, w))
                  by step <4>4 property mem_cons hypothesis H2
             <4>f qed by step <4>5 hypothesis Hind definition of mdm_
                         property mem_cons
            <3>f conclude
           <2>f qed by step <2>1 definition of mdm_
\end{lstlisting}
This proof is simple but rather long, essentially because we have to explicit a lot of
transformations such as
$\fid{cons}(\fid{head}(\fid{cons}(h,t)),\fid{tail}(\fid{cons}(h,t)))=\fid{cons}(h,t)$.
Remember indeed that \code{cons}, \code{head} and \code{tail} are not yet defined, but are
just described by properties. Once these functions implemented, those simplifications can be
the result of computations, with shorter proofs; yet having only an axiomatised form of these
functions for now, we have to provide some guidance. In this case, a more straightforward
approach (jumping directly to an implementation by inductive lists) is likely to be simpler.

A similar approach is chosen for the second subgoal:
\begin{lstlisting}
 logical let mde_(v : Val, w : Val, l : Self) =
  mem(v, del(l, w)) -> ~ Val!( = )(v, w) ;

 theorem mem_del_eq : all v w : Val, all l : Self, mde_(v, w, l)
 proof = <1>1 assume v w : Val,
              prove mde_(v, w, nil)
              by definition of mde_, del property isnil_nil, mem_nil
         <1>2 assume v w : Val, t : Self,
              hypothesis Hind : mde_(v, w, t), assume h : Val,
              prove mde_(v, w, cons(h, t))
          <2>1 hypothesis H1 : mem(v, del(cons(h, t), w)),
               prove ~ Val!( = )(v, w)
           <3>1 hypothesis H2 : Val!( = )(w, h),
                prove ~ Val!( = )(v, w)
            <4>1 prove mem(v, del(t, w))
             <5>1 prove ~ isnil(cons(h, t))
                 by property isnil_cons
             <5>2 prove Val!( = )(w, head(cons(h, t)))
                  by property head_cons hypothesis H2
             <5>3 prove mem(v, del(tail(cons(h, t)), w))
                  by hypothesis H1 definition of del step <5>1, <5>2
             <5>f qed by step <5>3 property tail_cons
            <4>f qed by step <4>1 hypothesis Hind definition of mde_
           <3>2 hypothesis H2 : ~ Val!( = )(w, h),
                prove ~ Val!( = )(v, w)
            <4>1 hypothesis H3 : Val!( = )(v, h),
                 prove ~ Val!( = )(v, w)
                 by hypothesis H2, H3
                    property Val!eq_symm, Val!eq_tran
            <4>2 hypothesis H3 : ~ Val!( = )(v, h),
                 prove ~ Val!( = )(v, w)
             <5>1 prove ~ isnil(cons(h, t))
                  by property isnil_cons
             <5>2 prove ~ Val!( = )(w, head(cons(h, t)))
                  by property head_cons hypothesis H2
             <5>3 prove mem(v, cons(head(cons(h, t)),
                                    del(tail(cons(h, t)), w)))
                  by hypothesis H1 definition of del step <5>1, <5>2
             <5>4 prove mem(v, cons(h, del(t, w)))
                  by step <5>3 property head_cons, tail_cons
             <5>5 prove mem(v, del(t, w))
                  by step <5>4 property mem_cons hypothesis H3
             <5>f qed by step <5>5 hypothesis Hind definition of mde_
            <4>f conclude
           <3>f conclude
          <2>f qed by definition of mde_ step <2>1
         <1>3 assume v w : Val,
              prove mde_(v, w, nil) ->
                    (all t : Self, mde_(v, w, t) ->
                     all h : Val, mde_(v, w, cons(h, t))) ->
                    all l : Self, mde_(v, w, l)
              assumed (* Standard induction principle *)
         <1>f conclude ;
\end{lstlisting}

The third subgoal is left as an exercise to the reader; note that we have slightly adapted the
statement of \code{mem_del_inv} to ease the proof by induction:
\begin{lstlisting}
 theorem mem_del_inv : all v w : Val, ~ Val!( = )(v, w) ->
                       all l : Self, mem(v, l) -> mem(v, del(l, w))
  proof = assumed (* TODO *) ;
\end{lstlisting}

Finally we prove the global result
\begin{lstlisting}
 proof of mem_del = by property mem_del_mem, mem_del_eq, mem_del_inv
                       definition of mdm_, mde_ ;
\end{lstlisting}

%----------------------------------------------------------------------------------------------
\subsection{Inductive Lists}
\label{reflist_inductive}

It is now time to define the species \code{InductiveList} which refines, or more precisely
implements, the species \code{EnumList} and therefore the species \code{FiniteList} and
\code{CoList}.

\subsubsection{Non Empty Supersets}
\label{reflist_inductive_supersetwitness}

First, taking into account the comment at the end of Sub. \ref{speclist_co}, we append to
the file \textbf{superset.fcl} the definition of a new species \code{SupersetWitness} as
follows:
\begin{lstlisting}
species SupersetWitness  =

 inherit Superset ;

 signature witness : Self ;

end ;;
\end{lstlisting}
The \code{witness} constant ensures that a \code{SupersetWitness} is never empty. More
specifically, it is used in our case as the default value returned e.g. when the \code{head}
function is applied to the empty list. Do not forget to recompile \textbf{superset.fcl} once
edited.

\subsubsection{A Type for Lists}

Back to the file \textbf{mylist.fcl}, we now define the representation of the species as being
% Didou: lists
a list
; yet rather than using the predefined \code{list} type provided in \textbf{basics.fcl},
we introduce our own definition for the sake of illustration. Type definitions are allowed in
\foc, but only at top level, that is outside a species:
\begin{lstlisting}
type mylist('a) = | FNil | FCons ('a, mylist('a)) ;;
\end{lstlisting}
Such an inductive definition
% Didou
, called a ``sum type'',
claims that the type only contains values build from the
constructors (surjectivity), and that two values of this type are equal
% Didou: iff
if and only if
they are
structurally equal (injectivity);
% Didou: well-foundedness
well-foundation
(finiteness of the constructs) is also
ensured.

\begin{tip}\small[Syntax] Type definitions are only authorised at top-level.
\end{tip}

This is also a polymorphic type: \code{'a} represents a type variable parametrisation our
definition, \code{mylist(int)} or \code{mylist(char)} being examples of types obtained by
instantiating this variable. Note that the \foc syntax requires
% Didou: constructor
sum type constructor
identifiers to
start by an uppercase character.


% Didou
% \begin{tip}\small[Syntax] Constructor identifiers have to start with an uppercase character.
\begin{tip}\small[Syntax] Sum type constructor identifiers have to start with an uppercase character.
\end{tip}

\subsubsection{Inductive Lists}

Provided this type, it is possible to introduce associated top level definitions and theorems
(but not declarations and properties, in the absence of inheritance mechanism). In this
tutorial, however, we stick to the species vision and include all the definitions and theorems
in a new species. Starting with trivial definitions, the \code{IndList} species, to be
inserted in the file \textbf{mylist.fcl}, is as follows:
\begin{lstlisting}
species IndList(Val is SupersetWitness) =

 inherit FiniteList(Val) ;

 representation = mylist(Val) ;

 let nil = #FNil ;
 let cons(h : Val, t : Self) = #FCons(h, t) ;
 let isnil(l : Self) = match l with | #FNil -> true | _ -> false ;
 let head(l : Self) =
  match l with | #FCons(h, _) -> h | _ -> Val!witness ;
 let tail(l : Self) = match l with | #FCons(_, t) -> t | _ -> #FNil ;

end ;;
\end{lstlisting}
The keyword \code{representation} is used to associate a concrete datatype to the species;
once the representation is defined it cannot be changed at inheritance. Note that we
parametrise \code{mylist} with \code{Val} (that is its representation).

For the functions \code{isnil}, \code{head} and \code{tail} we use pattern matching (as in
\ocaml and \coq). In \foc pattern matching has to be complete and without redundancy; in other
words, all cases should be addressed, and patterns included in previous patterns are not
allowed.

\subsubsection{Higher-Order Proofs}

We have now a definition for all the declared methods, but we are still lacking the proofs of
the properties. Let's first address the higher-order \code{induction} property; as mentioned,
\zenon is not able to tackle it. The most simple approach is to admit this property as an
axiom:
\begin{lstlisting}
 proof of finite =  assumed (* Requires a Coq proof *) ;
\end{lstlisting}
The keyword \code{assumed} is the logical backdoor in \foc: the proof is not provided, but the
property is accepted as true. It is recommended to associate, in such a case, a comment
justifying why no proof is given. That is, it is possible to introduce axioms in a \foc
development, but those axioms are traced and documented.

The alternative approach is to provide the \coq proof. It is beyond the scope of this tutorial
to describe how to build such a proof; ideally the use of \coq scripts should be limited to
the standard library of \foc, and only expert users should try to use \coq. We just provide
in the rest of this paragraph a sketch of the process, that can be skipped by most readers.

The first step is to identify the relevant properties and definitions to derive the proof --
exactly as when \zenon is used. This is required to make these definitions and properties
visible from the proof context. It is then possible to build a template for the \coq proof, as
follows:
\begin{lstlisting}
 proof of finite =  coq proof definition of nil, cons
                    {* *} ;
\end{lstlisting}
The keywords \code{coq proof} indicates that the proof is attached and has to be checked
directly by \coq, without assistance of \zenon. It is followed by the tips, in this case we
need to make sure that the definition of the methods \code{nil} and \code{cons} are visible,
as we will in fact reason on the inductive definition of \code{mylist} and on the constructors
\code{Nil} and \code{Cons}.
% Didou
These tips will not be used by \coq but serves \foc to be aware that the following proof
(that it cannot analyse) will depend on the listed functions, properties and theorems. During
dependencies computation, this allows to invalidate the proof in children species who will have
redefined some of these listed dependencies (or inductively, dependencies on these dependencies).
The \coq script is then expected to follow, between \code{\{*} and
\code{*\}}; here it is empty, just marking a hole that we will fill later using \coq.

It is then possible to compile the file with the command \textbf{focalizec mylist.fcl}; the
compilation of course fails when \coq checks the proof file \textbf{mylist.v}. Using \coq (and
providing the path to the \foc and \zenon libraries, that are prompted at compilation time)
the proof can then be completed by hand, and copied into the \foc source file:
\begin{lstlisting}
 proof of finite =  coq proof definition of nil, cons
                    {* Proof.
                        intros.
                        unfold abst_cons, abst_nil, cons, nil in *.
                        induction l.
                         trivial.
                        apply H0; apply IHl.
                       Qed. *} ;
\end{lstlisting}
A new compilation of \textbf{mylist.fcl} now succeeds.

\subsubsection{Proofs on Inductive Types}

Other properties related to inductive definitions can be done in \coq, or can be assumed. We
choose for this tutorial the second option:
\begin{lstlisting}
 proof of isnil_nil = assumed (* TODO *) ;

 proof of isnil_cons = assumed (* TODO *) ;

 proof of head_cons = assumed (* TODO *) ;

 proof of tail_cons = assumed (* TODO *) ;

 proof of list_dec = assumed (* TODO *) ;
\end{lstlisting}

Note however that \zenon is currently evolving to provide some support for such results, by
allowing for a new form of tip, \code{type}, when the very definition of a type provides the
required information. The expected use is as follows:
\begin{lstlisting}
 proof of isnil_nil = by type mylist definition of nil, isnil ;
\end{lstlisting}
It can be experimentally tested, but only at top level for now.

\subsubsection{Structural Recursion}

In the species \code{IndList}, we redefine the methods \code{mem} and \code{del} as follows:
\begin{lstlisting}
 let mem(v : Val, l : Self) =
  let recstruct mem_(l : Self) =
   match l with
   | #FNil -> false
   | #FCons(h, t) -> if Val!( = )(v, h) then true else mem_(t)
  in mem_(l) ;

 proof of mem_nil = assumed (* TODO *) ;

 proof of mem_cons = assumed (* TODO *) ;

 let recstruct del(l : Self, v : Val) =
  match l with
  | #FNil -> #FNil
  | #FCons(h, t) -> if Val!( = )(v, h)
                    then del(t, v)
                    else #FCons(h, del(t, v)) ;

 proof of mem_del =  assumed (* TODO *) ;
\end{lstlisting}
In this case, we are using the keyword \code{recstruct} to indicate a structurally recursive
function, that is recursive calls on subterms of the parameter -- termination then being
trivially true. For such functions, however, the decreasing parameter has to be the first one;
that's why we are using a \code{let in} construct, as the declaration of \code{mem} puts the
list as the second parameter.

Such redefinitions, of course, cause the deletion of the associated proofs for \code{mem_nil},
\code{mem_cons}, \code{mem_del}. Yet the use of pattern matching on an inductive type leads to
simpler proofs of these properties. For this tutorial, we just admit these properties.

Note that the keyword \code{recstruct} may become deprecated in future versions of \foc, as it
is an \emph{ad hoc} adaptation to deal with a simple form of recursion.


%----------------------------------------------------------------------------------------------
%----------------------------------------------------------------------------------------------
\section{A Complete Implementation}
\label{implem}

We are now at the final stage of the development, in which we will define collections. A
\emph{collection} is a frozen implementation obtained by abstracting the concrete
representation of a \emph{complete species}, that is a species which has a concrete datatype
representation, a definition for every signature and a proof for every property.

Provided what has been developed up to now, there are still two approaches to implement
subsets as lists. In both case, we need a collection implementing a complete species, but
this species can either:
\begin{itemize}
\item inherits from both \code{ExtSubset} and \code{IndList}.
\item inherits from \code{ExtSubset} and be parametrised by \code{IndList}.
\end{itemize}
These two approaches have similarities -- for example, in both cases the membership method
\code{<<} of subsets is implemented as the membership method \code{mem} of lists -- but they
represent very different intentions.

In the first case, using multiple inheritance, we claim that any subset is also a list. So the
methods of both interfaces can be used to manipulate these values; a set can be created with
two elements, then it can be considered as a list in which we exchange the position of these
elements.

In the second case, using parametrisation, we use lists to represent subsets, but emphasising
that lists and subsets are different in nature: manipulating a subset can only be done using
the methods of the subset interface -- and it is possible to ensure that some ``dangerous''
methods defined for lists are not used when dealing with subsets.

We discuss in the rest of this section of the strategy to build an executable program using
the implementation of subsets with the second approach, parametrisation by lists. As a
collection is never parametrised, and always implement one and only one complete species
whose parameters are instantiated by collections, we have to follow a strict discipline: to
create a collection from a given species, we have to create a collection for its parameters.
That is creating a subset collection requires creating a list collection, which itself
requires creating a superset collection.

We edit a new file, named \textbf{main.fcl}, and include the required information with the
\code{use} and \code{open} directives:
\begin{lstlisting}
 use "basics" ;;
 open "superset" ;;
 open "subset" ;;
 open "mylist" ;;
\end{lstlisting}
The rest of the section deals with a superset collection, a list collection and a subset
collection, that are implementing complete species.

%----------------------------------------------------------------------------------------------
\subsection{Integers}
\label{implem_integer}

\subsubsection{Adding Inputs and Outputs to Supersets}
\label{implem_integer_hmi}

We can implement collections with the currently defined interfaces, unfortunately this would
not be very demonstrative in the absence of HMI. As a first mandatory step toward a program we
therefore create a new species with methods to import or export values:
\begin{lstlisting}
species PrintParseSuperset =

 inherit SupersetWitness ;

 signature print : Self -> basics#unit ;

 signature parse : basics#string -> Self ;

end ;;
\end{lstlisting}
Of course, it is possible to add, at any point of the development, a \code{print} and a
\code{parse} methods to any species -- as illustrated later in this section for subsets. But
the definition of a \code{PrintParseSpecies} interface is required. Indeed, to implement for
example the \code{print} method for a subset, one has to be able to rely on the \code{print}
method of its superset parameter, that is to ensure that such method exists in the interface.

\subsubsection{A Complete Integer Species}
\label{implem_integer_complete}

Provided the specification \code{PrintParseSuperset}, we develop a complete species
representing machine integers:
\begin{lstlisting}
species Int =

 inherit PrintParseSuperset ;

 representation = basics#int ;

 let ( = )(x : Self, y : Self) = basics#( = )(x, y) ;
 proof of eq_refl = by definition of ( = ) ;
 proof of eq_symm = by definition of ( = ) ;
 proof of eq_tran = by definition of ( = ) ;

 let witness = 0 ;

 let print(s : Self) = basics#print_int(s) ;

 let parse(s : basics#string) = basics#int_of_string(s) ;

end ;;
\end{lstlisting}
In this species, the representation is defined, all the declared methods are
defined\footnote{For \code{=} it is possible to use the \code{int} equality, denoted
\code{=0x} in \foc, but \zenon does not know much about it and cannot prove \code{eq_refl},
\code{eq_symm} and \code{eq_tran}.} and all the properties are proved.

\subsubsection{An Integer Collection}
\label{implem_integer_collection}

The species \code{Int} being complete, it can be transformed into a collection:
\begin{lstlisting}
collection Int_Coll = implement Int; end ;;
\end{lstlisting}
\code{Int} now denotes a form of abstract data type, whose representation is hidden. It is not
possible to inherit from \code{Int}, or to modify otherwise its structure; of course, it is
still possible to use \code{Int} as a parameter.

\begin{tip}[Implementation] Collections are the final entities of \foc developments and
represent executable implementations. They cannot be parametrised or inherited from, but they
can be used as parameters for other species.
\end{tip}

%----------------------------------------------------------------------------------------------
\subsection{Lists if Integers}
\label{implem_lists}

As we want to create a collection from the parametrised species \code{ExtSubset}, we need
first to create a collection for the parameter itself, that is a collection representing the
species \code{IndList}:
\begin{lstlisting}
collection IntList_Coll = implement IndList(Int_Coll) ; end ;;
\end{lstlisting}
Remember that a collection cannot be parametrised, and that it has to implement a species
whose all formal parameters are instantiated by a collection, as it is the case here.

\begin{tip}[Implementation] A collection can only implements a complete species whose
parameters are instantiated by collections.
\end{tip}

%----------------------------------------------------------------------------------------------
\subsection{Subsets}
\label{implem_subset}

\subsubsection{A Complete Subset Species}
\label{implem_subset_complete}

Having chosen to implement subsets as a species parametrised by lists, it is straightforward
to complete \code{ExtSubset} in a new species \code{ListSubset}. We just provide some proofs,
assuming most of the properties, and we add a \code{print} method for subsets:
\begin{lstlisting}
species ListSubset(Val is PrintParseSuperset, Sup is IndList(Val)) =

 inherit ExtSubset(Val) ;

 representation = Sup ;

 let ( << )(v : Val, s : Self) = Sup!mem(v, s) ;

 let empty = Sup!nil ;
 proof of mem_empty = by definition of ( << ), empty
                         property Sup!mem_nil ;

 let ( + )(s : Self, v : Val) =
  if v << s then s else Sup!cons(v, s) ;

 proof of mem_insert =
  <1>1 assume v1  v2 : Val, s : Self, hypothesis H1 : v1 << s + v2,
       prove Val!( = )(v1, v2) \/ v1 << s
       by definition of ( << ), ( + )
          property Sup!mem_cons hypothesis H1
  <1>2 assume v1 v2 : Val, s : Self,
       hypothesis H1 : Val!( = )(v1, v2),
       prove v1 << s + v2
       assumed (* TODO *)
  <1>3 assume v1 v2 : Val, s : Self, hypothesis H1 : v1 << s,
       prove v1 << s + v2
       assumed (* TODO *)
  <1>f conclude ;

 let ( - )(s : Self, v : Val) = Sup!del(s, v) ;

 proof of mem_remove = assumed (* TODO *) ;

 proof of mem_congr = assumed (* TODO *) ;

 let rec ( <: )(s1 : Self, s2 : Self) =
  if Sup!isnil(s1)
  then true
  else if Sup!head(s1) << s2
       then Sup!tail(s1) <: s2
       else false ;
 (* TODO : Termination Proof *)

 proof of mem_incl = assumed (* TODO *) ;

 let rec ( = )(s1 : Self, s2 : Self) =
  if Sup!isnil(s1)
  then Sup!isnil(s2)
  else let h1 = Sup!head(s1) and h2 = Sup!head(s2) in
       if basics#( && )(h1 << s2, h2 << s1)
       then ((s1 - h1) - h2) = ((s2 - h1) - h2)
       else false ;
 (* TODO : Termination Proof *)

 proof of eq_incl = assumed (* TODO *) ;

 let print(s : Self) =
  let x = basics#print_string("{") in
  let rec print_(s : Self)=
   if Sup!isnil(s)
   then basics#print_string("}")
   else let y = Val!print(Sup!head(s)) in print_(s - Sup!head(s))
  in print_(s) ;

end ;;
\end{lstlisting}

\subsubsection{A Subset Collection}
\label{implem_subset_collection}

We can finally implement \code{ListSubset} in a collection \code{IntSubset_Coll} as follows:
\begin{lstlisting}
collection IntSubset_Coll =

 implement ListSubset(Int_Coll, IntList_Coll) ;

end ;;
\end{lstlisting}

%----------------------------------------------------------------------------------------------
\subsection{Using Subsets}
\label{implem_program}

\subsubsection{Top Level Use}
\label{implem_program_use}

We have now an implementation of finite subsets of integers, whose concrete representation is
hidden. To use this implementation, we can for example use the \ocaml files produced by \foc
as
% Didou: proven
proved
libraries for \ocaml programs. But it is also possible to use directly the various
methods of any collection from the \foc top level, as follows:
\begin{lstlisting}
basics#print_string("\n") ;;
basics#print_string("Subsets of Integers :\n") ;;
basics#print_string("---------------------\n") ;;

let subset1 = IntSubset_Coll!empty ;;
basics#print_string("Creating empty set : ") ;;
IntSubset_Coll!print(subset1) ;;
basics#print_string("\n") ;;

let subset2 = IntSubset_Coll!( + )(subset1, Int_Coll!parse("1")) ;;
basics#print_string("Inserting 1 : ") ;;
IntSubset_Coll!print(subset2) ;;
basics#print_string("\n") ;;

let subset3 = IntSubset_Coll!( + )(subset2, Int_Coll!parse("2")) ;;
basics#print_string("Inserting 2 : ") ;;
IntSubset_Coll!print(subset3) ;;
basics#print_string("\n") ;;

let subset4 = IntSubset_Coll!( + )(subset3, Int_Coll!parse("3")) ;;
basics#print_string("Inserting 3 : ") ;;
IntSubset_Coll!print(subset4) ;;
basics#print_string("\n") ;;

let subset5 = IntSubset_Coll!( + )(subset4, Int_Coll!parse("2")) ;;
basics#print_string("Inserting 2 : ") ;;
IntSubset_Coll!print(subset5) ;;
basics#print_string("\n") ;;

let subset6 = IntSubset_Coll!( - )(subset5, Int_Coll!parse("2")) ;;
basics#print_string("Removing 2 : ") ;;
IntSubset_Coll!print(subset6) ;;
basics#print_string("\n") ;;

let subset7 = IntSubset_Coll!( - )(subset6, Int_Coll!parse("3")) ;;
basics#print_string("Removing 3 : ") ;;
IntSubset_Coll!print(subset7) ;;
basics#print_string("\n") ;;
\end{lstlisting}

\subsubsection{Producing an Executable}
\label{implem_program_compile}

Once all \foc files have been compiled by \textbf{focalizec}, there is still a final stage of
compilation required to produce an executable, an \ocaml compilation, with the following
command\footnote{Modulo the path for the \textbf{stdlib}, and the use of the correct version
of the \ocaml compiler.}:

\begin{tabular}{l}
\textbf{ocamlc -I ~/focalize/focalizec/src/stdlib/} \\
\quad\textbf{-o main}\\
\quad\textbf{ml\_builtins.cmo basics.cmo}\\
\quad\textbf{superset.cmo subset.cmo mylist.cmo main.cmo}
\end{tabular}

This produces an executable file named \textbf{main}, which can be executed:
\begin{tabular}{l}
\textbf{Creating empty set: \{\}}\\
\textbf{Inserting 1: \{1\}}\\
\textbf{Inserting 2: \{21\}}\\
\textbf{Inserting 3: \{321\}}\\
\textbf{Inserting 2: \{321\}}\\
\textbf{Removing 2: \{31\}}\\
\textbf{Removing 3: \{1\}}\\
\end{tabular}

%----------------------------------------------------------------------------------------------
%----------------------------------------------------------------------------------------------
\section{Some Remarks}
\label{remark}

We discuss here more theoretical aspects, considering extensions and alternatives to our
development and their consequences.

%----------------------------------------------------------------------------------------------
\subsection{Over-specifications}
\label{remark_over}

We have considered, at the end of Sub. \ref{specsubset_generic}, the opportunity to declare
additional methods and properties for the species \code{Subset}. Our concern was not to enrich
the specification of \code{Subset} to enforces inhabitants of the species to be finite, but to
provide more features -- recognising that it may causes unnecessary burden for the developer
if these features are not used.

For example, one could ask for the union, the intersection or the complement operations. If
the two formers seems to be straightforward, the situation of the latter if more complex.
Indeed the complement, for any subset $S$, returns the subset of elements of the superset not
belonging to $S$. This is a perfectly valid requirement, and as we will see later in this
section, we can even implement directly the species \code{Subset} enriched with this
operation. Yet the complement operation would also be part of all the other inheriting
species, including \code{ExtSubset}, \code{ListSubset}. This would clearly cause some
difficulties, preventing the parameter \code{Val} to be infinite, as in such a case for
$S$ a finite subset, the complement of $S$ would not be finite. In other words, provided an
infinite superset, complement would not be an internal operation on finite subsets.

It is not always easy to identify well in advance this type of traps, and a re-engineering of
the inheritance tree can be required when facing similar problems during development. Yet to
make such difficulties less likely to happen as well as to ease possible modifications of the
species structure, a good recommendation is to multiply the inheritance steps and branches,
introducing very gradually new methods and properties. Remember that early branching is not a 
problem in a system such as \foc, as it supports multiple inheritance: having for example
species \code{ExtSubset} and \code{SubsetComplement} (inheriting from \code{Subset} and
enriched with the complement operation) does not prevent a later definition of a species
\code{CoSubset} inheriting from both if we are able to define an appropriate concrete
representation.
% Didou
Although deep and multiple inheritance can seem difficult to understand, we must not forget
that the compiler is able to provide traces of methods provenance which greatly help
understanding in case of a complex development.

%----------------------------------------------------------------------------------------------
\subsection{Closure Reasoning's}
\label{remark_closure}

It is worth mentioning that a species defines an interface which can be completed later,
therefore closure reasoning does not apply for such an interface.

Taking the example of the species \code{Subset}, we only provide a few methods to build
subsets: the empty subset, the insertion and the removal of an element. Using only these
methods it is not possible to build an infinite subset. But that does not mean that the
species \code{Subset} only describes finite subsets -- nothing prevent the developer to
introduce through inheritance more powerful methods such as a complement operation. The fact
that the species \code{Subset} only requires methods building finite subsets actually ensures
that it is indeed valid as an ancestor for the species representing finite subsets.

For the same reason, species invariants have to be handled with care. It is possible to write
a fully defined species, with methods such that all returned values of the species are of a
specific form (for example sorted lists). It is however important to note that inheritance
may, either by creation of new functions or by redefinition of existing ones, break such
invariants. The only mechanism provided in \foc for preventing such modification is to
freeze a complete species by transforming it into a collection.

%----------------------------------------------------------------------------------------------
\subsection{Observability Considerations}
\label{remark_observability}

We consider here the creation of a new species inheriting from \code{ExtSubset},
providing a choice operator. This is a standard notion in set theory: the choice operator
returns a value belonging to a subset, provided this subset is not empty. One of the interest
of such an operator in our development is to define a method for enumerating all elements of
a subset, as indicated by the theorem \code{enumerate}:
\begin{lstlisting}
species ExtSubsetChc(Val is Superset) =

 inherit ExtSubset(Val) ;

 signature chc : Self -> Val ;
 property mem_chc : all s : Self, ~(s = empty) -> chc(s) << s ;
 theorem enumerate : all s : Self,
                     s = empty \/ s = (s - chc(s)) + chc(s)
  proof = by property mem_choice, remove_insert ;

end;;
\end{lstlisting}
The choice operator has interesting properties. Indeed, it is able to extract a value from any
non empty subset, but we have no indication about which value will be returned provided the
subset passed as a parameter contains at least two of them. In fact, whereas the choice
operator has to be implemented as a \foc function at some point in the development, it may
not be a function in the sense of set theory. Indeed, provided $S_1$ and $S_2$ two sets that
are extensionally equal, \emph{i.e.} containing the same elements, nothing enforces to have
$\fid{chc}(S_1)\!=\!\fid{chc}(S_2)$. Of course, we can prevent those strange behaviours by
adding a property requiring extensional equality to be a congruence w.r.t. \code{chc}:
\begin{lstlisting}
 property chc_congr : all s1 s2 : Self,
                      s1 = s2 -> Val!( = )(chc(s1), chc(s2)) ;
\end{lstlisting}
But this deserves additional consideration, as it may be inadequate. Indeed, any
implementation of our subset species has to be parametrised by a collection having an
interface compatible with \code{Superset}. That is, values in our subsets can be of any sort,
as soon as we can define an equivalence relation (representing a form of equality). The point
is that even if other operations are available for the provided collection, such as for
example a comparison, they will not be visible (usable) from the species \code{ExtSubsetChc};
as a consequence, it may not be possible to satisfy the property \code{chc_congr}.

Consider an implementation of \code{ExtSubsetChc} using lists: thanks to \code{=}, it is
possible to build injective lists (no repetition of elements), but without a comparison
operator there is no way to sort such lists. So for example the set $\{1,2\}$ has two possible
implementations, $\fid{cons}(1,\fid{cons}(2,\fid{nil}))$ and
$\fid{cons}(2,\fid{cons}(1,\fid{nil}))$. Whereas we can distinguish them using the structural
equality \code{basics#( = )}, there is no way to characterise a good one and a bad one. If we
now try to implement \code{chc}, we have to exhibit a function that provided a non-empty list,
returns an element of this list. Remember that beyond an equality, we have no
% Didou: operations
operation
on
values, that is for example we cannot code \code{chc} such that it always
returns the smallest element of the list. The only criteria which is available is the position
of the value in the list, and \code{chc} can for example return the head value. This
definition complies with \code{mem_chc} but not with \code{chc_congr}.

It is possible to provide an implementation of \code{ExtSubsetChoice} with the property
\code{chc_congr}, but we need to slightly
% Didou: complexify our inheritance tree.
make our inheritance tree more complex.
For example, we can
define a species \code{OrderedSuperset} inheriting from \code{Superset}, with a total order
between values, and a species \code{NormalisedSubset} as follows:
\begin{lstlisting}
species NormalisedSubset(Val is OrderedSuperset) =

 inherits ExtSubsetChc(Val) ;

 property eq_congr : all s1 s2 : Self,
                     s1 = s2 -> basics#( = )(s1, s2) ;

end ;;
\end{lstlisting}
With this approach, any collection implementing \code{OrderedSubset} of course implements
\code{ExtSubsetChc} as well, but is parametrised by a collection providing a total order;
using this order, a normal form for subsets as well as a \code{chc} operator can be defined,
ensuring that the property \code{chc_congr} is indeed satisfied, for example returning the
smallest element of the subset.

%----------------------------------------------------------------------------------------------
\subsection{Functional Representations}
\label{remark_functional}

For the sake of illustration, we consider in this subsection a totally different
implementation of subsets based on a functional representation. Our intent is not to advice to
use such type of code -- which is generally considered as inefficient -- but rather to
emphasise the freedom offered to a developer facing an abstract specification, provided this
specification is appropriate.

As mentioned in Sub. \ref{specsubset_generic}, a subset is characterised by the values it
contains, in other words a subset is characterised by its membership function. It is therefore
possible to represent a subset by this very function, \foc being a functional (higher order)
language. One of the interest of this representation is that we can represent infinite
subsets: the function deciding if a natural value is even or not defines the subset of even
values. The complement operation is also naturally supported, the difficulties identified in
Sub. \ref{remark_over} being irrelevant here. The description is straightforward:
\begin{lstlisting}
species SubsetFun(Val is Superset) =

 inherit Subset(Val) ;

 representation = Val -> basics#bool ;

 let ( << )(v : Val, s : Self) = s(v) ;

 let empty = let empty_in(v : Val) = false in empty_in ;
 proof of mem_empty = assumed (* TODO *) ;

 let ( + )(s : Self, v : Val) =
  if v << s
  then s
  else let inner(w : Val) = if Val!( = )(w, v) then true else w << s
       in inner ;
 proof of mem_insert = assumed (* TODO *) ;

 let ( - )(s : Self, v : Val) =
  if v << s
  then let inner(w : Val) = if Val!( = )(w, v) then false else w << s
       in inner
  else s;
 proof of mem_remove = assumed (* TODO *) ;

 let comp(s : Self) =
  let inner(v : Val) = basics#( ~~ )(v << s) in inner ;

end ;;
\end{lstlisting}
Being higher order this encoding is not supported by \zenon, so the properties have to be
assumed or proved in \coq; yet this is just a technical concern. More fundamentally,
\code{SubsetFun} is not an extensional representation -- it is clearly not compatible with the
interface of \code{ExtSubset}. Inclusion or equality between subsets cannot be implemented in
the general case. It is also interesting to note that the \code{choice} operator, described in
Sub. \ref{remark_observability}, is not implementable as well. Finally, a property such as
\code{mem_congr}, defined in Par. \ref{specsubset_extensional_trivial}, is true but not
provable because once we have represented a subset by a function, we cannot later analyse this
function to check its structure.

%----------------------------------------------------------------------------------------------
%----------------------------------------------------------------------------------------------
\appendix

%----------------------------------------------------------------------------------------------
%----------------------------------------------------------------------------------------------
\newpage\section{Inheritance Graph}

Inheritance in our development is as follows:
\begin{small}
\[\xymatrix{
\fid{CoList}\ar@{=>}[d] & \fid{Superset}\ar@{=>}[d] & \fid{Subset}\ar@{=>}[d]\ar@{=>}[dr] &\\
\fid{FiniteList}\ar@{=>}[d] & \fid{SupersetWitness}\ar@{=>}[d]\ar@{=>}[r] & \fid{ExtSubset}\ar@{=>}[d] & \fid{\underline{FunSubset}} \\
\fid{EnumList}\ar@{=>}[d] & \fid{PrintParseSuperset}\ar@{=>}[d]\ar@{=>}[r] & \fid{\underline{ListSubset}}\ar@{=>}[d]&\\
\fid{\underline{IndList}}\ar@{=>}[d] & \fid{\underline{Int}}\ar@{=>}[d] & \fid{\fbox{IntSubset\_Coll}} \\
\fid{\fbox{IntList\_Coll}} & \fid{\fbox{Int\_Coll}} & \\
}\]
\end{small}
If $B$ inherits of $A$ then an arrow is drawn from $A$ to $B$. Complete species are
underlined, and collections are boxed. Note that parametrisation is not represented here to
avoid over-complexification.

We have slightly amended the inheritance relation, compared to what is described in this
tutorial. For example, the species \code{ExtSubset} inherits not from the species
\code{Superset} as in Par. \ref{specsubset_extensional_equal}, but from the species
\code{SupersetWitness} introduced in Par. \ref{reflist_inductive_supersetwitness}. Indeed,
we have noted that whatever his parameters is, the species \code{Superset} is never empty,
something characteristic of the species \code{SupersetWitness}. This requires of course to add
the definition \code{let witness = empty}. Similarly, the species \code{ListSubset} now
inherits from the species \code{PrintParseSuperset}; it indeed offers a \code{print} method,
and the \code{parse} method can be implemented for example by the stub
\code{let parse(s : basics#string) = empty}.

Once fine tuned, our development is more consistent and allows for easy extensions. This is
illustrated in the final code given at annex \ref{source}. It includes a new collection
implementing subsets of subsets of integers at a minimal cost (2 collections, 6 lines of
code); executing this program, we get:

\begin{tabular}{lll}
\begin{tabular}{l}
\textbf{\underline{Subsets of Integers :}}\\
\textbf{Creating empty set : \{\}}\\
\textbf{Inserting 1 : \{1\}}\\
\textbf{Inserting 2 : \{21\}}\\
\textbf{Inserting 3 : \{321\}}\\
\textbf{Inserting 2 : \{321\}}\\
\textbf{Removing 2 : \{31\}}\\
\textbf{Removing 3 : \{1\}}\\
\end{tabular}
& \quad &
\begin{tabular}{l}
\textbf{\underline{Subsets of Subsets of Integers :}}\\
\textbf{Creating empty set : \{\}}\\
\textbf{Inserting \{\} : \{\{\}\}}\\
\textbf{Inserting \{1\} : \{\{1\}\{\}\}}\\
\textbf{Inserting \{21\} : \{\{21\}\{1\}\{\}\}}\\
\textbf{Removing \{1\} : \{\{21\}\{\}\}}\\
\textbf{Creating \{12\} : \{12\}}\\
\textbf{Inserting \{12\} : \{\{21\}\{\}\}}\\
\end{tabular}
\end{tabular}

%----------------------------------------------------------------------------------------------
%----------------------------------------------------------------------------------------------
\newpage\section{Full Sources}
\label{source}

\lstset{basicstyle=\scriptsize\mdseries\itshape\ttfamily,
        frame=single} %numbers=left, numberstyle=\tiny,

\subsection{superset.fcl}
\label{source_superset}

\begin{lstlisting}
use "basics" ;;

species Superset =
 signature ( = ) : Self -> Self -> basics#bool ;
 property eq_refl : all x : Self, x = x ;
 property eq_symm : all x y : Self, x = y -> y = x ;
 property eq_tran : all x y z : Self, x = y -> y = z -> x = z ;
end ;;

species SupersetWitness  =
 inherit Superset ;
 signature witness : Self ;
end ;;

species PrintParseSuperset =
 inherit SupersetWitness ;
 signature print : Self -> basics#unit ;
 signature parse : basics#string -> Self ;
end ;;
\end{lstlisting}

\subsection{subset.fcl}
\label{source_subset}

\begin{lstlisting}
use "basics" ;;
open "superset" ;;

species Subset(Val is Superset) =
 signature ( << ) : Val -> Self -> basics#bool ;
 signature empty : Self ;
 property mem_empty : all v : Val, ~(v << empty) ;
 signature ( + ) : Self -> Val -> Self ;
 property mem_insert : all v1 v2 : Val, all s : Self,
                       v1 << s + v2 <-> (Val!( = )(v1, v2) \/ v1 << s) ;
 signature ( - ) : Self -> Val -> Self ;
 property mem_remove : all v1 v2 : Val, all s : Self,
                       v1 << s - v2 <-> (~(Val!( = )(v1, v2)) /\ v1 << s) ;
end ;;

species ExtSubset(Val is Superset) =
 inherit SupersetWitness, Subset(Val);
 property mem_congr : all v1 v2 : Val, Val!( = )(v1, v2) ->
                      (all s : Self, (v1 << s) <-> (v2 << s)) ;
 signature ( <: ) : Self -> Self -> basics#bool ;
 property mem_incl : all s1 s2 : Self,
                     s1 <: s2 <-> all v : Val, v << s1 -> v << s2 ;
 theorem incl_refl : all s : Self, s <: s
  proof = by property mem_incl ;
 theorem incl_tran : all s1 s2 s3 : Self, s1 <: s2 -> s2 <: s3 -> s1 <: s3
  proof = by property mem_incl ;
 property eq_incl : all s1 s2 : Self, s1 = s2 <-> s1 <: s2 /\ s2 <: s1 ;
 proof of eq_refl = by property eq_incl, incl_refl ;
 proof of eq_symm = by property eq_incl ;
 proof of eq_tran = by property eq_incl, incl_tran ;
 theorem mem_eq : all s1 s2 : Self, s1 = s2 <->
                  (all v : Val, v << s1 <-> v << s2)
  proof = by property eq_incl, mem_incl ;
 theorem incl_empty : all s : Self, empty <: s
  proof = by property mem_incl, mem_empty ;
 theorem incl_insert : all s : Self, all v : Val, s <: s + v
  proof = by property mem_insert, mem_incl ;
 theorem incl_remove : all s : Self, all v : Val, s - v <: s
  proof = by property mem_remove, mem_incl ;
 theorem incl_insert_mem : all s : Self, all v : Val,
                           v << s -> s + v <: s
  proof = by property mem_congr, mem_insert, mem_incl ;
 theorem incl_remove_mem : all s : Self, all v : Val,
                           ~(v << s) -> s <: s - v
  proof = <1>1 assume s : Self, v : Val, hypothesis Hv : ~(v << s),
               prove s <: s - v
           <2>1 assume w : Val, hypothesis Hw : w << s,
                prove w << s - v
            <3>1 prove ~(Val!( = )(w, v)) /\ w << s
             <4>1 prove ~(Val!( = )(w, v))
                  by property mem_congr hypothesis Hv, Hw
             <4>2 prove w << s
                  by hypothesis Hw
             <4>f conclude
            <3>f qed by property mem_remove step <3>1
           <2>f qed by property mem_incl step <2>1
          <1>f conclude ;
 theorem remove_insert : all s : Self, all v : Val,
                         v << s -> s = (s - v) + v
  proof = <1>1 assume s : Self, v : Val, hypothesis Hv : v << s,
               prove s = (s - v) + v
           <2>1 assume w : Val, hypothesis Hw : w << s,
                prove w << (s - v) + v
            <3>1 prove ~(Val!( = )(w, v)) -> w << s - v
                 by property mem_remove hypothesis Hw
            <3>f qed by property mem_insert step <3>1
           <2>2 assume w : Val, hypothesis Hw : w << (s - v) + v,
                prove w << s
            <3>1 prove Val!( = )(w, v) -> w << s
                 by property mem_congr hypothesis Hv
            <3>2 prove w << s - v -> w << s
                 by property mem_remove
            <3>f qed by property mem_insert hypothesis Hw step <3>1, <3>2
           <2>f qed by property eq_incl, mem_incl step <2>1, <2>2
          <1>f qed by step <1>1 ;
 let witness = empty;
end ;;

species SubsetFun(Val is Superset) =
 inherit Subset(Val) ;
 representation = Val -> basics#bool ;
 let ( << )(v : Val, s : Self) = s(v) ;
 let empty = let empty_in(v : Val) = false in empty_in ;
 proof of mem_empty = assumed (* TODO *) ;
 let ( + )(s : Self, v : Val) =
  if v << s
  then s
  else let inner(w : Val) = if Val!( = )(w, v) then true else w << s
       in inner ;
 proof of mem_insert = assumed (* TODO *) ;
 let ( - )(s : Self, v : Val) =
  if v << s
  then let inner(w : Val) = if Val!( = )(w, v) then false else w << s
       in inner
  else s;
 proof of mem_remove = assumed (* TODO *) ;
 let comp(s : Self) = let inner(v : Val) = basics#( ~~ )(v << s) in inner ;
end ;;
\end{lstlisting}

\subsection{mylist.fcl}
\label{source_mylist}

\begin{lstlisting}
use "basics" ;;
open "superset" ;;

species CoList(Val is Superset) =
 signature nil : Self ;
 signature cons : Val -> Self -> Self ;
 signature isnil : Self -> basics#bool ;
 signature head : Self -> Val ;
 signature tail : Self -> Self ;
 property isnil_nil : all l : Self, isnil(l) <-> basics#( = )(l, nil) ;
 property isnil_cons : all v : Val, all l : Self, ~ isnil(cons(v, l)) ;
 property head_cons : all v : Val, all l : Self,
                      basics#( = )(head(cons(v, l)), v) ;
 property tail_cons : all v : Val, all l : Self,
                      basics#( = )(tail(cons(v, l)), l) ;
 property list_dec : all l : Self,
                     ~ isnil(l) -> basics#( = )(l, cons(head(l), tail(l))) ;
 theorem cons_left : all l1 l2 : Self, all v1 v2 : Val,
                     basics#( = )(cons(v1, l1), cons(v2, l2)) ->
                     basics#( = )(v1, v2)
  proof = <1>1 assume t1 t2 : Self, h1 h2 : Val,
               hypothesis Heq : basics#( = )(cons(h1, t1), cons(h2, t2)),
               prove basics#( = )(h1, h2)
           <2>1 prove basics#( = )(head(cons(h1, t1)), head(cons(h2, t2)))
                by hypothesis Heq
           <2>2 prove basics#( = )(h1, head(cons(h2, t2)))
                by step <2>1 property head_cons
           <2>f qed by step <2>2 property head_cons
          <1>f qed by step <1>1 ;
 theorem cons_right : all l1 l2 : Self, all v1 v2 : Val,
                      basics#( = )(cons(v1, l1), cons(v2, l2)) ->
                      basics#( = )(l1, l2)
  proof = <1>1 assume l1 l2 : Self, v1 v2 : Val,
               hypothesis Heq : basics#( = )(cons(v1, l1), cons(v2, l2)),
               prove basics#( = )(l1, l2)
           <2>1 prove basics#( = )(tail(cons(v1, l1)), tail(cons(v2, l2)))
                by hypothesis Heq
           <2>2 prove basics#( = )(l1, tail(cons(v2, l2)))
                by step <2>1 property tail_cons
           <2>f qed by step <2>2 property tail_cons
          <1>f qed by step <1>1 ;
end ;;

species FiniteList(Val is Superset) =
 inherit CoList(Val) ;
 property finite :
  all f : (Self -> basics#bool),
  f(nil) ->
  (all l : Self, f(l) -> all v : Val, f(cons(v, l))) ->
  all l : Self, f(l) ;
 signature mem : Val -> Self -> basics#bool ;
 property mem_nil : all v : Val, ~ mem(v, nil) ;
 property mem_cons : all v h : Val, all t : Self,
                     mem(v, cons(h, t)) <-> Val!( = )(v, h) \/ mem(v, t) ;
 signature del : Self -> Val -> Self ;
 property mem_del : all v w : Val, all l : Self,
                    mem(v, del(l, w)) <-> ~ Val!( = )(v, w) /\ mem(v, l) ;
end ;;

species EnumList(Val is Superset) =
 inherit FiniteList(Val) ;
 let rec mem(v : Val, l : Self) =
  if isnil(l)
  then false
  else (if Val!( = )(v, head(l)) then true else mem(v, tail(l))) ;
 (* TODO : Termination Proof *)
 proof of mem_nil = by definition of mem property isnil_nil ;
 proof of mem_cons =
  <1>1 assume v h : Val, t : Self, hypothesis H : Val!( = )(v, h),
       prove mem(v, cons(h, t))
   <2>1 prove Val!( = )(v, head(cons(h, t)))
        by property head_cons hypothesis H
   <2>f qed by definition of mem property isnil_cons step <2>1
  <1>2 assume v h : Val, t : Self, hypothesis H : mem(v, t),
       prove mem(v, cons(h, t))
   <2>1 prove mem(v, tail(cons(h, t)))
        by property tail_cons hypothesis H
   <2>f qed by step <2>1 definition of mem property isnil_cons
  <1>3 assume v h : Val, t : Self, hypothesis H : mem(v, cons(h, t)),
       prove Val!( = )(v, h) \/ mem(v, t)
   <2>1 hypothesis H2 : ~ Val!( = )(v, h),
        prove mem(v, tail(cons(h, t)))
        by hypothesis H, H2 definition of mem property isnil_cons, head_cons
   <2>f qed by step <2>1 property tail_cons
  <1>f conclude ;
 let rec del(l : Self, v : Val) =
  if isnil(l)
  then nil
  else if Val!( = )(v, head(l))
       then del(tail(l), v)
       else cons(head(l) , del(tail(l), v)) ;
 (* TODO : Termination Proof *)
 logical let mdm_(v : Val, w : Val, l : Self) =
  mem(v, del(l, w)) -> mem(v, l) ;
 theorem mem_del_mem : all v w : Val, all l : Self, mdm_(v, w, l)
  proof = <1>1 assume v w : Val,
               prove mdm_(v, w, nil)
               by definition of mdm_, del property isnil_nil
          <1>2 assume v w : Val, t : Self, hypothesis Hind : mdm_(v, w, t),
               assume h : Val,
               prove mdm_(v, w, cons(h, t))
           <2>1 hypothesis H1 : mem(v, del(cons(h, t), w)),
                prove mem(v, cons(h, t))
            <3>1 hypothesis H2 : Val!( = )(v, h),
                 prove mem(v, cons(h, t))
                 by property mem_cons hypothesis H2
            <3>2 hypothesis H2 : ~ Val!( = )(v, h), H3 : Val!( = )(w, h),
                 prove mem(v, cons(h, t))
             <4>1 prove ~ isnil(cons(h, t))
                  by property isnil_cons
             <4>2 prove Val!( = )(w, head(cons(h, t)))
                  by property head_cons hypothesis H3
             <4>3 prove mem(v, del(tail(cons(h, t)), w))
                  by definition of del step <4>1, <4>2 hypothesis H1
             <4>4 prove mem(v, del(t, w))
                  by step <4>3 property tail_cons
             <4>f qed by step <4>4 hypothesis Hind definition of mdm_
                         property mem_cons
            <3>3 hypothesis H2 : ~ Val!( = )(v, h), H3 : ~ Val!( = )(w, h),
                 prove mem(v, cons(h, t))
             <4>1 prove ~ isnil(cons(h, t))
                  by property isnil_cons
             <4>2 prove ~ Val!( = )(w, head(cons(h, t)))
                  by property head_cons hypothesis H3
             <4>3 prove mem(v, cons(head(cons(h, t)),
                                    del(tail(cons(h, t)), w)))
                  by definition of del step <4>1, <4>2 hypothesis H1
             <4>4 prove mem(v, cons(h, del(t, w)))
                  by step <4>3 property head_cons, tail_cons
             <4>5 prove mem(v, del(t, w))
                  by step <4>4 property mem_cons hypothesis H2
             <4>f qed by step <4>5 hypothesis Hind definition of mdm_
                         property mem_cons
            <3>f conclude
           <2>f qed by step <2>1 definition of mdm_
          <1>3 assume v w : Val,
               prove mdm_(v, w, nil)->
                     (all t : Self, mdm_(v, w, t) ->
                      all h : Val, mdm_(v, w, cons(h, t))) ->
                     all l : Self, mdm_(v, w, l)
               assumed (* Standard induction principle *)
          <1>f conclude ;
 logical let mde_(v : Val, w : Val, l : Self) =
  mem(v, del(l, w)) -> ~ Val!( = )(v, w) ;
 theorem mem_del_eq : all v w : Val, all l : Self, mde_(v, w, l)
 proof = <1>1 assume v w : Val,
              prove mde_(v, w, nil)
              by definition of mde_, del property isnil_nil, mem_nil
         <1>2 assume v w : Val, t : Self, hypothesis Hind : mde_(v, w, t),
              assume h : Val,
              prove mde_(v, w, cons(h, t))
          <2>1 hypothesis H1 : mem(v, del(cons(h, t), w)),
               prove ~ Val!( = )(v, w)
           <3>1 hypothesis H2 : Val!( = )(w, h),
                prove ~ Val!( = )(v, w)
            <4>1 prove mem(v, del(t, w))
             <5>1 prove ~ isnil(cons(h, t))
                 by property isnil_cons
             <5>2 prove Val!( = )(w, head(cons(h, t)))
                  by property head_cons hypothesis H2
             <5>3 prove mem(v, del(tail(cons(h, t)), w))
                  by hypothesis H1 definition of del step <5>1, <5>2
             <5>f qed by step <5>3 property tail_cons
            <4>f qed by step <4>1 hypothesis Hind definition of mde_
           <3>2 hypothesis H2 : ~ Val!( = )(w, h),
                prove ~ Val!( = )(v, w)
            <4>1 hypothesis H3 : Val!( = )(v, h),
                 prove ~ Val!( = )(v, w)
                 by hypothesis H2, H3 property Val!eq_symm, Val!eq_tran
            <4>2 hypothesis H3 : ~ Val!( = )(v, h),
                 prove ~ Val!( = )(v, w)
             <5>1 prove ~ isnil(cons(h, t))
                  by property isnil_cons
             <5>2 prove ~ Val!( = )(w, head(cons(h, t)))
                  by property head_cons hypothesis H2
             <5>3 prove mem(v, cons(head(cons(h, t)),
                                    del(tail(cons(h, t)), w)))
                  by hypothesis H1 definition of del step <5>1, <5>2
             <5>4 prove mem(v, cons(h, del(t, w)))
                  by step <5>3 property head_cons, tail_cons
             <5>5 prove mem(v, del(t, w))
                  by step <5>4 property mem_cons hypothesis H3
             <5>f qed by step <5>5 hypothesis Hind definition of mde_
            <4>f conclude
           <3>f conclude
          <2>f qed by definition of mde_ step <2>1
         <1>3 assume v w : Val,
              prove mde_(v, w, nil) ->
                    (all t : Self, mde_(v, w, t) ->
                     all h : Val, mde_(v, w, cons(h, t))) ->
                    all l : Self, mde_(v, w, l)
              assumed (* Standard induction principle *)
         <1>f conclude ;
 theorem mem_del_inv : all v w : Val, ~ Val!( = )(v, w) ->
                       all l : Self, mem(v, l) -> mem(v, del(l, w))
  proof = assumed (* TODO *) ;
 proof of mem_del = by property mem_del_mem, mem_del_eq, mem_del_inv
                       definition of mdm_, mde_ ;
end ;;

type mylist('a) = | FNil | FCons ('a, mylist('a)) ;;

species IndList(Val is SupersetWitness) =
 inherit FiniteList(Val) ;
 representation = mylist(Val) ;
 let nil = #FNil ;
 let cons(h : Val, t : Self) = #FCons(h, t) ;
 let isnil(l : Self) = match l with | #FNil -> true | _ -> false ;
 let head(l : Self) = match l with | #FCons(h, _) -> h | _ -> Val!witness ;
 let tail(l : Self) = match l with | #FCons(_, t) -> t | _ -> #FNil ;
 proof of finite =  coq proof definition of nil, cons
                    {* Proof.
                        intros.
                        unfold abst_cons, abst_nil, cons, nil in *.
                        induction l.
                         trivial.
                        apply H0; apply IHl.
                       Qed. *} ;
 proof of isnil_nil = assumed (* TODO *) ;
 proof of isnil_cons = assumed (* TODO *) ;
 proof of head_cons = assumed (* TODO *) ;
 proof of tail_cons = assumed (* TODO *) ;
 proof of list_dec = assumed (* TODO *) ;
 let mem(v : Val, l : Self) =
  let recstruct mem_(l : Self) =
   match l with
   | #FNil -> false
   | #FCons(h, t) -> if Val!( = )(v, h) then true else mem_(t)
  in mem_(l) ;
 proof of mem_nil = assumed (* TODO *) ;
 proof of mem_cons = assumed (* TODO *) ;
 let recstruct del(l : Self, v : Val) =
  match l with
  | #FNil -> #FNil
  | #FCons(h, t) -> if Val!( = )(v, h)
                    then del(t, v)
                    else #FCons(h, del(t, v)) ;
 proof of mem_del =  assumed (* TODO *) ;
end ;;
\end{lstlisting}

\subsection{main.fcl}
\label{source_main}

\begin{lstlisting}
use "basics" ;;
open "superset" ;;
open "subset" ;;
open "mylist" ;;

species Int =
 inherit PrintParseSuperset ;
 representation = basics#int ;
 let ( = )(x : Self, y : Self) = basics#( = )(x, y) ;
 proof of eq_refl = by definition of ( = ) ;
 proof of eq_symm = by definition of ( = ) ;
 proof of eq_tran = by definition of ( = ) ;
 let witness = 0 ;
 let print(s : Self) = basics#print_int(s) ;
 let parse(s : basics#string) = basics#int_of_string(s) ;
end ;;

collection Int_Coll = implement Int; end ;;

collection IntList_Coll =  implement IndList(Int_Coll) ; end ;;

species ListSubset(Val is PrintParseSuperset, Sup is IndList(Val)) =
 inherit ExtSubset(Val), PrintParseSuperset ;
 representation = Sup ;
 let ( << )(v : Val, s : Self) = Sup!mem(v, s) ;
 let empty = Sup!nil ;
 proof of mem_empty = by definition of ( << ), empty property Sup!mem_nil ;
 let ( + )(s : Self, v : Val) = if v << s then s else Sup!cons(v, s) ;
 proof of mem_insert =
  <1>1 assume v1  v2 : Val, s : Self, hypothesis H1 : v1 << s + v2,
       prove Val!( = )(v1, v2) \/ v1 << s
       by definition of ( << ), ( + )
          property Sup!mem_cons hypothesis H1
  <1>2 assume v1 v2 : Val, s : Self, hypothesis H1 : Val!( = )(v1, v2),
       prove v1 << s + v2
       assumed (* TODO *)
  <1>3 assume v1 v2 : Val, s : Self, hypothesis H1 : v1 << s,
       prove v1 << s + v2
       assumed (* TODO *)
  <1>f conclude ;
 let ( - )(s : Self, v : Val) = Sup!del(s, v) ;
 proof of mem_remove = assumed (* TODO *) ;
 proof of mem_congr = assumed (* TODO *) ;
 let rec ( <: )(s1 : Self, s2 : Self) =
  if Sup!isnil(s1)
  then true
  else if Sup!head(s1) << s2
       then Sup!tail(s1) <: s2
       else false ;
 (* TODO : Termination Proof *)
 proof of mem_incl = assumed (* TODO *) ;
 let rec ( = )(s1 : Self, s2 : Self) =
  if Sup!isnil(s1)
  then Sup!isnil(s2)
  else let h1 = Sup!head(s1) and h2 = Sup!head(s2) in
       if basics#( && )(h1 << s2, h2 << s1)
       then ((s1 - h1) - h2) = ((s2 - h1) - h2)
       else false ;
 (* TODO : Termination Proof *)
 proof of eq_incl = assumed (* TODO *) ;
 let print(s : Self) =
  let x = basics#print_string("{") in
  let rec print_(s : Self)=
   if Sup!isnil(s)
   then basics#print_string("}")
   else let y = Val!print(Sup!head(s)) in print_(s - Sup!head(s))
  in print_(s) ;
 let parse(s : basics#string) = empty;
 let witness = empty ;
end ;;

collection IntSubset_Coll =
 implement ListSubset(Int_Coll, IntList_Coll) ;
end ;;

collection IntSubsetList_Coll = implement IndList(IntSubset_Coll) ; end ;;

collection IntSubset2_Coll =
 implement ListSubset(IntSubset_Coll, IntSubsetList_Coll) ;
end ;;

basics#print_string("\n") ;;
basics#print_string("Subsets of Integers :\n") ;;
basics#print_string("---------------------\n") ;;

let subset1 = IntSubset_Coll!empty ;;
basics#print_string("Creating empty set : ") ;;
IntSubset_Coll!print(subset1) ;;
basics#print_string("\n") ;;

let subset2 = IntSubset_Coll!( + )(subset1, Int_Coll!parse("1")) ;;
basics#print_string("Inserting 1 : ") ;;
IntSubset_Coll!print(subset2) ;;
basics#print_string("\n") ;;

let subset3 = IntSubset_Coll!( + )(subset2, Int_Coll!parse("2")) ;;
basics#print_string("Inserting 2 : ") ;;
IntSubset_Coll!print(subset3) ;;
basics#print_string("\n") ;;

let subset4 = IntSubset_Coll!( + )(subset3, Int_Coll!parse("3")) ;;
basics#print_string("Inserting 3 : ") ;;
IntSubset_Coll!print(subset4) ;;
basics#print_string("\n") ;;

let subset5 = IntSubset_Coll!( + )(subset4, Int_Coll!parse("2")) ;;
basics#print_string("Inserting 2 : ") ;;
IntSubset_Coll!print(subset5) ;;
basics#print_string("\n") ;;

let subset6 = IntSubset_Coll!( - )(subset5, Int_Coll!parse("2")) ;;
basics#print_string("Removing 2 : ") ;;
IntSubset_Coll!print(subset6) ;;
basics#print_string("\n") ;;

let subset7 = IntSubset_Coll!( - )(subset6, Int_Coll!parse("3")) ;;
basics#print_string("Removing 3 : ") ;;
IntSubset_Coll!print(subset7) ;;
basics#print_string("\n") ;;

basics#print_string("\n") ;;
basics#print_string("Subsets of Subsets of Integers :\n") ;;
basics#print_string("--------------------------------\n") ;;

let power1 = IntSubset2_Coll!empty ;;
basics#print_string("Creating empty set : ") ;;
IntSubset2_Coll!print(power1) ;;
basics#print_string("\n") ;;

let power2 = IntSubset2_Coll!( + )(power1, subset1) ;;
basics#print_string("Inserting {} : ") ;;
IntSubset2_Coll!print(power2) ;;
basics#print_string("\n") ;;

let power3 = IntSubset2_Coll!( + )(power2, subset2) ;;
basics#print_string("Inserting {1} : ") ;;
IntSubset2_Coll!print(power3) ;;
basics#print_string("\n") ;;

let power4 = IntSubset2_Coll!( + )(power3, subset3) ;;
basics#print_string("Inserting {21} : ") ;;
IntSubset2_Coll!print(power4) ;;
basics#print_string("\n") ;;

let power5 = IntSubset2_Coll!( - )(power4, subset2) ;;
basics#print_string("Removing {1} : ") ;;
IntSubset2_Coll!print(power5) ;;
basics#print_string("\n") ;;

let subset8 =
 IntSubset_Coll!( + )
  (IntSubset_Coll!( + )(IntSubset_Coll!empty, Int_Coll!parse("2")),
   Int_Coll!parse("1"));;
basics#print_string("Creating {12} : ") ;;
IntSubset_Coll!print(subset8) ;;
basics#print_string("\n") ;;

let power6 = IntSubset2_Coll!( + )(power5, subset8) ;;
basics#print_string("Inserting {12} : ") ;;
IntSubset2_Coll!print(power6) ;;
basics#print_string("\n") ;;

basics#print_string("\n") ;;
\end{lstlisting}

%----------------------------------------------------------------------------------------------
%----------------------------------------------------------------------------------------------

\end{document}

% LocalWords:  supersets superset boolean bool focalizec fo zv zvtov ocamlc pfc
% LocalWords:  eq fcl refl symm tran mem datatype ExtSubset subgoals subgoal yn
% LocalWords:  congr qed Hw Hv subproofs TODO reusability mylist CoList isnil
% LocalWords:  dec Heq FiniteList del EnumList InductiveList recstruct mdm FNil
% LocalWords:  SupersetWitness FCons surjectivity injectivity IndList Coq coq
% LocalWords:  abst IHl PrintParseSuperset PrintParseSpecies IntList ListSubset
% LocalWords:  IntSubset stdlib builtins cmo SubsetComplement CoSubset chc
% LocalWords:  ExtSubsetChc injective ExtSubsetChoice OrderedSuperset SubsetFun
% LocalWords:  NormalisedSubset OrderedSubset
