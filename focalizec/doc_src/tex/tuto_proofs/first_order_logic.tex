\section{Dealing with first order logic theorems}

The first part of this tutorial intends teaching how to use \zenon\ on
simple boolean properties. The aim is to show how \zenon\ can help
making whole part of basic inference steps one usually makes explicit
in tools like \coq. It must clearly understood that \zenon\ is not a proof
checker but a theorem prover. Obviously it will not au\-to\-ma\-ti\-cal\-ly
demonstrate itself any property. However, combined with the \focal\
proof language, it will automate tedious combinations of ``sub-lemmas''
one usually ``think intuitively feasible''.

In the following examples, we won't create species. Instead, to get
rid of com\-ple\-xi\-ty induced by \focal\ structures, we will state and
prove theorems at ``top-level''.


\subsection{A so simple property: fully automated proof}
Let's first address the following property:
$\forall a, b : {\tt boolean}, a \Rightarrow b \Rightarrow a$.
We write this in \focal\ as follows, in a source file
\textbf{focalizec ex\_implications.fcl}:

{\scriptsize
\lstinputlisting[caption=ex\_implications.fcl]{ex_implications.fcl}}

Here, we stated a property and directly asked \zenon\ to find a proof
without any direction. \zenon\ then uses its internal knowledge of
first order logic to solve the goal. At this stage it is possible to
compile the program using the command \textbf{focalizec ex\_implications.fcl}
and get few messages with not error:

{\scriptsize
\begin{verbatim}
Invoking ocamlc...
>> ocamlc -I /usr/local/lib/focalize -c ex_implications.ml
Invoking zvtov...
>> zvtov -zenon /usr/local/bin/zenon -new ex_implications.zv
Invoking coqc...
>> coqc -I /usr/local/lib/focalize -I /usr/local/lib/zenon ex_implications.v
\end{verbatim}}

Compilation, code generation and \coq\ verification were
successful. Let's investigate the tree of this proof, where we wrote
hypotheses (context) in \green{green} and goals in \blue{blue}. The
intuition is that we lift the two implications as hypotheses and from
the hypothesis \lstinline"a" we trivially prove our goal.

$$
\inferrule*[Right=($\Rightarrow$-intro)]
   {\inferrule*[Right=($\Rightarrow$-intro)] {\green{a, b} \vdash \blue{a}}
   { \green{a} \vdash  \blue{b \Rightarrow a}}}
   { \vdash \blue{a \Rightarrow b \Rightarrow a}}
$$

In \focal, this {\em proof} is achieved as a simple hierarchical sequence of
in\-ter\-me\-dia\-te {\em steps}. A proof step starts with a {\em proof bullet}, which
gives its nesting level. The top-level of a proof is 0. In a
{\em compound proof}, the steps are at level one plus the level of the proof
itself.

{\scriptsize
\lstinputlisting[caption=ex\_implications.fcl (2)]{ex_implications2.fcl}}

Here, the {\em steps} \lstinline"<1>1" and \lstinline"<1>2" are at level 1
and form the compound proof of the top-level theorem.  Step \lstinline"<1>1"
also has a compound proof (whose goal is \lstinline"b -> a"), made of
steps \lstinline"<2>1" 
and \lstinline"<2>2".  These latter are at level 2 (one more than the level of
their enclosing step).

\medskip
After the proof bullet comes the {\em statement} of the step,
introduced by the keyword \lstinline{prove}.  This is the
proposition that is asserted and proved by {\em this step}.  At the end of
this step's proof, it becomes available as a {\em fact} for the next steps
of this proof and deeper levels sub-goals.  In our example, step
\lstinline"<2>1" is available in the proof of \lstinline"<2>2", and
\lstinline"<1>1" is available in the proof of \lstinline"<1>2".  Note
that \lstinline"<2>1" is {\bf not} available in the proof of \lstinline"<1>2"
since \lstinline"<1>2" is located at a strictly lower nesting level
than \lstinline"<2>1".

\medskip
After the statement is the {\em proof of the step}. This is where
either you ask \zenon\ to do the proof from {\em facts} (hints) you
give it, or you decide to split the proof in ``sub-steps'' on which
\zenon\ will finally by called and that will serve (still with \zenon)
to finally prove the current goal by combining these ``sub-steps''
lemmas.

For instance, the proof of the whole theorem (which is itself a
statement) is not directly asked to \zenon\ as it was the case in the
first example (with the simple fact \lstinline"proof = conclude").  It has been decided to
split it in one sub-goal \lstinline"<1>1": prove \lstinline"b -> a". The same
structure is applied to this goal which is split in the sub-goals
\lstinline"<2>1" and \lstinline"<2>2".

\medskip
In the proof of sub-goal \lstinline"<2>1", appears a {\em fact}:
\lstinline"by hypothesis h1". Here \zenon\ is asked to find a proof of
the current goal, using hints it is provided with. You are responsible
in giving \zenon\ facts it will need to finally find a proof. It will
combine them in accordance with logical rules, but if it is missing
material for the proof, it will never succeed. Here, \zenon\ is told
that it should be able to prove the goal only using the hypothesis
\lstinline"h1" we introduced. It will obviously succeed since the goal
is exactly the hypothesis \lstinline"h1".

From the proof of \lstinline"a", i.e. the step \lstinline"<2>1", we can conclude the
en\-clo\-sing goal (\lstinline"prove b -> a"). This is done by the \lstinline"qed"
step whose aim is to close the enclosing proof by mean of the provided
facts (here the intermediate lemma stated by the step \lstinline"<2>1").

Finally, coming back to the remaining part of the proof, i.e. the
previous nesting level, we want to solve its goal (which is the whole
theorem). In the same manner, from the step \lstinline"<1>1" that proved
that \lstinline"b -> a" under the hypothesis \lstinline"a", we can conclude. We
then invoke the statement \lstinline"conclude" which is equivalent to
tell \zenon\ to use as facts, all the available proof steps in the
scope. Hence this is equivalent here to write
\lstinline"qed by step <1>1". Note that with \lstinline"conclude" the
\lstinline"qed" is optional since this statement implicitly marks the
end of the proof related to the current goal.

Having a look backward to compare our \coq\ and \focal\ proofs, we can
clearly see the same processing order. We introduced the implications as
hypotheses, with \verb"intro" in \coq\ and \lstinline"assume" in
\focal. Then we used the hypothesis \lstinline"h1", with \verb"exact h1" in 
\coq\ and \lstinline"by hypothesis h1" in \focal. The implicit nested
structure of the proof is made explicit in \focal.



\subsection{Still simple but easier with \zenon}
Let's continue with simple first-order logic properties and let's try
to demonstrate the following statement:
$\forall a, b : {\tt boolean}, (a \logand b) \Rightarrow (a \logor b)$.
We can easily build the tree of this proof as follows:

$$
\inferrule* [Right=($\Rightarrow$-intro)]
   {\inferrule* [Right=($\logand$-intro)] {\green{a \logand b} \vdash \blue{a \logand b}}
      {\inferrule* [Right=($\logor$-elim)] {\green{a \logand b} \vdash \blue{b}}
     { \green{a \logand b} \vdash \blue{a \logor b}}}}
   { \vdash \blue{a \logand b \Rightarrow a \logor b}}
$$

Note that we chose to prove \verb"b" but we could have chose to prove
\verb"a" instead. We now address this property in \focal\ again,
making explicit all the steps of the proof we did.

{\scriptsize
\lstinputlisting[caption=and\_or.fcl]{and_or.fcl}}

We can see that step \lstinline"<2>1" is directly proven by hypothesis
\verb"h1", without making explicit the $\logand${\tt -ELIM} rule of
the proof tree. Here we rely on \zenon\ to find the proof of this
step. Again, this property is simple enough to get it proved in a
fully automated way by \zenon.

{\scriptsize
\begin{lstlisting}
open "basics" ;;

theorem and_or : all a b : bool, (a /\ b) -> (a \/ b)
proof = conclude ;;
\end{lstlisting}}
