\section{Forewords}

\subsection{Content}
This document is a tutorial for \focal, describing how to
develop proofs of properties using \zenon. Differently from other
tutorials, we won't focalize on mathematical developments, preferring
to show the language in action on programs closer to what ``usual
programers'' develop in the ``everyday life''.

To get in touch with basic \zenon\ capabilities, we will first address
very simple first order logic properties with their proofs. This will
allow introducing the notion of hierarchical proofs. Then, we will
program a simple 3 traffic signals controller to apply these skills on
properties directly related to the program we will write. The aim is to
show what are the kind of properties one may want to state and how
their proofs get related to the types and functions definition of a
program.

\subsection{Notations and Recommandations}
In the rest of this tutorial, pieces of \focal\ code will be presented in
frames, as in this example:

{\scriptsize
\begin{lstlisting}
use "basics" ;;
species Controller =
\end{lstlisting}}

When introduced in the running text, \focal\ keywords will appear in a
special font like \lstinline"property". Terms representing specific
concepts of \focal\ are introduced using an emphasized font, for
example \emph{collection}. Finally, commands and file names are in
bold font, for example \textbf{focalizec fo\_logic.fcl}.
