\documentclass[11pt,a4paper,twoside,onecolumn,fullpage]{article}
\pagenumbering{arabic}
\pagestyle{headings}

\usepackage{amsmath}
\usepackage{amsfonts}
\usepackage{listings}
\lstset{basicstyle=\footnotesize\mdseries\itshape\ttfamily,
        frame=single} %numbers=left, numberstyle=\tiny,
\usepackage{mathpartir}
\usepackage{times}   %% Don't know why, but required to have listings beautified.
\usepackage{color}
\usepackage{url}

\newcommand{\ocaml}{{\sf OCaml}}
\newcommand{\focal}{{\sf Fo\-Ca\-Li\-ze}}
\newcommand{\focalizec}{{\sf focalizec}}
\newcommand{\zenon}{{\sf Zenon}}
\newcommand{\coq}{{\sf Coq}}

\newcommand{\logand}{\wedge}
\newcommand{\logor}{\vee}

\definecolor{green}{rgb}{0.0,0.6,0.2}
\definecolor{brown}{rgb}{0.75,0.37,0.12}
\definecolor{orange}{rgb}{1.0,0.4,0.2}
\definecolor{darkblue}{rgb}{0.28,0.28,0.73}
\newcommand{\green}[1]{{\color{green}{#1}}}
\newcommand{\red}[1]{{\color{red}{#1}}}
\newcommand{\blue}[1]{{\color{blue}{#1}}}
\newcommand{\orange}[1]{{\color{orange}{#1}}}
\newcommand{\brown}[1]{{\color{brown}{#1}}}
\newcommand{\magenta}[1]{{\color{magenta}{#1}}}
\newcommand{\cyan}[1]{{\color{cyan}{#1}}}
\newcommand{\darkblue}[1]{{\color{darkblue}{#1}}}

\lstdefinelanguage{FoCaLiZe}
  {morekeywords={alias, all, and, as, assume, assumed, begin, by, caml,
      collection, coq, coq_require, definition, else, end,
      ex, external, false, function, hypothesis, if, in,
      inherit, internal, implement, is, let, lexicographic,
      local, logical, match, measure, not, notation, of, open,
      on, or, order, proof, prop, property, prove, qed, rec,
      representation, Self, signature, species, step,
      structural, termination, then, theorem, true, type, use,
      with, conclude},
    otherkeywords={->, :, /\\, \\/, =>, \~, \#, ;, \&, \|,=,<>},
    sensitive=false,
    morecomment=[n]{(*}{*)},  %% Autoriser les nested comments
    morecomment=[n]{(**}{*)}, %% Autoriser les nested comments
    morestring=[b]",
  }

\newcommand{\setlangfocalize}{
\lstset{
  language=FoCaLiZe, tabsize=2, frame=single, breaklines=true,
  basicstyle=\ttfamily, framexleftmargin=1mm, xleftmargin=1mm
}
}
%% By default, use FoCaLize as default listing language.
\setlangfocalize

%% To have fancier listings for Coq code.
\lstdefinelanguage{MyCoq}
  {morekeywords={Variable, Theorem, Section, assert, End, Qed, Let,
      forall, ex, Chapter, Record, Parameter, Definition, Hypothesis,
      let, match, if, then, else, end, Function, Module, Inductive,
      Require, type, Fixpoint, fix, struct,inversion,discriminate,
      auto, trivial, intro, intros, induction, exact, elim, left,
      right, destruct},
    otherkeywords={->, :, :=, =>, ;, \|,<>,=, /\\, \\/},
    sensitive=false,
    morecomment=[n]{(*}{*)},   %% Allow nested comments.
    morecomment=[n]{(**}{*)},  %% Allownested comments.
    morestring=[b]",
  }

\newcommand{\setlangcoq}{
\lstset{
  language=MyCoq, tabsize=2, frame=single, breaklines=true,
  basicstyle=\ttfamily, framexleftmargin=1mm, xleftmargin=1mm
}
}



\author{Fran\c{c}ois Pessaux\\
  ENSTA ParisTech\\
  \url{francois.pessaux@ensta-paristech.fr}}
\title{Another Tutorial for {F}o{C}a{L}ize:\\
  Playing with Proofs}
\date{December 2013}

\begin{document}

\maketitle

%  Copyright 2006 INRIA
%  $Id: intro.tex,v 1.3 2006-03-01 14:39:03 doligez Exp $

\chapter{Introduction}\label{chap:intro}

Zenon is an automatic theorem-prover whose main claim to fame is that
it produces actual proofs of theorems.
Correctness is an explicit non-goal in the design of Zenon: the proof
output needs to be checked by another program before the theorem is
considered proved.
  As a consequence, Zenon is not
designed for direct use by humans, but rather for interfacing into
formal proof systems, such as interactive proof assistants and
non-interactive proof checkers.

This document gives a detailed description of all the information
needed by users of Zenon.  Chapter~\ref{chap:install} describes how to
compile and install Zenon from source code.  Chapter~\ref{chap:options}
describes all the command-line options accepted by Zenon.
Chapter~\ref{chap:input-zen} describes the native input syntax of
Zenon; chapter~\ref{chap:input-tptp} describes the TPTP input syntax;
chapter~\ref{chap:input-coq} describes the Coq-style input
syntax.  Chapter~\ref{chap:messages} describes the error and warning
messages that Zenon may output when searching for a proof.

\section{Dealing with first order logic theorems}

The first part of this tutorial intends teaching how to use \zenon\ on
simple boolean properties. The aim is to show how \zenon\ can help
making whole part of basic inference steps one usually makes explicit
in tools like \coq. It must clearly understood that \zenon\ is not a proof
checker but a theorem prover. Obviously it will not au\-to\-ma\-ti\-cal\-ly
demonstrate itself any property. However, combined with the \focal\
proof language, it will automate tedious combinations of ``sub-lemmas''
one usually ``think intuitively feasible''.

In the following examples, we won't create species. Instead, to get
rid of com\-ple\-xi\-ty induced by \focal\ structures, we will state and
prove theorems at ``top-level''.


\subsection{A so simple property: fully automated proof}
Let's first address the following property:
$\forall a, b : {\tt boolean}, a \Rightarrow b \Rightarrow a$.
We write this in \focal\ as follows, in a source file
\textbf{focalizec ex\_implications.fcl}:

{\scriptsize
\lstinputlisting[caption=ex\_implications.fcl]{ex_implications.fcl}}

Here, we stated a property and directly asked \zenon\ to find a proof
without any direction. \zenon\ then uses its internal knowledge of
first order logic to solve the goal. At this stage it is possible to
compile the program using the command \textbf{focalizec ex\_implications.fcl}
and get few messages with not error:

{\scriptsize
\begin{verbatim}
Invoking ocamlc...
>> ocamlc -I /usr/local/lib/focalize -c ex_implications.ml
Invoking zvtov...
>> zvtov -zenon /usr/local/bin/zenon -new ex_implications.zv
Invoking coqc...
>> coqc -I /usr/local/lib/focalize -I /usr/local/lib/zenon ex_implications.v
\end{verbatim}}

Compilation, code generation and \coq\ verification were
successful. Let's investigate the tree of this proof, where we wrote
hypotheses (context) in \green{green} and goals in \blue{blue}. The
intuition is that we lift the two implications as hypotheses and from
the hypothesis \lstinline"a" we trivially prove our goal.

$$
\inferrule*[Right=($\Rightarrow$-intro)]
   {\inferrule*[Right=($\Rightarrow$-intro)] {\green{a, b} \vdash \blue{a}}
   { \green{a} \vdash  \blue{b \Rightarrow a}}}
   { \vdash \blue{a \Rightarrow b \Rightarrow a}}
$$

In \focal, this {\em proof} is achieved as a simple hierarchical sequence of
in\-ter\-me\-dia\-te {\em steps}. A proof step starts with a {\em proof bullet}, which
gives its nesting level. The top-level of a proof is 0. In a
{\em compound proof}, the steps are at level one plus the level of the proof
itself.

{\scriptsize
\lstinputlisting[caption=ex\_implications.fcl (2)]{ex_implications2.fcl}}

Here, the {\em steps} \lstinline"<1>1" and \lstinline"<1>2" are at level 1
and form the compound proof of the top-level theorem.  Step \lstinline"<1>1"
also has a compound proof (whose goal is \lstinline"b -> a"), made of
steps \lstinline"<2>1" 
and \lstinline"<2>2".  These latter are at level 2 (one more than the level of
their enclosing step).

\medskip
After the proof bullet comes the {\em statement} of the step,
introduced by the keyword \lstinline{prove}.  This is the
proposition that is asserted and proved by {\em this step}.  At the end of
this step's proof, it becomes available as a {\em fact} for the next steps
of this proof and deeper levels sub-goals.  In our example, step
\lstinline"<2>1" is available in the proof of \lstinline"<2>2", and
\lstinline"<1>1" is available in the proof of \lstinline"<1>2".  Note
that \lstinline"<2>1" is {\bf not} available in the proof of \lstinline"<1>2"
since \lstinline"<1>2" is located at a strictly lower nesting level
than \lstinline"<2>1".

\medskip
After the statement is the {\em proof of the step}. This is where
either you ask \zenon\ to do the proof from {\em facts} (hints) you
give it, or you decide to split the proof in ``sub-steps'' on which
\zenon\ will finally by called and that will serve (still with \zenon)
to finally prove the current goal by combining these ``sub-steps''
lemmas.

For instance, the proof of the whole theorem (which is itself a
statement) is not directly asked to \zenon\ as it was the case in the
first example (with the simple fact \lstinline"proof = conclude").  It has been decided to
split it in one sub-goal \lstinline"<1>1": prove \lstinline"b -> a". The same
structure is applied to this goal which is split in the sub-goals
\lstinline"<2>1" and \lstinline"<2>2".

\medskip
In the proof of sub-goal \lstinline"<2>1", appears a {\em fact}:
\lstinline"by hypothesis h1". Here \zenon\ is asked to find a proof of
the current goal, using hints it is provided with. You are responsible
in giving \zenon\ facts it will need to finally find a proof. It will
combine them in accordance with logical rules, but if it is missing
material for the proof, it will never succeed. Here, \zenon\ is told
that it should be able to prove the goal only using the hypothesis
\lstinline"h1" we introduced. It will obviously succeed since the goal
is exactly the hypothesis \lstinline"h1".

From the proof of \lstinline"a", i.e. the step \lstinline"<2>1", we can conclude the
en\-clo\-sing goal (\lstinline"prove b -> a"). This is done by the \lstinline"qed"
step whose aim is to close the enclosing proof by mean of the provided
facts (here the intermediate lemma stated by the step \lstinline"<2>1").

Finally, coming back to the remaining part of the proof, i.e. the
previous nesting level, we want to solve its goal (which is the whole
theorem). In the same manner, from the step \lstinline"<1>1" that proved
that \lstinline"b -> a" under the hypothesis \lstinline"a", we can conclude. We
then invoke the statement \lstinline"conclude" which is equivalent to
tell \zenon\ to use as facts, all the available proof steps in the
scope. Hence this is equivalent here to write
\lstinline"qed by step <1>1". Note that with \lstinline"conclude" the
\lstinline"qed" is optional since this statement implicitly marks the
end of the proof related to the current goal.

Having a look backward to compare our \coq\ and \focal\ proofs, we can
clearly see the same processing order. We introduced the implications as
hypotheses, with \verb"intro" in \coq\ and \lstinline"assume" in
\focal. Then we used the hypothesis \lstinline"h1", with \verb"exact h1" in 
\coq\ and \lstinline"by hypothesis h1" in \focal. The implicit nested
structure of the proof is made explicit in \focal.



\subsection{Still simple but easier with \zenon}
Let's continue with simple first-order logic properties and let's try
to demonstrate the following statement:
$\forall a, b : {\tt boolean}, (a \logand b) \Rightarrow (a \logor b)$.
We can easily build the tree of this proof as follows:

$$
\inferrule* [Right=($\Rightarrow$-intro)]
   {\inferrule* [Right=($\logand$-elim)] {\green{a \logand b} \vdash \blue{a \logand b}}
      {\inferrule* [Right=($\logor$-elim)] {\green{a \logand b} \vdash \blue{b}}
     { \green{a \logand b} \vdash \blue{a \logor b}}}}
   { \vdash \blue{a \logand b \Rightarrow a \logor b}}
$$

Note that we chose to prove \verb"b" but we could have chose to prove
\verb"a" instead. We now address this property in \focal\ again,
making explicit all the steps of the proof we did.

{\scriptsize
\lstinputlisting[caption=and\_or.fcl]{and_or.fcl}}

We can see that step \lstinline"<2>1" is directly proven by hypothesis
\verb"h1", without making explicit the $\logand${\tt -ELIM} rule of
the proof tree. Here we rely on \zenon\ to find the proof of this
step. Again, this property is simple enough to get it proved in a
fully automated way by \zenon.

{\scriptsize
\begin{lstlisting}
open "basics" ;;

theorem and_or : all a b : bool, (a /\ b) -> (a \/ b)
proof = conclude ;;
\end{lstlisting}}


\section{Playing with programs}
Let's now introduce more material than simple first-order logic
formulae. In this section we will first introduce functions, then
inductive types in stated properties. Finally we will see that
such previously stated properties can be used as lemmas to prove further
theorems.

\subsection{Introducing functions}
One may want to prove that logical or ($\logor$) is commutative,
i.e. that $(a \logor b) \Rightarrow (b \logor a)$. But, on atomic
properties, this would again be trivial for \zenon. Instead, of making
again explicit trivial proof steps, we will now extend this formula
with the (trivial again) identity function. Hence we want to prove
that if \verb"id" is defined as $\lambda x.x$, then
$\forall a, b, c, d : int, (id (a) = b \logor id (c) = d) \Rightarrow (c = id (d) \logor a = id (b))$.
We write this in \focal\ like:

{\scriptsize
\lstinputlisting[numbers=left,caption=or\_id\_com.fcl]{or_id_com.fcl}}

As shown in the previous code snippet, we optimistically asked \zenon\ to
automatically handle the proof. So, let's invoke the compilation
command: \textbf{focalizec or\_id\_com.fcl} and we get:

{\scriptsize
\begin{verbatim}
Invoking ocamlc...
>> ocamlc -I /usr/local/lib/focalize -c or_id_com.ml
Invoking zvtov...
>> zvtov -zenon zenon -new or_id_com.zv
File "or_id_com.fcl", line 7, characters 8-16:
Zenon error: exhausted search space without finding a proof
### proof failed
\end{verbatim}}

We should have not been so optimistic: line 7, where we invoked \lstinline"conclude",
\zenon\ did not find any proof despite it natively knows equality and
basic logic. We will investigate this point incrementally, and once
understood, we will see that we could have fixed the proof more
quickly (and more lazily). The aim here is to make again hierarchical
proof steps explicit to train splitting proofs in intermediate cases
(which will quickly become mandatory with realistic proofs).

\medskip
What is the sketch of the proof ? We basically want to assume that
$id (a) = b \logor id (c) = d$ then prove $c = id (d) \logor a = id
(b)$. We will now build the structure of the proof incrementally,
adding steps from our intuition, but leaving temporarily them
unproved. By this mean, we do not yet focus on each sub-goal, rather
on the global scheme of the proof. In some sense, we add intermediate
lemmas and want to ensure that (obviously provided they will be proven)
\zenon\ can find a proof of the global theorem combining these
lemmas. So, let's simply add a step assuming $id (a) = b \logor id (c) = d$
and ``fake-prove'' that $c = id (d) \logor a = id (b)$. Then, we ask
\zenon\ to conclude the whole theorem by this step.

{\scriptsize
\lstinputlisting[numbers=left,caption=or\_id\_com.fcl (2)]{or_id_com1.fcl}}

We remark the apparition of the keyword \lstinline"assumed" whose aim
is to loosely make a ``fake'' proof. In a sense, this allows stating
the related goal as being an axiom. Obviously, this is cheating since
the proof gets admitted, hence do not reflect anymore a really holding
property. Decent \focal\ developments should not have such ``proofs''
remaining. However, this can be the only solution when dealing with
properties that can't be proved because relying of third-party code,
not available in \focal, or when properties deal with higher-order
(\zenon\ doesn't handle this aspect). In the present case, we
only use it as a temporary placeholder to help us refining our proof
from the general idea to the fine-grain sequence of steps.

At this point the source file can be compiled invoking
\textbf{focalizec or\_id\_com.fcl} which hopefully gives not
error. It is pretty satisfactory that from the only step of the proof,
having lifted the left part of the implication as hypothesis, the
whole theorem can be proved !

\medskip
It remains now to really prove that $c = id (d) \logor a = id(b)$ under
our hypothesis. This is achieved by proving it in both cases where we
have the left and the right parts of our disjunctive hypothesis. We
can then add these new steps, still assuming their proofs, just to
ensure that our intuition of the scheme is consistent.

{\scriptsize
\lstinputlisting[numbers=left,caption=or\_id\_com.fcl (3)]{or_id_com2.fcl}}

We introduced steps \verb"<2>1" and \verb"<2>2" and said that they
should be sufficient for \zenon\ to prove the enclosing goal.
To conclude
step \lstinline"<2>3" , we must make explicit that the 2 steps
\lstinline"<2>1" and \lstinline"<2>2" are performed under assumptions
being the two parts of the disjunction we had in hypothesis
\lstinline"h1", otherwise these 2 cases are not relevant (in other
words, why did we state and prove them). Hence, step
\lstinline"<2>3" is only missing this information:
\lstinline"by ... hypothesis h1" !
We again
compile the program and get pretty happy to see that, provided these
two steps, \zenon\ really succeeds.

\medskip
So, we now need to continue our incremental process and really prove
that on one side \lstinline"c = id (d)" and on the other
\lstinline"a = id (b)". Since \zenon\ looks smart, why not asking him
to \lstinline"conclude" ? Let's try\ldots

{\scriptsize
\lstinputlisting[numbers=left,caption=or\_id\_com.fcl (4)]{or_id_com3.fcl}}

It is now time to compile the program, again with the command
\textbf{focalizec or\_id\_com.fcl} and we get:

{\scriptsize
\begin{verbatim}
Invoking ocamlc...
>> ocamlc -I /usr/local/lib/focalize -c or_id_com3.ml
Invoking zvtov...
>> zvtov -zenon zenon -new or_id_com3.zv
File "or_id_com3.fcl", line 14, characters 12-20:
Zenon error: exhausted search space without finding a proof
### proof failed
\end{verbatim}}

\noindent clearly stating that \zenon\ didn't find any proof. Let's
just inspect the proof tree we tried to build:

$$
\inferrule* [Right=($\Rightarrow$-intro)]
  {\inferrule* [Right=($\logor$-elim)]
    { {\inferrule*
         {
           \inferrule* [Left=($\logor$-introl)]
           { {\begin{tabular}[b]{l}
          How to know id (a) is equal to a ? \\     How to know b is equal to id (b) ?
         \end{tabular}}}
           {\green{id (a) = b} \vdash \blue{a = id (b)}}
        }
        { \green{id (a) = b} \vdash \blue{c = id (d) \logor a = id (b)} }} \\
      {\inferrule*
         {
          \inferrule* [Right=($\logor$-intror)]
           { {\begin{tabular}[b]{l}
          How to know id (c) is equal to c ? \\   How to know d is equal to id (d) ?
         \end{tabular}}}
          {\green{id (c) = d} \vdash \blue{c = id (d)}}
        }
        { \green{id (c) = d} \vdash \blue{c = id (d) \logor a = id (b)} }}
    }
    { \green{id (a) = b \logor id (c) = d} \vdash \blue{c = id (d) \logor a = id (b)} }}
  { \vdash \blue{(id (a) = b \logor id (c) = d) \Rightarrow (c = id (d) \logor a = id (b))}}
$$

The blocking point is that the proof strongly rely on the fact that
\lstinline"id" being the identity, $id(a) = a, id(b) = b, id(c) = c$ and
$id(d) = d$, but \zenon\ is not aware of this. What \zenon\ needs is
to know about the {\em definition} of the function \lstinline"id".

Here comes
a new fact (in addition to the already seen facts \lstinline"conclude",
\lstinline"hypothesis" and \lstinline"step"): the
\lstinline"definition of" stating that \zenon\ must consider a whole
function (i.e. including its body -- its {\em definition}) to try
finding a proof. Hence, our proof of each intermediate steps
\lstinline"<2>1" and \lstinline"<2>2" will be done
\lstinline"by definition of id". Moreover, as shown in our above proof
tree, both goals (\lstinline"<2>1" and \lstinline"<2>2") rely on their
related hypothesis (\lstinline"h2" and \lstinline"h3").

{\scriptsize
\lstinputlisting[numbers=left,caption=or\_id\_com.fcl (5)]{or_id_com_good.fcl}}

We now compile our whole and definitive program and get proofs finally
done and accepted by \coq:

{\scriptsize
\begin{verbatim}
Invoking ocamlc...
>> ocamlc -I /usr/local/lib/focalize -c or_id_com.ml
Invoking zvtov...
>> zvtov -zenon zenon -new or_id_com.zv
Invoking coqc...                                                
>> coqc  -I /usr/local/lib/focalize  -I /usr/local/lib/zenon or_id_com.v
\end{verbatim}}

Now we suffered enough, splitting the proof of this theorem in several
parts and learned the \lstinline{by definition of} fact, let's just
discover that all the intermediate steps we did, dealing with {\em basic
logic combinations} \ldots could again be automatically done by
\zenon\ and that, only telling it that it should use the definition of
\lstinline{id} would have been sufficient !

{\scriptsize
\lstinputlisting[numbers=left,caption=or\_id\_com\_shortest.fcl]{or_id_com_shortest.fcl}}

We can invoke the compiler on this shortened version of our program
(assuming the source file is \textbf{or\_id\_com\_shortest.fcl}):
\textbf{focalizec or\_id\_com\_shortest.fcl} and get the same successful
happy end:

{\scriptsize
\begin{verbatim}
Invoking ocamlc...
>> ocamlc -I /usr/local/lib/focalize -c or_id_com_shortest.ml
Invoking zvtov...
>> zvtov -zenon zenon -new or_id_com_shortest.zv
Invoking coqc...                                               
>> coqc  -I /usr/local/lib/focalize  -I /usr/local/lib/zenon or_id_com_shortest.v
\end{verbatim}}


\subsection{Introducing pairs}

\focal\ natively provides the type of tuples. \zenon\ knows only about
{\em pairs} (i.e. 2-components tuples). However, until enhancements of
\focal\ and/or \zenon, it is possible to encode general tuples as
nested pairs. For instance, instead of manipulating the type
\lstinline"(int * bool * string)", one will manipulate
\lstinline"(int * (bool * string))" even if it is a bit cumbersome.

\medskip
We will now study some proofs dealing with pairs, see what \zenon\ is
able to handle and how we can explicitly write such proofs. We first
start by the initial type definition, aliasing pairs of
\lstinline"int"s to a type \lstinline"int_pair_t". Such a definition
is written:

{\scriptsize
\begin{lstlisting}
type int_pair_t = alias (int * int) ;;
\end{lstlisting}}

\noindent and simply declares a new type constructor compatible with
\lstinline "(int * int)".

\zenon\ natively knows about \lstinline"fst : ('a * 'b) -> 'a" and
\lstinline"snd : ('a * 'b) -> 'b" functions, extracting the first and
second component of a pair. For instance, it will be able to prove
that extracting components of one pair with 2 equal components will lead
to 2 equal values:

{\scriptsize
\lstinputlisting[numbers=left,caption=same\_comps.fcl]{same_comps.fcl}}



\subsection{Playing with pairs}

We will now prove another simple property to continue using the hierarchical way
to write proofs, hence train to make explicit steps for later, when
such splits will be mandatory. We want to prove the property:
$$
\forall v_1, v_2 : {\tt int}, \forall v : {\tt int\_pair\_t},
\ v = (v1, v2) \Rightarrow\ \sim (v1 = v2) \Rightarrow ~ ({\tt fst} (v) = {\tt snd} (v))
$$

This obviously can be proven by \zenon\ as shows the following
formulation in \focal:

{\scriptsize
\lstinputlisting[numbers=left,caption=diff\_comps.fcl]{diff_comps.fcl}}

However, we want to prove it ourselves (nearly, \zenon\ will finally still
provide the glue between our steps)! We first need to expose
the sketch of the proof: first assume the 2 implications, then prove
$\sim ({\tt fst} (v) = {\tt snd} (v))$. To do so, we will demonstrate that in fact
${\tt fst} (v) = v_1$, that ${\tt snd} (v) = v_2$, and conclude by the hypothesis
that $v_1 \not = v_2$.

{\scriptsize
\lstinputlisting[numbers=left,caption=diff\_comps.fcl (2)]{diff_comps2.fcl}}

In lines 10 and 11, we lift the implications premises as hypotheses,
then the re\-mai\-ning goal is \lstinline"prove ~ (fst (v) = snd (v))".
Then we prove in step \lstinline"<2>1" that \lstinline"fst (v) = v1"
which is obtained from the fact that \lstinline"v" is a pair (hypothesis
\lstinline"h1") and \zenon's knowledge about \lstinline"fst". We prove
that \lstinline"snd (v) = v2" by the same means. And finally from
these 2 intermediate steps and the hypothesis that $v_1 \not = v_2$
(\lstinline"h2") we achieve demonstration of the goal \lstinline"<1>1".


\subsection{Introducing inductive types}

Realistic programs usually do not only involve integers and pairs:
inductive type de\-fi\-ni\-tions are a powerful mean to model
data-structures. \focal\ doesn't escape this rule and \zenon\ makes
possible to reason on such type definitions to automate
proofs.
An inductive type definition introduces several {\em value constructors} for a type.
For instance:

{\scriptsize
\begin{lstlisting}
type signal_t = | Red | Orange | Green ;;
\end{lstlisting}}

\noindent declares the {\bf new} type \lstinline"signal_t" as containing the
{\bf only} 3 values \lstinline"Red", \lstinline"Orange" and
\lstinline"Green". These values are all different from each other.

Moreover, an inductive type definition can introduce parametrised
constructors, possibly by values of the type itself: we have a recursive
type definition:

{\scriptsize
\begin{lstlisting}
type peano_t = | Z | S (peano_t) ;;
\end{lstlisting}}

\noindent declares the {\bf new} type \lstinline"peano_t" as containing the
{\bf only} 2 values \lstinline"Z" and \lstinline"S", this latter
embedding a value of type \lstinline"peano_t" itself. We recognize
here the usual definition of Peano's integers.

\medskip
As a summary, inductive definitions natively introduce 2 important
concepts used all over proofs:
\begin{itemize}
\item {\em The injectivity of value constructors}: a value of such a type is
  one of its cons\-truc\-tors and nothing else, constructors being all
  different from each other.
\item {\em The induction principle}: assuming a property holding on constant
  value constructors, if this property holds for any parametrised
  value constructor, then it holds for any values of this type. This is
  indeed a generalization of the well-known recurrence principle on
  natural numbers.
\end{itemize}

\medskip
We will first show that \zenon\ greatly helps by knowing injectivity
of constructors. The aim will be to demonstrate that any value of type
\lstinline"signal_t" is equal to either \lstinline"Red", or \lstinline"Orange"
or \lstinline"Green". Hence we state the theorem:

{\scriptsize
\lstinputlisting[numbers=left,caption=signal.fcl]{signal1.fcl}}

\noindent and invoke the compiler to get:

{\scriptsize
\begin{verbatim}
Invoking ocamlc...
>> ocamlc -I /usr/local/lib/focalize -c signal.ml
Invoking zvtov...
>> zvtov -zenon zenon -new signal.zv
File "signal.fcl", line 7, characters 10-18:
Zenon error: exhausted search space without finding a proof
### proof failed
\end{verbatim}}

\zenon\ didn't find any proof despite we promised it knew how to
reason on inductive types! In fact, it was given no fact, no clue, so
how could it guess that this property was induced by the underlying
type definition? It it important to keep in mind that \zenon\ only implicitly uses
basic logic combinations: it will never use all the material
available in a program! So, we just need to tell him that this proof
can be deduced from the type definition of \lstinline"signal_t".

Here comes a new fact  (in addition to the already seen facts \lstinline"conclude",
\lstinline"hypothesis", \lstinline"step" and \lstinline"definition of"): the
\lstinline"type" fact stating that the definition of the following
type must be used. We modify our program just inserting this new fact
and get:

{\scriptsize
\lstinputlisting[numbers=left,caption=signal.fcl (2)]{signal2.fcl}}

\noindent which is perfectly proven now. This could seem not so
wonderful on such an obvious property, but this means that using
\zenon, such intrinsic property of inductive type definitions is
natively understood, as long as \zenon\ is told to use it by the fact
\lstinline"by ... type ...". There is no need to explicitly invoke
and manipulate the induction principle.

\medskip
We can also show that mutual exclusion of value constructors are
native for \zenon: let's prove that if a value of type
\lstinline"signal_t" is equal to \lstinline"Red", then it is different
of \lstinline"Green". This can appear more than obvious, such a
property, often used while reasoning by cases, requires some
intermediate steps (mostly applying the induction principle and
discriminations on the constructors). Let's state and prove this
property in \focal:

{\scriptsize
\lstinputlisting[numbers=left,caption=signal2.fcl]{signal3.fcl}}


\subsection{When automatic induction fails}
In the previous examples, we proved very simple facts and \zenon\ directly found
proofs in one shot \lstinline"by type ...", i.e. implicitly using
induction on the related type. However it is not always the case. It
may be needed to explicitly write some proofs, proving the base cases
then each inductive cases. In such a configuration, the
\lstinline"by type ..." won't apply alone.

It must be clear that using \lstinline"by type ..." alone (again,
implying simple induction of the type) only applies in case where the
goal has a shape \lstinline"all x : t, P (x)". This especially means
that a ``one shot proof'' must not start by eliminating the
\lstinline"all x : t" as we usually did.

The fact \lstinline"by type ..." is not reduced to induction: it also
states that a proof needs to know about the constructors of a
type. Hence, that's not because a ``one shot proof'' failed that the
fact \lstinline"by type ..." will not be needed.

\subsubsection{Simple example}
We first start with a simple theorem stating that a function \lstinline"zero"
defined recursively always return 0. We try to directly ask \zenon\ to apply
the induction principle to solve the goal:

{\scriptsize
\lstinputlisting[numbers=left,caption=zero.fcl (1)]{zero1.fcl}}

\noindent and see that the proof failed:
{\scriptsize
\begin{verbatim}
Invoking ocamlc...
>> ocamlc -I /usr/local/lib/focalize -c zero.ml
Invoking zvtov...
>> zvtov -zenon zenon -new zero.zv
File "zero.fcl", line 12, characters 8-42:                     
Zenon error: could not find a proof within the memory size limit
### proof failed
\end{verbatim}}

We now need to split the proof in 3 steps: one for the base case, one
for the inductive case, and the final \lstinline"qed" combining the
two former. As usually, we leave the difficult part (the induction
case) initially \lstinline"assumed" to ensure that our idea of the
proof passes with \zenon.  We only really prove the base case since it
is very simple and only depends on the definition of the function
\lstinline"zero" and the type \lstinline"peano_t".

{\scriptsize
\lstinputlisting[numbers=left,caption=zero.fcl (2)]{zero2.fcl}}

The proof is now found. {\bf The most important point to understand is that
\zenon\ could apply the induction principle in step \lstinline"<1>3"
because it has two steps of the form:
\begin{itemize}
\item P (base case)
\item all $y$ : t, P ($y$) $->$ P (inductive case using $y$)
\end{itemize}
}

Our property being \lstinline"zero (...) = 0", the step \lstinline"<1>1"
is the base case: \lstinline"zero (Z) = 0" and the step \lstinline"<1>2"
is the induction case:
\lstinline"all y : peano_t, zero (y) = 0 -> zero (S (y)) = 0".

We can now end the proof by really proving the step \lstinline"<1>2". We
first introduce \lstinline"y" and the induction hypothesis \lstinline"inH"
in the context, then we must prove \lstinline"zero (S (y)) = 0". This last
goal is simply a consequence of the induction hypothesis, the definition of
the function \lstinline"zero" and the definition of the type
\lstinline"peano_t". Note that in this step, the type \lstinline"peano_t"
is not used for induction: \zenon\ only needs it to know the constructor
\lstinline"S".

{\scriptsize
\lstinputlisting[numbers=left,caption=zero.fcl (3)]{zero3.fcl}}

The proof is now complete and the compilation is a success:

{\scriptsize
\begin{verbatim}
Invoking ocamlc...
>> ocamlc -I /usr/local/lib/focalize -c zero.ml
Invoking zvtov...
>> zvtov -zenon zenon -new  -script zero.zv
Invoking coqc...                                                
>> coqc  -I /usr/local/lib/focalize  -I /usr/local/lib/zenon zero.v
\end{verbatim}}


\subsubsection{More complex example}
In this next example, we propose to prove that given a structure of binary
tree, mirroring it twice is the identity (result tree is the same than
initial tree. We first define the tree structure, the mirror function
and state our property asking \zenon\ to prove it itself.

{\scriptsize
\lstinputlisting[numbers=left,caption=tree\_mirror1.fcl]{tree_mirror1.fcl}}

\noindent As planned and unfortunately, after a while, \zenon\ does not find any
proof with so much material :

{\scriptsize
\begin{verbatim}
Invoking ocamlc...
>> ocamlc -I /usr/local/lib/focalize -c tree_mirror1.ml
Invoking zvtov...
>> zvtov -zenon zenon -new tree_mirror1.zv
File "tree_mirror1.fcl", line 14, characters 8-46:              
Zenon error: could not find a proof within the memory size limit
### proof failed
\end{verbatim}}

As in the previous example, we need to split the proof in 3 steps: one for the
base case, one for the inductive case, and the final \lstinline"qed" combining
the two former. We postpone the difficult part (the induction case) for later
and mark it \lstinline"assumed". We however prove the base case since it
is very simple and only depends on the definition of the function
\lstinline"mirror" and the type \lstinline"bintree_t". Note that for
the induction case, we however state the induction hypotheses (\lstinline"ir1"
and \lstinline"ir2") before stating the goal of this case.

{\scriptsize
\lstinputlisting[numbers=left,caption=tree\_mirror2.fcl]{tree_mirror2.fcl}}

Invoking \focalizec\ on this program succeeds and it is now time to
complete our last proof. This proof, even if intuitive, can't be solved
directly by \zenon. Hence we have to detail it. Basically the proof
sketch is to ``unfold'' twice the function \lstinline"mirror" (i.e. to
look at the result of \lstinline"mirror (Node (...))") then to then use induction
hypotheses (\lstinline"ir1" and \lstinline"ir2") to finally get the
effective equality of our two terms.

{\scriptsize
\lstinputlisting[numbers=left,caption=tree\_mirror.fcl]{tree_mirror.fcl}}


\subsection{Introducing lemmas}
Until now we stated properties and demonstrated them writing
``all-in-one'' proofs, i.e. using intermediate (hence nested) steps,
hypotheses, types and functions definitions. However, depending on the
complexity of the property to prove, it may be easier to define
intermediate lemmas, or even involve previously demonstrated
theorems. This answers a need for modularity (intermediate lemmas can
be used for other proofs) and readability (intermediate lemmas can
make proofs more numerous but smaller) when writing proofs.

\medskip
Still addressing proofs on programs, we now want to prove that the
absolute value of a difference is always \ldots positive. The only
thing is, we won't write the program computing such a value using a
predefined {\tt abs} function bringing its property stating it always
returns a positive value. Instead, we write this function using a test
and a subtraction:

{\scriptsize
\begin{lstlisting}[caption=lemmas.fcl]
open "basics" ;;

let abs_diff (x, y) = if x >0x y then x - y else y - x ;;
\end{lstlisting}}

In this program and the following, relational operators on integers
are written suffixed by \lstinline{0x}. These are the \focal\
operators on \lstinline{int} ( \lstinline{>0x}, \lstinline{<0x},
\lstinline{>=0x} \ldots). We now state the property we want to
demonstrate, and as always we initially state it as \lstinline{assumed}:

{\scriptsize
\lstinputlisting[numbers=left,caption=lemmas.fcl (2)]{lemmas1.fcl}}

We must now elaborate the sketch of the proof to introduce
intermediate steps. From the definition of our function
\lstinline"abs_diff", it is clear that we must reason by cases, one if
$x > y$ and one if $\sim (x > y)$, i.e. $x \le y$. Hence, we will
introduce 2 steps for these cases, and an ending one using the former
to conclude the goal.

{\scriptsize
\lstinputlisting[numbers=left,caption=lemmas.fcl (3)]{lemmas2.fcl}}

In both intermediate steps \lstinline"<2>1" and \lstinline"<2>2" the
goal is the same than the global one: we did not yet split it, changed
it by any refinement. However, we introduced 2 different (and
complementary) hypotheses.
Having in mind that having covered cases $x > y$
and  $x \le y$ we covered all the cases of integers, we run the
compiler and get:

{\scriptsize
\begin{verbatim}
Invoking ocamlc...
>> ocamlc -I /usr/local/lib/focalize -c lemmas.ml
Invoking zvtov...
>> zvtov -zenon zenon -new lemmas.zv
File "lemmas2.fcl", line 16, characters 16-34:
Zenon error: exhausted search space without finding a proof
\end{verbatim}}

Oops, it goes wrong, \zenon\ didn't find any proof! So what? So why ?
As naively said in the above paragraph, ``{\em Having in mind that having
covered cases $x > y$ and  $x \le y$ we covered all the cases of
integers}'', we assume that it is obvious that 2 integers are either
greater or lower-or-equal together. But, this fact is {\bf not}
obvious: \zenon\ does not known arithmetic! So we need to give it
such a property as a fact to hope it will finally find a proof.

\medskip
We are currently trying to make a proof, and now we need to prove
another property. So, first we don't want to spread our effort in
several directions. We need to have this other property: why not state
it, not prove it yet, and check that our current proof pass with this
new property ? We just need a lemma to make our proof, so we will
introduce some. We then write the theorem
\lstinline"two_ints_are_gt_or_le"
stating that $\forall x, y : int,\ x \le y \logor x > y$ and give it
as a new fact to \zenon.

Here comes a new fact  (in addition to the
already seen facts \lstinline"conclude", \lstinline"hypothesis",
\lstinline"step", \lstinline"definition of" and \lstinline"type": the
\lstinline"property" fact, sta\-ting that \zenon\ should use the given
property (i.e. logical statement) to find a proof.

{\scriptsize
\lstinputlisting[numbers=left,caption=lemmas.fcl (4)]{lemmas2_1.fcl}}

We now compile again our development and see that with this new fact,
\zenon\ finally succeeded.

{\scriptsize
\begin{verbatim}
Invoking ocamlc...
>> ocamlc -I /usr/local/lib/focalize -c lemmas.ml
Invoking zvtov...
>> zvtov -zenon zenon -new lemmas.zv
Invoking coqc...                                                
>> coqc  -I /usr/local/lib/focalize  -I /usr/local/lib/zenon lemmas.v
\end{verbatim}}

Obviously, we get one more theorem to demonstrate. However, we know
that provided this theorem holds, the proof of our main program
property also holds. We can now go on further on it, leaving the new
lemma for later.

But\ldots by a wonderful coincidence, \focal\ comes with a ``standard
library''! And looking among available theorems (in
\textbf{basics.fcl}) we find a theorem:

{\scriptsize
\begin{lstlisting}
theorem int_gt_or_le : all x y : int, (x >0x y) \/ (x <=0x y)
\end{lstlisting}}

\noindent exactly fitting what we need! So, proof of our lemma will be
trivial since it will simply be done \lstinline"by property basics.int_gt_or_le".
But, we can make even simpler: in our proof, let's just use this
theorem from the library instead of aliasing it by our lemma! Hence,
in our proof, we change \lstinline"<2>3 qed" by adding
\lstinline"by ... int_gt_or_le" instead of 
\lstinline"by ... two_ints_are_gt_or_le" and remove this latter from
our source code.

\medskip
Now, let's going on with our proof. We need to really prove the 2
steps \lstinline"<2>1" and \lstinline"<2>2". For \lstinline"<2>1", we
can prove that ${\tt abs\_diff} (x, y) = x - y$ and that
$x - y \ge 0$. This way, we will really have proved that
${\tt abs\_diff} (x, y) \ge 0$: our sub-proof will then be conclude ``by
these 2 steps'' as show below in step \lstinline"<3>2"

Similarly, for \lstinline"<2>2" we will prove something like that
${\tt abs\_diff} (x, y) = y - x$ and $y - x \ge 0$. We let this second case
aside for the moment (i.e. \lstinline"assumed"), only dealing with the
first one.

{\scriptsize
\lstinputlisting[numbers=left,caption=lemmas.fcl (5)]{lemmas3.fcl}}

Compiling our program, we will see that the proof continues passing:
our idea of sub-proofs was correct. So, we now want to really prove
the new intermediate steps \lstinline"<3>1" and \lstinline"<3>2". The
sketch of the proof is to prove that ${\tt abs\_diff} (x, y) = x - y$ and
that $x - y \ge 0$ knowing we are in the context of hypothesis
\lstinline"h1" stating that $x > y$.

Lets start by step \lstinline"<3>1". We want to prove that
${\tt abs\_diff} (x, y) = x - y$.
This is a direct consequence of the definition of the function
\lstinline"abs_diff" since we are in the case of hypothesis
\lstinline"h1". Hence, this proof is simply done
\lstinline"by definition of abs_diff hypothesis h1".

We now address step \lstinline"<3>2".  What do we have as material? We
know by hypothesis \lstinline"h1" that $x>y$. From this point, it looks
obvious to us that in effect,  $x - y \ge 0$. However, like above for
the ``trivial'' lemma on arithmetic, it won't probably be so for
\zenon. We can again introduce a new lemma, or \ldots have a look to
see if there would not already be a suitable theorem in the \focal\
standard library! And hopefully, we find in \textbf{basics.fcl} the
theorem:

{\scriptsize
\begin{lstlisting}
theorem int_diff_ge_is_pos : all x y : int, x >=0x y -> x - y >=0x 0
\end{lstlisting}}

It is nearly won, but not yet. In effect, having a deeper look at our
hypothesis \lstinline"h1", we see that it states that $x > y$ although
the theorem \lstinline"int_diff_ge_is_pos" requires as hypothesis that
$x \ge y$. However, our intuition immediately makes us thinking that if
$x > y$ then it is inevitable that $x \ge y$. Again, a new lemma to
introduce or a look to have in the standard library\ldots

Fortunately, we again discover a theorem fitting our expectations in
\textbf{basics.fcl}:

{\scriptsize
\begin{lstlisting}
theorem int_gt_implies_ge : all x y : int, x >0x y -> x >=0x y
\end{lstlisting}}

Note that in ``real life'', it will happen that the library do not already
contains the theorem you need: in this case, you will really state it
as a new theorem (lemma) and finally will need to prove it!

Now we found the 2 former theorems, our goal should be solved by
\zenon\ \lstinline"by property ..." of them and the hypothesis
\lstinline"h1".

{\scriptsize
\lstinputlisting[numbers=left,caption=lemmas.fcl (6)]{lemmas4.fcl}}

As planned, the proof is accepted and we go on, trying to prove the
remaining step \lstinline"<2>2". We will proceed in the same way,
proving that assuming hypothesis \lstinline"h2" we have
${\tt abs\_diff} (x, y) = y - x$ and $y - x \ge 0$.

However, we can note that the theorem \lstinline"int_diff_ge_is_pos"
we used above states  $(a \ge b) \Rightarrow (a - b \ge 0)$.
But, our hypothesis \lstinline"h2" states that $x \le y$ and we need
to prove that $y -x \ge 0$. But in fact, in our hypothesis, if we swap $x$ and
$y$ and replace $\le$ by $\ge$ we get into the right hypothesis of the
theorem. Again, we will need a theorem stating that
$x \le y \Rightarrow y \ge x$ which already exists as
\lstinline"int_le_ge_swap". We then have one more step than in the
previous case, to demonstrate that $y \ge x$
\lstinline"by property int_le_ge_swap hypothesis h2".

\medskip
We finally show the new form of the proof, skipping the ({\em a priori}
obvious) proof that {\tt abs\_diff (x, y) = y - x}. Although it seems
it is only a consequence of the definition of \lstinline{abs_diff}, we
will see later that it requires something more.

{\scriptsize
\lstinputlisting[numbers=left,caption=lemmas.fcl (7)]{lemmas5.fcl}}

Finally, it only remains to inspect step \lstinline"<3>1". As
previously mentioned, it looks trivial that it only depends on the
fact we are in hypothesis \lstinline"h2", i.e. $x \le y$ and the
definition of \lstinline"abs_diff" falling in the ``else-case''. Let
simply make the proof with these 2 facts:

{\scriptsize
\begin{lstlisting}
...
       <2>2 hypothesis h2: x <=0x y,
                 prove abs_diff (x, y) >=0x 0
            <3>1 prove abs_diff (x, y) = y - x
                      by definition of abs_diff hypothesis h2
...
\end{lstlisting}}

\noindent We compile and get:

{\scriptsize
\begin{verbatim}
Invoking ocamlc...
>> ocamlc -I /usr/local/lib/focalize -c lemmas.ml
Invoking zvtov...
>> zvtov -zenon zenon -new lemmas.zv
File "lemmas.fcl", line 21, characters 17-56:                
Zenon error: exhausted search space without finding a proof
### proof failed
\end{verbatim}}

Having closer look at our hypothesis \lstinline"h2: x <=0x y" and the
way \lstinline"abs_diff" has its conditional written
\lstinline"if x >0x y", we see that the \lstinline"if" tests $x >y$,
hence in the ``else-case'' we have $\sim (x >y)$ and not $x \le y$ as
stated in the hypothesis! In effect, in a ``else-branch'' the holding
property is ``{\bf not}-the-tested-condition''. And again, for \zenon,
it is not obvious that $\sim (x >y)$ is the same thing than $x \le y$.
Again, we need to guide \zenon\ with such a theorem which hopefully
exists in \focal\ standard library:

{\scriptsize
\begin{lstlisting}
theorem int_le_not_gt : all x y : int, (x <=0x y) -> ~ (x >0x y)
\end{lstlisting}}

At this point, adding the fact \lstinline"int_le_not_gt" to the proof
of our step \lstinline"<3>1" will finally conclude the whole proof:

{\scriptsize
\lstinputlisting[numbers=left,caption=lemmas.fcl (8)]{lemmas.fcl}}


\section{A first simple program}
After having seen how to write hierarchical proofs in \focal\ using
\zenon\ in the context of pretty ad-hoc properties, we will finally
apply previous technics on proving properties related to a (still very
simple) program.

We deliberately make no use of \focal\ advanced modeling features like
in\-he\-ri\-tan\-ce, parametrisation, incremental conception and refinement
mechanisms. We only consider a raw software model, obviously not
supporting evolution, but that's not the aim. More information on
these points can be found in [BLA].

\subsection{The goal}
We want to model a simplified traffic signals controller. The system will
be made of 3 signals with 3 states: \green{green}, \orange{orange} and
\red{red}. The controller will alternatively make each signal becoming
green along a predefined sequence, making so that other signals are
red. As any usual signals, they turn orange before turning red. We can
then simply model the controller as a finite state automaton representing
cycling sequences (where \red{R} stands for red, \green{G} for green
and \orange{O} for orange, representing the state of each managed traffic
signal):
$$\green{G}\red{R}\red{R} \rightarrow \orange{O}\red{R}\red{R}
\rightarrow \red{R}\green{G}\red{R} \rightarrow \red{R}\orange{O}\red{R}
\rightarrow \red{R}\red{R}\green{G} \rightarrow \red{R}\red{R}�\orange{O}$$


\subsection{Modeling data structures}
Without surprise, to represent the color of a signal, we define a sum
type with 3 values:

{\scriptsize
\begin{lstlisting}
open "basics" ;;

(** Type of signals colors. *)
type color_t = | C_green | C_orange | C_red ;;
\end{lstlisting}}

Obviously, the automaton having 6 states, we need to define a sum type
with as many values. For readability, we name each case ``S\_''
followed by the corresponding signals color initials. For instance
\lstinline"S_orr" stands for {\em ``State where signal 1 is orange, signal
2 is red and signal 3 is red''}.

{\scriptsize
\begin{lstlisting}
(** Type of states the automaton can be. Simply named with letters
     corresponding to the colors of signal 1, 2 and 3. *)
type state_t = | S_grr | S_orr | S_rgr | S_ror | S_rrg| S_rro ;;
\end{lstlisting}}

Finally, the state of the controller will consists in the current
state of the automaton and the state of each signal. We then embed
the controller inside a species whose \lstinline"representation"
reflects this data structure.

{\scriptsize
\begin{lstlisting}
(** Species embedding the automaton controlling the signals colors changes. *)
species Controller =
  (* Need to encode tuples as nested pairs because of limitations of  Coq
     and Zenon. *)
  representation = (state_t * (color_t * (color_t * color_t))) ;
end ;;
\end{lstlisting}}

One may note that instead of defining the \lstinline"representation"
as a 4-components tuple, we nested pairs up to have 4 components. The
reason is that currently \focal\ compiler and \zenon\ don't yet
transparently generalize pairs, hence making very difficult proofs to
be compiled to \coq. However, this do not reduce the expressivity of
the language: it only makes things a bit more cumbersome.

\subsection{The main algorithm}
The controller is now modeled as a transition function taking the
current state of the controller as input and returning the next
state. Roughly speaking, it will discriminate on the state of the
automaton (first component of the \lstinline"representation" which is
the state of the controller), then determine the new state as well as
the new states of the signals. Hence, the transition function \lstinline"run_step"
will have type \lstinline"Self -> Self".

Because we modelled the state of the controller as a tuple-like data
structure, we first define projection functions to access individual
components of the controller state (i.e. the automaton state and each
signal state -- color).

{\scriptsize
\lstinputlisting[caption=controller.fcl]{controller/controller_algo.fcl}}

We only defined the behavioral, computational aspects of our
controller: no pro\-per\-ties yet. However we can compile this program and
get a usable piece of software.

\subsection{Introducing the main property}

It is now time to {\em ``prove our program''}. Behind this unclear but
widely used expression is hidden the task of characterizing the safety
properties of a system, then prove they hold. In our very simple case,
one interesting property is that we never have 2 green signals at the
same time. Since we have 3 signals we will state this property as the
negation of 3 disjunctions, each stating 2 of the 3 signals are green:
$$
\begin{array}{clc}
\sim ( & (signal_1\ is\ green \logand signal_2\ is\ green)\ \logor & \\
          & (signal_1\ is\ green \logand signal_3\ is\ green)\ \logor & \\
          & (signal_2\ is\ green \logand signal_3\ is\ green) & )
\end{array}
$$

Such a property leads to the following \focal\ theorem, still left
unproven for the moment:

{\scriptsize
\begin{lstlisting}
  (** The complete theorem stating that no signals are green at the same time. *)
  theorem never_2_green :
    all s r : Self,
    r = run_step (s) ->
    ~ ((get_s1 (r) = C_green /\ get_s2 (r) = C_green) \/
       (get_s1 (r) = C_green /\ get_s3 (r) = C_green) \/
       (get_s2 (r) = C_green /\ get_s3 (r) = C_green))
  proof = assumed ;
\end{lstlisting}}

\subsection{Making the proof}

It is now time to prove our theorem. One sketch of the proof is to
prove that we never have $signal_1$ and $signal_2$ green at the same
time, neither $signal_1$ and $signal_3$ nor $signal_2$ and $signal_3$.
From these 3 properties, \zenon\ should be able to find the remaining
``glue''  and prove the whole theorem!

We then just try to see if our intuition is right. We define the 3
intermediate lemmas \lstinline"never_s1_s2_green",
\lstinline"never_s1_s3_green" and \lstinline"never_s2_s3_green", let them
unproven for the moment, and ask \zenon\ to prove our main theorem
\lstinline"never_2_green" \lstinline"by property ..." our 3 lemmas:

{\scriptsize
\lstinputlisting[caption=controller.fcl (2)]{controller/controller1.fcl}}

We invoke the compilation by the regular command:
\textbf{focalizec controller.fcl}:

{\scriptsize
\begin{verbatim}
Invoking ocamlc...
>> ocamlc -I /usr/local/lib/focalize -c controller.ml
Invoking zvtov...
>> zvtov -zenon zenon -new controller.zv
Invoking coqc...                                                
>> coqc  -I /usr/local/lib/focalize  -I /usr/local/lib/zenon controller.v
\end{verbatim}}

\noindent and see that our proof passed, assuming our 3 pending
lemmas. It will then be time to actually prove these lemmas. One
imagine easily that their proofs will be similar, since the only
change between statements is the involved signals.


\subsubsection{Proving the first lemma}
We will now address proving the first lemma, namely
\lstinline"never_s1_s2_green", using the incremental approach we
previously introduced: setting-up the proof sketch, the main
intermediate steps with their goal to prove left assumed, then
refining these steps until nothing remains assumed. Obviously, our
lemma won't be fully automatically proved by one \zenon\ step since
its statement is too complex. Hence, we forget a proof of the shape:

{\scriptsize
\begin{lstlisting}
theorem never_s1_s2_green :
  all s r : Self,
  r = run_step (s) ->
  ~ (get_s1 (r) = C_green /\ get_s2 (r) = C_green)
proof = by definition of ... type ... step ... hypothesis ... ;
\end{lstlisting}}

\noindent and prepare us to write a hierarchical one, whose first step is
the simple introduction of hypotheses of our theorem, leaving its goal
to prove (i.e. currently left \lstinline"assumed"):

{\scriptsize
\begin{lstlisting}
theorem never_s1_s2_green :
  all s r : Self,
  r = run_step (s) ->
  ~ (get_s1 (r) = C_green /\ get_s2 (r) = C_green)
proof =
<1>1 assume s : Self, r : Self,
     hypothesis h1 : r = run_step (s),
     prove ~ (get_s1 (r) = C_green /\ get_s2 (r) = C_green)
     assumed
<1>2 conclude ;
\end{lstlisting}}

The sketch of the proof is a study by cases on the values of the
automaton state, showing that in each state, the resulting state of
the signals 1 and 2 is never green for both (i.e. there is never 2
\lstinline"C_green" values in the 2\textsuperscript{nd} and
3\textsuperscript{rd} components of the controller state). 

How can we prove this ? Simply by exhibiting, in each case, that the result
contains at least one color not equal to
\lstinline"C_green". Obviously, for each case we chose to target the
signal whose value is really not green! Hence, we refine our proof and
state each case of the proof, as many as there are states in the
automaton, hence as many as there are cases in the transition function
\lstinline"run_step". In the first case, no signal is green since
$signal_1$ is orange, and $signal_2$ is red: we chose to prove that 
$signal_1$ is not green. Conversely, in the second case, $signal_2$ is
green: we do not have the choice and must prove that $signal_1$ is not.

{\scriptsize
\begin{lstlisting}
(** Lemma stating that s1 and s2 are never green together. It's 1/3 of the
    final property stating that no signals are green at the same time. *)
theorem never_s1_s2_green :
  all s r : Self,
  r = run_step (s) ->
  ~ (get_s1 (r) = C_green /\ get_s2 (r) = C_green)
proof =
<1>1 assume s : Self, r : Self,
     hypothesis h1 : r = run_step (s),
     prove ~ (get_s1 (r) = C_green /\ get_s2 (r) = C_green)

     (* Proof by cases on values of the "automaton state" of s.
        For each case, we will prove that one of the 2 signal at least is
        not green. *)
     <2>1 hypothesis h2: get_s (s) = S_grr,
          prove ~ (get_s1 (r) = C_green)
          assumed

     (* Same proof kind for all the cases of automaton state. *)
     <2>2 hypothesis h3: get_s (s) = S_orr,
          prove ~ (get_s1 (r) = C_green)
          assumed

     (* Same proof kind for all the cases of automaton state. *)
     <2>3 hypothesis h4: get_s (s) = S_rgr,
          prove ~ (get_s1 (r) = C_green)
          assumed

     <2>4 hypothesis h5: get_s (s) = S_ror,
          prove ~ (get_s1 (r) = C_green)
          assumed

     <2>5 hypothesis h6: get_s (s) = S_rrg,
          prove ~ (get_s1 (r) = C_green)
          assumed

     <2>6 hypothesis h7: get_s (s) = S_rro,
          prove ~ (get_s2 (r) = C_green)
          assumed

     <2>7 qed by
          step <2>1, <2>2, <2>3, <2>4, <2>5, <2>6
          definition of run_step
          hypothesis h1
          type state_t
<1>2 conclude ;
\end{lstlisting}}

The conclusion of our proof is step \lstinline"<2>7" and obviously
relies on the 6 preceding steps, but also on the definition of the function
\lstinline"run_steps", the type \lstinline"state_t" and the hypothesis
\lstinline"h1: r = run_step (s)".

In effect, the intermediate steps can only be combined by \zenon,
hoping to find a complete proof, if it knows that they represent all
the possible cases of the function \lstinline"run_steps", knows that the type
\lstinline"state_t" only contains the values on which
\lstinline"run_steps" discriminates and finally knows that the
\lstinline"r" used in all steps goals is the result of calling
\lstinline"run_steps" (so is the resulting controller state), i.e. the
hypothesis \lstinline"h1".

As usual, we can compile the source and will see that \zenon\ finds the
proof and the whole theorem gets accepted by \coq. Removing one of
the facts provided in step \lstinline"<2>7" really causes the whole
proof to fail.

\medskip
It remains now to refine our proof by removing all the
\lstinline"assumed" we set to ``prove'' intermediate steps 
\lstinline"<2>1" to \lstinline"<2>6". In each case, to prove that a
signal is not green, we simply prove it has an effective other color
value. From this exhibited value (obviously not being
\lstinline"C_green") and the definition of the type
\lstinline"color_t", \zenon\ can establish that -- this type being an
inductive definition -- all its constructors are different 2 by 2. In
other words, the fact that \lstinline"C_red" is not equal to
\lstinline"C_green" requires \zenon\ to know the underlying type
definition. For this reason, each proof requires the exhibition of the
computed color (\lstinline"step<3>1") and the type
\lstinline"color_t".

The way the proof exhibiting the effective color
value (the one different from green) works is still left assumed,
hence following our refinement tactic.

{\scriptsize
\begin{lstlisting}
(** Lemma stating that s1 and s2 are never green together. It's 1/3 of the
    final property stating that no signals are green at the same time. *)
theorem never_s1_s2_green :
  all s r : Self,
  r = run_step (s) ->
  ~ (get_s1 (r) = C_green /\ get_s2 (r) = C_green)
proof =
<1>1 assume s : Self, r : Self,
     hypothesis h1 : r = run_step (s),
     prove ~ (get_s1 (r) = C_green /\ get_s2 (r) = C_green)

     (* Proof by cases on values of the "automaton state" of s.
        For each case, we will prove that one of the 2 signal at least is
        not green. *)
     <2>1 hypothesis h2: get_s (s) = S_grr,
          prove ~ (get_s1 (r) = C_green)
          (* To prove the signal s1 is not green, we prove it is orange. *)
          <3>1 prove get_s1 (r) = C_orange
               assumed
          <3>2 qed by step <3>1 type color_t

     (* Same proof kind for all the cases of automaton state. *)
     <2>2 hypothesis h3: get_s (s) = S_orr,
          prove ~ (get_s1 (r) = C_green)
          (* To prove the signal s1 is not green, we prove it is red. *)
          <3>1 prove get_s1 (r) = C_red
               assumed
          <3>2 qed by step <3>1 type color_t

     (* Same proof kind for all the cases of automaton state. *)
     <2>3 hypothesis h4: get_s (s) = S_rgr,
          prove ~ (get_s1 (r) = C_green)
          <3>1 prove get_s1 (r) = C_red
               assumed
          <3>2 qed by step <3>1 type color_t

     <2>4 hypothesis h5: get_s (s) = S_ror,
          prove ~ (get_s1 (r) = C_green)
          <3>1 prove get_s1 (r) = C_red
               assumed
          <3>2 qed by step <3>1 type color_t

     <2>5 hypothesis h6: get_s (s) = S_rrg,
          prove ~ (get_s1 (r) = C_green)
          <3>1 prove get_s1 (r) = C_red
               assumed
          <3>2 qed by step <3>1 type color_t

     <2>6 hypothesis h7: get_s (s) = S_rro,
          prove ~ (get_s2 (r) = C_green)
          <3>1 prove get_s2 (r) = C_red
               assumed
          <3>2 qed by step <3>1 type color_t

     <2>7 qed by
          step <2>1, <2>2, <2>3, <2>4, <2>5, <2>6
          definition of run_step
          hypothesis h1
          type state_t
<1>2 conclude ;
\end{lstlisting}}

Finally, once the compilation shown that this new refinement passes
\zenon\ searches and \coq\ assessment, it is time to complete the last
holes of the proof, the last \lstinline"assumed" remaining. Each such
case aims at proving that the value we chose and exhibited as being
different from \lstinline"C_green" is really the one computed by the
related call to \lstinline"get_s1 (r)". In other words, we need to
demonstrate that in the case \lstinline"<2>1<3>1", we really have
\lstinline"get_s1 (r) = C_orange" holding (and similarly for the other
cases).

One may be easily convinced that this is intrinsically due to
the way the function \lstinline"run_step" is written! But not only:
this is also due to the way \lstinline"get_s1" is written since it
appears in the goal to prove.
Moreover, each property holds in the context of the hypothesis
representing the examined case of the pattern-matching
\lstinline"match get_s (state) with" of \lstinline"run_step" : the
hypothesis \lstinline"h2" in the first case (\lstinline"h3" in the
second, \lstinline"h4" in the third, and so on). Finally, our
hypothesis deals with a value of type \lstinline"state_t" and our goal
with a value of type \lstinline"color_t". Hence \zenon\ will for sure
need to know about them!

Giving \zenon\ all these facts, we hope it will find a proof for each
case, which will really be the case. Hence our complete proof of the
initial lemma is:

{\scriptsize
\begin{lstlisting}
(** Lemma stating that s1 and s2 are never green together. It's 1/3 of the
    final property stating that no signals are green at the same time. *)
theorem never_s1_s2_green :
  all s r : Self,
  r = run_step (s) ->
  ~ (get_s1 (r) = C_green /\ get_s2 (r) = C_green)
proof =
<1>1 assume s : Self, r : Self,
     hypothesis h1 : r = run_step (s),
     prove ~ (get_s1 (r) = C_green /\ get_s2 (r) = C_green)

     (* Proof by cases on values of the "automaton state" of s.
        For each case, we will prove that one of the 2 signal at least is
        not green. *)
     <2>1 hypothesis h2: get_s (s) = S_grr,
          prove ~ (get_s1 (r) = C_green)
          (* To prove the signal s1 is not green we prove it is orange. *)
          <3>1 prove get_s1 (r) = C_orange
               by hypothesis h1, h2
                  definition of get_s1, run_step
                  type state_t, color_t
          <3>2 qed by step <3>1 type color_t

     (* Same proof kind for all the cases of automaton state. *)
     <2>2 hypothesis h3: get_s (s) = S_orr,
          prove ~ (get_s1 (r) = C_green)
          (* To prove the signal s1 is not green, we prove it is red. *)
          <3>1 prove get_s1 (r) = C_red
               by hypothesis h1, h3
                  definition of get_s1, run_step
                  type state_t, color_t
          <3>2 qed by step <3>1 type color_t

     (* Same proof kind for all the cases of automaton state. *)
     <2>3 hypothesis h4: get_s (s) = S_rgr,
          prove ~ (get_s1 (r) = C_green)
          <3>1 prove get_s1 (r) = C_red
               by hypothesis h1, h4
                  definition of get_s1, run_step
                  type state_t, color_t
          <3>2 qed by step <3>1 type color_t

     <2>4 hypothesis h5: get_s (s) = S_ror,
          prove ~ (get_s1 (r) = C_green)
          <3>1 prove get_s1 (r) = C_red
               by hypothesis h1, h5
                  definition of get_s1, run_step
                  type state_t, color_t
          <3>2 qed by step <3>1 type color_t

     <2>5 hypothesis h6: get_s (s) = S_rrg,
          prove ~ (get_s1 (r) = C_green)
          <3>1 prove get_s1 (r) = C_red
               by hypothesis h1, h6
                  definition of get_s1, run_step
                  type state_t, color_t
          <3>2 qed by step <3>1 type color_t

     <2>6 hypothesis h7: get_s (s) = S_rro,
          prove ~ (get_s2 (r) = C_green)
          <3>1 prove get_s2 (r) = C_red
               by hypothesis h1, h7
                  definition of get_s2, run_step
                  type state_t, color_t
          <3>2 qed by step <3>1 type color_t

     <2>7 qed by
          step <2>1, <2>2, <2>3, <2>4, <2>5, <2>6
          definition of run_step
          hypothesis h1
          type state_t
<1>2 conclude ;
\end{lstlisting}}


\subsubsection{Proving other lemmas : THE END}
We initially decided to split our main safety property
\lstinline"never_2_green" into 3 lemmas. We proved above the first of
them. All of them having an identical structure, their proofs will
obviously be strongly similar. Hence, we do not detail again their
proofs but provide the complete source file implementing our
controller.

{\scriptsize
\lstinputlisting[caption=controller.fcl (3)] {controller/controller_complete.fcl}}

\section{Conclusion}
This tutorial illustrated the way proofs can be carried out in \focal,
using \zenon\ to make them easier. We addressed here development much
more ``algorithm-oriented'' than other documents more oriented toward
``mathematical-modeling''.

We didn't used powerful modeling constructs of \focal\ to only
concentrate on hierarchical split of proofs, intermediate lemmas
stating and kinds of facts available to guide \zenon\ in its proofs
searches and in which case to use them.


\bibliographystyle{abbrv}
\bibliography{bibli}

\end{document}

