
\section{A first simple program}
After having seen how to write hierarchical proofs in \focal\ using
\zenon\ in the context of pretty ad-hoc properties, we will finally
apply previous technics on proving properties related to a (still very
simple) program.

We deliberately make no use of \focal\ advanced modeling features like
in\-he\-ri\-tan\-ce, parametrisation, incremental conception and refinement
mechanisms. We only consider a raw software model, obviously not
supporting evolution, but that's not the aim. More information on
these points can be found in [BLA].

\subsection{The goal}
We want to model a simplified traffic signals controller. The system will
be made of 3 signals with 3 states: green, orange and red. The
controller will alternatively make each signal becoming green along a
predefined sequence, making so that other signals are red. As any
usual signals, they turn orange before turning red. We can then simply
model the controller as a finite state automaton representing
cycling sequences (where \red{R} stands for red, \green{G} for green
and \orange{O} for orange, representing the state of each managed traffic
signal):
$$\green{G}\red{R}\red{R} \rightarrow \orange{O}\red{R}\red{R}
\rightarrow \red{R}\green{G}\red{R} \rightarrow \red{R}\orange{O}\red{R}
\rightarrow \red{R}\red{R}\green{G} \rightarrow \red{R}\red{R}�\orange{O}$$


\subsection{Modeling data structures}
Without surprise, to represent the color of a signal, we define a sum
type with 3 values:

{\scriptsize
\begin{lstlisting}
open "basics" ;;

(** Type of signals colors. *)
type color_t = | C_green | C_orange | C_red ;;
\end{lstlisting}}

Obviously, the automaton having 6 states, we need to define a sum type
with as many values. For readability, we name each case ``S\_''
followed by the corresponding signals color initials. For instance
\lstinline"S_orr" stands for {\em ``State where signal 1 is orange, signal
2 is red and signal 3 is red''}.

{\scriptsize
\begin{lstlisting}
(** Type of states the automaton can be. Simply named with letters corresponding
    to the colors of signal 1, 2 and 3. *)
type state_t = | S_grr | S_orr | S_rgr | S_ror | S_rrg| S_rro ;;
\end{lstlisting}}

Finally, the state of the controller will consists in the current
state of the automaton and the state of each signal. We then embed
the controller inside a species whose \lstinline"representation"
reflects this data structure.

{\scriptsize
\begin{lstlisting}
(** Species embedding the automaton controlling the signals colors changes. *)
species Controller =
  (* Need to encode tuples as nested pairs because of limitations of Coq and
     Zenon. *)
  representation = (state_t * (color_t * (color_t * color_t))) ;
end ;;
\end{lstlisting}}

One may note that instead of defining the \lstinline"representation"
as a 4-components tuple, we nested pairs up to have 4 components. The
reason is that currently \focal\ compiler and \zenon\ don't yet
transparently generalize pairs, hence making very difficult proofs to
be compiled to \coq. However, this do not reduce the expressivity of
the language: it only makes things a bit more cumbersome.

\subsection{The main algorithm}
The controller is now modeled as a transition function taking the
current state of the controller as input and returning the next
state. Roughly speaking, it will discriminate on the state of the
automaton (first component of the \lstinline"representation" which is
the state of the controller), then determine the new state as well as
the new states of the signals. Hence, the transition function \lstinline"run_step"
will have type \lstinline"Self -> Self".

Because we modelled the state of the controller as a tuple-like data
structure, we first define projection functions to access individual
components of the controller state (i.e. the automaton state and each
signal state -- color).

{\scriptsize
\lstinputlisting[caption=controller.fcl]{controller/controller_algo.fcl}}

We only defined the behavioral, computational aspects of our
controller: no pro\-per\-ties yet. However we can compile this program and
get a usable piece of software.

\subsection{Introducing the main property}

It is now time to {\em ``prove our program''}. Behind this unclear but
widely used expression is hidden the task of characterizing the safety
properties of a system, then prove they hold. In our very simple case,
one interesting property is that we never have 2 green signals at the
same time. Since we have 3 signals we will state this property as the
negation of 3 disjunctions, each stating 2 of the 3 signals are green:
$$
\begin{array}{clc}
\sim ( & (signal_1\ is\ green \logand signal_2\ is\ green)\ \logor & \\
          & (signal_1\ is\ green \logand signal_3\ is\ green)\ \logor & \\
          & (signal_2\ is\ green \logand signal_3\ is\ green) & )
\end{array}
$$

Such a property leads to the following \focal\ theorem, still left
unproved for the moment:

{\scriptsize
\begin{lstlisting}
  (** The complete theorem stating that no signals are green at the same time. *)
  theorem never_2_green :
    all s r : Self,
    r = run_step (s) ->
    ~ ((get_s1 (r) = C_green /\ get_s2 (r) = C_green) \/
       (get_s1 (r) = C_green /\ get_s3 (r) = C_green) \/
       (get_s2 (r) = C_green /\ get_s3 (r) = C_green))
  proof = assumed ;
\end{lstlisting}}

\subsection{Making the proof}

It is now time to prove our theorem. One sketch of the proof is to
prove that we never have $signal_1$ and $signal_2$ green at the same
time, neither $signal_1$ and $signal_3$ nor $signal_2$ and $signal_3$.
From these 3 properties, \zenon\ should be able to find the remaining
``glue''  and prove the whole theorem!

We then just try to see if our intuition is right. We define the 3
intermediate lemmas \lstinline"never_s1_s2_green",
\lstinline"never_s1_s3_green" and \lstinline"never_s2_s3_green", let them
unproved for the moment, and ask \zenon\ to prove our main theorem
\lstinline"never_2_green" \lstinline"by property ..." our 3 lemmas:

{\scriptsize
\lstinputlisting[caption=controller.fcl (2)]{controller/controller1.fcl}}

We invoke the compilation by the regular command:
\textbf{focalizec controller.fcl}:

{\scriptsize
\begin{verbatim}
Invoking ocamlc...
>> /usr/local/bin/ocamlc -I /usr/local/lib/focalize -c controller.ml
Invoking zvtov...
>> /usr/local/bin/zvtov -zenon /usr/local/bin/zenon -new controller.zv
Invoking coqc...                                                
>> /usr/local/bin/coqc  -I /usr/local/lib/focalize  -I /usr/local/lib/zenon controller.v
\end{verbatim}}

\noindent and see that our proof passed, assuming our 3 pending
lemmas. It will then be time to actually prove these lemmas. One
imagine easily that their proofs will be similar, since the only
change between statements is the involved signals.


\subsubsection{Proving the first lemma}
We will now address proving the first lemma, namely
\lstinline"never_s1_s2_green", using the incremental approach we
previously introduced: setting-up the proof sketch, the main
intermediate steps with their goal to prove left assumed, then
refining these steps until nothing remains assumed. Obviously, our
lemma won't be fully automatically proved by one \zenon\ step since
its statement is too complex. Hence, we forget a proof of the shape:

{\scriptsize
\begin{lstlisting}
theorem never_s1_s2_green :
  all s r : Self,
  r = run_step (s) ->
  ~ (get_s1 (r) = C_green /\ get_s2 (r) = C_green)
proof = by definition of ... type ... step ... hypothesis ... ;
\end{lstlisting}}

\noindent and prepare us to write a hierarchical one, whose first step is
the simple introduction of hypotheses of our theorem, leaving its goal
to prove (i.e. currently left \lstinline"assumed"):

{\scriptsize
\begin{lstlisting}
theorem never_s1_s2_green :
  all s r : Self,
  r = run_step (s) ->
  ~ (get_s1 (r) = C_green /\ get_s2 (r) = C_green)
proof =
<1>1 assume s : Self, r : Self,
     hypothesis h1 : r = run_step (s),
     prove ~ (get_s1 (r) = C_green /\ get_s2 (r) = C_green)
     assumed
<1>2 conclude ;
\end{lstlisting}}

The sketch of the proof is a study by cases on the values of the
automaton state, showing that in each state, the resulting state of
the signals 1 and 2 is never green for both (i.e. there is never 2
\lstinline"C_green" values in the 2\textsuperscript{nd} and
3\textsuperscript{rd} components of the controller state). 

How can we prove this ? Simply by exhibiting, in each case, that the result
contains at least one color not equal to
\lstinline"C_green". Obviously, for each case we chose to target the
signal whose value is really not green! Hence, we refine our proof and
state each case of the proof, as many as there are states in the
automaton, hence as many as there are cases in the transition function
\lstinline"run_step". In the first case, no signal is green since
$signal_1$ is orange, and $signal_2$ is red: we chose to prove that 
$signal_1$ is not green. Conversely, in the second case, $signal_2$ is
green: we do not have the choice and must prove that $signal_1$ is not.

{\scriptsize
\begin{lstlisting}
(** Lemma stating that s1 and s2 are never green together. It's 1/3 of the
    final property stating that no signals are green at the same time. *)
theorem never_s1_s2_green :
  all s r : Self,
  r = run_step (s) ->
  ~ (get_s1 (r) = C_green /\ get_s2 (r) = C_green)
proof =
<1>1 assume s : Self, r : Self,
     hypothesis h1 : r = run_step (s),
     prove ~ (get_s1 (r) = C_green /\ get_s2 (r) = C_green)

     (* Proof by cases on values of the "automaton state" of s.
        For each case, we will prove that one of the 2 signal at least is
        not green. *)
     <2>1 hypothesis h2: get_s (s) = S_grr,
          prove ~ (get_s1 (r) = C_green)
          assumed

     (* Same proof kind for all the cases of automaton state. *)
     <2>2 hypothesis h3: get_s (s) = S_orr,
          prove ~ (get_s1 (r) = C_green)
          assumed

     (* Same proof kind for all the cases of automaton state. *)
     <2>3 hypothesis h4: get_s (s) = S_rgr,
          prove ~ (get_s1 (r) = C_green)
          assumed

     <2>4 hypothesis h5: get_s (s) = S_ror,
          prove ~ (get_s1 (r) = C_green)
          assumed

     <2>5 hypothesis h6: get_s (s) = S_rrg,
          prove ~ (get_s1 (r) = C_green)
          assumed

     <2>6 hypothesis h7: get_s (s) = S_rro,
          prove ~ (get_s2 (r) = C_green)
          assumed

     <2>7 qed by
          step <2>1, <2>2, <2>3, <2>4, <2>5, <2>6
          definition of run_step
          hypothesis h1
          type state_t
<1>2 conclude ;
\end{lstlisting}}

The conclusion of our proof is step \lstinline"<2>7" and obviously
relies on the 6 preceding steps, but also on the definition of the function
\lstinline"run_steps", the type \lstinline"state_t" and the hypothesis
\lstinline"h1: r = run_step (s)".

In effect, the intermediate steps can only be combined by \zenon,
hoping to find a complete proof, if it knows that they represent all
the possible cases of the function \lstinline"run_steps", knows that the type
\lstinline"state_t" only contains the values on which
\lstinline"run_steps" discriminates and finally knows that the
\lstinline"r" used in all steps goals is the result of calling
\lstinline"run_steps" (so is the resulting controller state), i.e. the
hypothesis \lstinline"h1".

As usual, we can compile the source and will see that \zenon\ finds the
proof and the whole theorem gets accepted by \coq. Removing one of
the facts provided in step \lstinline"<2>7" really causes the whole
proof to fail.

\medskip
It remains now to refine our proof by removing all the
\lstinline"assumed" we set to ``prove'' intermediate steps 
\lstinline"<2>1" to \lstinline"<2>6". In each case, to prove that a
signal is not green, we simply prove it has an effective other color
value. From this exhibited value (obviously not being
\lstinline"C_green") and the definition of the type
\lstinline"color_t", \zenon\ can establish that -- this type being an
inductive definition -- all its constructors are different 2 by 2. In
other words, the fact that \lstinline"C_red" is not equal to
\lstinline"C_green" requires \zenon\ to know the underlying type
definition. For this reason, each proof requires the exhibition of the
computed color (\lstinline"step<3>1") and the type
\lstinline"color_t".

The way the proof exhibiting the effective color
value (the one different from green) works is still left assumed,
hence following our refinement tactic.

{\scriptsize
\begin{lstlisting}
(** Lemma stating that s1 and s2 are never green together. It's 1/3 of the
    final property stating that no signals are green at the same time. *)
theorem never_s1_s2_green :
  all s r : Self,
  r = run_step (s) ->
  ~ (get_s1 (r) = C_green /\ get_s2 (r) = C_green)
proof =
<1>1 assume s : Self, r : Self,
     hypothesis h1 : r = run_step (s),
     prove ~ (get_s1 (r) = C_green /\ get_s2 (r) = C_green)

     (* Proof by cases on values of the "automaton state" of s.
        For each case, we will prove that one of the 2 signal at least is
        not green. *)
     <2>1 hypothesis h2: get_s (s) = S_grr,
          prove ~ (get_s1 (r) = C_green)
          (* To prove the signal s1 is not green, we prove it is orange. *)
          <3>1 prove get_s1 (r) = C_orange
               assumed
          <3>2 qed by step <3>1 type color_t

     (* Same proof kind for all the cases of automaton state. *)
     <2>2 hypothesis h3: get_s (s) = S_orr,
          prove ~ (get_s1 (r) = C_green)
          (* To prove the signal s1 is not green, we prove it is red. *)
          <3>1 prove get_s1 (r) = C_red
               assumed
          <3>2 qed by step <3>1 type color_t

     (* Same proof kind for all the cases of automaton state. *)
     <2>3 hypothesis h4: get_s (s) = S_rgr,
          prove ~ (get_s1 (r) = C_green)
          <3>1 prove get_s1 (r) = C_red
               assumed
          <3>2 qed by step <3>1 type color_t

     <2>4 hypothesis h5: get_s (s) = S_ror,
          prove ~ (get_s1 (r) = C_green)
          <3>1 prove get_s1 (r) = C_red
               assumed
          <3>2 qed by step <3>1 type color_t

     <2>5 hypothesis h6: get_s (s) = S_rrg,
          prove ~ (get_s1 (r) = C_green)
          <3>1 prove get_s1 (r) = C_red
               assumed
          <3>2 qed by step <3>1 type color_t

     <2>6 hypothesis h7: get_s (s) = S_rro,
          prove ~ (get_s2 (r) = C_green)
          <3>1 prove get_s2 (r) = C_red
               assumed
          <3>2 qed by step <3>1 type color_t

     <2>7 qed by
          step <2>1, <2>2, <2>3, <2>4, <2>5, <2>6
          definition of run_step
          hypothesis h1
          type state_t
<1>2 conclude ;
\end{lstlisting}}

Finally, once the compilation shown that this new refinement passes
\zenon\ searches and \coq\ assessment, it is time to complete the last
holes of the proof, the last \lstinline"assumed" remaining. Each such
case aims at proving that the value we chose and exhibited as being
different from \lstinline"C_green" is really the one computed by the
related call to \lstinline"get_s1 (r)". In other words, we need to
demonstrate that in the case \lstinline"<2>1<3>1", we really have
\lstinline"get_s1 (r) = C_orange" holding (and similarly for the other
cases).

One may be easily convinced that this is intrinsically due to
the way the function \lstinline"run_step" is written! But not only:
this is also due to the way \lstinline"get_s1" is written since it
appears in the goal to prove.
Moreover, each property holds in the context of the hypothesis
representing the examined case of the pattern-matching
\lstinline"match get_s (state) with" of \lstinline"run_step" : the
hypothesis \lstinline"h2" in the first case (\lstinline"h3" in the
second, \lstinline"h4" in the third, and so on). Finally, our
hypothesis deals with a value of type \lstinline"state_t" and our goal
with a value of type \lstinline"color_t". Hence \zenon\ will for sure
need to know about them!

Giving \zenon\ all these facts, we hope it will find a proof for each
case, which will really be the case. Hence our complete proof of the
initial lemma is:

{\scriptsize
\begin{lstlisting}
(** Lemma stating that s1 and s2 are never green together. It's 1/3 of the
    final property stating that no signals are green at the same time. *)
theorem never_s1_s2_green :
  all s r : Self,
  r = run_step (s) ->
  ~ (get_s1 (r) = C_green /\ get_s2 (r) = C_green)
proof =
<1>1 assume s : Self, r : Self,
     hypothesis h1 : r = run_step (s),
     prove ~ (get_s1 (r) = C_green /\ get_s2 (r) = C_green)

     (* Proof by cases on values of the "automaton state" of s.
        For each case, we will prove that one of the 2 signal at least is
        not green. *)
     <2>1 hypothesis h2: get_s (s) = S_grr,
          prove ~ (get_s1 (r) = C_green)
          (* To prove the signal s1 is not green, we prove it is orange. *)
          <3>1 prove get_s1 (r) = C_orange
               by hypothesis h1, h2
                  definition of get_s1, run_step
                  type state_t, color_t
          <3>2 qed by step <3>1 type color_t

     (* Same proof kind for all the cases of automaton state. *)
     <2>2 hypothesis h3: get_s (s) = S_orr,
          prove ~ (get_s1 (r) = C_green)
          (* To prove the signal s1 is not green, we prove it is red. *)
          <3>1 prove get_s1 (r) = C_red
               by hypothesis h1, h3
                  definition of get_s1, run_step
                  type state_t, color_t
          <3>2 qed by step <3>1 type color_t

     (* Same proof kind for all the cases of automaton state. *)
     <2>3 hypothesis h4: get_s (s) = S_rgr,
          prove ~ (get_s1 (r) = C_green)
          <3>1 prove get_s1 (r) = C_red
               by hypothesis h1, h4
                  definition of get_s1, run_step
                  type state_t, color_t
          <3>2 qed by step <3>1 type color_t

     <2>4 hypothesis h5: get_s (s) = S_ror,
          prove ~ (get_s1 (r) = C_green)
          <3>1 prove get_s1 (r) = C_red
               by hypothesis h1, h5
                  definition of get_s1, run_step
                  type state_t, color_t
          <3>2 qed by step <3>1 type color_t

     <2>5 hypothesis h6: get_s (s) = S_rrg,
          prove ~ (get_s1 (r) = C_green)
          <3>1 prove get_s1 (r) = C_red
               by hypothesis h1, h6
                  definition of get_s1, run_step
                  type state_t, color_t
          <3>2 qed by step <3>1 type color_t

     <2>6 hypothesis h7: get_s (s) = S_rro,
          prove ~ (get_s2 (r) = C_green)
          <3>1 prove get_s2 (r) = C_red
               by hypothesis h1, h7
                  definition of get_s2, run_step
                  type state_t, color_t
          <3>2 qed by step <3>1 type color_t

     <2>7 qed by
          step <2>1, <2>2, <2>3, <2>4, <2>5, <2>6
          definition of run_step
          hypothesis h1
          type state_t
<1>2 conclude ;
\end{lstlisting}}


\subsubsection{Proving other lemmas : THE END}
We initially decided to split our main safety property
\lstinline"never_2_green" into 3 lemmas. We proved above the first of
them. All of them having an identical structure, their proofs will
obviously be strongly similar. Hence, we do not detail again their
proofs but provide the complete source file implementing our
controller.

{\scriptsize
\lstinputlisting[caption=controller.fcl (3)] {controller/controller_complete.fcl}}
